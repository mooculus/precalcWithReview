\documentclass{ximera}

\input{../../preamble.tex}

\author{Elizabeth Campolongo}
\acknowledgement{https://www.stitz-zeager.com/szca07042013.pdf}

\begin{document}
\begin{exercise}

Find all exact solutions to the following equation.
%
$$3+\cot^2(\theta) = -3\csc(\theta).$$

\begin{enumerate}
\item First rewrite using trig identities: \smallskip\\
$3 + (\answer{\csc^2(\theta)} - \answer{1}) = -3\csc(\theta)$

\item \begin{exercise}
Now solve this quadratic equation in terms of $\csc(\theta)$. We can factor it as: \smallskip\\
$(\answer{\csc(\theta)} + 1)(\csc(\theta) + \answer{2}) = 0.$

\item \begin{exercise}
Now solve the two trig equations.
\begin{enumerate}
 \item $\csc(\theta) +1 = 0.$
 \begin{enumerate}
 \item The reference angle $\theta_R = \answer{\frac{\pi}{2}}$.
 
 \item The general solution is $\theta = \answer{\frac{3\pi}{2}} + 2\pi k$, for any integer $k$.
 \end{enumerate}
 
 \item $\csc(\theta) +2 = 0.$
 \begin{enumerate}
 \item The reference angle $\theta_R = \answer{\frac{\pi}{6}}$.
 
 \item The general solutions (from least to greatest) are \smallskip\\
 $\theta = \answer{\frac{7\pi}{6} + 2\pi k}$ and $\theta = \answer{\frac{11\pi}{6} + 2\pi k}$, for any integer $k$.
 \end{enumerate}
\end{enumerate}
\end{exercise}
\end{exercise}
\end{enumerate}

\end{exercise}
\end{document}