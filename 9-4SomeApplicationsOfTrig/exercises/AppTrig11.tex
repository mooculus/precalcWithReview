\documentclass{ximera}

\input{../../preamble.tex}

\author{Kenneth Berglund}
\acknowledgement{https://www.stitz-zeager.com/szca07042013.pdf}

\begin{document}
\begin{exercise}
The average daily temperature for a given month in Columbus, Ohio is given by $f(t)$ where $t$ is the $t$th month of the year. Assume the minimum average temperature for a given month is 30 degrees Fahrenheit, and this minimum occurs in January, the first month. Assume the maximum average temperature for a given month is 75 degrees Fahrenheit, and this maximum occurs in July, the seventh month. Assume the function $f$ is sinusoidal. 

The midline of $f$ is $\answer{\frac{105}{2}}$. The amplitude of $f$ is $\answer{22.5}$. The period of $f$ is $\answer{12}$. 

\begin{exercise}
Plugging the above information into the cosine function, we have $\answer{22.5}\cos\left(\answer{\frac{2\pi}{12}}(x - h)\right) + \answer{\frac{105}{2}}$.

\begin{exercise}
What should the value of $h$ be in the above function? That is, by how much should we shift $22.5\cos\left(\frac{2\pi}{12}x \right) + \frac{105}{2}$ to obtain $f$? $h = \answer{7}$
\end{exercise}

\end{exercise}

\end{exercise}
\end{document}
