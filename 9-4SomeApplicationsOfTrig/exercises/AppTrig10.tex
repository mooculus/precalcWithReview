\documentclass{ximera}

\input{../../preamble.tex}

\author{Kenneth Berglund}
\acknowledgement{https://www.stitz-zeager.com/szca07042013.pdf}

\begin{document}
\begin{exercise}
The fraction of the moon $f(t)$ illuminated at midnight on the $t$th day of the year (meaning $f(t) = 1$ for a full moon and $f(t) = 0$ for a new moon). For simplicity, we assume the lunar cycle is 28 days. Say the full moon occurs on January 4th. Assume the function $f$ is sinusoidal. 

The midline of $f$ is $\answer{.5}$. The amplitude of $f$ is $\answer{0.5}$. The period of $f$ is $\answer{28}$. 

\begin{exercise}
Plugging the above information into the cosine function, we have $\answer{0.5}\cos\left(\answer{\frac{2\pi}{28}}(x - h)\right) + \answer{0.5}$.

\begin{exercise}
What should the value of $h$ be in the above function? That is, by how much should we shift $\frac{1}{2}\cos\left(\frac{2\pi}{28}x \right) + \frac{1}{2}$ to obtain $f$? $h = \answer{4}$
\end{exercise}

\end{exercise}

\end{exercise}
\end{document}
