\documentclass[noauthor, nooutcomes]{ximera}

\input{../preamble}
\author{Kenneth Berglund}
\license{Creative Commons Attribution-ShareAlike 4.0 International License}
\acknowledgement{https://www.stitz-zeager.com/szca07042013.pdf}

\title{Solving Inequalities Graphically}

%This is so the pictures show up
\pgfplotsset{compat=1.5.1}

%This is also for pictures
\usetikzlibrary{calc}

%This makes hyperbolas. I found it at https://newbedev.com/can-one-draw-a-hyperbola-with-arguments-in-tikz
%
% #1 optional parameters for \draw
% #2 angle of rotation in degrees
% #3 offset of center as (pointx, pointy) or (name-o-coordinate)
% #4 length of plus (semi)axis, that is axis which hyperbola crosses
% #5 length of minus (semi)axis
% #6 how much of hyperbola to draw in degrees, with 90 you’d reach infinity
%
\newcommand\tikzhyperbola[6][thick]{%
    \draw [#1, rotate around={#2: (0, 0)}, shift=#3]
        plot [variable = \t, samples=1000, domain=-#6:#6] ({#4 / cos( \t )}, {#5 * tan( \t )});
    \draw [#1, rotate around={#2: (0, 0)}, shift=#3]
        plot [variable = \t, samples=1000, domain=-#6:#6] ({-#4 / cos( \t )}, {#5 * tan( \t )});
}

\begin{document}
\begin{abstract}
  
\end{abstract}
\maketitle
\licenseSZ

\begin{motivatingQuestions}\begin{itemize}
	\item What is a solution to an inequality?
	\item How can we use graphs to solve inequalities?
\end{itemize}\end{motivatingQuestions}

%\typeout{************************************************}
%\typeout{Introduction}
%\typeout{************************************************}
\section{Introduction}
Dabin and Melina are having a walking race. Dabin can walk 1 meter per second, but Melina can walk 2 meters per second. Since Melina is the faster walker, she gives Dabin a head start of 5 meters. At this point, we can ask a few questions about the race. Two questions we'll focus on are ``When is Dabin in the lead?'' and  ``When is Melina in the lead?'' both of which can be answered by considering inequalities. 

To start, let's define some relevant functions. The function $D$ defined by $D(t) = 5 + t$ represents how far (in meters) Dabin has walked $t$ seconds after the start of the race. Similarly, the function $M$ defined by $M(t) = 2t$ represents how far Melina has walked $t$ seconds after the start of the race. Here, asking when Dabin is in the lead is the same as asking for all $t$ such that $D(t) > M(t)$. In the vocabulary that we'll use, we want to solve the inequality $D(t) > M(t)$. 

\begin{definition}
Say $f$ and $g$ are functions. A \index{solution ! to an inequality}\dfn{solution} to the inequality $f(x) < g(x)$ is the set of $x$ values where $f(x) < g(x)$. Similarly, a solution to the inequality $f(x) > g(x)$ is the set of $x$ values where $f(x) > g(x)$.
\end{definition}

Note that we define a solution to an inequality as a set. We will often write the sets in interval notation. 

%%\typeout{************************************************}
%%\typeout{Solving inequalities graphically}
%%\typeout{************************************************}
\section{Solving inequalities graphically}
\begin{example}
Let $D$ be defined by $D(t) = 5 + t$, and $M$ be defined by $M(t) = 2t$. Find a solution to the inequality $D(t) < M(t)$. 
\end{example}
\begin{explanation}
This example asks us to find the set of $t$ values where $D(t) < M(t)$. One approach to inequalities of this form is to look at the graphs of the equations involved. The following figure shows the graphs of $y = D(t)$ and $y = M(t)$. 

\begin{image}
\begin{tikzpicture}
    \begin{axis}[xlabel=$t$,xmin=-1, xmax=10, ymin=-1, ymax=13, xtick={0, 1,..., 10},ytick={0, 1,..., 13}]
         \addplot[dashed, domain=-1:8, <->]{5 + x} node{};
	\addplot[domain=-0.5:6.5, <->]{2*x} node{}; 
	\node (d) at (axis cs:1.5,8) {$y=D(t)$};
	\node (m) at (axis cs:3.5,4) {$y=M(t)$};
	\addplot[penColor, mark=*, only marks] coordinates {(5, 10)};
	\node[left] (l) at (axis cs:5,10.2) {$(5, 10)$};
    \end{axis}
\end{tikzpicture}
\end{image}

Because of the way we draw the graphs of functions, if $D(t) < M(t)$ for some $y$ if and only if the graph of $D$ lies below the graph of $M$ at the point $t$. Using this information, we can see that if $t > 5$, then the graph of $D$ lies below the graph of $M$. Therefore, the set of all $t$ such that $t > 5$ is the solution to $D(t) < M(t)$. Writing this in interval notation, the solution is $(5, \infty)$. 

Putting this in terms of the scenario described at the beginning of the section, Melinda is in the lead after 5 seconds. 
\end{explanation}

\begin{exploration}
Find a solution to the inequality $D(t) < M(t)$. 
\end{exploration}

\begin{example}
Let $f$ and $g$ be functions whose graphs are shown below. Assume all important behavior of the functions is shown in the figure.

\begin{image}
\begin{tikzpicture}
    \begin{axis}[xmin=-3, xmax=3, ymin=-3, ymax=5, xtick={-2,1,..., 2},ytick={-2, -1,...,4}]
         \addplot[thick, domain=0:3, ->]{4-2*x} node{};
	\addplot[thick, domain=-3:0, <-]{4+2*x} node{};
	\addplot[domain=-1.55:1.55, <->, dashed]{2*x^2} node{}; 
	\node (g) at (axis cs:2,4) {$y=g(x)$};
	\node (f) at (axis cs:-1.5,-1) {$y=f(x)$};
	\addplot[penColor, mark=*, only marks] coordinates {(-1,2)};
	\addplot[penColor, mark=*, only marks] coordinates {(1,2)};
	\node[left] (l) at (axis cs:-1,2) {$(-1, 2)$};
	\node[right] (r) at (axis cs:1,2) {$(1,2)$};
    \end{axis}
\end{tikzpicture}
\end{image}
 
\begin{enumerate}
	\item Solve the inequality $f(x) < g(x)$.
	\item Solve the inequality $f(x) \ge g(x)$.
\end{enumerate}
\end{example}


\begin{explanation}
\begin{enumerate}
	\item  To solve $f(x) < g(x)$, we look for where the graph of $f$ is below the graph of $g$. This appears to happen for the $x$ values less than $-1$ and greater than $1$. Our solution is $(-\infty, -1) \cup (1,\infty)$.

	\item To solve $f(x) \geq g(x)$, we look for solutions to $f(x)=g(x)$ as well as $f(x) > g(x)$. To solve the former equation we can look at the $x$-coordinates of the intersection points. This yields $x = \pm 1$. To solve $f(x) > g(x)$, we look for where the graph of $f$ is above the graph of $g$.  This appears to happen between $x=-1$ and $x=1$, on the interval $(-1,1)$. Hence, our solution to $f(x) \geq g(x)$ is $[-1,1]$. 
\end{enumerate}
\end{explanation}

Now let's turn our attention to inequalities involving absolute values, which are often a source of confusion. The following theorem provides the complete story. As you read through the theorem, use the interactive figure below to help you interpret and make sense of each bullet point statement.

\begin{theorem}
Let $c$ be a real number.
\begin{itemize}

\item  For $c > 0$, $|x| < c$ is equivalent to $-c<x<c$.

\item  For $c > 0$, $|x| \leq c$ is equivalent to $-c \leq x \leq c$.

\item  For $c \leq 0$, $|x| < c$ has no solution, and for $c < 0$, $|x| \leq c$ has no solution.

\item  For $c \geq 0$, $|x| > c$ is equivalent to $x<-c$ or $x>c$.

\item  For $c \geq 0$, $|x| \geq c$ is equivalent to $x \leq -c$ or $x \geq c$.

\item  For $c < 0$, $|x| > c$ and $|x| \geq c$ are true for all real numbers.

\end{itemize}
\end{theorem}

In light of what we have developed in this section, we can understand these statements graphically. For instance, if $c > 0$, the graph of $y=c$ is a horizontal line which lies above the $x$-axis through $(0,c)$. To solve $|x| < c$, we are looking for the $x$ values where the graph of $y=|x|$ is below the graph of $y=c$. We know that the graphs intersect when $|x|=c$, which we know happens when $x=c$ or $x=-c$.  Graphing, we get

\begin{image}
\begin{tikzpicture}
    \begin{axis}[axis equal image, ymin=-1, ymax=7,ticks=none]
         \addplot[thick, domain=0:7, ->]{x} node{};
	\addplot[thick, domain=-7:0, <-]{(-1)*x} node{};
	\addplot[<->, dashed]{5} node{}; 
	\node (g) at (axis cs:3,5.5) {$y=c$};
	\node (f) at (axis cs:2,0.5) {$y=|x|$};
	\addplot[penColor, mark=*, only marks] coordinates {(5,5)};
	\addplot[penColor, mark=*, only marks] coordinates {(-5,5)};
	\node[left] (l) at (axis cs:-5,4.5) {$(-c,c)$};
	\node[right] (r) at (axis cs:5,4.5) {$(c,c)$};
    \end{axis}
\end{tikzpicture}
\end{image}

We see that the graph of $y=|x|$ is below $y=c$ for $x$ between $-c$ and $c$, and hence we get $|x| < c$ is equivalent to $-c < x < c$. The other properties in the theorem can be shown similarly. You can try changing the value of $c$ using Desmos.
\begin{center}  
\desmos{dbpb01aybm}{800}{600}  
\end{center}

\begin{example}
Solve the inequality $4 - 3|2x + 1| > -2$.
\end{example}
\begin{explanation}
Let's start by graphing both sides of the inequality on the same axes. 
\begin{image}
\begin{tikzpicture}
    \begin{axis}[xmin=-3,xmax=3,ymin=-5, ymax=5,xtick={-2,-1,...,2},ytick={-4,-3,...,4}]
         \addplot[thick, domain=-0.5:1, ->]{(-6)*x + 1} node{};
	\addplot[thick, domain=-2:-0.5, <-]{6*x + 7} node{};
	\addplot[domain=-3:3,<->, dashed]{-2} node{}; 
	\node (g) at (axis cs:2,-1.5) {$y = -2$};
	\node (f) at (axis cs:-1.75,3) {$y=4 - 3|2x + 1|$};
	\addplot[penColor, mark=*, only marks] coordinates {(0.5,-2)};
	\addplot[penColor, mark=*, only marks] coordinates {(-1.5,-2)};
	\node[left] (l) at (axis cs:-1.75,-3) {$\left(-\frac{3}{2}, -2\right)$};
	\node[right] (r) at (axis cs:0.75,-3) {$\left(\frac{1}{2}, -2\right)$};
    \end{axis}
\end{tikzpicture}
\end{image}

We see that the graph of $y=4-3|2x+1|$ is above $y=-2$ for $x$ values between $-\frac{3}{2}$ and $\frac{1}{2}$. Therefore, the solution in interval notation is $\left(-\frac{3}{2}, \frac{1}{2}\right)$.
\end{explanation}

\begin{example}
Solve the inequality $x^2 \le x$.
\end{example}
\begin{explanation}
We could start by graphing both sides of the inequality on the same graph, but here, we'll demonstrate another possible approach. Note that $x^2 \le x$ is equivalent to the inequality $x^2 - x \le 0$, by subtracting $x$ from both sides. Now, let's graph $y = x^2 - x$ and $y = 0$ on the same axes.

\begin{image}
\begin{tikzpicture}
    \begin{axis}[xmin=-3, xmax=4]
         \addplot[domain=-2.2:3.2, <->]{x^2 - x} node{};
	\addplot[<->, dashed, very thick]{0} node{}; 
	\node (g) at (axis cs:2,6) {$y=x^2 - x$};
	\node (f) at (axis cs:-2,0.5) {$y=0$};
	\addplot[penColor, mark=*, only marks] coordinates {(0,0)};
	\addplot[penColor, mark=*, only marks] coordinates {(1,0)};
	\node(l) at (axis cs:-.5,-1) {$(0,0)$};
	\node(r) at (axis cs:1,-1) {$(1,0)$};
    \end{axis}
\end{tikzpicture}
\end{image}

Notice that the two points of intersection are $(0, 0)$ and $(1, 0)$, so $x^2 - x = 0$ for $x = 0$ and $x = 1$. To find the solution to $x^2 - x \le < 0$, we can see that the graph of $y = x^2 - x$ lies below the graph of $y = 0$ between 0 and 1. Therefore, the solution is $[0, 1]$. 
\end{explanation}

The above example illustrates a common technique. Rather than considering two functions $f$ and $g$ and asking when one is greater than, less than, or equal to the other, we can move one function to the other side, and consider the function $f - g$. Now, the problem becomes one of finding when the function $f - g$ is positive, negative, or zero. 

\end{document}
