\documentclass{ximera}

\input{../../preamble.tex}

\author{Kenneth Berglund}
\acknowledgement{}

\begin{document}
Follow the steps below to solve the inequality $e^{-7x} \le 17$.
\begin{exercise}
An equivalent form of this inequality is $e^{-7x} - \answer{17} \le 0$. 

\begin{exercise}
The function $f$ defined by $f(x) = e^{-7x} - 17$ has $\answer{1}$ zero(s).

\begin{exercise}
The zero of $e^{-7x} - 17$ is $\answer{-\frac{\ln(17)}{7}}$.

\begin{exercise}
\begin{enumerate}
\item On the interval $\left(-\infty, -\frac{\ln(17)}{7}\right)$, $e^{-7x} - 17$ is 
\begin{multipleChoice}  
\choice[correct]{positive.}  
\choice{negative.} 
\end{multipleChoice} 

\item On the interval $\left(-\frac{\ln(17)}{7}, \infty\right)$, $e^{-7x} - 17$ is 
\begin{multipleChoice}  
\choice{positive.}  
\choice[correct]{negative.} 
\end{multipleChoice} 


\end{enumerate}

\begin{exercise}
The solution to the inequality $e^{-7x} \le 17$ in interval notation is $\left[\answer{-\frac{\ln(17)}{7}}, \answer{\infty}\right)$. Please write your interval(s) in increasing order. Use \verb+infty+ or \verb+oo+ if you need to enter the infinity symbol $\infty$.
\end{exercise}

\begin{exercise}
The solution to the inequality $e^{-7x} < 17$ in interval notation is $\left(\answer{-\frac{\ln(17)}{7}}, \answer{\infty}\right)$. Please write your interval(s) in increasing order. Use \verb+infty+ or \verb+oo+ if you need to enter the infinity symbol $\infty$.
\end{exercise}

\end{exercise}
\end{exercise}
\end{exercise}

\end{exercise}
\end{document}