\documentclass{ximera}

\input{../preamble}
\author{Ivo Terek}
\license{Creative Commons Attribution-ShareAlike 4.0 International License}
\acknowledgement{}

\title{Secant, Cosecant and Cotangent}

\begin{document}
\begin{abstract}
  
\end{abstract}
\maketitle


%\typeout{************************************************}
%\typeout{Motivating Questions}
%\typeout{************************************************}

\begin{motivatingQuestions}\begin{itemize}
\item How to use the reciprocal ratios of $\sin\theta$, $\cos\theta$, and $\tan\theta$, to also obtain information about a given right triangle?
\item What are the values of such reciprocal rations for the standard angles of $30^\circ$, $45^\circ$ and $60^\circ$?
\end{itemize}\end{motivatingQuestions}


%\typeout{************************************************}
%\typeout{Subsection Introduction}
%\typeout{************************************************}

\section{Introduction}

Briefly speaking, we have met three fundamental ratios between sides of a right triangle: sine, cosine, and tangent. But their reciprocals are also relevant ratios between the sides of the given triangle. Now, while such reciprocals ratios turn out to carry the same information as sine, cosine, and tangent, it is useful to know how to manipulate them as well. Later, when we study trigonometric functions as actual functions of a real parameter, discussing their graphs, symmetries, etc., more differences will become apparent.

%\typeout{************************************************}
%\typeout{Summary}
%\typeout{************************************************}

\section{Definitions and examples}

\begin{callout}
  {\bf Definition (right triangle trig -- bis):} Consider the following right triangle, with one angle $\theta$ indicated, and sides labeled $a$, $b$ and $c$.

  \begin{center}
    \includegraphics[scale=.3]{/figures/9-1-1-defn.png}
  \end{center}

  
  We define the {\bf secant}, {\bf cosecant}, and {\bf cotangent} of $\theta$, by
  \begin{itemize}
  \item $\displaystyle{\sec\theta = \frac{c}{b} = \frac{1}{\cos\theta} \left(= \frac{\rm hyp.}{\rm adj.}\right)}$;
  \item $\displaystyle{\csc\theta = \frac{c}{a}=\frac{1}{\sin\theta}\left(=\frac{\rm hyp.}{\rm opp.}\right)}$, and;
  \item $\displaystyle{\cot\theta = \frac{a}{b}=\frac{1}{\tan\theta} \left(= \frac{\rm adj.}{\rm opp.}\right)}$.
  \end{itemize}
\end{callout}

Note that since for acute angles we always have $\sin\theta$ and $\cos\theta$ between $0$ and $1$, the reciprocals $\csc\theta$ and $\sec\theta$ will always be bigger than $1$.


\begin{example}
  For each of the following triangles with a given angle $\theta$, identify the adjacent (adj.), opposite (opp.) and hypotenuse (hyp.), and compute $\sec \theta$, $\csc \theta$, and $\cot\theta$.
  \begin{enumerate}[label=\alph*.]
  \item  \begin{figure}[h]
      \centering
      \includegraphics[scale=.3]{/figures/9-1-2-triangle-15-8-17.png}
    \end{figure}
    \begin{explanation}
      We have ${\rm opp.} = 8$, ${\rm adj.} = 15$ and ${\rm hyp.} = 17$. This means that $$\sec\theta = \frac{17}{15}, \quad \csc \theta =\frac{17}{8}, \quad\mbox{and}\quad \cot\theta =\frac{15}{8}. $$Of course, you can find $\cos \theta$, $\sin \theta$, and $\tan\theta$ first, and then just flip all the fractions.
    \end{explanation}
  \item \begin{figure}[h]
      \centering
      \includegraphics[scale=.3]{/figures/9-1-2-triangle-20-21-29.png}
    \end{figure}
    \begin{explanation}
      This time, we have ${\rm opp.} = 21$, ${\rm adj.} = 20$ and ${\rm hyp.} = 29$. This means that $$\sec\theta = \frac{29}{20}, \quad \csc \theta =\frac{29}{21}, \quad\mbox{and}\quad \cot\theta =\frac{20}{21}. $$
    \end{explanation}
  \end{enumerate}
\end{example}

Next, we had the fundamental identity $\sin^2\theta + \cos^2\theta = 1$. It turns out that with this, we may obtain two extra useful identities.

\begin{callout}
  {\bf Theorem (Fundamental Identities -- bis):} For any given angle $\theta$, we have that $$1 + \cot^2\theta = \csc^2\theta\quad\mbox{and}\quad \tan^2\theta + 1 = \sec^2\theta,$$where $\cot^2\theta$ means $(\cot \theta)^2$, and similarly for all other functions.
\end{callout}

You should \emph{not} think of those two extra identities as something more to be memorized. The only identity worth the trouble is $\sin^2\theta + \cos^2\theta = 1$. The following strategy is something you can quickly reproduce on a scrap paper if you need to recall these formulas, once you have understood the idea once:

\begin{itemize}
\item Divide both sides of $\sin^2\theta+\cos^2\theta = 1$ by $\sin^2\theta$: $$\frac{\sin^2\theta+\cos^2\theta}{\sin^2\theta} = \frac{1}{\sin^2\theta} \implies1 + \frac{\cos^2\theta}{\sin^2\theta} = \csc^2\theta \implies 1+\cot^2\theta = \csc^2\theta.$$
\item Divide both sides of $\sin^2\theta+\cos^2\theta = 1$ by $\cos^2\theta$:  $$\frac{\sin^2\theta+\cos^2\theta}{\cos^2\theta} = \frac{1}{\cos^2\theta} \implies \frac{\sin^2\theta}{\cos^2\theta} + 1 = \sec^2\theta \implies \tan^2\theta+1=\sec^2\theta.$$
\end{itemize}

\begin{example}
  For each of the given triangles, given the value of a trigonometric function at the indicated angle $\theta$, find the lengths of the missing sides.
  \begin{enumerate}[label=\alph*.]
  \item Given: $\csc\theta = \sqrt{11}/2$ on
    \begin{figure}[h]
      \centering
      \includegraphics[scale=.3]{/figures/9-1-2-triangle-3sqrt11.png}
    \end{figure}
    
    \begin{explanation}
      We start with $$\frac{3\sqrt{11}}{a} = \csc\theta = \frac{\sqrt{11}}{2} \implies 6\sqrt{11} = a\sqrt{11} \implies a = 6.$$It remains to find the value of $b$. This can be done with the Pythagorean Theorem, as follows: $a^2+b^2=c^2$ becomes  $$36 + b^2 = (3\sqrt{11})^2 \implies 36+b^2 = 99 \implies b^2= 63 \implies b = 3\sqrt{7}.$$
    \end{explanation}
  \item Given: $\cot\theta = 2\sqrt{2}/5$ on     \begin{figure}[h]
      \centering
      \includegraphics[scale=.3]{/figures/9-1-2-triangle-4sqrt33.png}
    \end{figure}

    
    \begin{explanation}
      Let's start again using the trigonometric function we were given: $$\frac{a}{b} = \cot\theta = \frac{2\sqrt{2}}{5} \implies a= \frac{2b\sqrt{2}}{5}.$$We cannot conclude anything else about $a$ and $b$ just from this, so we must resort to the Pythagorean Theorem again. The relation $a^2+b^2=c^2$ gives us that $$\left(\frac{2b\sqrt{2}}{5}\right)^2+b^2 = (4\sqrt{33})^2 \implies \frac{8b^2}{25}+b^2 = 528 \implies \frac{33b^2}{25} = 528.$$Simplifying this, we have that $$b^2 = 25\times \frac{528}{33} = 25\times 16 \implies b = 5\times 4 \implies b=20.$$Now, we may go back and find $a$: $$a = \frac{2b\sqrt{2}}{5}= \frac{40\sqrt{2}}{5} \implies a = 8\sqrt{2}.$$
    \end{explanation}
  \end{enumerate}
\end{example}


\section{Values of trig functions for standard angles -- bis}

Previously, we have obtained the following table of standard values for sine, cosine, and tangent:

$$
\begin{array}{c||ccccc}
 & 0^\circ & 30^\circ & 45^\circ & 60^\circ & 90^\circ \\
\hline\hline \\[-1em]  
\sin \theta & 0 & \frac{1}{2} & \frac{\sqrt{2}}{2} & \frac{\sqrt{3}}{2} & 1 \\[1em]
 \cos \theta & 1 & \frac{\sqrt{3}}{2} & \frac{\sqrt{2}}{2} & \frac{1}{2} & 0 \\[1em]
\tan\theta& 0 & \frac{\sqrt{3}}{3} & 1 & \sqrt{3} & {\rm DNE}
\end{array}
$$

By simply inverting all of those values, we naturally obtain a similar table with standard values for secant, cosecant and cotangent. Of course, this is \emph{not} another table you have to memorize, but we'll list it here for completeness:

$$
\begin{array}{c||ccccc}
 & 0^\circ & 30^\circ & 45^\circ & 60^\circ & 90^\circ \\
\hline\hline \\[-1em]  
\csc \theta & {\rm DNE} & 2 & \sqrt{2} & \frac{2\sqrt{3}}{3} & 1 \\[1em]
 \sec \theta & 1 & \frac{2\sqrt{3}}{3} & \sqrt{2} & 2 & {\rm DNE} \\[1em]
\cot\theta& {\rm DNE} & \sqrt{3} & 1 & \frac{1}{\sqrt{3}} & 0
\end{array}
$$

Note the undefined ``extreme'' values: $\csc(0^\circ)$ is undefined, because that would be $1/\sin(0^\circ)$, but $\sin(0^\circ) = 0$ and we cannot divide by zero. Similarly for the other ones.


\begin{summary}\begin{itemize}
\item We have defined secant, cosecant, and cotangent, as the reciprocal ratios of cosine, sine, and tangent. For each angle $\theta$, we have the associated fundamental identities $1+\tan^2\theta=\sec^2\theta$ and $\cot^2\theta + 1 = \csc^2\theta$, which can be easily deduced from the good old $\sin^2\theta+\cos^2\theta=1$. Again, such identities can be used together with the Pythagorean Theorem to obtain information about sides of a right triangle.
\item We have summarized (again in a table) the standard values of secant, cosecant, and cotangent, for the most frequent angles of $30^\circ$, $45^\circ$, and $60^\circ$.
\end{itemize}\end{summary}




\end{document}
