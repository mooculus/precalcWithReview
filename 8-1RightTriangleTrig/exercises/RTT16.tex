\documentclass{ximera}

\input{../../preamble.tex}

\author{Ivo Terek}
\license{Creative Commons Attribution-ShareAlike 4.0 International License}

%\outcome{Calculating the rate of change.}
%\outcome{Discuss the meaning of antiderivatives of a position function.}

\begin{document}
\begin{exercise}

  If $\theta$ is the acute angle for which $\tan\theta=\sqrt{5}$, then the value of the expression $\sin\theta\sec\theta+\cot\theta$ is:

  \begin{multipleChoice}
    \choice{$\frac{4\sqrt{5}}{5}$}
    \choice{$\frac{1+\sqrt{5}}{5}$}
    \choice{$\sqrt{5}$}
    \choice[correct]{$\frac{6\sqrt{5}}{5}$}
    \choice{$\frac{6+\sqrt{5}}{5}$}
  \end{multipleChoice}

  Hint: there is a very quick way to solve this one without relying on triangles or any fundamental trigonometric identities. Can you see how?
\end{exercise}
\end{document}