\documentclass{ximera}

\input{../../preamble.tex}

\author{Kenneth Berglund}
\license{Creative Commons Attribution-ShareAlike 4.0 International License}


\begin{document}
\begin{exercise}

  If $\theta$ is the acute angle for which $\cos(\theta)=\frac{1}{5}$, then 

	\begin{enumerate}
		\item $\sin(\theta) = \answer{\frac{2\sqrt{6}}{5}}$
		\item $\tan(\theta) = \answer{2\sqrt{6}}$
		\item $\csc(\theta) = \answer{\frac{5}{2\sqrt{6}}}$
		\item $\sec(\theta) = \answer{5}$
		\item $\cot(\theta) = \answer{\frac{1}{2\sqrt{6}}}$
	\end{enumerate}

\end{exercise}

\begin{exercise}

  If $\theta$ is the acute angle for which $\sin(\theta)=\frac{2}{3}$, then 

	\begin{enumerate}
		\item $\cos(\theta) = \answer{\frac{\sqrt{5}}{3}}$
		\item $\tan(\theta) = \answer{\frac{2}{\sqrt{5}}}$
		\item $\csc(\theta) = \answer{\frac{3}{2}}$
		\item $\sec(\theta) = \answer{\frac{3}{\sqrt{5}}}$
		\item $\cot(\theta) = \answer{\frac{\sqrt{5}}{2}}$
	\end{enumerate}

\end{exercise}

\begin{exercise}

  If $\theta$ is the acute angle for which $\tan(\theta)= 2$, then 

	\begin{enumerate}
		\item $\sin(\theta) = \answer{\frac{2}{\sqrt{5}}}$
		\item $\cos(\theta) = \answer{\frac{1}{\sqrt{5}}}$
		\item $\csc(\theta) = \answer{\frac{\sqrt{5}}{2}}$
		\item $\sec(\theta) = \answer{\sqrt{5}}$
		\item $\cot(\theta) = \answer{\frac{1}{2}}$
	\end{enumerate}

\end{exercise}


\begin{exercise}

  If $\theta$ is the acute angle for which $\csc(\theta)= 6$, then 

	\begin{enumerate}
		\item $\sin(\theta) = \answer{\frac{1}{6}}$
		\item $\cos(\theta) = \answer{\frac{\sqrt{35}}{6}}$
		\item $\tan(\theta) = \answer{\frac{1}{\sqrt{35}}}$
		\item $\sec(\theta) = \answer{\frac{6}{\sqrt{35}}}$
		\item $\cot(\theta) = \answer{\sqrt{35}}$
	\end{enumerate}

\end{exercise}
\end{document}