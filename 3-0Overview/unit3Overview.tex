\documentclass{ximera}

\input{../preamble}
\author{Elizabeth Miller}
\license{Creative Commons Attribution-ShareAlike 4.0 International License}



\title{Unit 2: Lines and Exponentials}

\begin{document}
\begin{abstract}
\end{abstract}
\maketitle

\begin{overview}
\item
%\item Understanding Lines %Elizabeth 
%	\begin{enumerate}
%	\item Finding Linear Equations from tables
%	\item Slopes
%	\item Forms of a Line
%
%		\textit{Point-Slope Form, Slope-Intercept Form, Standard Form} 
%
%	\end{enumerate}
%	
%\item Linear Models %David
%	\begin{enumerate}
%	\item 
%	\end{enumerate}
%
%\item Exponential Models %David
%	\begin{enumerate}
%	\item 
%	\end{enumerate}
%
%\item Exponential Equations %(Elizabeth) 
%	\begin{enumerate}
%	\item Properties of Exponents 
%	\item Multiple bases, change of bases ($y = a*R^{x}$ to $y = a*e^{bx}$ )
%
%	\item End behavior of Exponential Functions 
%	\item Positive vs Negative Exponentials 
%	\end{enumerate} 

\end{overview}


\begin{objectives}
\item
%\item Excursions in Mathematics
%	\begin{itemize}
%	\item Linear  
%	\item Exponentials 
%	\end{itemize}
%\item Deep understanding in multiple representations (graphs, tables, equations, words, applications, data)  
%	\begin{itemize}
%	\item Number sense to Symbol (expression?) sense to Function sense. 
%	\item Recognize functions in everyday life and appreciate the ubiquity of functions 
%	\item Taking word problem and representing it mathematically; making sense of the situation rather than just looking for words that might indicate operation. 
%
%		\textit{Using graph to describe what is happening in the real world situation} 
%	\item choosing an appropriate function type for data?  Either from graphical representation, or from a description of what the data is representing. 
%
%		\textit{Finding and describing patterns from pictures and numbers toward symbols } 
%	\end{itemize}
%\item Familiarity with Famous Functions (tables and graphs ONLY, give expressions maybe but not expected to work with algebraically) 
%	\begin{itemize}
%	\item Linear 
%	\item Exponential 
%	\item Identify if a function is a linear or exponential function based on definition, evidence presented 
%	\end{itemize}
%\item Computations
%	\begin{itemize}
%	\item Work with Linear equations algebraically
%	\item Compute Slopes
%	\item Properties of Exponents 
%	\item Work with Exponential equations algebraically 
%	\end{itemize}
%\item Average Rate of Change
%	\begin{itemize}
%	\item compute between two points
%	\item Compare to slope 
%	\item linear stays the same, exponential changes 
%	\end{itemize}


\end{objectives}



\end{document}
