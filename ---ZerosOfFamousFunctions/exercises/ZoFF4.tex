\documentclass{ximera}

\input{../../preamble.tex}

\author{Carl Stitz \and Jeff Zeager \and  Bobby Ramsey}
\license{CC-By-SA-NC}
\acknowledgement{http://www.stitz-zeager.com/}

\begin{document}
\licenseSZ
\begin{exercise}
	The following is a rational function.
	$$f(x) = \frac{x^3-2x+1}{x-1} - \frac{1}{2} x + 1.$$
	
	$$ \text{How many zeros does this function have? } \quad \answer{2} $$
	\begin{exercise}
		List them in order from left to right.
		\[ x = \answer{-1/2}  \quad \text{ and } \quad x=\answer{0} \]
	\end{exercise}
\end{exercise}
\begin{hint}
	Try rewriting the function as a single fraction.
\end{hint}

\begin{exercise}
	The following is a rational function.
	$$ g(x) = \frac{1}{x+3} + \frac{1}{x-3} - \frac{x^2-3}{x^2-9}. $$
	
	$$ \text{How many zeros does this function have? } \quad \answer{1} $$
	\begin{exercise}
		It is at $x = \answer{-1}$.
		\begin{exercise}
			Why is $x=3$ NOT a zero of $g$?
			\begin{multipleChoice}
				\choice{Because $g(3)$ is a nonzero number.}
				\choice{Because $g(3)=0$.}
				\choice[correct]{Because $x=3$ is not in the domain of $g$.}
			\end{multipleChoice}
		\end{exercise}
	\end{exercise}
\end{exercise}
\begin{hint}
	Make sure to check your possible solutions are actually solutions.
\end{hint}

\begin{exercise}
	The following is a rational function.
	$$ h(x) = 1 - \frac{x^2-2x+1}{x^3+x^2-2x}. $$
	$$ \text{How many zeros does this function have? } \quad \answer{0} $$
	\begin{feedback}
		What are you left with if you rewrite $h$ as a single fraction and simplify? Are there any values of $x$ making that resulting fraction zero?	
	\end{feedback}		
\end{exercise}




\end{document}
