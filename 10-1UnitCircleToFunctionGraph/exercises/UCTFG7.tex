\documentclass{ximera}

\input{../../preamble.tex}

\author{Kenneth Berglund}
\acknowledgement{}

\begin{document}
We will algebraically find a candidate for the period of the tangent function, defined by $\tan(x) = \frac{\sin(x)}{\cos(x)}$.
\begin{exercise}
Using the angle sum identity, we know that for any real number $x$, 
$$
\sin(x + \pi)  = \sin(x)\cos(\answer{\pi}) + \cos(x)\sin(\answer{\pi})
$$
and
$$
\cos(x + \pi)  = \cos(x)\cos(\answer{\pi}) - \sin(x)\sin(\answer{\pi}).
$$

\begin{exercise}
Using knowledge of famous angles, we can simplify the following expressions as follows: $$\sin(x)\cos(\pi) + \cos(x)\sin(\pi) = \answer{-\sin(x)}$$ and $$\cos(x)\cos(\pi) -\sin(x)\sin(\pi) = \answer{-\cos(x)}$$.

\begin{exercise}
Using the information found above, $\tan(x + \pi) = \answer{\tan(x)}$. 

\begin{exercise}
We conclude that a possible period of the tangent function is $\answer{\pi}$. 



\end{exercise}

\end{exercise}
\end{exercise}
\end{exercise}
\end{document}