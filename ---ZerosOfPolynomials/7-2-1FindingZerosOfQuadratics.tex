\documentclass{ximera}

\input{../preamble.tex}
\author{David Kish}
\license{Creative Commons Attribution-ShareAlike 4.0 International License}
\acknowledgement{}

\title{Finding Zeros of Quadratics}

\begin{document}
\begin{abstract}
  
\end{abstract}
\maketitle


%\typeout{************************************************}
%\typeout{Motivating Questions}
%\typeout{************************************************}

%\begin{motivatingQuestions}\begin{itemize}
%\item 
%\item 
%\item 
%\end{itemize}\end{motivatingQuestions}

%\typeout{************************************************}
%\typeout{Subsection Introduction}
%\typeout{************************************************}

\section{Introduction}


%\typeout{************************************************}
%\typeout{Subsection Introduction}
%\typeout{************************************************}

\section{Factoring}

 We've seen multiplying polynomials before,
      such as when you start with
      $(x+2)(x+3)$ and obtain $x^2+5x+6$.
     This section,  is about the opposite process of factoring.
      For example, starting with
      $x^2+5x+6$ and obtaining $(x+2)(x+3)$.
      We will start with the simplest kind of factoring:
      for example starting with $x^2+2x$ and obtaining $x(x+2)$.

      When you write $x^2+2x$,
      you have an algebraic expression built with two terms that are
      added together.
      When you write $x(x+2)$,
      you have an algebraic expression built with two factors that are
      multiplied together.
      Factoring is useful, because sometimes
      (but not always)
      having your expression written as parts that are multiplied
      together makes it easy to simplify the expression.

  You've seen this with fractions.
      To simplify $\frac{15}{35}$,
      breaking down the numerator and denominator into factors is useful:
      $\frac{3\cdot5}{7\cdot5}$.
      Now you can see that the factors of $5$ cancel.

     There are other reasons to appreciate the value in factoring.
        One reason is that there is a relationship between a factored polynomial
        and the horizontal intercepts of its graph.
        For example in the graph of $y=(x+2)(x-3)$, the horizontal intercepts are $(-2,0)$ and $(3,0)$.
        Note the $x$-values are $-2$ and $3$,
        and think about what happens when you subsitutue those numbers in for $x$ in $y=(x+2)(x-3)$.
\begin{image}
\begin{tikzpicture}
                \begin{axis}[xmin=-4,xmax=5,ymin=-8,ymax=10,
                            xtick={-3,-2,...,4},
                            minor xtick={-4,-3,...,5},
                            ytick={-6,-4,...,8},
                            minor ytick={-8,-7,...,10},
                            ]
                    \addplot[firstcurve,domain=-3.531:4.531] {x^2-x-6};
                    \addplot[color=secondcolor,soliddot] coordinates {(3,0)} node[above left] {$(3,0)$};
                    \addplot[color=secondcolor,soliddot] coordinates {(-2,0)} node[above right] {$(-2,0)$};
                \end{axis}
              \end{tikzpicture}
\end{image}

 The most basic technique for factoring involves recognizing the
      \textbf{greatest common factor}
            between two expressions,
      which is the largest factor that goes in evenly to both expressions.
      For example,
      the greatest common factor between $6$ and $8$ is $2$, since $2$ divides nicely into both $6$ and $8$ and no larger number would divide nicely into both $6$ and $8$.

Similarly, the greatest common factor between $4x$ and $3x^2$ is $x$.
      If you write $4x$ as a product of its factors,
      you have $2\cdot 2 \cdot x$.
      And if you fully factor $3x^2$,
      you have $3\cdot x\cdot x$.
      The only factor they have in common is $x$,
      so that is the greatest common factor.
      No larger expression goes in nicely to both expressions.
\begin{example}
   What is the common factor between $6x^2$ and $70x$?
        Break down each of these into its factors:
\begin{center}
$
\begin{array}{cc}
    6x^2  =2\cdot3\cdot x\cdot x & 70x  =2\cdot5\cdot7\cdot x
\end{array}
$
\end{center}
And identify the common factors:
\begin{center}
$
\begin{array}{cc}
 6x^2  =\attention{2}\cdot3\cdot \attention{x}\cdot x & 70x  =\attention{2}\cdot5\cdot7\cdot \attention{x}
\end{array}
$
\end{center}
\end{example}
%\typeout{************************************************}
%\typeout{Subsection Factoring Trinomials with Leading Coefficient One}
%\typeout{************************************************}
\section{Factoring Trinomials with Leading Coefficient One}
  We've learned how to multiply binomials like
      $(x+2)(x+3)$ and obtain the trinomial $x^2+5x+6$.
      In this section, we will learn how to undo that.
      So we'll be starting with a trinomial like
      $x^2+5x+6$ and obtaining its factored form $(x+2)(x+3)$.
      The trinomials that we'll factor in this section all have leading coefficient $1$,
      but in a later section
      will cover some more general trinomials.
\begin{example}
      Consider the example $x^2+\firsthighlight{5}x+\secondhighlight{6}=(x+2)(x+3)$.
      There are at least three things that are important to notice:
\begin{itemize}
\item The leading coefficient of $x^2+\firsthighlight{5}x+\secondhighlight{6}$ is $1$.

   \item  The two factors on the right use the numbers $2$ and $3$,
            and when you \textit{multiply} these you get the $\secondhighlight{6}$.

            \item The two factors on the right use the numbers $2$ and $3$,
            and when you \textit{add} these you get the $\firsthighlight{5}$.
\end{itemize}
     So the idea is that if you need to factor
      $x^2+5x+6$ and you somehow discover that $2$ and $3$ are special numbers
      (because $2\cdot3=6$ and $2+3=5$),
      then you can conclude that
      $(x+2)(x+3)$ is the factored form of the given polynomial.

       Factor $x^2+13x+40$.
        Since the leading coefficient is $1$,
        we are looking to write this polynomial as
        $(x+\mathord{?})(x+\mathord{?})$ where the question marks are two possibly different,
        possibly negative, numbers.
        We need these two numbers to multiply to $40$ and add to $13$.
        How can you track these two numbers down?
        Since the numbers need to multiply to $40$,
        one method is to list all \textbf{factor pairs}
        of $40$ in a table just to see what your options are.
        We'll write every \textit{pair of factors}
        that multiply to $40$.
\begin{center}
$
\begin{array}{cc}
           1\cdot40 & -1\cdot(-40) \\
          2\cdot20 & -2\cdot(-20) \\
         4\cdot10 & -4\cdot(-10)\\
        5\cdot8 &  -5\cdot(-8) \\
 \end{array}
$
\end{center}
        We wanted to find \textit{all} factor pairs.
        To avoid missing any, we started using $1$ as a factor,
        and then slowly increased that first factor.
        The table skips over using $3$ as a factor,
        because $3$ is not a factor of $40$.
        Similarly the table skips using $6$ and $7$ as a factor.
        And there would be no need to continue with $8$ and beyond,
        because we already found ``large''
        factors like $8$ as the partners of
        ``small'' factors like $5$.

   There is an entire second column where the signs are reversed,
        since these are also ways to multiply two numbers to get $40$.
        In the end, there are eight factor pairs.

        We need a pair of numbers that also \textit{adds} to $13$.
        So we check what each of our factor pairs add up to:
\begin{center}
$
\begin{array}{c|l}
      \text{Factor Pair} & \text{Sum of the Pair} \\
\hline
   1\cdot40 & 41 \\
     2\cdot20 & 22 \\
   4\cdot10 & 14 \\
        5\cdot8 & 13 \text{(what we wanted)} \\
-1\cdot(-40) & \text{(no need to go this far)}\\
         -2\cdot(-20) & \text{(no need to go this far)}\\
 -4\cdot(-10) & \text{(no need to go this far}) \\
-5\cdot(-8) & \text{(no need to go this far)} \\
\end{array}
$
\end{center}
  The winning pair of numbers is $5$ and $8$.
        Again, what matters is that $5\cdot8=40$, and $5+8=13$.
        So we can conclude that $x^2+13x+40=(x+5)(x+8)$.
\end{example}
   To ensure that we made no mistakes,
      here are some possible checks.\\
\textbf{Multiply it Out}
   Multiplying out our answer
        $(x+5)(x+8)$ should give us $x^2+13x+40$.

 $(x+5)(x+8)=(x+5)\cdot x+(x+5)\cdot8
            =x^2+5x+8x+40
           = x^2+13x+40$\\
\textbf{Evaluating}
 If the answer really is $(x+5)(x+8)$,
        then notice how evaluating at $-5$ would result in $0$.
        So the original expression should also result in $0$ if we evaluate at $\substitute{-5}$.
        And similarly, if we evaluate it at $\substitute{-8}$,
        $x^2+13x+40$ should be $0$.
\begin{center}
$
\begin{array}{rl}
(\substitute{-5})^2+13(\substitute{-5})+40 & =0 \\
          25-65+40 & =0 \\
          0 &= 0 \text{ Confirmed}\\
(\substitute{-8})^2+13(\substitute{-8})+40&=0\\
64-104+40&=0\\
 0&=0 \text{ Confrimed}
\end{array}$
\end{center}
This also gives us evidence that the factoring was correct.

       Factor $y^2-11y+24$.
          The negative coefficient is a small complication from the previous example,
          but the process is actually still the same.

    We need a pair of numbers that multiply to $24$ and add to $-11$.
          Note that we \textit{do} care to keep track that they sum to a negative total.
\begin{center}
$
\begin{array}{c|l}
         \text{Factor Pair} & \text{Sum of the Pair}\\
\hline
   1\cdot24 &  25\\
2\cdot12 & 14 \\
          3\cdot8 & 11\text{(close; wrong sign)} \\
          4\cdot6 & 10 \\
           -1\cdot(-24) & -25\\
-2\cdot(-12) & -14 \\
       -3\cdot(-8) & -11 \text{(what we wanted)}\\
-4\cdot(-6) & \text{(no need to go this far)}
\end{array}
$
\end{center}
 So $y^2-11y+24=(y-3)(y-8)$.
       To confirm that this is correct, we should check.
          Either by multiplying out the factored form:
\begin{center}
$
\begin{array}{rl}
(y-3)(y-8)& =(y-3)\cdot y-(y-3)\cdot8\\
               &=y^2-3y-8y+24 \\
               &=y^2-11y+24 \text{ Confirmed}
\end{array}
$
\end{center}

 Or by evaluating the original expression at $\substitute{3}$ and $\substitute{8}$:
\begin{center}
$\begin{array}{rl}
    \substitute{3}^2-11(\substitute{3})+24 & = 0 \\
            9-33+24 & = 0 \\
            0 & = 0 \\
\substitute{8}^2-11(\substitute{8})+24 &=0\\
64-88+24 & =0\\
 0 & =0
\end{array}$
\end{center}
Our factorization passes the tests.

\begin{callout}
{\bf Warning:} Solving an equation like $x^2 = 4$ is very different than computing $\sqrt{4} = 2$. To solve the equation, one rewrites it as $x^2-4=0$, then factors it as $(x-2)(x+2) = 0$, and concludes that $x=-2$ or $x=2$. However, people often write this sloppily as $x = \pm 2$, and proceed to say {\bf wrong} things such as $\sqrt{4} = \pm 2$. Square roots are always non-negative, so $\sqrt{4}=2$ is the only correct equality. To relate this back with solving the equation $x^2=4$, one could alternatively take the square root of both sides, leading to $\sqrt{x^2} = \sqrt{4}$. The square root of a square is not the original number, but its absolute value instead. This means that instead of writing the next step as $x=2$ (which would lead us to miss the solution $x=-2$), we should write $|x| = 2$. Which real numbers have absolute value equal to $2$? Just $x=2$ and $x=-2$.
\end{callout}


%\typeout{************************************************}
%\typeout{Subsection Introduction}
%\typeout{************************************************}



%\typeout{************************************************}
%\typeout{Summary}
%\typeout{************************************************}

%\begin{summary}\begin{itemize}
%\item 
%\item 
%\item
%\end{itemize}\end{summary}




\end{document}
