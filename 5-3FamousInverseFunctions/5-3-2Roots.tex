\documentclass{ximera}

\input{../preamble.tex}
\author{Elizabeth Campolongo}
\license{Creative Commons Attribution-ShareAlike 4.0 International License}
\acknowledgement{https://spot.pcc.edu/math/orcca/ed2/html/section-technical-definition-of-a-function.html, https://activecalculus.org/prelude/sec-changing-functions-models.html, https://openstax.org/books/college-algebra/pages/3-1-functions-and-function-notation}

\title{Roots}

\begin{document}
\begin{abstract}
  
\end{abstract}
\maketitle

Coming Soon.

Recall in Section 3-2-3 that we briefly discussed the square root function..


%This exemplifies the fact that, 
In general, the square root is not one-to-one when used for {\em solving} equations. Instead, the output must be chosen. For instance, the square root of 9 is 3, since the square root {\em function} is only defined on $[0,\infty)$. However, if we wish to solve the equation $x^2=9$, the solution is not $x=3$, or, to be more precise, there are \textbf{two} solutions: $x=3$ \textbf{and} $x = -3$, since both $3^2$ and $(-3)^2$ are equal to 9. This can be demonstrated by solving the equation through factoring:
$$0=x^2-9 =(x+3)(x-3),$$
which has {\em two} solutions: $x=3$ and $x = -3$.

$\sqrt{}$ is not one-to-one so you must {\em choose} which value makes sense in the context, or list both.

\begin{definition}
For any real number $x$, the $n^{th}$ root of $x$ is denoted by 
$$\sqrt[n]{x} = x^{\frac{1}{n}}.$$
%
It is further defined as the real number $y$ such that 
$$y^n = x.$$
Note that if $n$ is \textbf{even}, then both $y^n$ and $(-y)^n$ will be equal to $x$, so the $n^{th}$ root is actually $|y|$, since we define the $n^{th}$ root {\em function} to produce values in $[0,\infty)$ for $n$ even as with the more-common square root. Further note that the input of an even $n^{th}$ root (the $x$ in $\sqrt[n]{x}$) must be in $[0,\infty)$ as well.

However, if $n$ is \textbf{odd}, then there is only one such $y$, and {\em both} the input and output of the $n^{th}$ root function are in $(-\infty,\infty)$.
\end{definition}

\begin{example}
We demonstrate a few common even and odd $n^{th}$ roots to highlight this distinction.
\begin{enumerate}
\item $\sqrt[3]{8} = 2$, since $2 \cdot 2 \cdot 2 = 8$.

\item $\sqrt[3]{-8} = - 2$, since $-2 \cdot (-2) \cdot (-2) = 4 \cdot (-2) = -8$.

\item $\sqrt[4]{16} = 2$, since $2 \cdot 2 \cdot 2 \cdot 2 \cdot 2 = 4 \cdot 4 = 16$. However, the $4^{th}$ root of -16 is not defined.

\item $\sqrt[4]{0} =0$, since zero times any number is always zero. This is {\em the} example of an even $n^{th}$ root that has only {\em one} solution.

\item Likewise, $\sqrt[125]{0} = 0$.
\end{enumerate}
\end{example}

Generally speaking, we use $n^{th}$ roots to solve equations, in which case we must be careful to understand the context of application as explained above with the square root.

If we are asked to find all values $x$ such that $x^2=4$, then the question is asking which values of $x$ multiplied by themselves give 4. In other words, find $x$ such that $x \cdot x$ is equal to 4. It is simple to see that there are two values which make this true:
$$x\cdot x = 1\cdot x \cdot x = (-1)^2 \cdot x \cdot x = -1 \cdot x \cdot (-1)\cdot x = -x \cdot (-x),$$
by the distributive property of multiplication. Thus, both $x$ and $-x$ will be solutions. In solving an equation, it is common to express this as follows.
\begin{align*}
x^2 &= 4 \\
\sqrt{x^2} &= \sqrt{4} \\
\pm x &= 2,
\end{align*}
which translates to $x = \pm 2$. This is because the square root of a number is always non-negative; {\em however}, we are looking for the value that makes this equation true, and since $f(x)=x^2$ is \textbf{not} one-to-one, there are two values of $x$ which make it equal to any positive number, as demonstrated above.
%do \textbf{not} restrict the range of the $n^{th}$ root to $[0,\infty)$, but allow it to produce a negative value. In this case, the $n^{th}$ root is no longer one-to-one for even $n$.


\begin{callout}
{\bf Warning:} Solving an equation like $x^2 = 4$ is very different than computing $\sqrt{4} = 2$. To solve the equation, one rewrites it as $x^2-4=0$, then factors it as $(x-2)(x+2) = 0$, and concludes that $x=-2$ or $x=2$. However, people often write this sloppily as $x = \pm 2$, and proceed to say {\bf wrong} things such as $\sqrt{4} = \pm 2$. Square roots are always non-negative, so $\sqrt{4}=2$ is the only correct equality. To relate this back with solving the equation $x^2=4$, one could alternatively take the square root of both sides, leading to $\sqrt{x^2} = \sqrt{4}$. The square root of a square is not the original number, but its absolute value instead. This means that instead of writing the next step as $x=2$ (which would lead us to miss the solution $x=-2$), we should write $|x| = 2$. Which real numbers have absolute value equal to $2$? Just $x=2$ and $x=-2$.

This is true for any {\em even} root, for instance, consider the equation $x^4 = 16$. If we take the fourth root ($\sqrt[4]{}$) of both sides, we then have $x = \pm 2$. However, if we have $x^3 = 8$, when we take the third root, we get only one solution, $x=2$. Factoring, we can see why this is the case: $x^3=8$ is equivalent to 
$$0=x^3 - 8=(x-2)(x^2+2x+4),$$
which has only one real solution, $x=2$.

However, {\em odd} roots have the unique property that they can be applied to {\em negative} numbers. Following the pattern above, $x^2=-4$ has no real solutions since the square root is only defined for values greater or equal to zero. However, $x^3=-8$ does have a solution: $x=-2$, we can check this easily by multiplying out 
$$(-2)^3=(-2)\cdot(-2)\cdot(-2) = 4 \cdot (-2) = -8.$$
It is important to note that the partentheses around the ``-2" terms are {\bf not} optional. 
\end{callout}

\end{document}
