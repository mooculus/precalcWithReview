\documentclass{ximera}

\input{../../preamble.tex}

\author{Bobby Ramsey}
\license{Creative Commons Attribution-ShareAlike 4.0 International License}

%\outcome{Calculating the rate of change.}
%\outcome{Discuss the meaning of antiderivatives of a position function.}

\begin{document}
\begin{exercise}

The graph of a line is given below.
\begin{image}
\begin{tikzpicture}
    \begin{axis}
    		[
    		width=0.75\linewidth,
                xmin=-6.25,xmax=6.25,
                ymin=-6.25,ymax=6.25,
                minor ytick=,
                minor xtick=,
                xtick={-6,...,6},
                ytick={-6,...,6},
                clip=false,
                ]
	\addplot+[domain=-6.25:6.25]{(2/3)*x+(1/3)};		
        \addplot[soliddot] coordinates {(-2,-1)} node[below right] {$(-2,-1)$};
        \addplot[soliddot] coordinates {(4,3)} node[above left] {$(4,3)$};
    \end{axis}
\end{tikzpicture}
\end{image}

\begin{enumerate}
	\item The line parallel to this graphed line, which passes through the point $(3, 1)$ has equation in point-slope form given by $y - \answer{1} = \answer{2/3}\left( x - \answer{3} \right)$.

	\item The line perpendicular to this graphed line, which passes through the point $(-2, 3)$ has equation in slope-intercept form given by $y = \answer{-3/2} x+ \answer{0}$.


\end{enumerate}



\end{exercise}
\end{document}