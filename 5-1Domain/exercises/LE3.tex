\documentclass{ximera}

\input{../../preamble.tex}

\author{Bobby Ramsey}
\license{Creative Commons Attribution-ShareAlike 4.0 International License}

%\outcome{Calculating the rate of change.}
%\outcome{Discuss the meaning of antiderivatives of a position function.}

\begin{document}
\begin{exercise}


\begin{image}
\begin{tikzpicture}
    \begin{axis}
    		[
    		width=0.75\linewidth,
                xmin=-6.25,xmax=6.25,
                ymin=-6.25,ymax=6.25,
                minor ytick=,
                minor xtick=,
                xtick={-6,...,6},
                ytick={-6,...,6},
                clip=false,
                ]
	\addplot+[domain=-6.25:6.25]{(2/5)*x+2};		
        \addplot[soliddot] coordinates {(-5,0)} node[above left] {$(-5,0)$};
        \addplot[soliddot] coordinates {(5,4)} node[above left] {$(5,4)$};
    \end{axis}
\end{tikzpicture}
\end{image}

\begin{enumerate}
\item The slope of this line is
	\begin{multipleChoice}
		\choice[correct]{positive because $y$ is increasing}
		\choice{positive because $y$ is decreasing}
		\choice{negative because $y$ is increasing}
		\choice{negative because $y$ is decreasing}
	\end{multipleChoice}

\item The slope of this line is $m = \answer{2/5}$.
	\begin{hint}
		Recall that the slope of the line is the rate of change between any two data points on the line, $m = \dfrac{\Delta y}{\Delta x}$.
	\end{hint}

\item The $y$-value of the point corresponding to $x=9$ is $\answer{28/5}$.
	\begin{hint}
		How much $y$ increases if $x$ increases by $1$? How much does $y$ increase if $x$ increases by $4$?
	\end{hint}

\end{enumerate}
%\begin{multipleChoice}
%\choice{constant function.}
%\choice{positive function. }
%\choice{negative function.}
%\choice{position function.}
%\choice[correct]{velocity function.}
%\end{multipleChoice}


\end{exercise}
\end{document}