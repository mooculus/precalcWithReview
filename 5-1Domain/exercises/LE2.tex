\documentclass{ximera}

\input{../../preamble.tex}

\author{Bobby Ramsey}
\license{Creative Commons Attribution-ShareAlike 4.0 International License}

%\outcome{Calculating the rate of change.}
%\outcome{Discuss the meaning of antiderivatives of a position function.}

\begin{document}
\begin{exercise}
A table of data is given below.
\begin{center}
$
\begin{array}{||c|c||}
\hline
x&y\\
\hline 
\hline
0&2\\
\hline
1&5\\
\hline
2&8 \\
\hline
5&  \\
\hline 
\hline
\end{array}
$
\end{center}

\begin{enumerate}
\item The rate of change from the top row to the second row is: $\answer{3}$.

\item The rate of change from the top row to the second row is: $\answer{3}$.

\item If this rate of change is maintained, whenever the $x$-value of a data point increases by $1$, the $y$-value of the data point must increase by $\answer{3}$.

\item If this rate of change is maintained, whenever the $x$-value of a data point increases by $3$, the $y$-value of the data point must increase by $\answer{9}$.

\item If this rate of change is maintained, the $x$-value $5$ corresponds to the $y$-value $\answer{17}$.

\item An equation that describes the pattern in the table is $y = \answer{3x+2}$.
\end{enumerate}

%\begin{multipleChoice}
%\choice{constant function.}
%\choice{positive function. }
%\choice{negative function.}
%\choice{position function.}
%\choice[correct]{velocity function.}
%\end{multipleChoice}


\end{exercise}
\end{document}
