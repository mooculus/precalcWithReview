\documentclass{ximera}

\input{../../preamble.tex}

\author{Bobby Ramsey}
\license{Creative Commons Attribution-ShareAlike 4.0 International License}

%\outcome{Calculating the rate of change.}
%\outcome{Discuss the meaning of antiderivatives of a position function.}

\begin{document}
\begin{exercise} Vertical Lines

\begin{callout}
Vertical lines do not have a slope. They consist of all points with the same $x$-coordinate. For example, the linear relationship with data given in this table consists of points with $x$-coordinate equal to $5$.
\begin{center}
$
\begin{array}{||c|c||}
\hline
x&y\\
\hline 
\hline
5&-1\\
\hline
5&0\\
\hline
5&1 \\
\hline
5& 2 \\
\hline 
\hline
\end{array}
$
\end{center}

Since this line does not have a slope, we can not express its equation in either point-slope or slope-intercept forms.  Instead, a vertical line has an equation of the form $x = C$, where $C$ is the common $x$-coordinate between all the points, meaning that the line given in the table above has equation $x=5$.
\end{callout}

\begin{enumerate}
	\item The line given by the following table of data:
		\begin{center}
			$ \begin{array}{||c|c||}
			\hline
			x&y\\
			\hline 
			\hline
			2&-3\\
			\hline
			2&1\\
			\hline
			2&1 \\
			\hline
			2& 3 \\
			\hline
			2& 5 \\
			\hline 
			\hline
			\end{array}$
		\end{center}
		has equation given by $x = \answer{2}$.

	\item The line given by the following table of data:
		\begin{center}
			$\begin{array}{||c|c||}
			\hline
			x&y\\
			\hline 
			\hline
			-3/4 &-8\\
			\hline
			-3/4 &-7\\
			\hline
			-3/4 &-6 \\
			\hline
			-3/4 & -5 \\
			\hline 
			\hline
			\end{array}$ 
		\end{center}
		has equation given by $x = \answer{-\frac{3}{4}}$.

\end{enumerate}


\end{exercise}
\end{document}
