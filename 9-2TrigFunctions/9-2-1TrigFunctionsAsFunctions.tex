\documentclass{ximera}

\input{../preamble}
\author{David Kish, Kenneth Berglund}
\license{Creative Commons Attribution-ShareAlike 4.0 International License}
\acknowledgement{}

\title{Trig Functions as Functions}

\begin{document}
\begin{abstract}
  
\end{abstract}
\maketitle

%\typeout{************************************************}
%\typeout{Subsection }
%\typeout{************************************************}


\begin{motivatingQuestions}
\begin{itemize}
\item How do trigonometric functions interact with other functions?
\item How do we find zeros of trigonometric functions?
\item What does average rate of change look like with trigonometric functions?
\end{itemize}
\end{motivatingQuestions}
\section{Trig Function Compositions}

Trigonometric functions can be composed with any of the types of functions that we have already seen. Just as with other function compostions, we need to be mindful of the domains and ranges of our functions.

\begin{example}
Let's consider the following functions: $f(x)=\sin{(x)}$ and $g(x)=3x^2$.
\\
Find the function below and state its domain and range.
\begin{enumerate}
\item $(f \circ g)(x)$
\item $(g \circ f)(x)$
\item $(f \circ f)(x)$
\end{enumerate}

\begin{explanation}
First let's find the domain and range of $f(x)$ and $g(x)$.\\
The domain for both $f(x)$ and $g(x)$ is $(-\infty , \infty)$. The range for $f(x)$ is $[-1,1]$ and the range for $g(x)$ is $[0,\inf)$. Now we can look at the compositions.
\begin{enumerate}
\item $f(g(x))= \sin(3x^2)$\\
The domain of $f \circ g$ consists of all inputs to $g$ whose corresponding outputs are in the domain of $f$.  Since the domain of $f$ is $(-\infty, \infty)$, all outputs of $g$ are in the domain of $f$, so the domain of $f \circ g$ is the entire domain of $g$, namely $(-\infty , \infty)$.

Finding the range of $f \circ g$ is a bit trickier. Since the range of $g$ is only non-negative numbers, only non-negative numbers will be plugged into $f$ when evaluating the composition. The question therefore becomes: what outputs of $f$ correspond to non-negative inputs? The answer is: all of them! By looking at the graph of the sine function, we can see that all $y$-values have a corresponding non-negative $x$-value. Therefore, the range of $f$, namely $[-1, 1]$, is the range of $f \circ g$.

\item $g(f(x)) =3\sin^2(x)$. Remember that $\sin^2(x)$ means $(\sin(x))^2$.\\
All real numbers can be plugged into $\sin(x)$, and the results can all be squared. The results of \textit{that} process can all be multiplied by 3, so the domain of $g \circ f$ is $(-\infty, \infty)$.

Let's start by finding the range of $\sin^2(x)$. Since we are squaring $\sin(x)$, its outputs are all non-negative. However, the maximum absolute value of $\sin(x)$ is 1, so the maximum value of $\sin^2(x)$ is 1. Since $\sin(x)$ ranges through all values in $[-1, 1]$, $\sin^2(x)$ ranges through all values in $[0, 1]$. Multiplying $\sin^2(x)$ by 3 results in a vertical stretch by a factor of 3, so the range of $3\sin^2(x)$ is $[0, 3]$. 
\item $f(f(x))= \sin(\sin(x))$\\
The domain of $\sin(x)$ is all real numbers, and therefore, all outputs of $\sin(x)$ can be plugged into $\sin(x)$, meaning that the domain of $f \circ f$ is all real numbers, $(-\infty, \infty)$. 

Since $\sin(x)$ increases from -1 to 1 as $x$ goes from $-\pi/2$ to $\pi/2$, $\sin(\sin(x))$ increases from $\sin(-1)$ to $\sin(1)$ as $x$ goes from $-\pi/2$ to $\pi/2$. Therefore, the range of $f \circ f$ is $[\sin(-1), \sin(1)]$. 
\end{enumerate}
\end{explanation}
\end{example}

\section{Finding Zeros of Trigonometric Funtions.}
\begin{example}
Let $f$ be a function defined by $f(x)=\sin^2(x)-1$. Find the zeros of $f$ for $0\leq x \leq 2\pi$.
\begin{explanation}
First we need to set $f(x)$ equal to $0$:\\
\[
\sin^2(x)-1=0
\]
Now we can recognize that we have a difference of squares, so we have the following:
\[
(\sin(x)+1)(\sin(x)-1) = 0.
\]
Now we can set each factor on the left equal to zero. So we have
\[
\sin(x)+1=0 \text{ and } \sin(x)-1 =0.
\]
After simplifying them a bit, we have
\[
\sin(x) =-1 \text{ and } \sin(x)=1.
\]
Because these are famous values of $\sin(x)$, we can find values for $x$ without using inverse trigonometry. The solution to $\sin(x) = -1$ is $x = \frac{3\pi}{2}$ and the solution to $\sin(x) = 1$ is $x = \frac{\pi}{2}$. Note that there are other $x$-values for which $\sin(x) = \pm 1$, but these are the only ones for which $0 \le x \le 2\pi$. 
\end{explanation}
\end{example}

\begin{example}
Let $f$ be a function defined by $f(x)=\sin^2(x)+\frac{5}{2}\sin(x)-\frac{3}{2}$. Find the zeros of $f$ for $0\leq x \leq 2\pi$
\begin{explanation}
First we need to set $f$ to equal $0$:\\
$\sin^2(x)+\frac{5}{2}\sin(x)-\frac{3}{2}=0$\\
Thankfully the left-hand side can factor nicely, so we have\\
\[
\sin^2(x)+\frac{5}{2}\sin(x)-\frac{3}{2}= \left(\sin(x)-\frac{1}{2}\right)(\sin(x)+3).
\]
Now we set each factor equal to zero:
\[
\sin(x)-\frac{1}{2}=0 \text{ and } \sin(x)+3 =0.
\]
After simplifying them a bit, we have
\[
\sin(x) = \frac{1}{2} \text{ and } \sin(x)=-3.
\]
$-3$ is outside of the range of $\sin(x)$ so we get no solutions to our equation from that factor. Our only concern is the famous value where $sin(x) = \frac{1}{2}$. This happens to occur in two places for $0\leq x \leq 2\pi$, so our answer is $x=\frac{\pi}{6},\frac{5\pi}{6}$.
\end{explanation}
\end{example}


\section{Average Rate of Change with Trigonometry}

We can still find average rate of change with trigonometric functions, but because they are periodic there can be some interesting results.

\begin{example}
Let $f(x) =\sin(x)$.
\begin{enumerate}
\item Find $\av_{[\frac{\pi}{6},\frac{3\pi}{4}]} $.
\item Find $\av_{[\frac{\pi}{3},\frac{2\pi}{3}]} $.
\end{enumerate}
\begin{explanation}
\begin{enumerate}
\item $$\av_{[\frac{\pi}{6},\frac{3\pi}{4}]}  = \frac{\sin(\frac{3\pi}{4})-\sin(\frac{\pi}{6})}{\frac{3\pi}{4} - \frac{\pi}{6}}$$
First we substitute in our trig values:
$$\frac{\sin(\frac{3\pi}{4})-\sin(\frac{\pi}{6})}{\frac{3\pi}{4} - \frac{\pi}{6}}=\frac{\frac{\sqrt{2}}{2}-\frac{1}{2}}{\frac{3\pi}{4} - \frac{\pi}{6}}$$
Now we simplify our fractions\\
$\frac{\frac{\sqrt{2}}{2}-\frac{1}{2}}{\frac{3\pi}{4} - \frac{\pi}{6}}= \frac{\frac{\sqrt{2}-1}{2}}{\frac{9\pi}{12} - \frac{2\pi}{12}}= \frac{\frac{\sqrt{2}-1}{2}}{\frac{7\pi}{12}}$\\
$\frac{\frac{\sqrt{2}-1}{2}}{\frac{7\pi}{12}}=\frac{6(\sqrt{2}-1)}{7\pi}$\\
$\av_{[\frac{\pi}{6},\frac{3\pi}{4}]} =\frac{6\sqrt{2}-6}{7\pi}$

\item $\av_{[\frac{\pi}{3},\frac{2\pi}{3}]} = \frac{\sin(\frac{2\pi}{3})-\sin(\frac{\pi}{3})}{\frac{2\pi}{3} - \frac{\pi}{3}}$\\
First we substitute our trig values\\
$\frac{\sin(\frac{2\pi}{3})-\sin(\frac{\pi}{3})}{\frac{2\pi}{3} - \frac{\pi}{3}} = \frac{\frac{\sqrt{3}}{2}-\frac{\sqrt{3}}{2}}{\frac{2\pi}{3} - \frac{\pi}{3}}$\\
Now we simplify\\
$\frac{\frac{\sqrt{3}}{2}-\frac{\sqrt{3}}{2}}{\frac{2\pi}{3} - \frac{\pi}{3}}=\frac{0}{\frac{\pi}{3}}$\\
We can see that we have $0$ in the numerator, which means that our average rate of change will be $0$. This is because we are working with a periodic function and picked values in similar positions. Even though we had positive and negative rates of change at some point between $\frac{\pi}{3}$ and $\frac{2\pi}{3}$, we ended up with the same $y$ value, so \textit{on average} there was no change.
\end{enumerate}
\end{explanation}
\end{example}



\end{document}
