\documentclass{ximera}

\input{../../preamble.tex}

\author{Kenneth Berglund}

\begin{document}


\begin{exercise}
Let $f$ be a circular function with amplitude 2, midline $-3$, and period 1 such that $f(2) = -1$.

A formula for $f$ in terms of the cosine function is $f(x) = \answer{2}\cos\left(\answer{2\pi}(x - 2)\right) + \answer{-3}$. 
\end{exercise} 

\begin{exercise}
Let $f$ be a circular function with amplitude 6, midline $1$, and period $\pi$ such that $f(1) = 1$.

A formula for $f$ in terms of the sine function is $f(x) = \answer{6}\cos\left(\answer{2}(x - 1)\right) + \answer{1}$. 
\end{exercise} 

\begin{exercise}
Let $f$ be a circular function with amplitude $\pi$, midline $-\pi$, and period $2\pi$ such that $f(-1) = 0$.

A formula for $f$ in terms of a sine or cosine function is $f(x) = \answer{\pi}$\wordChoice{\choice{$\sin$}\choice[correct]{$\cos$}}$\left(\answer{1}(x + 1)\right) + \answer{-\pi}$. 
\end{exercise} 
 
\end{document}