\documentclass{ximera}

\input{../../preamble.tex}

\author{Kenneth Berglund}

\begin{document}


\begin{exercise}
Let $f$ be a circular function defined by $f(x) = -4.5 \cos\left(\frac{2\pi}{7}(x - 1.8)\right) + 2.7$.

The midline of $f$ is $\answer{2.7}$. The amplitude of $f$ is $\answer{4.5}$. The period of $f$ is $\answer{7}$. 
\end{exercise} 

\begin{exercise}
Let $f$ be a circular function defined by $f(x) = \frac{1}{2}\sin\left(4(x - \pi)\right) - \frac{1}{4}$.

The midline of $f$ is $\answer{-\frac{1}{4}}$. The amplitude of $f$ is $\answer{\frac{1}{2}}$. The period of $f$ is $\answer{\frac{2\pi}{4}}$. 
\end{exercise} 

\begin{exercise}
Let $f$ be a circular function defined by $f(x) = \frac{\pi}{2}\sin\left(\pi x - \pi\right) - 22$.

The midline of $f$ is $\answer{-22}$. The amplitude of $f$ is $\answer{\frac{\pi}{2}}$. The period of $f$ is $\answer{2}$. 
\end{exercise} 

\begin{exercise}
Let $f$ be a circular function defined by $f(x) = 20\cos\left(\frac{1}{3}( x - 4)\right) + 10$.

The midline of $f$ is $\answer{10}$. The amplitude of $f$ is $\answer{20}$. The period of $f$ is $\answer{6\pi}$. 
\end{exercise} 

\begin{exercise}
Let $f$ be a circular function defined by $f(x) = 2\cos\left(-\frac{\pi}{3}( x - 4)\right) + 21$.

The midline of $f$ is $\answer{21}$. The amplitude of $f$ is $\answer{2}$. The period of $f$ is $\answer{6}$. 
\end{exercise} 
\end{document}