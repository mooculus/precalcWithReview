\documentclass{ximera}

\input{../../preamble.tex}

\author{Kenneth Berglund}

\begin{document}


\begin{exercise}
In this multi-part problem, we will explore the properties of a composition of a trigonometric function with one of our other famous functions. 

Consider the function $f$ defined by $f(x) = \csc(x^2)$. 

If $f = g \circ h$ with $g(x) = \csc(x)$, then $h(x) = \answer{x^2}$. 

\begin{exercise}
What are the domain and range of $g$ and $h$?

The domain of $g(x) = \csc(x)$ is $x \ne \answer{\pi k}$ for all integers $k$. The range of $g(x) = \csc(x)$ is $(\answer{-\infty}, \answer{-1}] \cup [\answer{1}, \answer{\infty})$. 

The domain of $h(x) = x^2$ is $(\answer{-\infty}, \answer{\infty})$. The range of $h(x) = x^2$ is $[\answer{0}, \answer{\infty})$. 


\end{exercise}
\begin{exercise}
Next we will consider the domain and range of $f = g \circ h$. 

We know from the previous part that the domain of $g$ is $x \ne \pi k$ for all integers $k$. This means that $\csc(x)$ is undefined when $x = \pi k$ for all integers $k$. However, we're considering $\csc(x^2)$, which is undefined when $x^2 = \pi k$ for all integers $k$. This means $\csc(x^2)$ is undefined when $x = \answer{\sqrt{\pi k}}$ for all integers $k$. Therefore, the domain of $f$ is $x \ne \answer{\sqrt{\pi k}}$. 

As $x$ ranges through all real numbers, $x^2$ ranges through all \wordChoice{\choice[correct]{positive}\choice{negative}} real numbers. As $x^2$ ranges through all positive real numbers, $\csc(x^2)$ ranges through \wordChoice{\choice[correct]{its entire}\choice{only some of its}} range. Therefore, the range of $\csc(x^2)$ is $(\answer{-\infty}, \answer{-1}] \cup [\answer{1}, \answer{\infty})$. 

\begin{exercise}
Find $\av_{[\sqrt{\pi/4}, \sqrt{\pi/2}]}$ for $f$. 

$\av_{[\sqrt{\pi/4}, \sqrt{\pi/2}]} = \answer{\frac{1 - \sqrt{2}}{\sqrt{\pi/2} - \sqrt{\pi/4}}}$.

\end{exercise}

\end{exercise}

\end{exercise} 
 
\end{document}