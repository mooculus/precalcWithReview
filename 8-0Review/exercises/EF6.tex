\documentclass{ximera}

\input{../../preamble.tex}

\author{Ivo Terek}
\license{Creative Commons Attribution-ShareAlike 4.0 International License}

%\outcome{Calculating the rate of change.}
%\outcome{Discuss the meaning of antiderivatives of a position function.}

\begin{document}
\begin{exercise}

  A population has $5000$ people at time $t=0$, where $t$ is measured in years.

  \begin{exercise}
    If the population increases by $200$ people by year, the population $P(t)$ after $t$ years equals $P(t) = \answer{5000+200t}$.
  \end{exercise}


  \begin{exercise}
      If the population increases by $7\%$ by year, the population $P(t)$ after $t$ years equals $P(t) = \answer{5000(1.07)^t}$.
  \end{exercise}
  
\end{exercise}
\end{document}