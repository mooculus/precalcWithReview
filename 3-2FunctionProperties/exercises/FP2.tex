\documentclass{ximera}

\input{../../preamble.tex}

\author{Kenneth Berglund}
\acknowledgement{}

\begin{document}
\begin{exercise}

Look at the following graphs of functions. Assume that all the important behavior of the functions is shown on the graphs below. 

\begin{image}
\begin{tikzpicture}
    \begin{axis}[
	title=Graph A,
        	xlabel=$x$,
    	ylabel=$y$,]
    	\draw[thick, <->, penColor] (axis cs: -7,1) -- (axis cs: 7,1); 
    \end{axis}
\end{tikzpicture}
\begin{tikzpicture}
    \begin{axis}[
	title=Graph B,
        	xlabel=$x$,
    	ylabel=$y$,]
	
	\draw[thick, <->, penColor] (axis cs: -7,-3) -- (axis cs: 7,3); 
    \end{axis}
	
\end{tikzpicture}
\end{image}

\begin{image}
\begin{tikzpicture}
    \begin{axis}[
	title=Graph C,
        	xlabel=$x$,
    	ylabel=$y$,]
	
    	\draw[thick, <->, penColor] (axis cs: -7,5) -- (axis cs: 7,-1); 
    \end{axis}
	
\end{tikzpicture}
\begin{tikzpicture}
    \begin{axis}[
	title=Graph D,
        	xlabel=$x$,
    	ylabel=$y$,]
	
	\draw[thick, <-, penColor] (axis cs: -4, 7) -- (axis cs: 0, -1); 
	\draw[thick, ->, penColor] (axis cs: 0, -1) -- (axis cs: 4, 7);
    \end{axis}
	
\end{tikzpicture}
\end{image}

\begin{image}
\begin{tikzpicture}
    \begin{axis}[
	title=Graph E,
        	xlabel=$x$,
    	ylabel=$y$,]
	
	\draw[thick, <-, penColor] (axis cs: -7, 2) -- (axis cs: -2, 0); 
	\draw[very thick, penColor] (axis cs: -2, 0) -- (axis cs: 2, 0); 
	\draw[thick, ->, penColor] (axis cs: 2, 0) -- (axis cs: 7, 2); 
    \end{axis}
	
\end{tikzpicture}
\begin{tikzpicture}
    \begin{axis}[
	title=Graph F,
        	xlabel=$x$,
    	ylabel=$y$,]
	\draw[thick, <-, penColor] (axis cs: -7, 2) -- (axis cs: -2, 0); 
	\draw[very thick, penColor] (axis cs: -2, 0) -- (axis cs: 2, 0); 
	\draw[thick, ->, penColor] (axis cs: 2, 0) -- (axis cs: 7, -2); 
    \end{axis}
	
\end{tikzpicture}
\end{image}

\begin{image}
\begin{tikzpicture}
    \begin{axis}[
	title=Graph G,
        	xlabel=$x$,
    	ylabel=$y$,]
	
	\draw[very thick, <->, penColor] (axis cs: -7,0) -- (axis cs: 7,0); 
    \end{axis}
	
\end{tikzpicture}
\begin{tikzpicture}
    \begin{axis}[
	title=Graph H,
        	xlabel=$x$,
    	ylabel=$y$,]
	\draw[thick, <-, penColor] (axis cs: -7, 4) -- (axis cs: -2, 0); 
	\draw[very thick, penColor] (axis cs: -2, 0) -- (axis cs: 2, 0); 
	\draw[thick, ->, penColor] (axis cs: 2, 0) -- (axis cs: 7, -3); 
    \end{axis}
	
\end{tikzpicture}
\end{image}

\begin{enumerate}
\item The function corresponding to Graph A is
\begin{multipleChoice}
\choice{neither even nor odd.}
\choice[correct]{even, but not odd.}
\choice{odd, but not even.}
\choice{both even and odd.}
\end{multipleChoice}

\item The function corresponding to Graph B is
\begin{multipleChoice}
\choice{neither even nor odd.}
\choice{even, but not odd.}
\choice[correct]{odd, but not even.}
\choice{both even and odd.}
\end{multipleChoice}

\item The function corresponding to Graph C is
\begin{multipleChoice}
\choice[correct]{neither even nor odd.}
\choice{even, but not odd.}
\choice{odd, but not even.}
\choice{both even and odd.}
\end{multipleChoice}

\item The function corresponding to Graph D is
\begin{multipleChoice}
\choice{neither even nor odd.}
\choice[correct]{even, but not odd.}
\choice{odd, but not even.}
\choice{both even and odd.}
\end{multipleChoice}

\item The function corresponding to Graph E is
\begin{multipleChoice}
\choice{neither even nor odd.}
\choice[correct]{even, but not odd.}
\choice{odd, but not even.}
\choice{both even and odd.}
\end{multipleChoice}

\item The function corresponding to Graph F is
\begin{multipleChoice}
\choice{neither even nor odd.}
\choice{even, but not odd.}
\choice[correct]{odd, but not even.}
\choice{both even and odd.}
\end{multipleChoice}

\item The function corresponding to Graph G is
\begin{multipleChoice}
\choice{neither even nor odd.}
\choice{even, but not odd.}
\choice{odd, but not even.}
\choice[correct]{both even and odd.}
\end{multipleChoice}

\item The function corresponding to Graph H is
\begin{multipleChoice}
\choice[correct]{neither even nor odd.}
\choice{even, but not odd.}
\choice{odd, but not even.}
\choice{both even and odd.}
\end{multipleChoice}

\end{enumerate}
\end{exercise}
\end{document}