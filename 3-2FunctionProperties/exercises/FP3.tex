\documentclass{ximera}

\input{../../preamble.tex}

\author{Kenneth Berglund}
\acknowledgement{https://www.stitz-zeager.com/szca07042013.pdf}

\begin{document}
\licenseSZ
\begin{exercise}
\begin{enumerate}
\item The function $f$ defined by $f(x) = 12x$ is
\begin{multipleChoice}
\choice{even.}
\choice[correct]{odd.}
\choice{neither even nor odd.}
\choice{both even and odd.}
\end{multipleChoice}

\item The function $f$ defined by $f(x) = 12x + 2$ is
\begin{multipleChoice}
\choice{even.}
\choice{odd.}
\choice[correct]{neither even nor odd.}
\choice{both even and odd.}
\end{multipleChoice}

\item The function $f$ defined by $f(x) = 12$ is
\begin{multipleChoice}
\choice[correct]{even.}
\choice{odd.}
\choice{neither even nor odd.}
\choice{both even and odd.}
\end{multipleChoice}

\item The function $f$ defined by $f(x) = 5x^2 - 4$ is
\begin{multipleChoice}
\choice[correct]{even.}
\choice{odd.}
\choice{neither even nor odd.}
\choice{both even and odd.}
\end{multipleChoice}

\item The function $f$ defined by $f(x) = 3x^3 - 5x$ is
\begin{multipleChoice}
\choice{even.}
\choice[correct]{odd.}
\choice{neither even nor odd.}
\choice{both even and odd.}
\end{multipleChoice}

\item The function $f$ defined by $f(x) = 0$ is
\begin{multipleChoice}
\choice{even.}
\choice{odd.}
\choice{neither even nor odd.}
\choice[correct]{both even and odd.}
\end{multipleChoice}

\end{enumerate}
\end{exercise}
\end{document}