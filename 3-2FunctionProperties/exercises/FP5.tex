\documentclass{ximera}

\input{../../preamble.tex}

\author{Kenneth Berglund}
\acknowledgement{}

\begin{document}
\begin{exercise}
Recall from the chapter that sine and cosine are periodic functions with period $2\pi$. This means that for all inputs $x$, $\sin(x + 2\pi) = \sin(x)$ and $\cos(x + 2\pi) = \cos(x)$. 

\begin{enumerate}
\item Consider the function $f$ defined by $f(x) = 2\sin(x)$. $f$ is
\begin{multipleChoice}
\choice{not periodic.}
\choice{periodic with period $\pi$.}
\choice[correct]{periodic with period $2\pi$.}
\choice{periodic with period $3\pi$.}
\choice{periodic with period $4\pi$.}
\end{multipleChoice}

\item Consider the function $g$ defined by $g(x) = \cos(x + 5)$. $g$ is
\begin{multipleChoice}
\choice{not periodic.}
\choice{periodic with period $\pi$.}
\choice[correct]{periodic with period $2\pi$.}
\choice{periodic with period $3\pi$.}
\choice{periodic with period $4\pi$.}
\end{multipleChoice}

\item Consider the function $h$ defined by $h(x) = \sin(x) + 2\cos(x)$. $h$ is
\begin{multipleChoice}
\choice{not periodic.}
\choice{periodic with period $\pi$.}
\choice[correct]{periodic with period $2\pi$.}
\choice{periodic with period $3\pi$.}
\choice{periodic with period $4\pi$.}
\end{multipleChoice}
\end{enumerate}

\end{exercise}
\end{document}