\documentclass{ximera}

\input{../../preamble.tex}

\author{Kenneth Berglund}
\acknowledgement{https://www.stitz-zeager.com/szct07042013.pdf}

\begin{document}
\begin{exercise}
Recall from the chapter that sine and cosine are periodic functions with period $2\pi$. This means that for all inputs $x$, $\sin(x + 2\pi) = \sin(x)$ and $\cos(x + 2\pi) = \cos(x)$. 

Many functions that can be built out of $\sin$ and $\cos$ are also periodic. In this exercise, we'll use Desmos to explore how the period can change. 

\begin{enumerate}

\item Consider the function $f$ defined by $f(x) = \sin(3x)$. For reference, here is a graph of $f$ on Desmos:
\begin{center}
\desmos{uc3meehtpv}{800}{600}
\end{center}

The period of $f$ is
\begin{multipleChoice}
\choice{$\pi$.}
\choice{$2\pi$.}
\choice{$3\pi$.}
\choice{$6\pi$.}
\choice{$\frac{\pi}{2}$.}
\choice[correct]{$\frac{2\pi}{3}$.}
\end{multipleChoice}

\item Consider the function $g$ defined by $g(x) = \cos\left(\frac{x}{3}\right)$. For reference, here is a graph of $g$ on Desmos:
\begin{center}
\desmos{364oqkoauu}{800}{600}
\end{center}

The period of $g$ is
\begin{multipleChoice}
\choice{$\pi$.}
\choice{$2\pi$.}
\choice{$3\pi$.}
\choice[correct]{$6\pi$.}
\choice{$\frac{\pi}{2}$.}
\choice{$\frac{2\pi}{3}$.}
\end{multipleChoice}

\item Consider the function $h$ defined by $h(x) = \sin(2x - \pi)$. For reference, here is a graph of $h$ on Desmos:
\begin{center}
\desmos{wha8ccbi93}{800}{600}
\end{center}

The period of $h$ is
\begin{multipleChoice}
\choice[correct]{$\pi$.}
\choice{$2\pi$.}
\choice{$3\pi$.}
\choice{$6\pi$.}
\choice{$\frac{\pi}{2}$.}
\choice{$\frac{2\pi}{3}$.}
\end{multipleChoice}

\end{enumerate}

\end{exercise}
\end{document}