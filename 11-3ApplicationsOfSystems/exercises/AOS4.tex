\documentclass{ximera}

\input{../../preamble.tex}

\author{Kenneth Berglund}
\acknowledgement{https://www.stitz-zeager.com/szca07042013.pdf}

\begin{document}
\licenseSZ
\begin{exercise}
A certain bacteria culture follows grows at an exponential rate.

The number of bacteria $N$ at a time $t$ is given by the formula
$$
N(t) = N_0 e^{kt},
$$
where $N(0) = N_0$ is the initial number of bacteria and $k > 0$ is a constant.

After 10 minutes, there are 10,000 bacteria. Five minutes later, there are 14,000 bacteria. 

\begin{enumerate}
\item How many bacteria were present initially? $\answer{\frac{250000}{49}}$
\item How long before there are 50,000 bacteria? $\answer{\frac{5\ln(49/5)}{\ln(7/5)}}$
\end{enumerate}


\end{exercise}
\end{document}