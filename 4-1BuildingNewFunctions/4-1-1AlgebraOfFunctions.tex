\documentclass{ximera}

\input{../preamble.tex}
\author{Bobby Ramsey}
\license{Creative Commons Attribution-ShareAlike 4.0 International License}
\acknowledgement{https://activecalculus.org/prelude/sec-changing-combining.html}

\title{Algebra of Functions}


%\item Algebra of Functions
%\begin{itemize}
%	\item Add, subtract, multiply, and divide functions 
%	\item Think of complicated functions as objects in their own right 
%	\item Evaluate complicated functions 
%	\item Break more complicated functions into famous function %(prep for product and quotient rule) 
%	\item Undertand functions via graphs, tables, algebraically, and abstractly 
%\end{itemize}

\begin{document}
\begin{abstract}
  
\end{abstract}
\licenseAPC
\maketitle

%\typeout{************************************************}
%\typeout{Motivating Questions}
%\typeout{************************************************}

\begin{motivatingQuestions} 
	\begin{itemize}
		\item We know that we can add, subtract, multiply, and divide numbers. What kinds of operations can we perform on functions?
	\end{itemize} 
\end{motivatingQuestions}


%\typeout{************************************************}
%\typeout{Subsection Introduction}
%\typeout{************************************************}

\section{Introduction}

	In arithmetic, we execute processes where we take two numbers to generate a new number. For example, 
	$2 + 3 = 5$. The number $5$ results from adding the numbers $2$ and $3$.
	Similarly, we can multiply two numbers to generate a new one: $2 \cdot 3 = 6$.
	
	We can work similarly with functions. Just as we can add, subtract, multiply, and divide numbers, we can also add, subtract, multiply, and divide 
	functions to create a new function from two or more given functions.

\section{Algebra of Functions}

	In most mathematics up until calculus, the main object we study is \emph{numbers}. We ask questions such as
	\begin{itemize}
		\item ``What number(s) form solutions to the equation $x^2 - 4x - 5 = 0$?''\\
		\item ``What number is the slope of the line represented by $3x - 4y = 7$?''\\
		\item ``What number is generated as output by the function $f(x) = \sqrt{x^2 + 1}$ by the input $x = -2$?''
	\end{itemize}
	
	Certainly we also study overall patterns as seen in functions and equations, but this usually occurs through an examination of numbers themselves, 
	and we think of numbers as the main objects being acted upon.
	
	This changes in calculus.  In calculus, the fundamental objects being studied are functions themselves.  A function is a much more sophisticated 
	mathematical object than a number, in part because a function can be thought of in terms of its graph, which is an infinite collection of ordered 
	pairs of the form $(x,f(x))$.
	
	It is often helpful to look at a function's formula and observe algebraic structure.  For instance, given the quadratic function
	$$q(x) = -3x^2 + 5x - 7$$
	we might benefit from thinking of this as the sum of three simpler functions:  the constant function $c(x) = -7$, the linear function $s(x) = 5x$ that 
	passes through the point $(0,0)$ with slope $m = 5$, and the concave down basic quadratic function $w(x) = -3x^2$.  
	Indeed, each of the simpler functions $c$, $s$, and $w$ contribute to making $q$ be the function that it is.  Likewise, if we were interested in 
	the function $p(x) = (3x^2 + 4)(9 - 2x^2)$, it might be natural to think about the two simpler functions $f(x) = 3x^2 + 4$ and $g(x) = 9 - 2x^2$ 
	that are being multiplied to produce $p$.
	
	We thus naturally arrive at the ideas of adding, subtracting, multiplying, or dividing two or more functions, and hence introduce the following definitions and notation.
	\begin{definition}
		Let $f$ and $g$ be functions. 
		\begin{itemize}[label=\textbullet]
			\item The \dfn{sum of $f$ and $g$} is the function $f + g$ defined by $(f+g)(x) = f(x) + g(x)$.\\
			\item The \dfn{difference of $f$ and $g$} is the function $f - g$ defined by $(f-g)(x) = f(x) - g(x)$.\\
			\item The \dfn{product of $f$ and $g$} is the function $f \cdot g$ defined by $(f \cdot g)(x) = f(x) \cdot g(x)$.\\
			\item The \dfn{quotient of $f$ and $g$} is the function $\frac{f}{g}$ defined by $\left( \frac{f}{g} \right)(x) = \frac{f(x)}{g(x)}$ for all $x$ such that $g(x) \neq 0$.
		\end{itemize}
	\end{definition}
	We are thinking here about $f$ and $g$ being functions with real numbers as outputs. Performing these operations on the functions means applying 
	the corresponding operation to the output values of the functions.
	
	\begin{example}
		Consider the functions $f$ and $g$ defined by the table of values below. \\
		\begin{minipage}{.5\textwidth}
		  	\begin{center}
				$\begin{array}{||c|c||}
					\hline 
					\hline
					x & f(x)\\
					\hline
					2&0\\
					\hline
					4&3\\
					\hline
					6&7\\
					\hline
					8& -2\\
					\hline 
					\hline
				\end{array}$
			\end{center}
		\end{minipage}% This must go next to `\end{minipage}`
		\begin{minipage}{.5\textwidth}
		  	\begin{center}
				$\begin{array}{||c|c||}
					\hline 
					\hline
					x & g(x)\\
					\hline
					1&5\\
					\hline
					2&9\\
					\hline
					3&-1\\
					\hline
					4& 4\\
					\hline 
					\hline
				\end{array}$
			\end{center}
		\end{minipage}
		\begin{enumerate}
			\item Determine the value of $(f+g)(2)$.\\
			\item Determine the value of $(f-g)(4)$.\\
			\item Determine the value of $(f \cdot g)(2)$.\\
			\item Determine the value of  $\left( \frac{f}{g} \right)(4)$.\\
			\item What can we say about the value of $(f+g)(3)$?\\
		\end{enumerate}
		\begin{explanation}
			\begin{enumerate}
				\item We know that $(f+g)(2) = f(2)+g(2)$. From the tables above $f(2) = 0$ and $g(2)=9$, so $(f+g)(2)=0+9 = 9$.
				
				\item Since $f(4)=3$ and $g(4)=4$, we know $(f-g)(4)=f(4) - g(4) = 3-4 = -1$.

				\item $(f\cdot g)(2) = f(2)\cdot g(2) = 0 \cdot 9 = 0$.

				\item $\left( \frac{f}{g}\right)(4) = \frac{f(4)}{g(4)}=\frac{3}{4}$.
				
				\item The value of $(f+g)(3)$ would be given by $f(3)+g(3)$. we are given the value of $g(3)$ in the table above, but there is no listed
					value for $f(3)$. That means $f(3)$ is undefined, since $3$ is not a valid input. Therefore, $(f+g)(3)$ is undefined.
			\end{enumerate}
		\end{explanation}
	\end{example}	


	\begin{example}
		Consider the functions $f$ and $g$ defined by \\
		
		\begin{minipage}{.5\textwidth}
		  	\begin{center}
				\begin{tikzpicture}
					\begin{axis}[
						xmin=-3.3, xmax=3.3, ymin=-2.3,ymax=5.3,    
						width=0.75\linewidth,
		                		minor ytick=,minor xtick=,
			                	xtick={-3,...,3}, ytick={-2,...,5},
		                		clip=false,
						grid style={dashed, gray!40}
						]
						\addplot[color=penColor, very thick, domain=-3.3:-1]{-0.5*x+2.5} node[pos=0.5, above]{\Large{$f$}}; 
		        			\addplot[soliddot] coordinates {(-1,3)};
		
						\path[draw, color=penColor, very thick] (axis cs:-1,1) --  (axis cs:1,4);
		               			\addplot[hollowdot] coordinates {(-1,1)};
		        			\addplot[soliddot, color=penColor] coordinates {(1,4)};
		
						\addplot[color=penColor, very thick, domain=1:3.3]{(-3/2)*(x-1)+1}; 
						\addplot[hollowdot] coordinates {(1,1)};
		               			\addplot[hollowdot] coordinates {(2,-0.5)};
					\end{axis}
				\end{tikzpicture}
			\end{center}
		\end{minipage}% This must go next to `\end{minipage}`
		\begin{minipage}{.5\textwidth}
		  	\begin{center}
				\begin{tikzpicture}
					\begin{axis}[
						xmin=-3.3, xmax=3.3, ymin=-2.3,ymax=5.3,    
						width=0.75\linewidth,
		                		minor ytick=,minor xtick=,
			                	xtick={-3,...,3}, ytick={-2,...,5},
		                		clip=false,
						grid style={dashed, gray!40}
						]
						\addplot[color=penColor, very thick, domain=-2.5:1]{-x^2+4} node[pos=0.55, above left]{\Large{$g$}}; 
		        			\addplot[soliddot] coordinates {(1,3)};
		               			\addplot[hollowdot] coordinates {(0,4)};
		        			\addplot[soliddot, color=penColor] coordinates {(0,1)};
		
						\addplot[color=penColor, very thick, domain=1:3.3]{-x+3}; 
		               			\addplot[hollowdot] coordinates {(1,2)};
	             				\addplot[hollowdot] coordinates {(2,1)};    					
					\end{axis}
				\end{tikzpicture}
			\end{center}
		\end{minipage}
		\begin{enumerate}
			\item Determine the exact value of $(f+g)(0)$.\\
			\item Determine the exact value of $(g-f)(1)$.\\
			\item Determine the exact value of $(f \cdot g)(-1)$.\\
			\item Are there any values of $x$ for which $\left( \frac{f}{g} \right)(x)$ is undefined?  
				If not, explain why.  If so, determine the values and justify your answer.\\
			\item For what values of $x$ is $(f \cdot g)(x) = 0$?  Why?\\
		\end{enumerate}
		\begin{explanation}
			\begin{enumerate}
				\item The notation $(f+g)(0)$ means we are plugging the input $0$ into both functions $f$ and $g$, then \emph{adding} the results.
					That is, $(f+g)(0) = f(0) + g(0)$. From the graphs above we see $f(0) = \frac{5}{2}$ and $g(0) = 1$. That means
					$(f+g)(0) = \frac{5}{2} + 1 = \frac{7}{2}$.

				\item The notation $(g-f)(1)$ means we are plugging the input $1$ into both functions $g$ and $f$, then \emph{subtracting} the
					 results.That is, $(g-f)(1) = g(1) - f(1)$. From the graphs above we see $f(1) = 4$ and $g(1) = 3$. That means
					$(g-f)(1) = 3 - 4 = -1$.					

				\item The notation $(f\cdot g)(-1)$ means we are plugging the input $-1$ into both functions $f$ and $g$, then \emph{multiplying}
					 the results. That is, $(f\cdot g)(-1) = f(-1) \cdot g(-1)$. 
					 From the graphs above we see $f(-1) = 3$ and $g(-1) = 3$, which tells us 
					$(f\cdot g)(-1) = 3 \cdot 3 = 9$.

				\item For any valid value of the input $x$,  $\left(\frac{f}{g}\right)(x) = \frac{f(x)}{g(x)}$. In order for that fraction to be defined
					$f(x)$ has to exist, $g(x)$ has to exist, and $g(x) \neq 0$ since division by zero is undefined. 
					From the graphs above, $f$ is defined for all $x$-values except $x=2$, and $g$ is defined for all $x$-values except $x=2$. That
					tells us that $\left( \frac{f}{g} \right)(2)$ is undefined. Notice that $g(-2)=0$ and $g(3)=0$? That means 
					$\left( \frac{f}{g}\right)(x)$ is undefined at $x=-2$ and $x=3$ as well
					
				\item Since $( f \cdot g)(x) = f(x) \cdot g(x)$, if an $x$-value makes $(f \cdot g)(x)=0$, then $f(x)\cdot g(x)= 0$. The only
					way a product of two real numbers can be zero is if at least one of the factors is itself zero. That means we are looking
					for all of the $x$-values satisfying either $f(x)=0$ or $g(x)=0$. (In other words, we're looking for the $x$-intercepts of 
					these graphs.)\\
					
					From the graph of $g$ we see that $g(-2)=0$ and $g(3)=0$. The graph of $f$ crosses the $x$-axis somewhere between the
					points $(1,0)$ and $(2,0)$, but we will have to be more careful to find the exact value we are looking for. \\
					
					Notice that the graph of $f$ looks to be a straight line if we only look at those $x$-values with $x>1$. The straight line that
					$f$ follows travels through the point $(1,1)$ and $(3, -2)$. Its slope is given by $m = \frac{-2 - 1}{3 - 1} = -\frac{3}{2}$. Since
					the line contains the point $(1,1)$, the point-slope form of the equation of the line can be written as $y-1 = -\frac{3}{2}(x-1)$.
					This line crosses the $x$-axis when its $y$-coordinate is zero. Solving for the corresponding $x$-value gives us:
					\begin{align*}
						0 - 1 &= -\frac{3}{2}(x-1)\\
						-1 &= -\frac{3}{2}(x-1)\\
						-\frac{2}{3} \cdot (-1) &= -\frac{2}{3} \cdot \left( -\frac{3}{2}(x-1) \right)\\
						\frac{2}{3} &= x-1\\
						\frac{2}{3} + 1 &= x\\
						\frac{5}{3} &= x.
					\end{align*}
					That means the point $\left( \frac{5}{3}, 0 \right)$ is on the graph of $f$, so $f\left( \frac{5}{3} \right) = 0$.\\
					
					The $x$-values with $(f \cdot g)(x) = 0$ are $x=-2$, $x=\frac{5}{3}$, and $x=3$.
			\end{enumerate}
		\end{explanation}	
	\end{example}

	\begin{remark}
	The only way a product of two real numbers can  be zero is if at least one of the factors is itself zero. That is, if $ab = 0$ for two numbers $a$ and $b$, then $a= 0$ or $b = 0$. This technique does not work for numbers other than zero. 
	\end{remark}

	Consider the functions $f(x) = 2x$ and $g(x) = x$. These are functions whose graphs are straight lines with slopes $2$ and $1$ respectively.
  	\begin{image}
		\begin{tikzpicture}
			\begin{axis}[
				xmin=-4.3, xmax=4.3, ymin=-10.3,ymax=10.3,    
				width=0.75\linewidth,
                		minor ytick=,minor xtick=,
	                	xtick={-4,...,4}, ytick={-10,-8,...,10},
                		clip=false,
				grid style={dashed, gray!40}
				]
				\addplot+[color=penColor, very thick, domain=-4.3:4.3]{2*x} node[pos=0.85, above]{\Large{$f$}}; 
				\addplot+[color=penColor2, very thick, domain=-4.3:4.3]{x} node[pos=0.85, below right]{\Large{$g$}}; 
			\end{axis}
		\end{tikzpicture}
	\end{image}

	Let's look at a few points on these graphs. Since $f(1)=2$ and $g(1)=1$, the sum of those output values is $3$, so we'll mark the point $(1,3)$.
	Similarly $f(2)=4$ and $g(2)=2$, with $4+2=6$ so we'll mark the point $(2,6)$. As $f(3)=6$ and $g(3)=3$, we'll also mark $(3,9)$.
  	\begin{image}
		\begin{tikzpicture}
			\begin{axis}[
				xmin=-4.3, xmax=4.3, ymin=-10.3,ymax=10.3,    
				width=0.75\linewidth,
                		minor ytick=,minor xtick=,
	                	xtick={-4,...,4}, ytick={-10,-8,...,10},
                		clip=false,
				grid style={dashed, gray!40}
				]
				\addplot+[color=penColor, very thick, domain=-4.3:4.3]{2*x} node[pos=0.85, above]{\Large{$f$}}; 
				\addplot+[color=penColor2, very thick, domain=-4.3:4.3]{x} node[pos=0.85, below right]{\Large{$g$}}; 
%				\addplot+[color=penColor3, very thick, domain=-3.3:3.3]{3*x} node[pos=0.85, above left]{\Large{$f+g$}}; 
								
        			\addplot[soliddot, color=penColor] coordinates {(2,4)};
         			\addplot[soliddot, color=penColor2] coordinates {(2,2)};
         			\addplot[soliddot, color=penColor3] coordinates {(2,6)} node[above]{$(2.6)$};

        			\addplot[soliddot, color=penColor] coordinates {(1,2)};
         			\addplot[soliddot, color=penColor2] coordinates {(1,1)};
         			\addplot[soliddot, color=penColor3] coordinates {(1,3)} node[above]{$(1, 3)$};
	
	       			\addplot[soliddot, color=penColor] coordinates {(3,6)};
         			\addplot[soliddot, color=penColor2] coordinates {(3,3)};
         			\addplot[soliddot, color=penColor3] coordinates {(3,9)} node[above]{$(3,9)$};
			\end{axis}
		\end{tikzpicture}
	\end{image}
	Notice that the points we've marked $(1,3)$, $(2, 6)$, and $(3,9)$ are starting to form a straight line. 
	Let's connect those dots to examine the line constructed this way.
  	\begin{image}
		\begin{tikzpicture}
			\begin{axis}[
				xmin=-4.3, xmax=4.3, ymin=-10.3,ymax=10.3,    
				width=0.75\linewidth,
                		minor ytick=,minor xtick=,
	                	xtick={-4,...,4}, ytick={-10,-8,...,10},
                		clip=false,
				grid style={dashed, gray!40}
				]
				\addplot+[color=penColor, very thick, domain=-4.3:4.3]{2*x} node[pos=0.85, above]{\Large{$f$}}; 
				\addplot+[color=penColor2, very thick, domain=-4.3:4.3]{x} node[pos=0.85, below right]{\Large{$g$}}; 
				\addplot+[color=penColor3, very thick, domain=-3.3:3.3]{3*x} node[pos=0.85, above left]{\Large{$f+g$}}; 
								
        			\addplot[soliddot, color=penColor] coordinates {(2,4)};
         			\addplot[soliddot, color=penColor2] coordinates {(2,2)};
         			\addplot[soliddot, color=penColor3] coordinates {(2,6)};

        			\addplot[soliddot, color=penColor] coordinates {(1,2)};
         			\addplot[soliddot, color=penColor2] coordinates {(1,1)};
         			\addplot[soliddot, color=penColor3] coordinates {(1,3)};
	
	       			\addplot[soliddot, color=penColor] coordinates {(3,6)};
         			\addplot[soliddot, color=penColor2] coordinates {(3,3)};
         			\addplot[soliddot, color=penColor3] coordinates {(3,9)};
			\end{axis}
		\end{tikzpicture}
	\end{image}
	The graph obtained this way is the graph of $f+g$.  This graph is a straight line passing through the points $(0,0)$ and $(1,3)$, so the line has
	equation $y=3x$.
	
	For a given value of $x$, we know $(f+g)(x) = f(x)+g(x)$. This means $(f+g)(x) =2x + x = 3x$ by combining the like terms. Notice that this 
	aligns with the graph we found above. This example shows that we can work with these operations through formulas for our functions as well.
	
	\begin{example}
		Let $f$ and $g$ be the functions given by the formulas $f(x) = 2\sin(x)$ and $g(x)=5x-4$.
		\begin{enumerate}
			\item Find the value of $( f - g)\left( \frac{\pi}{3} \right)$.\\
			\item Find a formula for $( f + g)(x)$.\\
			\item Find a formula for $( f \cdot g)(x)$.\\
			\item Find a formula for $\left( \frac{f}{g}\right)(x)$.\\
		\end{enumerate}
		\begin{explanation}
			\begin{enumerate}
				\item $(f-g)\left( \frac{\pi}{3} \right) = f\left( \frac{\pi}{3} \right)-g\left( \frac{\pi}{3} \right)$.
					Sine is one of our famous functions, which has $\sin\left( \frac{\pi}{3} \right) = \frac{\sqrt{3}}{2}$, so 
					$f\left( \frac{\pi}{3} \right) = 2 \sin\left( \frac{\pi}{3} \right) = 2 \frac{\sqrt{3}}{2} = \sqrt{3}$.
					$g\left( \frac{\pi}{3} \right) = 5\left( \frac{\pi}{3} \right)-4 = \frac{5\pi}{3}-4$.
					
					Together this gives $(f-g)\left( \frac{\pi}{3} \right) = \sqrt{3} - \frac{5\pi}{3}+4$
				
				\item $( f + g)(x) = f(x) + g(x) = 2\sin(x) + 5x - 4$.\\
					
				\item $( f \cdot g)(x) = 2\sin(x)\left(5x-4\right)$.\\

				\item $\left( \frac{f}{g}\right)(x) = \frac{2\sin(x)}{5x-4}$.\\
			\end{enumerate}
		\end{explanation}
	\end{example}
	
	

	
	When we work in applied settings with functions that model phenomena in the world around us, it is often useful to think carefully about the units of 
	various quantities. Analyzing units can help us both understand the algebraic structure of functions and the variables involved, as well as assist us in 
	assigning meaning to quantities we compute. We have already seen this with the notion of average rate of change: if a function $P(t)$ measures the 
	population in a city in year $t$ and we compute $\av_{[5, 11]}$, then the units on $\av_{[5, 11]}$ are ``people per year,'' and the value of 
	$\av_{[5, 11]}$ is telling us the average rate at which the population changes in people per year on the time interval from year $5$ to year $11$.


	\begin{example}
		Say that an investor is regularly purchasing stock in a particular company. Let $N(t)$ represent the number of shares owned on day $t$,
		 where $t = 0$ represents the first day on which shares were purchased. Let $S(t)$ give the value of one share of the stock on day $t$; 
		 note that the units on $S(t)$ are dollars per share. How is the total value, $V(t)$, of the held stock on day $t$ determined?
		\begin{explanation}
			Observe that the units on $N(t)$ are ``shares'' and the units on $S(t)$ are ``dollars per share''.  Thus when we compute the product
			$$N(t) \, \text{shares}  \cdot S(t) \, \text{dollars per share}\text{,}$$
			it follows that the resulting units are ``dollars'', which is the total value of held stock.  Hence,
			$$V(t) = N(t) \cdot S(t)\text{.} $$
		\end{explanation}
	\end{example}



	\begin{exploration}
		Let $f$ be a function that measures a car's fuel economy in the following way. Given an input velocity $v$ in miles per hour, $f(v)$ is the number 
		of gallons of fuel that the car consumes per mile (i.e., ``gallons per mile''). We know that $f(60) = 0.04$.
		\begin{enumerate}
			\item What is the meaning of the statement ``$f(60) = 0.04$'' in the context of the problem? That is, what does this say about 
				the car's fuel economy? Write a complete sentence.
			\item Consider the function $g(v) = \frac{1}{f(v)}$. What is the value of $g(60)$? What are the units on $g$? What does $g$ measure? 
			\item Consider the function $h(v) = v \cdot f(v)$. What is the value of $h(60)$? What are the units on $h$? What does $h$ measure?
			\item Do $f(60)$, $g(60)$, and $h(60)$ tell us fundamentally different information, or are they all essentially saying the same thing?  Explain.
			\item Suppose we also know that $f(70) = 0.045$. Find the average rate of change of $f$ on the interval $[60,70]$. What are the units 
				on the average rate of change of $f$? What does this quantity measure? Write a complete sentence to explain.
		\end{enumerate}
	\end{exploration}	
	
	
	Taking a complicated function and determining how it is constructed out of simpler ones is an important skill to develop. At the beginning of this section 
	we split the function $q(x) =  -3x^2 + 5x - 7$ into the sum/difference of three simple functions, $-3x^2$, $5x$, and $7$. Let us experiment with 
	splitting a few more complicated functions.
	\begin{example}
		\begin{enumerate}	
			\item Find functions $h$ and $k$ so that $f(x) = 4x^2 \sin(x)$ can be written as $(h\cdot k)(x)$.\\
			\item Find functions $f$ and $g$ so that $h(x) = \frac{2x+3}{x-1}$ can be written as $\left(\frac{f}{g}\right)(x)$.\\
			\item Find functions $r$ and $s$ so that $t(x) = \frac{x+1}{3x-1}+ x e^x$ can be written as $\left(r + s \right)(x)$.\\
			\item Find functions $f$, $g$, and $h$ so that $k(x) = \frac{\sin(x) \sqrt{2x+3}}{1+\ln(x)}$ can be written as 
				$\left( \frac{f\cdot g}{h}\right)(x)$.\\
		\end{enumerate}
		
		
		\begin{explanation}

			Before working through these questions, we want to remind you that the answers we give are not unique. There are many different, 
			equally valid choices for the simpler functions requested. For the first question, we will mention multiple possibilities. We leave it to you
			to find other answers for the remaining questions.
			\begin{enumerate}	
				\item Notice that if $h(x) = 4x^2$ and $k(x) = \sin(x)$, then $\left( h \cdot k \right)(x) = 4x^2 \sin(x)$, so this choice of $h$ and $k$ 
					is one valid answer. What if we had chosen $h(x) = 4x$ and $k(x) = x\sin(x)$ instead? Then 
					$\left( h \cdot k\right)(x) = (4x)(x\sin(x))=4x^2 \sin(x)$, so this choice would also be a valid answer. Another possibility
					would have been to choose $h(x) = 4$ and $k(x) = x^2 \sin(x)$. \\
				\item If we set $f(x) = 2x+3$ and $g(x) = x-1$, then $\left(\frac{f}{g}\right)(x) = \frac{2x+3}{x-1}$.\\
				\item Since $t(x) = \frac{x+1}{3x-1} + xe^x$ is written as a sum of two terms, take $r(x) = \frac{x+1}{3x-1}$ to be the first term
					and $s(x) = xe^x$ to be the seond term. Then $(r+s)(x) = t(x)$.
				\item We are trying to identify $ \frac{\sin(x) \sqrt{2x+3}}{1+\ln(x)}$ as a fraction with the numerator a product. The denominator is 
					$1 + \ln(x)$ and the numerator is a product of $\sin(x)$ and $\sqrt{2x+3}$. We will choose $f(x) = \sin(x)$, 
					$g(x) = \sqrt{2x+3}$, and $h(x) = 1 + \ln(x)$.
			\end{enumerate}
		\end{explanation}
	\end{example}
	
	\begin{exploration}
		\begin{enumerate}
			\item Find functions $f$ and $g$ so that $h(x) = \frac{5 e^x}{1+\sin(x)}$ can be written as $\left(f \cdot g\right)(x)$.\\
			\item Find functions $h$ and $k$ so that $f(x) = 3x^2 - \sqrt{x+1}$ can be written as $(h + k)(x)$. Find two other choices for $h$ and $k$.\\
			\item If $f$ and $g$ are functions, we know that $f+g$ is the function given by $(f+g)(x) = f(x)+g(x)$. What function do you think the 
				notation $f + 3g$ means? Find functions $f$ and $g$ so that $h(x) = 4x^2-5\sqrt{x} + 7\cdot 2^x$ can be written as 
				$\left( f + 3g\right)(x)$.\\
		\end{enumerate}
	\end{exploration}
	
%\typeout{************************************************}
%\typeout{Summary}
%\typeout{************************************************}

%\begin{summary}
%\item 
%\item 
%\item
%\end{summary}




\end{document}
