\documentclass{ximera}

\input{../../preamble.tex}

\author{Elizabeth Campolongo, Kenneth Berglund}
\acknowledgement{https://www.stitz-zeager.com/szprecalculus07042013.pdf}

\begin{document}
In this multi-step exercise, we will solve an equation involving the absolute value: 
$$|x - 3| - |2x + 1| = 0.$$
%%%%%%%%%%%%%%%%%%%%%%

\begin{exercise}
Moving $|2x + 1|$ to the other side of the equation, we obtain $|x -3| = \answer{|2x + 1|}$.

\begin{exercise}
Since $|x - 3| = |2x + 1|$, we know that $x - 3 = \pm \answer{|2x + 1|}$. 

\begin{exercise}
This gives us two equations:
\begin{enumerate}
\item $x - 3 = |2x + 1|$
\item $x - 3 = - |2x + 1|$. 
\end{enumerate}
From the first equation, we know that $2x + 1 = \pm(\answer{x - 3})$.
\begin{exercise}
If $2x + 1 = x - 3$, then $x = \answer{-4}$. If $2x + 1 = -(x - 3)$, then $x = \answer{\frac{2}{3}}$. 
\end{exercise}
From the second equation, we know that $-(x - 3) = |2x + 1|$, so therefore, $2x + 1 = \pm(\answer{-(x - 3)})$. 
\begin{exercise}
If $2x + 1 = -(x - 3)$, then $x = \answer{\frac{2}{3}}$. If $2x + 1 = x - 3$, then $x = \answer{-4}$. 
\end{exercise}
\begin{exercise}
What are all the solutions (from smaller to larger)?\\
$x=\answer{-4}$ and $x = \answer{\frac{2}{3}}$. 

\end{exercise}
\end{exercise}
\end{exercise}
\end{exercise}

\end{document}