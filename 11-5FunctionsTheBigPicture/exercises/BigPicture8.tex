\documentclass{ximera}

\input{../../preamble.tex}

\author{Elizabeth Campolongo}
\acknowledgement{https://www.stitz-zeager.com/szprecalculus07042013.pdf}

\begin{document}
\begin{exercise}
For each of the following function values give exact solutions ({\bf simplified} as much as possible) or write ``undefined" if the function is not defined for the given value.
%
\begin{enumerate}
%
\item Define $f(x)= 3x+4.$
\begin{enumerate}
\item $f(3) = \answer{13}$

\item $f(-4) = \answer{-8}$

\item $f(2+a)=\answer{10}+\answer{3}a$

\item $f(b-1) = \answer{3}b+\answer{1}$

\item $f(7c-2) = \answer{21}c+\answer{-2}$

\item $f(x^2) = \answer{3x^2+4}$

\item $f(3x^3+1) = \answer{9}x^3 +\answer{7}$
\end{enumerate}

\item Define $g(x)= x^2+2x-5.$
\begin{enumerate}
\item $g(2) = \answer{3}$

\item $g(2+a)=\answer{1}a^2+\answer{6}a+\answer{3}$

\item $g(b-1) = \answer{1}b^2+ \answer{-6}$

\item $g(3c-2) = \answer{9}c^2 +\answer{-6}c + \answer{-5}$

\item $g(d+1)+4 = \answer{1}d^2+\answer{4}d+\answer{2}$
\end{enumerate}


\item Define $h(x)= \cos(2x).$
\begin{enumerate}
\item $h\!\left(\frac{5\pi}{4}\right) = \answer{0}$

\item $h(2+a)=\answer{\cos(4+4a)}$

\item $h(x+\pi) = \answer{\cos(2x)}$

\item $h\!\left(x-\frac{\pi}{4}\right) = \answer{\sin(2x)}$

\item $h\!\left(x-\frac{\pi}{2}\right) = \answer{-\cos(2x)}$

\item $h\!\left(x+\frac{3\pi}{4}\right) = \answer{\sin(2x)}$
\end{enumerate}


\item Define $k(x)= 2|x-4|+1$.
\begin{enumerate}
\item $k(-3) = \answer{15}$

\item $k(2) -2 = \answer{3}$

\item $k(4) = \answer{1}$

\item $k(1+a)=\answer{2|a-3| +1}$

\item $k(2-b) +1 = \answer{2|-b-2|+2}$

\item For $b\geq0$, $k(2-b) +1 = \answer{2b+6}$

\item $k(c+5) = \answer{2|c+1|+1}$

\item $k(7-3d) = \answer{6|1-d|+1}$
\end{enumerate}

\item Define $f(x)= e^{x+1} -2$.
\begin{enumerate}
\item $f(0) = \answer{e-2}$

\item $f(-1) = \answer{-1}$

\item $f(3+a)=\answer{e^{a+4}-2}$

\item For $b<0$, $f(\ln(b)) = \answer{\text{undefined}}$

\item For $b>0$, $f(\ln(b)) = b\answer{e} + \answer{-2}$

\item $f(\ln(4)) = \answer{4}e + \answer{-2}$
\end{enumerate}

\item Define $g(x)= \ln(x-1)$.
\begin{enumerate}
\item $g(1) = \answer{\text{undefined}}$

\item $g(2) = \answer{0}$

\item For $a>0$, $g(a+1) = \answer{\ln(a)}$

\item $g\!\left(\frac{1}{3}\right) = \answer{\text{undefined}}$

%\item $g(2+a)=\answer{a^2+6a+3}$

\item For $b<0$, $g(4-b) = \answer{\ln(3-b)}$

\item For $b\ge 3$, $g(4-b) = \answer{\text{undefined}}$
%\item $g(3c-2) = \answer{9c^2-6c-5}$
\end{enumerate}

\item Define \begin{equation*}
	h(x) = \begin{cases}
		\frac{2x}{x^2-1},  & x < -1 \\
		\\
		\frac{x+1}{x^2+3x+2},  & x> -1 
		\end{cases}
		\end{equation*}
	\begin{enumerate}
	\item $h(1) = \answer{\frac{1}{3}}$
	
	\item $h(-1) = \answer{\text{undefined}}$
	
	\item $h(0) = \answer{\frac{1}{2}}$
	
	\item For $a>0$, $h(a-1) = \frac{1}{\answer{a+1}}$
	
	\item For $b \leq -4$, $h(b+2) =\frac{2b + 4}{\answer{(b+ 1)(b+3)}}$
	\end{enumerate}
\end{enumerate}

\end{exercise}
\end{document}
