\documentclass{ximera}

\input{../../preamble.tex}

\author{Elizabeth Campolongo}
\acknowledgement{https://www.stitz-zeager.com/szprecalculus07042013.pdf}

\begin{document}
\begin{exercise}
For the following word problems, determine what type of function is needed to solve them (e.g., linear, quadratic, cubic, logarithmic, exponential, etc.) and give the function. 
%


%%%%%%%%%%%%%%%%%%%%%%
The information entropy $H$, in bits, of a randomly generated password consisting
of $n$ characters is given by $H = n\log_2(N)$, where $N$ is the number of possible symbols for each
character in the password. In general, the higher the entropy, the stronger the password.

If a 7 character case-sensitive password is comprised of letters and numbers only, find the
associated information entropy.
What type of equation would be needed to solve this?\\
\wordChoice{\choice{linear}\choice{quadratic}\choice{cubic}\choice[correct]{logarithmic}\choice{exponential}}.
\begin{exercise}
Write out the equation to solve this question. For the last set of blanks, convert your answer to only use natural log.\\
$H = \answer{7}\log_{\answer{2}}(\answer{62}) = \answer{7}\frac{\answer{\ln(62)}}{\ln(\answer{2})}$.

\begin{exercise}
How many possible symbol options per character is required to produce a 7 character password
with an information entropy of 50 bits?\\
Again, write out the equation to solve this question.\\
$\answer{50} = \answer{7}\log_{\answer{2}}(\answer{N})$.
\end{exercise}
\end{exercise}


\end{exercise}
\end{document}