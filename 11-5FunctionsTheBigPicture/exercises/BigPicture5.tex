\documentclass{ximera}

\input{../../preamble.tex}

\author{Elizabeth Campolongo}
\acknowledgement{https://www.stitz-zeager.com/szprecalculus07042013.pdf}

\begin{document}
How many solutions are there to each of the following equations, and what are they?
\begin{enumerate}
\item
\begin{exercise}
 $2+\sqrt{3-x}=-2$ \\
This equation has $\answer{0}$ solution(s).
\end{exercise}
%
%%%%%%%%%%%%%%%%%%%%
\item 
\begin{exercise}
$4x^3-2=0$ \\
This equation has $\answer{1}$ solution(s).
%
\begin{exercise}
What is the solution?
$x=\answer{(0.5)^{1/3}}$
\end{exercise}
\end{exercise}
%%%%%%%%%%%%%%%%%%%%%
\item 
\begin{exercise}
$3x^2-4x+1=0$ \\
This equation has $\answer{2}$ solution(s).
\begin{exercise}
What are the solutions (from smaller to larger)?\\
$x=\answer{\frac{1}{3}}$ and $x=\answer{1}$
\end{exercise}
\end{exercise}

%%%%%%%%%%%%%%%%%%%%%%
%\item 
%\begin{exercise}
%$|4x-3|=x^2$ \\
%This equation has $\answer{4}$ solution(s).
%\begin{exercise}
%What are the solutions (from smaller to larger)?\\
%$x = \answer{-2-\sqrt{7}}$, $x=\answer{-2+\sqrt{7}}$, $x=\answer{1}$, and $x = \answer{3}$. 
%\end{exercise}
%\end{exercise}

%%%%%%%%%%%%%%%%%%%%%%
\item 
\begin{exercise}
$x^2-6x=-10$ \\
This equation has $\answer{0}$ solution(s).
\end{exercise}


%%%%%%%%%%%%%%%%%%%%%%
\item 
\begin{exercise}
$4x^6+5=9$ \\
This equation has $\answer{2}$ solution(s).
\begin{exercise}
What are the solutions (from smaller to larger)?\\
$x=\answer{-1}$ and $x=\answer{1}$
\end{exercise}
\end{exercise}

%%%%%%%%%%%%%%%%%%%%%%
%\item 
%\begin{exercise}
%$|x-3|-|2x+1|=0$ \\
%This equation has $\answer{2}$ solution(s).
%\begin{exercise}
%What are the solutions (from smaller to larger)?\\
%$x=\answer{-4}$ and $x=\answer{\frac{2}{3}}$
%\end{exercise}
%\end{exercise}

%%%%%%%%%%%%%%%%%%%%%%
%\item $4x^3-8x^2-11x=3$ \\
\item 
\begin{exercise}
$(2x+1)(2x^2-5x-3)=0$ \\
This equation has $\answer{2}$ solution(s).
\begin{exercise}
What are the solutions (from smaller to larger)?\\
$x=\answer{-\frac{1}{2}}$ and $x=\answer{3}$
\end{exercise}
\end{exercise}

%%%%%%%%%%%%%%%%%%%%%%
\item 
\begin{exercise}
$e^{\frac{1}{3} + 2x}=\sqrt{x}-2$ \\
This equation has $\answer{0}$ solution(s).
\end{exercise}
%%%%%%%%%%%%%%%%%%%%%%
%\item $\log_2(x-1) =\sqrt{x+2}-2$. \\
%This equation has $\answer{1}$ solution(s).
%\begin{exercise}
%What is the solution?
%$x=\answer{2}$
%\end{exercise}


\end{enumerate}

\end{document}