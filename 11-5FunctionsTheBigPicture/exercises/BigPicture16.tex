\documentclass{ximera}

\input{../../preamble.tex}

\author{Elizabeth Campolongo}
\acknowledgement{https://www.stitz-zeager.com/szprecalculus07042013.pdf}

\begin{document}
In this multi-step exercise, we will solve an equation involving the absolute value: 
$$|4x-3|=x^2.$$
%%%%%%%%%%%%%%%%%%%%%%

\begin{exercise}
If $|4x - 3| = x^2$, that means that $4x - 3 = \pm \answer{x^2}$.

\begin{exercise}
This gives us two equations to solve: $4x - 3 = x^2$ and $4x - 3 = -x^2$. 

These equations have $\answer{4}$ solutions altogether.
\begin{exercise}
What are the solutions (from smaller to larger)?\\
$x = \answer{-2-\sqrt{7}}$, $x=\answer{-2+\sqrt{7}}$, $x=\answer{1}$, and $x = \answer{3}$. 
\end{exercise}
\end{exercise}
\end{exercise}


\end{document}