\documentclass{ximera}

\input{../../preamble.tex}

\author{Ivo Terek}
\acknowledgement{}

\begin{document}
\begin{exercise}
  Let's approximate the are under the graph of $f(x) = x^2+x$, over the interval $[1,4]$.

  \begin{enumerate}
  \item Dividing $[1,4]$ into $n$ equal subintervals, the length of each one of them is $\Delta x = \frac{\answer{3}}{n}$.
  \item Writing $$x_0 = 1,\quad x_1 = 1+\Delta x, \quad x_2 = 1+2\Delta x,\quad\ldots,\quad, x_n = 1+ n\Delta x = 4,$$ we have that $x_k = 1+\frac{\answer{3}k}{n}$.
  \item Setting up a left Riemann sum, we have that $$\mbox{area under curve} \approx\sum_{k=0}^{n-1} f(x_k)\Delta x = \sum_{k=0}^{n-1} \left(\left(1+\frac{\answer{3}k}{n}\right)^2+\left(1+\frac{\answer{3}k}{n}\right)\right) \frac{\answer{3}}{n}.$$
  \item We may reorganize this sum as $$\frac{\answer{6}}{n^{\answer{1}}} \sum_{k=0}^{n-1} 1 + \frac{\answer{27}}{n^{\answer{2}}}\sum_{k=0}^{n-1}k+\frac{\answer{27}}{n^{\answer{3}}}\sum_{k=0}^{n-1}k$$
  \item Using formulas for sums of powers, this boils down to $$\frac{\answer{57}}{\answer{2}} +\frac{\answer{-33}}{n}+\frac{\answer{27}}{n^2}.$$
  \item The desired area equals $\frac{\answer{57}}{\answer{2}}$.
  \end{enumerate}
\end{exercise}
\end{document}