\documentclass{ximera}

\input{../../preamble.tex}

\author{David Kish}
\license{Creative Commons Attribution-ShareAlike 4.0 International License}



\begin{document}
\begin{exercise}
A researcher performs an experiment to test a hypothesis that involves the nutrients niacin and retinol. She feeds one group of laboratory rats a daily diet of precisely $36.54$ units of niacin and $24600$ units of retinol. She uses two types of commercial pellet foods. Food A contains $0.13$ unit of niacin and $110$ units of retinol per gram. Food B contains $0.32$ unit of niacin and $100$ units of retinol per gram. Using a system of equations set up two equations to help you figure out how many grams of each food she feeds this group of rats each day. Let $a$ represent grams of Food A and $b$ represent grams of Food B.

\[
\begin{cases}
\text{Formula for niacin} \answer{.13}a + \answer{.32}b = \answer{36.54}\\
\text{Formula for retinol} \answer{110}a + \answer{100}b = \answer{24600}
\end{cases}
\]
Use the previous equations and solve for the grams of Food A and B.
\[
a = \answer{190} \text{grams}
\]
\[
b = \answer{37} \text {grams}
\]
\end{exercise}

\end{document}
