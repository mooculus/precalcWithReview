\documentclass{ximera}

\input{../../preamble.tex}

\author{Ivo Terek}
\license{Creative Commons Attribution-ShareAlike 4.0 International License}

%\outcome{Calculating the rate of change.}
%\outcome{Discuss the meaning of antiderivatives of a position function.}

\begin{document}
\begin{exercise}

 Imagine that a certain population, in danger of becoming extinct, decreases by $12\%$ per week. Knowing that the original population was of $960$, fill the following table (rounding results up to the next integer, before proceeding to the next month):

  $$
\begin{array}{cc}
\text{Week } t& \text{Population}\\
\hline
0 & 960 \\
1& \answer{845}\\
2& \answer{744}\\
3& \answer{655}\\
4&\answer{577}
\end{array}
$$

\begin{exercise}
  If the population $P$ were a \emph{linear} function $P(t) = mt+n$ of the week $t$, based on the first two values of the above table, what would be the coefficients $m$ and $n$? Answer: $P(t) = \answer{-115}t+\answer{960}$.

  \begin{exercise}
    Is $P(3)$ equal to $655$? Based on your answer, is a linear model the most adequate for this situation?
    \begin{multipleChoice}
      \choice{Yes}
      \choice[correct]{No}
    \end{multipleChoice}
  \end{exercise}
  
\end{exercise}

\end{exercise}

\end{document}
