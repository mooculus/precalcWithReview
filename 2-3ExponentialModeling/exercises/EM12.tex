\documentclass{ximera}

\input{../../preamble.tex}

\author{Ivo Terek}
\license{Creative Commons Attribution-ShareAlike 4.0 International License}

%\outcome{Calculating the rate of change.}
%\outcome{Discuss the meaning of antiderivatives of a position function.}

\begin{document}
\begin{exercise}
  Imagine that a certain product, due to supply and demand, has its price increase by $15\%$ per month. Knowing that the original price was $\$180$, fill the following table (rounding prices up to the closest cent, before proceeding to the next month):

  $$
\begin{array}{cc}
\text{Month } t& \text{Price}\\
\hline
0 & \${180}\\
1& \$\answer{207}\\
2&\$\answer{238.05}\\
3&\$\answer{273.76}\\
4&\$\answer{314.82}
\end{array}
$$

\begin{exercise}
  If the price $P$ were a \emph{linear} function $P(t) = mt+n$ of the month $t$, based on the first two values of the above table, what would be the coefficients $m$ and $n$? Answer: $P(t) = \answer{27}t+\answer{180}$.

  \begin{exercise}
    Is $P(3)$ equal to $273.76$? Based on your answer, is a linear model the most adequate for this situation?
    \begin{multipleChoice}
      \choice{Yes}
      \choice[correct]{No}
    \end{multipleChoice}
  \end{exercise}
  
\end{exercise}

\end{exercise}
\end{document}
