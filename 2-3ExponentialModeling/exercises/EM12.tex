\documentclass{ximera}

\input{../../preamble.tex}

\author{Ivo Terek}
\license{Creative Commons Attribution-ShareAlike 4.0 International License}

%\outcome{Calculating the rate of change.}
%\outcome{Discuss the meaning of antiderivatives of a position function.}

\begin{document}
\begin{exercise}
  Imagine that a certain product, due to supply and demand, has its price increase by $15\%$ per month. Knowing that the original price was $\$180$, fill the following table (rounding prices up to the closest cent, before proceeding to the next month):

  $$
\begin{array}{cc}
\text{Month } t& \text{Price}\\
\hline
0 & \${180}\\
1& \$\answer{207}\\
2&\$\answer{238.05}\calcHW \\
3&\$\answer{273.76}\calcHW \\
4&\$\answer{314.82}\calcHW
\end{array}
$$

\begin{exercise}
  If the price $P$ were a \emph{linear} model given by the formula $P = mt+n$ of the month $t$, based on the first two values of the above table, what would be the coefficients $m$ and $n$? Answer: $P = \answer{27}t+\answer{180}$.

  \begin{exercise}
    Is $P$ equal to $273.76$ when $t=3$? Based on your answer, is a linear model the most adequate for this situation?
    \begin{multipleChoice}
      \choice{Yes}
      \choice[correct]{No}
    \end{multipleChoice}
 
\begin{exercise}
  If the price $P$ were an \emph{exponential} model given by the formula $P = ab^t$ of the month $t$, based on the first two values of the above table, what would be the values of $a$ and $b$? Answer: $P = \answer{180} \cdot \answer{1.15}^t$.

  \begin{exercise}
    Is $P$ equal to $273.76$ when $t=3$ using an exponential model?  Based on your answer, is an exponential model a good model for this situation?
    \begin{multipleChoice}
      \choice[correct]{Yes}
      \choice{No}
    \end{multipleChoice}
 
\begin{exercise}
 An exponential model is a good choice for this data because...
    \begin{multipleChoice}
      \choice{The output data is a constant value.}
	\choice{The average rate of change of the data is constant.}
	\choice[correct]{To get from one output value to the next on the table, you multiply by a constant factor.}
      \choice{To get from one output value to the next on the table, you add a constant number.}
    \end{multipleChoice}
 



 \end{exercise}

 \end{exercise}

 \end{exercise}
  


 \end{exercise}
  
\end{exercise}

\end{exercise}
\end{document}
