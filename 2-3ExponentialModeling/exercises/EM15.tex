\documentclass{ximera}

\input{../../preamble.tex}

\author{Ivo Terek}
\license{Creative Commons Attribution-ShareAlike 4.0 International License}

%\outcome{Calculating the rate of change.}
%\outcome{Discuss the meaning of antiderivatives of a position function.}

\begin{document}
\begin{exercise}
 Let's say that a car factory, for the mass production of a certain piece needed for making engines, has a fixed cost of $\$320$, plus $\$50$ per unit. Fill the following table with prices in terms of distances:

  $$
\begin{array}{cc}
u\text{ units}& \text{Price}\\
\hline
0 & \$320 \\
5& \$\answer{570}\\
10& \$\answer{820}\\
15& \$\answer{1070}\\
20 & \$\answer{1320}
\end{array}
$$

\begin{exercise}
  What does seem more adequate to model this situation?
  \begin{multipleChoice}
    \choice[correct]{A linear function}
    \choice{An exponential function}
  \end{multipleChoice}
  \begin{exercise}
    Find a linear formula for the total cost $C$ in terms of the amount $u$ of units produces. Answer: $C(u) = \answer{50}u+\answer{320}$.
  \end{exercise}
\end{exercise}

\end{exercise}
\end{document}