\documentclass{ximera}

\input{../preamble}
\author{David Kish}
\license{Creative Commons Attribution-ShareAlike 4.0 International License}
\acknowledgement{https://activecalculus.org/prelude/sec-changing-in-tandem.html}

\title{Exponential Modeling: Exponent Rules}

\begin{document}
\begin{abstract}
We review the rules of exponents.
\end{abstract}
\maketitle

 
 %\section{Exponent Rules}
   %%%%%%%%%%%%%%%%
    \section{Product Rule}
        If we write out $3^5\cdot 3^2$ without using exponents,
        we'd have:
\[
 3^5 \cdot 3^2 = \left(3 \cdot 3\cdot 3\cdot 3\cdot 3\right) \cdot \left(3 \cdot 3\right)
\]
        If we then count how many $3$s are being multiplied together,
        we find we have $5+2=7$, a total of seven $3$s.
        So $3^5\cdot 3^2$ simplifies like this:
       \[
          3^5\cdot 3^2 = 3^{5+2} = 3^7
\]
\begin{example}          
Simplify $x^2\cdot x^3$. \\
\begin{explanation}       
          To simplify $x^2\cdot x^3$,
          we write this out in its expanded form,
          as a product of $x$'s, we have
\[
            x^2\cdot x^3 =(x\cdot x)(x \cdot x \cdot x)
            =x\cdot x\cdot x \cdot x \cdot x
            =x^5
   \]   
          Note that we obtained the exponent of $5$ by adding $2$ and $3$.
  \end{explanation}
\end{example}

      This example demonstrates our first exponent rule,
      the Product Rule:\\
\begin{callout}
\textbf{ \Large Product Rule of Exponents} \\
      When multiplying two expressions that have the same base,
      we can simplify the product by adding the exponents.
        \[
x^m \cdot x^n = x^{m+n}
\]
\end{callout}
   Recall that $x=x^1$. It helps to remember this when multiplying certain expressions together.
\begin{example}        
  Multiply $x(x^3+2)$ by using the distributive property.\\
\begin{explanation}
          According to the distributive property,
        $x(x^3+2)=x\cdot x^3 + x\cdot2$
          How can we simplify that term $x\cdot x^3$?
          It's really the same as $x^1\cdot x^3$,
          so according to the Product Rule, it is $x^4$.
          So we have:
        \[
            x(x^3+2)=x\cdot x^3 + x\cdot2
            =x^4+2x
\]
\end{explanation}
\end{example}         

%%%%%%%%%%%%%%%%
 \section{Power to a Power Rule}

        If we write out $\left(3^5\right)^2$ without using exponents,
        we'd have $3^5$ multiplied by itself:
   \[
         \left(3^5\right)^2 = \left(3^5\right)\cdot \left(3^5\right)
         = \left(3\cdot 3\cdot 3\cdot 3 \cdot 3 \right) \cdot \left(3 \cdot 3\cdot 3\cdot 3\cdot 3\right)
      \] 
        If we again count how many $3$s are being multiplied,
        we have a total of two groups each with five $3$s.
        So we'd have $2\cdot 5=10$ instances of a $3$.
        So $\left(3^5\right)^2$ simplifies like this:
   
          $\left(3^5\right)^2 = 3^{2\cdot 5}$
          $= 3^{10}$


\begin{example}
          Simplify $\left(x^2\right)^3$.\\
\begin{explanation}
          To simplify $\left(x^2\right)^3$,
          we write this out in its expanded form,
          as a product of $x$'s, we have
            $\left(x^2\right)^3 =\left(x^2\right) \cdot \left(x^2\right)\cdot\left(x^2\right)$
            $=(x \cdot x)\cdot (x \cdot x)\cdot (x \cdot x)$
            $=x^6$
     \end{explanation}
          Note that we obtained the exponent of $6$ by multiplying $2$ and $3$.
\end{example}
      This demonstrates our second exponent rule,
      the Power to a Power Rule:
\begin{callout}
\textbf{ \Large Power to a Power Rule} \\
      when a base is raised to an exponent and that expression is raised to another exponent,
      we multiply the exponents.
   \[
      \left(x^m\right)^n = x^{m \cdot n}
   \]
\end{callout}

%%%%%%%%%%%%%%%%%%
      \section{ Product to a Power Rule}
        The third exponent rule deals with having multiplication inside a set of parentheses and an exponent outside the parentheses.
        If we write out $\left(3t\right)^5$ without using an exponent,
        we'd have $3t$ multiplied by itself five times:
\[
      (3t)^5= (3t)(3t)(3t)(3t)(3t)
\]
        Keeping in mind that there is multiplication between every $3$ and $t$,
        and multiplication between all of the parentheses pairs,
        we can reorder and regroup the factors:
          $\left(3t\right)^5 = (3\cdot t)\cdot (3\cdot t)\cdot (3\cdot t)\cdot (3\cdot t)\cdot (3\cdot t)$
          $= \left(3\cdot 3\cdot 3\cdot 3\cdot 3 \right) \cdot \left(t \cdot t \cdot t \cdot t \cdot t\right)$
          $= 3^5 t^5$
        We could leave it written this way if $3^5$ feels especially large.
        But if you are able to evaluate $3^5=243$,
        then perhaps a better final version of this expression is $243t^5$.
 
        We essentially applied the outer exponent to each factor inside the parentheses.
        It is important to see how the exponent $5$ applied to \textbf{both} the $3$ \textbf{and} the $t$,
        not just to the $t$.
   

 \begin{example}   
          Simplify $(xy)^5$.
    
          To simplify $(xy)^5$,
          we write this out in its expanded form,
          as a product of $x$'s and $y$'s, we have

            $(xy)^5 =(x \cdot y) \cdot (x \cdot y) \cdot (x \cdot y) \cdot (x \cdot y) \cdot (x \cdot y)$
            $=(x \cdot x \cdot x \cdot x \cdot x) \cdot (y \cdot y \cdot y \cdot y \cdot y)$
            $=x^5 y^5$

          Note that the exponent on $xy$ can simply be applied to both $x$ and $y$.
\end{example}


      This demonstrates our third exponent rule,
      the Product to a Power Rule:
\begin{callout}
\textbf{ \Large Product to a Power Rule} \\
      When a product is raised to an exponent,
      we can apply the exponent to each factor in the product.
\[
        \left(x\cdot y\right)^n = x^{n}\cdot y^{n}
 \]
\end{callout}


%%%%%%%%%%%%%%%%%
      \section{Summary of the Rules of Exponents for Multiplication}
  


          If $a$ and $b$ are real numbers,
          and $m$ and $n$ are positive integers,
          then we have the following rules:
 
\textbf{Product Rule}
\[
            a^{m} \cdot a^{n} = a^{m+n}
\]
\textbf{Power to a Power Rule}
   \[
           (a^{m})^{n} = a^{m\cdot n}
   \]
 \textbf{ Product to a Power Rule}
   \[
           (ab)^{m} = a^{m} \cdot b^{m}
   \]  
      Many examples will make use of more than one exponent rule.
      In deciding which exponent rule to work with first,
      it's important to remember that the order of operations still applies.
\begin{example} Simplify the following expression.
                $\left(3^7r^5\right)^4$\\
    \begin{explanation}
                Since we cannot simplify anything inside the parentheses, we'll begin simplifying this expression using the Product to a Power rule.
                We'll apply the outer exponent of 4 to each factor inside the parentheses.
                Then we'll use the Power to a Power Rule to finish the simplification process.
      \[          
                  \left(3^7r^5\right)^4 = \left(3^7\right)^4 \cdot \left(r^5\right)^4
                  = 3^{7\cdot4} \cdot r^{5\cdot 4}
                  = 3^{28}r^{20}
         \]      
                Note that $3^{28}$ is too large to actually compute, even with a calculator,
                so we leave it written as $3^{28}$.
\end{explanation}
\end{example}
\begin{example}
Simplify the following expression.
       $\left(t^3\right)^2\cdot \left(t^4\right)^5$\\
\begin{explanation}
                According to the order of operations,
                we should first simplify any exponents before carrying out any multiplication.
                Therefore, we'll begin simplifying this by applying the Power to a Power Rule and then finish using the Product Rule.
           \[
                  \left(t^3\right)^2\cdot \left(t^4\right)^5 = t^{3\cdot2}\cdot t^{4\cdot5}
                  = t^6 \cdot t^{20}
                  = t^{6+20}
                  = t^{26}
\]
\end{explanation}
\end{example}
 \begin{remark} 
        We cannot simplify an expression like $x^2y^3$ using the Product Rule,
        as the factors $x^2$ and $y^3$ do not have the same base.
\end{remark}
%%%%%%%%%
\section{Quotient to a Power Rule}
  One rule we have learned is the product to a power rule,
      as in $(2x)^{3}=2^{3}x^{3}$.
      When two factors are multiplied and the product is raised to a power,
      we may apply the exponent to each of those factors individually.
      We can use the rules of fractions to extend this property to a
      quotient raised to a power.\\
   Let $y$ be a real number, where $y \neq 0$.\\
          Find another way to write $\left(\frac{5}{y}\right)^4$.\\
     Writing the expression without an exponent and then simplifying, we have:
  \begin{center}
            $\left( \frac{5}{y} \right)^4 = \left( \frac{5}{y} \right) \left( \frac{5}{y} \right) \left( \frac{5}{y} \right) \left( \frac{5}{y} \right)$\\
            $= \frac{5 \cdot 5 \cdot 5 \cdot 5}{y \cdot y \cdot y \cdot y}$\\
            $= \frac{5^4}{y^4}$\\
            $= \frac{625}{y^4}$\\
\end{center}
        Similar to the product to a power rule,
          we essentially applied the outer exponent to the factors
          inside the parentheses to factors of the numerator and
          factors of the denominator.
\begin{callout}
          The general rule is:
       For real numbers $a$ and $b$
          (with $b \neq 0$)
          and natural number $m$,
        $
            \left( \frac{a}{b}  \right)^{m} = \frac{a^{m}}{b^{m}}
          $
\end{callout}
 This rule says that when you raise a fraction to a power,
      you may separately raise the numerator and denominator to that power.
      In Example,
      this means that we can directly calculate $\left( \frac{5}{y} \right)^4$:
   \begin{center}
        $\left( \frac{5}{y} \right)^4 = \frac{5^4}{y^4}$\\
        $=\frac{625}{y^4}$
\end{center}
\begin{example}
\begin{enumerate}
\item Simplify $\left(\dfrac{p}{2}\right)^6$.
        \item  Simplify $\left(\frac{5^6w^7}{5^2w^4}\right)^9$.
                  If you end up with a large power of a specific number,
                  leave it written that way.
\item     Simplify $\frac{\left(2r^5\right)^7}{\left(2^2 r^8\right)^3}$.
                  If you end up with a large power of a specific number,
                  leave it written that way.
\end{enumerate}
\begin{explanation}
\begin{enumerate}
\item    We can use the quotient to a power rule:
\begin{center}
                    $\left(\frac{p}{2}\right)^6=\frac{p^6}{2^6}$\\
                    $=\frac{p^6}{64}$
\end{center}
        \item If we stick closely to the order of operations,
                  we should first simplify inside the parentheses and then work with the outer exponent.
                  Going this route,
                  we will first use the quotient rule:
\begin{center}
                   $\left(\frac{5^6w^7}{5^2w^4}\right)^9 = \left(5^{6-2}w^{7-4}\right)^9$\\
                    $=  \left(5^{4}w^{3}\right)^9$\\
                    Now we can apply the outer exponent to each factor inside the parentheses using theproduct to a power rule. \\
                    $= \left(5^{4}\right)^9\cdot \left(w^{3}\right)^9$ \\
                    To finish, we need to use the >power to a power rule.\\
                    $= 5^{4\cdot 9}\cdot w^{3 \cdot 9}$\\
                    $= 5^{36}\cdot w^{27}$
\end{center}
       \item According to the order of operations,
                  we should simplify inside parentheses first,
                  then apply exponents, then divide.
                  Since we cannot simplify inside the parentheses,
                  we must apply the outer exponents to each factor inside the respective set of parentheses first:
              \begin{center}
                    $\frac{\left(2r^5\right)^7}{\left(2^2 r^8\right)^3}
                    =  \frac{2^7 \left(r^5\right)^7}{\left(2^2\right)^3 \left(r^8\right)^3}$\\
                   At this point, we need to use the power-to-a-power rule:\\
                    $\phantom{\frac{\left(2r^5\right)^7}{\left(2^2 r^8\right)^3}}
                    = \frac{2^{7} r^{5\cdot 7}}{ 2^{2\cdot 3} r^{8 \cdot 3}}$\\
                    $= \frac{2^{7} r^{35}}{ 2^{6} r^{24}}$\\
                   To finish simplifying, we'll conclude with the quotient rule:\\
                    $\phantom{\frac{\left(2r^5\right)^7}{\left(2^2 r^8\right)^3}}
                    = 2^{7-6} r^{35-24}$\\
                    $= 2^{1} r^{11}$\\
                    $= 2 r^{11}$
\end{center}
\end{enumerate}
\end{explanation}
\end{example}
%%%%%%%%%%
\section{Zero as an Exponent}
So far, we have been working with exponents that are natural numbers
      ($1, 2, 3, \ldots$).
      By the end of this section,
      we will expand our understanding to include exponents that are any integer,
      as with $5^{0}$ and $12^{-2}$.
      As a first step,
      let's explore how $0$ should behave as an exponent
      by considering the pattern of decreasing powers of $2$ in <xref ref="figure-descending-powers-of-two">Figure</xref>.    
 Simplify the following expressions.
              Assume all variables represent non-zero real numbers


To simplify any of these expressions,
              it is critical that we remember an exponent only applies to what it is touching or immediately next to.


     In the expression $\left(173 x^4 y^{251}\right)^0$,
                    the exponent $0$ applies to everything inside the parentheses.
$
                      \left(173 x^4 y^{251}\right)^0 =  1
                    $

       In the expression $(-8)^0$ the exponent applies to everything inside the parentheses,
                    $-8$.
                    $
                      (-8)^0  =  1
                    $

       In contrast to the previous example,
                    the exponent only applies to the $8$.
                    The exponent has a higher priority than negation in the order of operations.
                    We should consider that $-8^0 = -\left(8^0\right)$, and so:
                   
                      $-8^0  =  -\left(8^0\right)$
                      $=  -1$

      In the expression $3x^0$,
                    the exponent $0$ only applies to the $x$:
                      $3x^0 = 3\cdot x^0$
                      $= 3\cdot 1$
                      $= 3$
%%%%%%%%
\section{Negative Exponents}
   We understand what it means for a variable to have a natural number exponent.
      For example, $x^5$ means $\overbrace{x\cdot x\cdot x\cdot x\cdot x}^{\text{five times}}$.
      Now we will try to give meaning to an exponent that is a negative integer,
      like in $x^{-5}$.

        To consider what it could possibly mean to have a negative integer exponent,
        let's extend the pattern we saw in <xref ref="figure-descending-powers-of-two">Figure</xref>.
        In that table, each time we move down a row,
        we reduce the power by $1$ and we divide the value by $2$.
        We can continue this pattern in the power and value columns,
        going all the way down into when the exponent is negative.
\begin{center}
$
  \begin{array}{ccc}
            Power & Result & \\
        \hline
            2^3 & 8 & \\
\\
              2^2 & 4 & \text{(divide by 2)} \\
\\
               2^1 & 2 & \text{(divide by 2)} \\
\\
                     2^0 & 1 & \text{(divide by 2)} \\
\\
           2^{-1} & \frac{1}{2}=\frac{1}{2^1} & \text{(divide by 2)} \\   
\\       
           2^{-2} & \frac{1}{4}=\frac{1}{2^2} & \text{(divide by 2)} \\
\\
                   2^{-3} & \frac{1}{8}=\frac{1}{2^3} & \text{(divide by 2)} \\          
     \end{array}
$
\end{center}

  We see a pattern where $2^{\text{negative number}}$ is equal to $\frac{1}{2^{\text{positive number}}}$.
      Note that the choice of base $2$ was arbitrary,
      and this pattern works for all bases except $0$,
      since we cannot divide by $0$ in moving from one row to the next.

\begin{remark}
     Negative Integers as Exponents
          For any non-zero real number $a$ and any natural number $n$,
          we define $a^{-n}$ to mean the reciprocal of $a^n$. That is,
\[      
 a^{-n} = \frac{1}{a^n}
   \]
     
    \end{remark}

\end{document}
