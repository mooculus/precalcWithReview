\documentclass{ximera}

\input{../../preamble.tex}

\author{Kenneth Berglund}
\acknowledgement{}

\begin{document}
\begin{exercise}
The reference angle of $\theta = \frac{13\pi}{12}$ is $\theta_R = \answer{\frac{\pi}{12}}$.

When drawn in standard position, the terminal side of $\theta$ lies in Quadrant $\answer{III}$ (give your answer in Roman numerals). 

\begin{exercise}
Using formulas you will learn later, you will be able to compute that $\sin\left(\frac{\pi}{12}\right) = \frac{\sqrt{6} - \sqrt{2}}{4}$.

Using this information, you can conclude that $\sin\left(\frac{13\pi}{12}\right) = $\wordChoice{\choice{$+$}\choice[correct]{$-$}}$\frac{\sqrt{6} - \sqrt{2}}{4}$.

\begin{exercise}
$\cos\left(\frac{13\pi}{12}\right) = -\sqrt{\frac{\answer{2} + \sqrt{\answer{3}}}{4}}$.

(Hint: Use the fact that $\cos^2(\theta) + \sin^2(\theta) = 1$.)
\end{exercise}
\end{exercise}

\end{exercise}
\end{document}