\documentclass{ximera}

\input{../preamble}
\author{Elizabeth Miller}
\license{Creative Commons Attribution-ShareAlike 4.0 International License}


\title{The Importance of the Equals Sign}

\begin{document}
\begin{abstract}
 We explore the similarities and differences between expressions, equations, and functions.  We also look at when it is and is not appropriate to use the equals sign.
\end{abstract}
\maketitle


%\typeout{************************************************}
%\typeout{Motivating Questions}
%\typeout{************************************************}

\begin{motivatingQuestions}\begin{itemize}
\item What are some similarities and differences between mathematical expressions, equations, and functions?
\item How is solving an equation related to finding the zero of a function?
\item When should we and when shouldn't we use an equals sign?
\end{itemize}\end{motivatingQuestions}


%\typeout{************************************************}
%\typeout{Subsection Introduction}
%\typeout{************************************************}

\section{Introduction}
Now that we are exploring the zeros of functions, one issue that often comes up for students (and for teachers reading students' work!) is when you should and should not use an equals sign.  We are going to review when is and is not ok to use equals.

First, it is helpful to review a few important terms.  In mathematics, it is important to use these and other terms precisely so that you are communicating clearly and saying what you intend to say.  Speaking precisely using mathematical terms can be difficult to learn and takes some practice!

\section{Expressions}

\begin{definition}
An \dfn{algebraic expression} is any combination of variables and numbers using arithmetic operations such as addition, subtraction, multiplication, division, and exponentiation.
\end{definition}

Here are some examples of algebraic expressions:
$$
5x^2-17  \hspace{.5in}
\frac{56x}{\sqrt{17x}}  \hspace{.5in}
2x+3y+7z
$$

The important thing to notice is that there are no equals signs in an expression.  There are also no inequality signs.

\begin{definition}
An \dfn{mathematical expression}, or just an \dfn{expression}, is similar to an algebraic expression, but can contain other mathematical objects such as $\sin(x)$ or $\ln(x)$ or similar objects that you will learn about in future classes.  In particular, it does not contain an equals sign or an inequality sign.
\end{definition}

Here are some examples of mathematical expressions:
$$
\frac{\sin(x)}{\cos(x)}  \hspace{.5in}
5x+\ln(x)-12  \hspace{.5in}
2x+3y+7z
$$

Every algebraic expression is also a mathematical expression.

\begin{definition}
\dfn{Evaluating an expression} is substituting in a particular value for the variable in a mathematical expression.
\end{definition}

Here is an example of evaluating an expression.  Consider the expression $5x^2-17$.  Let's evaluate that expression at $x=1$.
$$
5(1)^2-17=5-17=-12
$$

Notice that when evaluating this expression at a particular point, we can use an equals sign.  This is a good use of the equals sign and shows us simplifying.  But, we should not put an equals sign between $5x^2-17$ and $5(1)^2-17$ as these two expressions are only equal when $x=1$.

\section{Equations}
When we use an equals sign to say that two different mathematical expressions give the same value, we are creating an equation.

\begin{definition}
An \dfn{equation} is a statement that two mathematical expressions are equal.
\end{definition}

Here are some examples of equations:
$$
5x^2-17=-12  \hspace{.5in}
\frac{56x}{\sqrt{17x}}=12x  \hspace{.5in}
2x+3y+7z=\frac{x}{y+z}
$$

When we are given an equation in a problem, we often want to know what value of the variable will make the equation true.  That is, what value of the variable will make both sides give the same value.

\begin{definition}
\dfn{Solving an equation} is the process of determining precisely what value of a variable makes the equation true.
\end{definition}

Here is an example of solving an equation.  Let's solve $5x^2-17=-12$.

\begin{align*}
5x^2-17&=-12\\
5x^2&=5\\
x^2&=1\\
x=1 &\text{  or  }  x=-1
\end{align*}

Notice that this is the reverse process of evaluating the expression $5x^2-17$.  When evaluating the expression, we knew the $x$-value and substituted it in.  When solving the equation, we knew what the output should be and had to find the $x$-value that would produce that output.  In fact, we found two such values!

\begin{remark}
Notice that when solving an equation, we don't put equals between the steps.  This is very important.  In many cases, the steps are not equal! 
\end{remark}

This is a key observation.  Notice that if we naively wrote $$5x^2-17=-12=5x^2=5=...,$$ we would be saying something not true.  In particular, we would be claiming that $-12=5$!

The best thing is to do when solving an equation is to make a new line for each step, but if you need to write your steps on a single line, you can use an arrow to show the next step.  For example, we could write 
$$5x^2-17=-12 \rightarrow 5x^2=5 \rightarrow x^2=1 \rightarrow x=1 \text{  or  }  x=-1.$$  
We could also use connecting words between equations.  For example, we could write:
$$5x^2-17=-12 \text{ so } 5x^2=5 \text{ thus } x^2=1 \text{ therefore } x=1 \text{  or  }  x=-1.$$  

\section{Zeros of Functions Revisited}
Notice that when we are working with functions, we are also working with equations and expressions.  
\begin{itemize}
\item When we write $f$, we are referencing the function by name.  
\item When we write $f(x)$, this is an expression for the output of the function at $x$.
\item When we write $f(x)=5x^2-17$, we are defining the way the function produces outputs.
\item When we want to find the zeros of this function, we set up the equation $f(x)=0$.  In our case, this would mean solving the equation $5x^2-17=0$.
\begin{align*}
5x^2-17&=0\\
5x^2&=17\\
x^2&=\frac{17}{5}\\
x=\sqrt{\frac{17}{5}} &\text{  or  }  x=-\sqrt{\frac{17}{5}}
\end{align*}  
Notice that this is solving an equation so we do not write equals signs between the steps.
\end{itemize}

Another important connection between finding zeros of functions and solving equations is that every equation can be thought of as the zero of a function.  Consider the following example.

\begin{example}
Rewrite the equation $5x+7=6-x^2$ as the zero of a function.  You do not need to find the zero.

\begin{explanation}
In order to rewrite this problem so that solving this equation is equivalent to finding the zero of a function, we want to move all the terms to the same side and combine like terms.  For our example, this means
\begin{align*}
5x+7&=6-x^2\\
5x+7 -(6-x^2)&=0\\
-6x^2+5x+1&=0
\end{align*}
Now we let
$$f(x)=-6x^2+5x+1$$
Now, the $x$ values which are zeros of $f$ will be the same $x$-values that solve $5x+7=6-x^2$.  We will learn to find zeros of quadratic equations in the the next section.
\end{explanation}
\end{example}

\begin{summary}\begin{itemize}
\item You should not write an equals sign between two things which do not have the same value.  Equals does not mean ``next step''!  Instead, to indicate a next step, you may use an arrow, a new line, or connecting words like ``so'' and``thus''.
\item  Every equation can be thought of as the zero of a function by moving all the terms to one side and then defining a function to be the output of that side.
\end{itemize}\end{summary}


%\typeout{************************************************}
%\typeout{Summary}
%\typeout{************************************************}

%\begin{summary}\begin{itemize}
%\item 
%\item 
%\item 
%\end{itemize}\end{summary}




\end{document}
