\documentclass{ximera}

\input{../../preamble.tex}

\author{Elizabeth Campolongo}
\acknowledgement{https://www.stitz-zeager.com/szprecalculus07042013.pdf}

\begin{document}
\begin{exercise}
Use a triangle to rewrite the following values as an algebraic expression of $x$ (without trig functions) and give the domain on which this equivalence is valid.%
\begin{enumerate}
\item $\sin(\arccos(x)) = \sqrt{\answer{1-x^2}}$ given $\answer{-1} \leq x \leq \answer{1}$
%
\item  $\sec(\arctan(x)) = \sqrt{\answer{1+x^2}}$ given $\answer{-\infty} \leq x \leq \answer{\infty}$
%
\item $\tan(\arcsin(x)) = \frac{x}{\answer{\sqrt{1-x^2}}}$ given $\answer{-1} < x < \answer{1}$
%
\item $\cos(2\arctan(x)) = \frac{\answer{1-x^2}}{1+\answer{x^2}}$ given $\answer{-\infty} \leq x \leq \answer{\infty}$
%
\item $\cos\!\big(\!\arcsin\big(\frac{x}{2}\big)\big) = \frac{\answer{\sqrt{4-x^2}}}{2}$ given $\answer{-2} \leq x \leq \answer{2}$
%
\item $\cos(2\arcsin(4x)) = 1-\answer{32}x^2$ given $\answer{-\frac{1}{4}} \leq x \leq \answer{\frac{1}{4}}$
%
\item $\tan(\arcsec(x)) = \sqrt{\answer{x^2 - 1}}$ given $\answer{1} \leq x \leq \infty$ and \\
$\tan(\arcsec(x)) = -\sqrt{\answer{x^2-1}}$ given $-\infty \leq x \leq \answer{-1}$.
\end{enumerate}
\end{exercise}


\end{document}


