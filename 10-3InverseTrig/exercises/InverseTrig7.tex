\documentclass{ximera}

\input{../../preamble.tex}

\author{Elizabeth Campolongo}
\acknowledgement{https://activecalculus.org/prelude/sec-trig-inverse.html}

\begin{document}
\begin{exercise}
Consider a right triangle where the two legs measure $5$ and $12$ respectively and $\alpha$ is the angle opposite the shorter leg and $\beta$ is the angle opposite the longer leg.
%
\begin{enumerate}
\item What is the exact value of $\cos(\alpha)$? \\
$\cos(\alpha) = \cos\!\Big(\!\arctan\!\Big(\answer{\frac{5}{12}}\Big)\Big) = \answer{\frac{12}{13}}$.
%
\item What is the exact value of $\sin(\beta)$?\\
$\sin(\beta) = \sin\!\Big(\!\arctan\!\Big(\answer{\frac{12}{5}}\Big)\Big) = \answer{\frac{12}{13}}$.
%
\item What is the exact value of $\tan(\beta)$? \\ 
of $\tan(\alpha)$?\\
$\tan(\beta) = \answer{\frac{12}{5}}$. \\
$\tan(\alpha) = \answer{\frac{5}{12}}$.
%
\item What is the exact radian measure of $\alpha$?\\
$\alpha = \answer{\arctan\!\Big(\frac{5}{12}\Big)}$.
%
\item What is the exact radian measure of $\beta$?\\
$\beta = \answer{\arctan\!\Big(\frac{12}{5}\Big)}$.
%
\item True or false: For any two angles $\theta$ and $\gamma$ such that $\theta + \gamma = \frac{\pi}{2}$ (radians), it follows that $\cos(\theta) = \sin(\gamma)$.
%
\begin{multipleChoice}
\choice[correct]{True}
\choice{False}
\choice{Not enough information to determine.}
\end{multipleChoice}
%
\end{enumerate}
\end{exercise}
\end{document}