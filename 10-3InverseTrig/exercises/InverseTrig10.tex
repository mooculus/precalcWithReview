\documentclass{ximera}

\input{../../preamble.tex}

\author{Kenneth Berglund}
%\acknowledgement{https://www.stitz-zeager.com/szprecalculus07042013.pdf}

\begin{document}
\begin{exercise}
Solve the equation $\sin(t) = \frac{5}{6}$. 

One solution, found by applying $\arcsin$ to both sides of the equation, is given by $\answer{\arcsin\left(\frac{5}{6}\right)}$ and lies in Quadrant $\answer{I}$ (give your answer as a Roman numeral). 

\begin{exercise}
Since sine has period $2\pi$, one period of sine is given by $[0, \answer{2\pi}]$. Since $\sin(t) > 0$, the other solution on this period has to lie in 
\begin{multipleChoice}
\choice{Quadrant I.}
\choice[correct]{Quadrant II.}
\choice{Quadrant III.}
\choice{Quadrant IV.}
\choice{There is no other solution on this period.}
\end{multipleChoice} 

\begin{exercise}
The second solution to the equation on the period $[0, 2\pi]$ is given by $\answer{\pi - \arcsin\left(\frac{5}{6}\right)}$. 

\begin{exercise}
A list of all solutions is $\pi - \arcsin\left(\frac{5}{6}\right) + \answer{2\pi k}$ and $\arcsin\left(\frac{5}{6}\right) + \answer{2\pi k}$ for all integers $k$. 
\end{exercise}
\end{exercise}
\end{exercise}
\end{exercise}

\end{document}


