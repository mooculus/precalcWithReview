\documentclass{ximera}

\input{../../preamble.tex}

\author{Kenneth Berglund}
%\acknowledgement{https://www.stitz-zeager.com/szprecalculus07042013.pdf}

\begin{document}
\begin{exercise}
Solve the equation $\tan(t) = -6$. 

One solution, found by applying $\arctan$ to both sides of the equation, is given by $\answer{\arctan(-6)}$ and lies in Quadrant $\answer{IV}$ (give your answer as a Roman numeral). 

\begin{exercise}
Since tangent has period $\pi$, one period of tangent is given by $[-\pi, \answer{0}]$. Since $\tan(t) < 0$, the other solution on this period has to lie in 
\begin{multipleChoice}
\choice{Quadrant I.}
\choice{Quadrant II.}
\choice{Quadrant III.}
\choice{Quadrant IV.}
\choice[correct]{There is no other solution on this period.}
\end{multipleChoice} 

\begin{exercise}
A list of all solutions is $\arctan(-6) + \answer{\pi k}$ for all integers $k$. 
\end{exercise}
\end{exercise}

\end{exercise}

\end{document}


