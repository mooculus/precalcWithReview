\documentclass{ximera}

\input{../../preamble.tex}

\author{Elizabeth Campolongo}
\acknowledgement{https://activecalculus.org/prelude/sec-trig-inverse.html}

\begin{document}
\begin{exercise}
At an airshow, a pilot is flying low over a runway while maintaining a constant altitude of 2000 feet and a constant speed.  On a straight path over the runway, the pilot observes on her laser range-finder that the distance from the plane to a fixed building adjacent to the runway is 7500 feet.  Five seconds later, she observes that distance to the same building is now 6000 feet.%
\par
%
\begin{enumerate}
\item What is the angle of depression from the plane to the building when the plane is 7500 feet away from the building? (The angle of depression is the angle that the pilot's line of sight makes with the horizontal.) \\
%
The angle of depression from the plane to the building when the plane is 7500 feet away from the building is $\answer{\arcsin(\frac{20}{75})}$.
%
\item What is the angle of depression when the plane is 6000 feet from the building?\\
%
The angle of depression from the plane to the building when the plane is 6000 feet away from the building is $\answer{\arcsin(\frac{1}{3})}$.
%
\item How far did the plane travel during the time between the two different observations?\\
$\answer{7500\cos(\arcsin(\frac{20}{75}))} - \answer{6000\cos(\arcsin(\frac{1}{3}))}$ feet.
%
\item What is the plane's velocity (in miles per hour)? \\
The plane is traveling $\answer{1500\cdot\frac{5\cos(\arcsin(\frac{20}{75})) - 4\cos(\arcsin(\frac{1}{3}))}{5}}$ feet per second. \\
Note, there are 5280ft in a mile and 360 seconds in an hour. \\
Hence, the plane is traveling $\answer{10800\cdot\frac{5\cos(\arcsin(\frac{20}{75})) - 4\cos(\arcsin(\frac{1}{3}))}{528}}$ miles per hour.
%
\end{enumerate}
\end{exercise}
\end{document}