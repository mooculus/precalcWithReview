\documentclass{ximera}

\input{../../preamble.tex}

\author{Elizabeth Campolongo}
\acknowledgement{https://activecalculus.org/prelude/sec-trig-inverse.html}

\begin{document}
\begin{exercise}
On a calm day, a photographer is filming a hot air balloon.  When the balloon launches, the photographer is stationed 850 feet away from the balloon.
%
\begin{enumerate}
\item When the balloon is 200 feet off the ground, what is the angle of elevation of the camera?\\
The angle of elevation of the camera is $\answer{\arctan(\frac{4}{17})}$.
%
\item When the balloon is 275 feet off the ground, what is the angle of elevation of the camera?\\
The angle of elevation of the camera is $\answer{\arctan(\frac{11}{34})}$.
%
\item Let $\theta$ represent the camera's angle of elevation when the balloon is at an arbitrary height $h$ above the ground.  Express $\theta$ as a function of $h$. \\
$\theta = \answer{\arctan(\frac{h}{850})}$.
%
\item Determine $\av_{[200,275]}$ for $\theta$ (as a function of $h$) and write at least one sentence to carefully explain the meaning of the value you find, including units.\\
%
$$\av_{[200,275]} = \frac{\big(\answer{\arctan(\frac{4}{17})} - \answer{\arctan(\frac{11}{34})} \big)}{75}$$
%
The units of this expression are 
%
\begin{multipleChoice}
\choice{feet per radian}
\choice[correct]{radians per foot}
\choice{degrees per foot}
\choice{feet per degree}
\end{multipleChoice}
%
\end{enumerate}
\end{exercise}
\end{document}