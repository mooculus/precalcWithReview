\documentclass{../ximera}

\input{../../preamble.tex}

\author{Sam Henke}
\acknowledgement{}

\begin{document}
\licenseSZ
\begin{exercise}
    The function \( f(x) = 5(x - 1)^3 \) is one-to-one, so it has an inverse function. In this problem, we explore the relationship between the operations making up \( f \) and the operations making up \( f^{-1} \).

    \begin{enumerate}
        \item The function \( f \) is composed of three basic operations:
        \begin{itemize}
            \item Multiply by \( 5 \): \( \answer{3} \)
            \item Subtract \( 1 \): \( \answer{1} \)
            \item Cube (raise to the exponent \( 3 \)): \( \answer{2} \)
        \end{itemize}
        Order these operations according to the order of operations. (Enter ``1'' for the first operation, ``2'' for the second, and ``3'' for the third.)

        \item The inverse function is composed of the inverses of the basic operations above:
        \begin{itemize}
            \item Divide by \( 5 \): \( \answer{1} \)
            \item Add \( 1 \): \( \answer{3} \)
            \item Cube root (raise to the exponent \( 1/2 \)): \( \answer{2} \)
        \end{itemize}
        The inverse function is composed of the inverses of each basic operation in \emph{reverse order}. Order the inverse operations according to the reverse order of operations.

        \item Compose these operations to obtain a formula for \( f^{-1} \).
        \[
            f^{-1}(x) = \answer{(x/5)^{1/3} + 1}
        \]
    \end{enumerate}
\end{exercise}
\end{document}
