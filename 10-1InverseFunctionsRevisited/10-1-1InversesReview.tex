\documentclass[nooutcomes]{ximera}

\input{../preamble}
\author{Kenneth Berglund}
\license{Creative Commons Attribution-ShareAlike 4.0 International License}
\acknowledgement{}


\title{Review of Inverse Functions}

\begin{document}
%\licenseSZ
\begin{abstract}
  
\end{abstract}
\maketitle


%\typeout{************************************************}
%\typeout{Motivating Questions}
%\typeout{************************************************}

%\begin{motivatingQuestions}\begin{itemize}
%\item What are inverse functions?
%\end{itemize}\end{motivatingQuestions}


%\typeout{************************************************}
%\typeout{Introduction}
%\typeout{************************************************}
In Section 3-2-2, we briefly introduced the concept of \emph{inverse functions}. Recall that for a one-to-one function $f$, we can define the inverse function $f^{-1}$. If we think of $f$ as a process that takes some input $x$ and produces some output $f(x)$, then providing $f(x)$ as an input to $f^{-1}$ produces the original input $x$, and vice versa. Symbolically, we wrote that $f^{-1}(f(x)) = x$ and $f(f^{-1}(x)) = x$. 

We learned several important principles, which we summarize below.
\begin{itemize}
\item
A function $f$ has an inverse function if and only if there exists a function $g$ that undoes the work of $f$: that is, there is some function $g$ for which $g(f(x)) = x$ for each $x$ in the domain of $f$, and $f(g(y)) = y$ for each $y$ in the range of $f$. We call $g$ the inverse of $f$, and write $g = f^{-1}$.%
\item
A function $f$ has an inverse function if and only if the graph of $f$ passes the {\it Horizontal Line Test}.
\item
A function $f$ has an inverse function if and only if $f$ is a {\it one-to-one} function.
\item
When $f$ has an inverse, we know that writing ``$y = f(t)$'' and ``$t = f^{-1}(y)$''  are two different perspectives on the same statement.
\item If $(x, y)$ is a point on the graph of $f$, then $(y, x)$ is a point on the graph of $f^{-1}$. 
\item The graph of $f^{-1}$ is the graph of $f$ reflected across the line $y = x$.
\item The domain of $f$ is the range of $f^{-1}$ and the range of $f$ is the domain of $f^{-1}$.
\item If $f^{-1}$ is the inverse of $f$, then $f$ is the inverse of $f^{-1}$.

\end{itemize}

In this section, we'll explore inverse functions more in-depth.
\end{document}
