\documentclass{ximera}

\input{../../preamble.tex}

\author{Kenneth Berglund}
\acknowledgement{https://activecalculus.org/prelude/sec-circular-traversing.html}

\begin{document}
\licenseAPC
\begin{exercise}
The London Eye is a Ferris wheel 135 meters in diameter. It is boarded at its lowest point (6 o'clock) from a platform which is 6 meters above ground. The wheel makes one full rotation every 30 minutes, and at time $t = 0$ you board at the loading platform (6 o'clock). Let $d = g(t)$ denote your horizontal distance from the diameter of the wheel perpendicular to the ground in meters after $t$ minutes.

\begin{enumerate}
\item The period of the function $d = g(t)$ is $\answer{15}$ minutes.

\item The midline of the function $d = g(t)$ is $\answer{33.75}$ meters.

\item The amplitude of the function $d = g(t)$ is $\answer{33.75}$ meters.

\item Which of the following graphs is the graph of $d=g(t)$?
\begin{figure}[!h]
\begin{image}
\includegraphics[width=.4\linewidth]{ex3-a.png}
\hspace{20mm}
\includegraphics[width=.4\linewidth]{ex3-b.png}

\end{image}
\caption{A on the left and B on the right}
\end{figure}

\begin{figure}[!h]
\begin{image}
\includegraphics[width=.4\linewidth]{ex3-c.png}
\hspace{20mm}
\includegraphics[width=.4\linewidth]{ex3-d.png}
\end{image}
\caption{C on the left and D on the right}
\end{figure}

\begin{multipleChoice}
\choice{A}
\choice{B}
\choice[correct]{C}
\choice{D}
\end{multipleChoice}
\end{enumerate}

\end{exercise}
\end{document}