\documentclass{ximera}

\input{../preamble.tex}
\author{Bobby Ramsey}
\license{Creative Commons Attribution-ShareAlike 4.0 International License}


\title{Creating a New Function: Tangent}

\begin{document}
\begin{abstract}
  
\end{abstract}
\maketitle


%\typeout{************************************************}
%\typeout{Motivating Questions}
%\typeout{************************************************}

%\begin{motivatingQuestions}
%	\item 
%\end{motivatingQuestions}


%\typeout{************************************************}
%\typeout{Subsection Introduction}
%\typeout{************************************************}

\section{Introduction}
	We are now going to determine the graph of the tangent function by analyzing what we now know about the sine and cosine functions. As a reminder, here is a graph of those functions with some
	important points marked. Specifically the points at all multiples of $\frac{\pi}{2}$ have been marked, as well as at the standard points 
	$x=\frac{\pi}{6}$, $\frac{\pi}{4}$, $\frac{\pi}{3}$, $\frac{2\pi}{3}$, $\frac{3\pi}{4}$, and $\frac{5\pi}{6}$.
	
	\begin{image}
		\begin{tikzpicture}
			\begin{axis}[
		            xmin=-6.75,xmax=6.75,ymin=-1.5,ymax=1.5,
		            axis lines=center,
		            xtick={-6.28, -4.71, -3.14, -1.57, 0, 1.57, 3.142, 4.71, 6.28},
		            xticklabels={$-2\pi$,$-3\pi/2$,$-\pi$, $-\pi/2$, $0$, $\pi/2$, $\pi$, $3\pi/2$, $2\pi$},
		            ytick={-1,1},
		            minor ytick=,minor xtick=,
		            width=0.75\linewidth,
		            height=0.375\linewidth,
		            xlabel=$x$, ylabel=$x$,
			    clip=false,
			    grid style={dashed, gray!40}
		          ]        
		          \addplot [very thick, penColor, samples=300,smooth, domain=(-6.75:6.75)] {sin(deg(x))};
		          \node at (axis cs:-1.57,.75) [penColor2] {$\sin(x)$};
			  \addplot [soliddot, color=penColor] coordinates {(-6.28, 0) (-4.71, 1) (-3.142, 0) (-1.57, -1) (0,0) (1.57, 1) (3.142, 0) (4.71, -1) (6.28, 0)};
			  \addplot [soliddot, color=penColor] coordinates {(0.5236, 0.5) (0.785, 0.707) (1.047, 0.866) (2.094, 0.866) (2.356, 0.707) (2.618, 0.5)};
		        \end{axis}
		\end{tikzpicture}
	\end{image}
	
	\begin{image}
		\begin{tikzpicture}
			\begin{axis}[
		            xmin=-6.75,xmax=6.75,ymin=-1.5,ymax=1.5,
		            axis lines=center,
		            xtick={-6.28, -4.71, -3.14, -1.57, 0, 1.57, 3.142, 4.71, 6.28},
		            xticklabels={$-2\pi$,$-3\pi/2$,$-\pi$, $-\pi/2$, $0$, $\pi/2$, $\pi$, $3\pi/2$, $2\pi$},
		            ytick={-1,1},
		            minor ytick=,minor xtick=,
		            width=0.75\linewidth,
		            height=0.375\linewidth,
		            xlabel=$x$, ylabel=$x$,
			    clip=false,
			    grid style={dashed, gray!40}
		          ]        
		          \addplot [very thick, penColor, samples=300,smooth, domain=(-6.75:6.75)] {cos(deg(x))};
		          \node at (axis cs:-1.57,.75) [penColor2] {$\cos(x)$};
			  \addplot [soliddot, color=penColor] coordinates {(-6.28, 1) (-4.71, 0) (-3.142, -1) (-1.57, 0) (0,1) (1.57, 0) (3.142, -1) (4.71, 0) (6.28, 1)};
			  \addplot [soliddot, color=penColor] coordinates {(0.5236, 0.866) (0.785, 0.707) (1.047, 0.5)  (2.094, -0.5) (2.356, -0.707) (2.618, -0.866)};
		        \end{axis}
		\end{tikzpicture}
	\end{image}

\section{Graph of $f(x) = \sin(x) + \cos(x)$}

	Before considering the function $\tan(x)=\frac{\sin(x)}{\cos(x)}$, lets consider the (possibly) more straightforward $f(x) = \sin(x) + \cos(x)$.  Let's practice with this function creating a graph of a sum of two known functions.  

What is the value of $f(0)$?  We know $f(0) = \sin(0) + \cos(0) = 0 + 1 = 1$. 

	We can easily calculate the values of $f$ at all of the important points marked in the graphs above. Let us plot the points of $f$ corresponding to 
	them.
	\begin{image}
		\begin{tikzpicture}
			\begin{axis}[
		            xmin=-6.75,xmax=6.75,ymin=-1.5,ymax=1.5,
		            axis lines=center,
		            xtick={-6.28, -4.71, -3.14, -1.57, 0, 1.57, 3.142, 4.71, 6.28},
		            xticklabels={$-2\pi$,$-3\pi/2$,$-\pi$, $-\pi/2$, $0$, $\pi/2$, $\pi$, $3\pi/2$, $2\pi$},
		            ytick={-1,1},
		            minor ytick=,minor xtick=,
		            width=0.75\linewidth,
		            height=0.375\linewidth,
		            xlabel=$x$, ylabel=$x$,
			    clip=false,
			    grid style={dashed, gray!40}
		          ]        
		          \node at (axis cs:-1.57,.75) [penColor2] {$f(x)$};
			  \addplot [soliddot, color=penColor] coordinates {(-6.28, 1) (-4.71, 1) (-3.142, -1) (-1.57, -1) (0,1) (1.57, 1) (3.142, -1) (4.71, -1) (6.28, 1)};
			  \addplot [soliddot, color=penColor] coordinates {(0.5236, 1.366) (0.785, 1.414) (1.047, 1.366)  (2.094, 0.366) (2.356, 0) (2.618, -0.366)};

		        \end{axis}
		\end{tikzpicture}
	\end{image}
	Those extra points plotted between $x=0$ and $x=\pi$ show us the behavior of this function $f$. Notice that between $x=0$ and $x=\frac{\pi}{2}$,
	the graph increases to a peak, then decreases in a very sinusoidal manner. If we continue this with standard values and ``connect the dots'', we end
	up with the following graph.
	\begin{image}
		\begin{tikzpicture}
			\begin{axis}[
		            xmin=-6.75,xmax=6.75,ymin=-1.5,ymax=1.5,
		            axis lines=center,
		            xtick={-6.28, -4.71, -3.14, -1.57, 0, 1.57, 3.142, 4.71, 6.28},
		            xticklabels={$-2\pi$,$-3\pi/2$,$-\pi$, $-\pi/2$, $0$, $\pi/2$, $\pi$, $3\pi/2$, $2\pi$},
		            ytick={-1,1},
		            minor ytick=,minor xtick=,
		            width=0.75\linewidth,
		            height=0.375\linewidth,
		            xlabel=$x$, ylabel=$x$,
			    clip=false,
			    grid style={dashed, gray!40}
		          ]        
		          \addplot [very thick, penColor, samples=300,smooth, domain=(-6.75:6.75)] {cos(deg(x))+ sin(deg(x))};

		          \node at (axis cs:-1.57,.75) [penColor2] {$f(x)$};
			  \addplot [soliddot, color=penColor] coordinates {(-6.28, 1) (-4.71, 1) (-3.142, -1) (-1.57, -1) (0,1) (1.57, 1) (3.142, -1) (4.71, -1) (6.28, 1)};
			  \addplot [soliddot, color=penColor] coordinates {(0.5236, 1.366) (0.785, 1.414) (1.047, 1.366)  (2.094, 0.366) (2.356, 0) (2.618, -0.366)};

		        \end{axis}
		\end{tikzpicture}
	\end{image}
	We've ended up with another periodic function that looks like a stretched and shifted version of sine or cosine.
	
	Now, let's try again using division instead of addition.

\section{Determining the Graph of Tangent}	
	Recall that $\tan(x)=\frac{\sin(x)}{\cos(x)}$.
	
	Notice that this function is undefined at all $x$-values with $\cos(x) = 0$. That means the function $\tan(x)$ is not defined for $x=\pm \frac{\pi}{2}$, 
	$\pm \frac{3\pi}{2}$, $\pm \frac{5\pi}{2}$, $\ldots$.
	
	We can calculate values of $\tan(x)$ for other inputs. $\tan\left( 0\right) = \frac{\sin\left( 0\right)}{\cos\left(0\right)} = \frac{0}{1} = 0$.
	The following table lists some of the other values arising from this division. Notice that in this table we have chosen to rationalize the denominators
	of the fractions that have appeared. That is, we have written $\frac{1}{\sqrt{2}}$ as $\frac{\sqrt{2}}{2}$, by multiplying the fraction by $1$ written in 
	the form $\frac{\sqrt{2}}{\sqrt{2}}$. Similarly $\frac{1}{\sqrt{3}}$ is written as $\frac{\sqrt{3}}{3}$. 

  	\begin{center}
		$\begin{array}{||c|c|c|c||}
			\hline 
			\hline
			& & & \\
			x & \sin(x) & \cos(x) & t(x) = \frac{\sin(x)}{\cos(x)}\\ 
			& & & \\
			\hline
			& & & \\
			-\frac{\pi}{3}&  -\frac{\sqrt{3}}{2}&\frac{1}{2} & -\sqrt{3}\\
			& & & \\
			\hline 
			& & & \\
			-\frac{\pi}{4}&-\frac{\sqrt{2}}{2}&\frac{\sqrt{2}}{2}&-1\\
			& & & \\
			\hline
			& & & \\
			-\frac{\pi}{6}&-\frac{1}{2} & \frac{\sqrt{3}}{2} & -\frac{\sqrt{3}}{3}\\ 
			& & & \\
			\hline
			& & & \\
			0&0 & 1 & 0\\ 
			& & & \\
			\hline
			& & & \\
			\frac{\pi}{6}&\frac{1}{2} & \frac{\sqrt{3}}{2} & \frac{\sqrt{3}}{3}\\ 
			& & & \\
			\hline
			& & & \\
			\frac{\pi}{4}&\frac{\sqrt{2}}{2}&\frac{\sqrt{2}}{2}&1\\
			& & & \\
			\hline
			& & & \\
			\frac{\pi}{3}&  \frac{\sqrt{3}}{2}&\frac{1}{2} & \sqrt{3}\\
			& & & \\
			\hline 
			\hline
		\end{array}$
	\end{center}
	If we plot these points, we find the following graph.
	\begin{image}
		\begin{tikzpicture}
			\begin{axis}[
		            xmin=-2,xmax=2,ymin=-2.5,ymax=2.5,
		            axis lines=center,
		            xtick={-1.57, -0.785, 0, 0.785, 1.57},
		            xticklabels={$-\pi/2$, $-\pi/4$, $0$, $\pi/4$, $\pi/2$},
		            ytick={-2,...,2},
		            minor ytick=,minor xtick=,
		            width=0.75\linewidth,
		            height=0.75\linewidth,
		            xlabel=$x$, ylabel=$x$,
%			    clip=false,
			    grid style={dashed, gray!40}
		          ]        
%		          \addplot [very thick, penColor, samples=100,smooth, domain=(-1.56:1.56)] {tan(deg(x))};
			  \addplot [soliddot, color=penColor] coordinates { (0,0) };
			  \addplot [soliddot, color=penColor] coordinates {(-0.5236, -0.5774) (0.5236, 0.5774) (0.785, 1) (-0.785, -1)  (1.047, 1.732)  (-1.047, -1.732)};
			  \node at (axis cs:-0.7,2) [penColor2] {$t(x)$};
		        \end{axis}
		\end{tikzpicture}
	\end{image}
	Let's think about what happens if $x$ is a number really close to $\frac{\pi}{2}$, but just a little bit smaller than $\frac{\pi}{2}$. Notice from the graphs 
	that the value of $\sin(x)$ will be a positive number that is really close to $1$ and the value of $\cos(x)$ will be really close to $0$ but still positive.
	\begin{image}
		\begin{tikzpicture}
			\begin{axis}[
		            xmin=-6.75,xmax=6.75,ymin=-1.5,ymax=1.5,
		            axis lines=center,
		            xtick={-6.28, -4.71, -3.14, -1.57, 0, 1.57, 3.142, 4.71, 6.28},
		            xticklabels={$-2\pi$,$-3\pi/2$,$-\pi$, $-\pi/2$, $0$, $\pi/2$, $\pi$, $3\pi/2$, $2\pi$},
		            ytick={-1,1},
		            minor ytick=,minor xtick=,
		            width=0.75\linewidth,
		            height=0.375\linewidth,
		            xlabel=$x$, ylabel=$x$,
			    clip=false,
			    grid style={dashed, gray!40}
		          ]        
		          \addplot [very thick, penColor, samples=300,smooth, domain=(-6.75:6.75)] {sin(deg(x))} node[pos=0.6, color=penColor2, above right] {$\sin(x)$};
		          \addplot [very thick, penColor3, samples=300,smooth, domain=(-6.75:6.75)] {cos(deg(x))} node[pos=0.5, color=penColor2, above left] {$\cos(x)$};

		        \end{axis}
		\end{tikzpicture}
	\end{image}	
	
	What happens if we take $1$ and divide it by a small positive number? Let's look at a table of values to see.
  	\begin{center}
		$\begin{array}{||c|c||}
			\hline 
			\hline
			&  \\
			z & \frac{1}{z}\\ 
			&  \\
			\hline
			&  \\
			1&  1\\
			&  \\
			\hline 
			&  \\
			\frac{1}{2}&2\\
			&  \\
			\hline
			&  \\
			\frac{1}{3}&3\\ 
			&  \\
			\hline
			&  \\
			\frac{1}{10}&10\\ 
			&  \\
			\hline
			&  \\
			\frac{1}{100}&100\\ 
			&  \\
			\hline
			&  \\
			\frac{1}{1000}&1000\\ 
			&  \\
			\hline 
			\hline
		\end{array}$
	\end{center}
	Notice that as the numbers $z=1$, $1/2$, $1/3$, $\ldots$ got smaller and smaller, the values of $\frac{1}{z} = 1$, $2$, $3$, $\ldots$ got larger and 
	larger? That is the same thing we are noticing in the graph of the function $t$ we are building above. For values of $x$ really close to $\pi/2$, but still 
	less than $\pi/2$, the value of $t(x)$ is basically $1$ divided by a very small positive number. This table of values tells us that the smaller that 
	denominator gets, the larger the fraction becomes. Adding this behavior to the graph of $t$ gives the following.
	
	\begin{image}
		\begin{tikzpicture}
			\begin{axis}[
		            xmin=-2,xmax=2,ymin=-5.5,ymax=5.5,
		            axis lines=center,
		            xtick={-1.57, -0.785, 0, 0.785, 1.57},
		            xticklabels={$-\pi/2$, $-\pi/4$, $0$, $\pi/4$, $\pi/2$},
		            ytick={-5,...,5},
		            minor ytick=,minor xtick=,
		            width=0.75\linewidth,
		            height=0.75\linewidth,
		            xlabel=$x$, ylabel=$x$,
%			    clip=false,
			    grid style={dashed, gray!40}
		          ]        
		          \addplot [very thick, penColor, samples=100,smooth, domain=(-1.56:1.56)] {tan(deg(x))};
			  \addplot [soliddot, color=penColor] coordinates { (0,0) };
			  \addplot [soliddot, color=penColor] coordinates {(-0.5236, -0.5774) (0.5236, 0.5774) (0.785, 1) (-0.785, -1)  (1.047, 1.732)  (-1.047, -1.732)};
			  \node at (axis cs:-0.7,2) [penColor2] {$t(x)$};
		        \end{axis}
		\end{tikzpicture}
	\end{image}
	
	By repeating similar calculations for other standard inputs, we arrive at the following graph.
	\begin{image}
		\begin{tikzpicture}
			\begin{axis}[
		            xmin=-6.75,xmax=6.75,ymin=-5.5,ymax=5.5,
		            axis lines=center,
		            xtick={-6.28, -4.71, -3.14, -1.57, 0, 1.57, 3.142, 4.71, 6.28},
		            xticklabels={$-2\pi$,$-3\pi/2$,$-\pi$, $-\pi/2$, $0$, $\pi/2$, $\pi$, $3\pi/2$, $2\pi$},
		            ytick={-5,...,5},
		            minor ytick=,minor xtick=,
		            width=0.75\linewidth,
		            height=0.75\linewidth,
		            xlabel=$x$, ylabel=$x$,
%			    clip=false,
			    grid style={dashed, gray!40}
		          ]        
		          \addplot [very thick, penColor, samples=100,smooth, domain=(-1.56:1.56)] {tan(deg(x))};% node [pos=0.51, penColor2, right] {$t(x)$};
		          \addplot [very thick, penColor, samples=100,smooth, domain=(1.58:4.7)] {tan(deg(x))};
		          \addplot [very thick, penColor, samples=100,smooth, domain=(4.9:6.28)] {tan(deg(x))};
		          \addplot [very thick, penColor, samples=100,smooth, domain=(-4.7:-1.58)] {tan(deg(x))};
		          \addplot [very thick, penColor, samples=100,smooth, domain=(-6.28:-4.9)] {tan(deg(x))};          		          
		        \end{axis}
		\end{tikzpicture}
	\end{image}

	
	As you can see from the graph, $\tan(x)$ is an odd, periodic function with period $\pi$ (not $2\pi$ like sine and cosine).
%\typeout{************************************************}
%\typeout{Summary}
%\typeout{************************************************}

%\begin{summary}
%\item 
%\item 
%\item
%\end{summary}




\end{document}
