\documentclass{ximera}

\input{../preamble}
\author{Kenneth Berglund}
\license{Creative Commons Attribution-ShareAlike 4.0 International License}
\acknowledgement{}

\title{What is Calculus?}

\begin{document}
\begin{abstract}
\end{abstract}
\maketitle
%
%
%
\begin{motivatingQuestions}\begin{itemize}
\item What are the main ideas of calculus?
\end{itemize}\end{motivatingQuestions}
%
%
%\typeout{************************************************}
%\typeout{Subsection Introduction}
%\typeout{************************************************}
%
\section{Introduction}
%
This course aims to provide you with a background in all the tools you'll need to be successful in calculus. In this section, we'll provide a brief overview of some of the major ideas of calculus. 

Some of the questions calculus can help to answer are:
\begin{itemize}
\item What is the area of a region bounded by two curves?
\item What is the speed of a runner at mile 2 who runs $f(t) = \sqrt{\frac{t}{6}}$ miles in $t$ minutes? 
\item Can you avoid speeding if you drive 6 miles in 5 minutes, and the speed limit is 60 miles per hour?
\item Is there a way to write sine and cosine using simpler functions?
\item How can we find relative maxima and minima of functions?
\item How can you find the volume of a 3D solid whose cross sections all have the same shape?
\end{itemize}


\section{Rates of Change}
One extremely useful topic in calculus is the idea of instantaneous rates of change. We've often computed average rates of change in this course, so it's important to understand the difference.



\end{document}
