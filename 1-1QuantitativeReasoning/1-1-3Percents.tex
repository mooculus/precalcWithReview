\documentclass{ximera}
\input{../preamble}


\author{David Kish, Debi Stout}
% Source: 1116 Materials



\title{Quantitative Reasoning: Percents}

\begin{document}

\begin{abstract}
We build some ``real-world'' math skills and lay the foundation for future course topics by looking at percentages.
\end{abstract}
\maketitle

%\typeout{************************************************}
%\typeout{Percentages}
%\typeout{************************************************}

\section{Percentages}
Percentages are everywhere!  They are used to describe discounts, markups, commissions, statistics, information, change, and on and on.  They are an important topic and so we want to make sure that we really understand them.  Let’s start by just breaking down the word “percent.”  “Per” and “cent.”  There are some specific words that often translate into certain mathematical operations.  For example, if a question asks, “how many apples and oranges are there?” what operation are you going to do with the number of apples and oranges?  Let’s make a short list of which words usually translate into which operation: \\
If you read a math problem: “There are 5 apples and 6 oranges.  How many apples and oranges are there?”  What math operation are you going to do with the number of apples (5) and oranges (6)?
\begin{center}
    “and” often translates to: $\answer{\text{addition}}$
\end{center}
Let’s change the problem a little bit: “There are 4 apples and 7 oranges.  What is the difference between the number of apples and number of oranges?”  What math operation are you going to do with the number of apples (4) and oranges (7)?
\begin{center}
    “difference” often translates to $\answer{\text{subtraction}}$
\end{center}

Let’s change it again: “We want each of the 3 people in our group to have 5 apples.  How many apples are we going to need?”  What math operation are you going to do with the number of people (3) and apples (5)?
\begin{center}
  “each of” often translates to: $\answer{multiplication}$
\end{center}

Note: We will also see in the following problems that specifically with percentages, “of” often translates to this same operation. \\
Another change: “We have a group of 5 people and a total of 15 apples.  How many apples per person are there?”  What math operation are you going to do with the number of apples (15) and people (5)?
\begin{center}
    “per” often translates to: $\answer{division}$
\end{center}
Let’s change it one last time: “The total number of apples and oranges is 13.  If the number of apples is “x” and the number of oranges is “y,” write an equation for the total number of apples and oranges.”  What math symbol did you put in for the word “is”?
\begin{center}
    “is” often translates to: $\answer{equals}$
\end{center}
Now “cent.”  Where have we seen this word before?  How many cents are in a dollar?  How many years are in a century?  How many centimeters are in a meter?  
\begin{center}
    Wherever you see the word “cent”, it is likely representing the number $\answer{100}$
\end{center}
\begin{center}
Hence, “percent” can be interpreted as $\answer{divide}$ by $\answer{100}$
\end{center}
So, for example, 13\% = $\answer{\frac{13}{100}}$ = 0.$\answer{13}$\\

\section{Percent Increase and Decrease}
In the problems where percent increases or decreases are calculated, we calculate not only the percent change, but also the “actual value” change as well.  There are specific vocabulary terms to describe these two “measures” of changes: absolute change and relative change. 
\begin{itemize}
    \item \textbf{Absolute change} is the “actual value” change that occurred.  For example, in question 2, the absolute change was \$14.69 -- the actual dollar amount that the price changed. 
    \item \textbf{Relative change} is the percent change that occurred.  For example, again in question 2, the relative change was 30\% decrease -- the percent or proportion that the dollar amount was changed. 
\end{itemize}
\begin{example}
A shop owner raises the price of a \$100 pair of shoes by 50\%. After a few weeks, because of falling sales, the owner reduces the price of the shoes by 50\%.
What is the new price of the shoes (after both percent changes have occurred)?
\\
\begin{explanation}
First, we must account for the percent increase. We can find 50\% of \$100 and then add it to the original amount
\[
0.5 \times \$100 = \$50
\]
\[
\$ 100 + \$50 = \$ 150
\]
Now we can deal with the percent decrease. Remember to decrease from the new amount (\$150) and not the original (\$100).
\[
0.5 \times \$150 = \$75
\]
After subtracting this amount from \$150 we will have the final amount.
\[
\$ 150 - \$75 = \$ 75
\]
After raising the \$100 price by 50\% and lowering it be 50\% the final price became \$75 an overall decrease of 25\%!
\end{explanation}
\end{example}
\begin{exploration}
The annual number of burglaries in a town rose by $40\%$ in 2012 and fell by $30\%$ in 2013.
\begin{itemize}
\item[a.] What was the total percent change in burglaries over the two years?
Let’s first try this problem with a specific number of burglaries to start with.  In your group, find the $\%$ change if there were certain number of burglaries in 2011 (choose a 3 digit number that does NOT end with a 0).  Then try to think about how to do this problem without knowing a specific number of burglaries.
\item[b.] It might be tempting to say that the change over the two years was a $10\%$ decrease.  Why might someone think it is a $10\%$ decrease (i.e., how do you come up with $10\%$ from the numbers in the problem?)
\item[c.] Why is $10\%$ incorrect?
\end{itemize}
\end{exploration}

\end{document}