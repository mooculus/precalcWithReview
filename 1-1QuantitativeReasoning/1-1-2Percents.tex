\documentclass{ximera}
\input{../preamble}


\author{David Kish, Debi Stout}
% Source: 1116 Materials



\title{Quantitative Reasoning: Percents}

\begin{document}

\begin{abstract}
We practice communicating mathematical explanations through fractions and percents.
\end{abstract}
\maketitle

%\typeout{************************************************}
%\typeout{Percentages}
%\typeout{************************************************}

In this section, we're going to practice two key aspects of mathematics: 1) Using drawing to reason and justify mathematics and 2) Communicating mathematical explanations to others.

\section{Fractions}

\begin{example}
Explain fractions to your neice.

\begin{explanation}

Fractions are writen in the form $\frac{A}{B}$ where $A$ and $B$ are numbers. For example, $\frac{2}{6}$.  To interpret this fraction, we need to know what the `whole' is. Are we talking about a fraction of a $7$ sweet rolls? $1$ sweet roll? Let's say $1$ sweet roll is the whole. Now we'll divide the $1$ sweet roll into $6$ equal size pieces because the denominator is $6$. Each of those pieces is $\frac{1}{6}$ of the $1$ sweet roll. So shading $2$ pieces of size $\frac{1}{6}$ gives us $\frac{2}{6}$ of $1$ sweet roll.

\begin{image}
    \includegraphics[width=3in]{fractionpic.png}
\end{image}
\end{explanation}
\end{example}


\begin{exploration}
Explain to your nephew how to solve the following problems. These may require some creativity to solve.
\begin{itemize}
\item Camila has invited 9 guests over for breakfast. However, she only made 8 churros. On top of that, only 7 guests like cinnamon sugar on their churro.  Draw a picture to illustrate how Camila can ensure each guest gets the same amount of churro and gets their desired cinnamon sugar/none on their churro. Then use the picture to determine what fraction of the churros should be coated with with cinnamon sugar.
\item Chris made $2$ sweet potato pies to split between him and $5$ friends. But he accidentally used to different sized pie dishes so they're different sizes! How can Chris split up the pies so each person gets the same amount of pie?
\end{itemize}
\end{exploration}

\section{Percentages}
Percentages are a special type of ratio that can be expressed as a fraction in the form $\frac{A}{100}$.  For example,
\begin{center} $\frac{25}{100} = \frac{1}{4} =0.25=25\%$ \end{center}

The word `percent' comes from the Latin `per centum' which mean `by the one-hundred'. The prefix `cent' is all around us. For example, a `century' is $100$ years, `centimeter' is $\frac{1}{100}$ of a meter, a `cent' is $\frac{1}{100}$ of a dollar. In spanish, `cien' is $100$. \\



\begin{exploration}
Explain to your nephew how to solve the following problems.
\begin{itemize} 
\item Chandice has started collecting small toy vehicals and has $25$ toy vehicals. Of the $25$ vehicals, $16$ are race cars, $4$ are pick-up trucks, $4$ are semi-trucks, and $2$ are fire trucks.  Without using a calculator, what percent of Chandice's toy vehicals are race cars? Draw a picture that supports your computations.
\item Kajal wants to give a $20\%$ tip for food delivery but needs to enter the tip amount as a dollar amount. She purchased $\$13.00$ worth of food. Without using a calculator, what is the amount Kajal should give as tip?
\end{itemize}
\end{exploration} 

\begin{MM}
Commnicating mathematics is a powerful learning tool.  When working on homework with friends, practice verbally explaining the ideas from the section and how to solve the problems to others.  Especially if you ask for help on a problem, explain it back to the person to check for understanding. When reviewing for quizzes or exams, explaining math, even  to a cat, dog, or plant, will help catch concepts and processes we haven't yet grasped.  
\end{MM}

\section{Percent Increase and Decrease}


\begin{example}
In the problems where percent increases or decreases are calculated, we calculate not only the percent change, but also the ``actual value" change as well.  There are specific vocabulary terms to describe these two ``measures" of changes: absolute change and relative change. Using an example, wxplain this terminology to a friend.
\begin{explanation}
Suppose a bus pass regularly costs $\$62$ for a $31$-day pass. Seniors receive a $30\%$ discount bus passes. How much do seniors pay for a $31$-day bus pass?

Our whole $100\%$ is $\$62$. In this case, we can reason using a picture of $100$ pieces of $10$ pieces\\
\begin{center}$\frac{1}{100} = 1\%$ of $\$62$ is $\$.62$\\
$\frac{10}{100} = 10\%$ of $\$62$ is $\$6.2$\\ \end{center}

We need $3$ pieces of size $\frac{10}{100}$ 

\begin{center} $\frac{3}{10} = \frac{30}{100}$ = $30\%$ of $\$62$ is $\$18.6$ \end{center}

\begin{image}
    \includegraphics[width=3in]{percentpic.png}
\end{image}

$\$18.6$ is the discount.  Seniors pay: $\$62 - \$18.6 = \$43.4$ for a $31$-day bus pass.
\end{explanation}
\end{example}


\begin{itemize}
    \item \textbf{Absolute change} is the ``actual value" change that occurred. The absolute change for the bus pass was $\$18.6$ -- the actual dollar amount that the price changed. 
    \item \textbf{Relative change} is the percent change that occurred. The relative changefor the bus pass was 30\% decrease -- the percent or proportion that the dollar amount was changed. 
\end{itemize}



\begin{example}
A shop owner raises the price of a \$100 skateboard by 50\%. After a few weeks, because of falling sales, the owner reduces the price of the skateboard by 50\%.
What is the new price of the skateboard (after both percent changes have occurred)? Explain how to solve this problem using computations supported with a drawing.

\begin{explanation}
  First, we must account for the percent increase. We can find 50\% of \$100 and then add it to the original amount

$$
0.5 \times \$100 = \$50
$$
$$
\$ 100 + \$50 = \$ 150
$$

\begin{image}
    \includegraphics[width=3in]{skateboard1.png}
\end{image}

Now we can deal with the percent decrease. Remember to decrease from the new amount (\$150) and not the original (\$100).
$$
0.5 \times \$150 = \$75
$$
After subtracting this amount from \$150 we will have the final amount.
$$
\$ 150 - \$75 = \$ 75
$$

\begin{image}
    \includegraphics[width=4in]{skateboard2.png}
\end{image}



After raising the \$100 price by 50\% and lowering it be 50\% the final price became \$75 an overall decrease of 25\%!
\end{explanation}
\end{example}



\begin{exploration}
The annual number of attendees at the pride parade in a town rose by $20\%$ in 2012 and by $30\%$ in 2013.
\begin{itemize}
\item[a.] What was the total percent change in attendees over the two years?
Let's first try this problem with a specific number of attendees to start with.  In your group, find the $\%$ change if there were certain number of attendees in 2011 (choose a 3 digit number that does NOT end with a 0).  Then try to think about how to do this problem without knowing a specific number of attendees.
\item[b.] It might be tempting to say that the change over the two years was a $50\%$ increase.  Why might someone think it is a $50\%$ increase (i.e., how do you come up with $50\%$ from the numbers in the problem?)
\item[c.] Why is $50\%$ incorrect?
\end{itemize}
\end{exploration}
\begin{MM}
There are many strategies for problem solving in mathematics, several of which we've already introduced.
\begin{itemize}
\item Understand the problem
\begin{itemize}
\item Restate the problem in your own words
\item Use drawings to understand the problem
\item Consider the definitions of mathematical terms and knowledge related to those terms, such as theorems.  Ask, what do I know about this topic that might help me solve this problem? 
\end{itemize}
\item Create a plan
\begin{itemize}
\item Break the problem down into smaller pieces
\item Test numbers
\item Look for patterns
\item Check your assumptions
\end{itemize}
\item Carry out the plan while monitoring your progress
\begin{itemize}
\item Throughout the process, check that your steps make sense
\item Check for arithmetic or algebra mistakes
\item If your plan doesn't work, that's okay! Create another plan.
\end{itemize}
\item Evaluate your process and solution
\begin{itemize}
\item Attend to units to check if your answer makes sense
\item Use an estimate to check if your answer makes sense
\item What did or didn't go well with your plan? How can that inform your future work on problems?
\item What mistakes did you make along the way? How can you catch those mistakes in the future?
\end{itemize}


\end{itemize}
\end{MM}




\end{document}
