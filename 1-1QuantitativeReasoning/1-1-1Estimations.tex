\documentclass{ximera}
\input{../preamble}


\author{David Kish, Debi Stout}
% Source: 1116 Materials



\title{Quantitative Reasoning: Estimates}

\begin{document}
\licenseAPC
\begin{abstract}
We build some ``real-world'' math skills and quantitative reasoning by roughly estimating some story problems.
\end{abstract}
\maketitle

%\typeout{************************************************}
%\typeout{Rough Estimages}
%\typeout{************************************************}

%\section{Rough Estimates}

%replace color

\section{Tips for Doing Rough Estimates} 
\begin{itemize}
\item Estimations are \textbf{NOT} guesses.  They can sometimes be educated guesses, but estimations, at least in this course, will always still require some sort of calculation(s).
\item Rough estimations are meant to be estimations which can be calculated mentally, meaning without calculators, or even pen and paper.  You will still be expected to record/write down your work and thought process for your instructor, but it should be work that you are able to do mentally.
\item Everyone has different skill and comfort levels with their mental calculations -- you may need to round values a lot to make them something you are comfortable doing in your head.  This is completely fine, but the important thing is that in your solutions/write-ups you explain what numbers you rounded and why.
\end{itemize}


\begin{exploration}[In Class Activity]Pizza Party: 
Now let’s give this a try but working with your group to determine what you would buy for a pizza party in the following scenario: 
You and your roommate are going to have some people over later and so you go to the grocery store to grab some snacks.  Everyone has agreed to pitch in \$10 to pay for pizza and snacks for the night, and you think about 12 people are coming over.  Here are the prices of various snacks from the grocery store: 
\begin{itemize}
    \item Bag of tortilla chips - \$3.99 
    \item Salsa - \$3.79 
    \item Bag of “normal” chips - \$2.99 
    \item Dozen cookies from bakery - \$4.99 
    \item Veggie Tray - \$14.99 
    \item 12 pack of soda - \$5.79 
\end{itemize}
You want to get at least one of each of these items, but you’re also going to order some pizza and breadsticks. 
Here are the prices from your pizza place: 
\begin{itemize}
    \item Cheese: 
    \begin{itemize}
        \item Medium - \$9.99 
        \item Large - \$11.99 
    \end{itemize}
    \item Pepperoni: 
    \begin{itemize}
        \item Medium - \$11.49 
        \item Large - \$13.49 
    \end{itemize}
    \item Breadsticks: 
    \begin{itemize}
        \item 5 for \$4.99 
    \end{itemize}
    \item \$2.49 delivery charge.  Don’t forget tip! 
\end{itemize}


You are a good person and do not plan to pocket any of the money that your friends are going to give you to pay for these pizzas and snacks.  Plus, you and your roommate are going to pay for your fairshare (also each pitch in \$10).  Decide what you’re going to get! 
\end{exploration}


\begin{example}
Roughly estimate how many gallons of gasoline you might use to drive from here to Chicago, IL.

\begin{image}
    \includegraphics[width=4in]{ColumbusChicago.png}
\end{image}

\begin{explanation}
Distance to Chicago, IL: about 350 miles (google maps)
My vehicle gets about 30 miles per gallon (30 miles = 1 gallon), so $350$ mi $\times \frac{1 gal}{30 mi}$ is about ${360 \over 30} = 12$ gallons
\end{explanation}
\end{example}


\begin{exploration}
Roughly estimate how many seconds you’ve been alive. 
\begin{itemize}
    \item Determine a number that would definitely be way too low for even a rough estimate for each question, but still requires some calculation. Explain why you think this would be unreasonably low. (Still do not use a calculator, use mental arithmetic.) 
    \item Determine a number that would definitely be way too high for even a rough estimate for each question. Explain why you think this would be unreasonably high.
    \item For each of those problems, do you think your original rough estimation is an underestimate or overestimate? Explain why you think this based off of your calculations. Or if you’re not sure if it’s under or over, explain why you are unsure.
    \item If possible, determine a more exact value by using a calculator and not rounding values to get an exact number, or see if google has any estimate(s). Compare this with your original estimations: were you under or over estimating, or can’t tell? If you have values to compare, how different are the two values? 
\end{itemize}
\end{exploration}

\end{document}
