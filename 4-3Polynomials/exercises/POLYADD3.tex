\documentclass{ximera}

\input{../../preamble.tex}

\author{Kenneth Berglund}




\begin{document}
In this exercise, we will rewrite the following quadratic in vertex form $a(x - h)^2 + k$, using the method of completing the square:
\[
f(x)=x^2-7x+1.
\]

\begin{exercise}
We want to rewrite $x^2 - 7x$ as a square of a linear polynomial. We need to add a constant term in order to make $x^2 - 7x$ into a square. 

We want to find numbers $m$ and $n$ so $(x + m)^2 = x^2 - 7x + n$. That is, we want to find a number $n$ that we can add onto $x^2 - 7x$ so it is the square of the linear polynomial $x + m$. 

$$
m = \answer{\frac{-7}{2}} \text{ and } n = \answer{\frac{49}{4}}
$$
\begin{hint}
Try multiplying out $(x + m)^2$ and matching the coefficients of each term with the coefficients of the corresponding term in $x^2 - 7x + n$. 
\end{hint}


\begin{exercise}
Modify the quadratic so that the square appears, in the disguise of $x^2 - 7x + \frac{49}{4}$.
$$
\left(x^2 - 7x + \answer{\frac{49}{4}} - \answer{\frac{49}{4}}\right) +1
$$ 

Use positive numbers in your answer.

\begin{hint}
Remember that if you add a term to an expression, to preserve equality, you must subtract the same term.
\end{hint}

\begin{exercise}
Now rewrite $x^2 - 7x + \frac{49}{4}$ as a square of a linear polynomial!
$$
\left(\left(x + \answer{\frac{-7}{2}}\right)^2 - \frac{49}{4}\right) +1
$$ 
\begin{hint}
What square is equal to $x^2 - 7x + \frac{49}{4}$? We found it before!
\end{hint}

\begin{exercise}
Finally, get rid of the outer parentheses and simplify to finish writing the quadratic in vertex form.
$$
\left(x -\frac{7}{2}\right)^2 - \answer{\frac{45}{4}}
$$ 

\begin{exercise}
As a bonus, here's what the manipulation would look like when done all at once:
\begin{align*}
x^2 - 7x +1 & = \left(x^2 - 7x + \frac{49}{4} - \frac{49}{4}\right) +1 \\
& = \left(\left(x - \frac{7}{2}\right)^2 - \frac{49}{4}\right) +1 \\
& = \left(x -\frac{7}{2}\right)^2 - \frac{49}{4} +\frac{4}{4}\\
& =\left(x -\frac{7}{2}\right)^2 - \frac{45}{4}.
\end{align*}
\end{exercise}
\end{exercise}
\end{exercise}
\end{exercise}
\end{exercise}




\end{document}