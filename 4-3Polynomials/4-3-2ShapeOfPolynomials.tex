\documentclass{ximera}

\input{../preamble.tex}
\author{David Kish}
\license{Creative Commons Attribution-ShareAlike 4.0 International License}


\title{Shape of Polynomials}

\begin{document}
\begin{abstract}
We explore polynomial functions.
\end{abstract}
\maketitle

We know that linear functions are the simplest of all funcitons we can consider:  their graphs have the simplest shape, their average rate of change is always constant (regardless of the interval chosen), and their formula is elementary.  Moreover, computing the value of a linear function only requires multiplication and addition.%

If we think of a linear function as having formula $L(x) = b + mx$, and the next-simplest functions, quadratic functions, as having form $Q(x) = c + bx + ax^2$, we can see immediate parallels between their respective forms and realize that it's natural to consider slightly more complicated functions by adding additional power functions.%

Indeed, if we instead view linear functions as having form
$$
L(x) = a_0 + a_1 x
$$
(for some constants $a_0$ and $a_1$) and quadratic functions as having form%
$$
Q(x) = a_0 + a_1 x + a_2 x^2\text{,}
$$
then it's natural to think about more general functions of this same form, but with additional power functions included.%
%polynomial function

\begin{definition}
Given real numbers $a_0, a_1, \ldots, a_n$ where $a_n \neq 0$, we say that the function a function $P$ is a \dfn{polynomial} of degree $n$ if
it can be written in the form%
$$
P(x) = a_0 + a_1 x + a_2 x^2 + \cdots + a_{n-1}x^{n-1} + a_n x^n, 
$$
\end{definition}
We say that the $a_i$ are the \emph{coefficients} of the polynomial and the individual power functions $a_i x^i$ are the \emph{terms} of the polynomial.   Any value of $x$ for which $P(x) = 0$ is called a \emph{zero} of the polynomial. 


\begin{example}%

The polyomial function $P(x) = 3 - 7x + 4x^2 - 2x^3 + 9x^5$ has degree $5$, its constant term is $3$, and its linear term is $-7x$.
\end{example}

Since a polynomial is simply a sum of constant multiples of various power functions with positive integer powers, we often refer to those individual terms by referring to their individual degrees:  the linear term, the quadratic term, and so on.  In addition, since the domain of any power function of the form $p(x) = x^n$ where $n$ is a positive whole number is the set of all real numbers, it's also true the the domain of any polynomial function is the set of all real numbers.%
\begin{exploration}
Point your browser to the \emph{Desmos} worksheet at \link{http://gvsu.edu/s/0zy}.  There you'll find a degree $4$ polynomial of the form $p(x) = a_0 + a_1x + a_2x^2 + a_3x^3 + a_4x^4$, where $a_0, \ldots, a_4$ are set up as sliders.  In the questions that follow, you'll experiment with different values of $a_0, \ldots, a_4$ to investigate different possible behaviors in a degree $4$ polynomial.%
\begin{enumerate}
\item What is the largest number of distinct points at which $p(x)$ can cross the $x$-axis?%
For a polynomial $p$, we call any value $r$ such that $p(r) = 0$ a zero of the polynomial.  Report the values of $a_0, \ldots, a_4$ that lead to that largest number of zeros for $p(x)$.%
\item What other numbers of zeros are possible for $p(x)$?  Said differently, can you get each possible number of fewer zeros than the largest number that you found in (a)? Why or why not?
\item We say that a function has a turning pointif the function changes from decreasing to increasing or increasing to decreasing at the point.  For example, any quadratic function has a turning point at its vertex.%
What is the largest number of turning points that $p(x)$ (the function in the Desmos worksheet) can have?  Experiment with the sliders, and report values of $a_0, \ldots, a_4$ that lead to that largest number of turning points for $p(x)$.%
\item What other numbers of turning points are possible for $p(x)$? Can it have no turning points?  Just one?  Exactly two? Experiment and explain.%
\item What long-range behavior is possible for $p(x)$?  Said differently, what are the possible results for $\displaystyle \lim_{x \to -\infty} p(x)$ and $\displaystyle \lim_{x \to \infty} p(x)$?%
\item What happens when we plot $y = a_4 x^4$ in and compare $p(x)$ and $a_4 x^4$?  How do they look when we zoom out? (Experiment with different values of each of the sliders, too.)
\end{enumerate}
\end{exploration}

We know that each of the power functions $x$, $x^2$, $\ldots$, $x^n$ grow without bound as $x \to \infty$.  Intuitively, we sense that $x^5$ grows faster than $x^4$ (and likewise for any comparison of a higher power to a lower one).  This means that for large values of $x$, the most important term in any polynomial is its highest order term. when we compared $p(x) = a_0 + a_1 x + a_2 x^2 + a_3 x^3 + a_4 x^4$ and $y = a_4 x^4$.%


For any degree $n$ polynomial $p(x) = a_0 + a_1 x + \cdots + a_{n-1}x^{n-1} + a_n x^n$, its long-range behavior is the same as its highest-order term $q(x) = a_n x^n$.  Thus, any polynomial of even degree appears ``U-shaped'' ($\cup$ or $\cap$, like $x^2$ or $-x^2$) when we zoom way out, and any polynomial of odd degree appears ``chair-shaped'' (like $x^3$ or $-x^3$) when we zoom way out.%

\begin{image}
\includegraphics{poly-degree-7-near.jpg}
\end{image}

\begin{image}
\includegraphics{poly-degree-7-far.jpg}
\end{image}


In the second graph we see how the degree $7$ polynomial pictured there (and in the first graph as well) appears to look like $q(x) = -x^7$ as we zoom out.%


\begin{summary}\begin{itemize}
\item Polynomial
\end{itemize}\end{summary}

 
% %\section{Exponent Rules}
%   %%%%%%%%%%%%%%%%
%    \section{Product Rule}
%        If we write out $3^5\cdot 3^2$ without using exponents,
%        we'd have:
%$$
% 3^5 \cdot 3^2 = \left(3 \cdot 3\cdot 3\cdot 3\cdot 3\right) \cdot \left(3 \cdot 3\right)
%$$
%        If we then count how many $3$s are being multiplied together,
%        we find we have $5+2=7$, a total of seven $3$s.
%        So $3^5\cdot 3^2$ simplifies like this:
%       $$
%          3^5\cdot 3^2 = 3^{5+2} = 3^7
%$$
%\begin{example}          
%Simplify $x^2\cdot x^3$. \\
%\begin{explanation}       
%          To simplify $x^2\cdot x^3$,
%          we write this out in its expanded form,
%          as a product of $x$'s, we have
%$$
%            x^2\cdot x^3 =(x\cdot x)(x \cdot x \cdot x)
%            =x\cdot x\cdot x \cdot x \cdot x
%            =x^5
%   $$   
%          Note that we obtained the exponent of $5$ by adding $2$ and $3$.
%  \end{explanation}
%\end{example}
%
%      This example demonstrates our first exponent rule,
%      the Product Rule:\\
%\begin{callout}
%\textbf{ \Large Product Rule of Exponents} \\
%      When multiplying two expressions that have the same base,
%      we can simplify the product by adding the exponents.
%        $$
%x^m \cdot x^n = x^{m+n}
%$$
%\end{callout}
%   Recall that $x=x^1$. It helps to remember this when multiplying certain expressions together.
%\begin{example}        
%  Multiply $x(x^3+2)$ by using the distributive property.\\
%\begin{explanation}
%          According to the distributive property,
%        $x(x^3+2)=x\cdot x^3 + x\cdot2$
%          How can we simplify that term $x\cdot x^3$?
%          It's really the same as $x^1\cdot x^3$,
%          so according to the Product Rule, it is $x^4$.
%          So we have:
%        $$
%            x(x^3+2)=x\cdot x^3 + x\cdot2
%            =x^4+2x
%$$
%\end{explanation}
%\end{example}         
%
%%%%%%%%%%%%%%%%%
% \section{Power to a Power Rule}
%
%        If we write out $\left(3^5\right)^2$ without using exponents,
%        we'd have $3^5$ multiplied by itself:
%   $$
%         \left(3^5\right)^2 = \left(3^5\right)\cdot \left(3^5\right)
%         = \left(3\cdot 3\cdot 3\cdot 3 \cdot 3 \right) \cdot \left(3 \cdot 3\cdot 3\cdot 3\cdot 3\right)
%      $$ 
%        If we again count how many $3$s are being multiplied,
%        we have a total of two groups each with five $3$s.
%        So we'd have $2\cdot 5=10$ instances of a $3$.
%        So $\left(3^5\right)^2$ simplifies like this:
%   
%          $\left(3^5\right)^2 = 3^{2\cdot 5}$
%          $= 3^{10}$
%
%
%\begin{example}
%          Simplify $\left(x^2\right)^3$.\\
%\begin{explanation}
%          To simplify $\left(x^2\right)^3$,
%          we write this out in its expanded form,
%          as a product of $x$'s, we have
%            $\left(x^2\right)^3 =\left(x^2\right) \cdot \left(x^2\right)\cdot\left(x^2\right)$
%            $=(x \cdot x)\cdot (x \cdot x)\cdot (x \cdot x)$
%            $=x^6$
%     \end{explanation}
%          Note that we obtained the exponent of $6$ by multiplying $2$ and $3$.
%\end{example}
%      This demonstrates our second exponent rule,
%      the Power to a Power Rule:
%\begin{callout}
%\textbf{ \Large Power to a Power Rule} \\
%      when a base is raised to an exponent and that expression is raised to another exponent,
%      we multiply the exponents.
%   $$
%      \left(x^m\right)^n = x^{m \cdot n}
%   $$
%\end{callout}
%
%%%%%%%%%%%%%%%%%%%
%      \section{ Product to a Power Rule}
%        The third exponent rule deals with having multiplication inside a set of parentheses and an exponent outside the parentheses.
%        If we write out $\left(3t\right)^5$ without using an exponent,
%        we'd have $3t$ multiplied by itself five times:
%$$
%      (3t)^5= (3t)(3t)(3t)(3t)(3t)
%$$
%        Keeping in mind that there is multiplication between every $3$ and $t$,
%        and multiplication between all of the parentheses pairs,
%        we can reorder and regroup the factors:
%          $\left(3t\right)^5 = (3\cdot t)\cdot (3\cdot t)\cdot (3\cdot t)\cdot (3\cdot t)\cdot (3\cdot t)$
%          $= \left(3\cdot 3\cdot 3\cdot 3\cdot 3 \right) \cdot \left(t \cdot t \cdot t \cdot t \cdot t\right)$
%          $= 3^5 t^5$
%        We could leave it written this way if $3^5$ feels especially large.
%        But if you are able to evaluate $3^5=243$,
%        then perhaps a better final version of this expression is $243t^5$.
% 
%        We essentially applied the outer exponent to each factor inside the parentheses.
%        It is important to see how the exponent $5$ applied to \textbf{both} the $3$ \textbf{and} the $t$,
%        not just to the $t$.
%   
%
% \begin{example}   
%          Simplify $(xy)^5$.
%    
%          To simplify $(xy)^5$,
%          we write this out in its expanded form,
%          as a product of $x$'s and $y$'s, we have
%
%            $(xy)^5 =(x \cdot y) \cdot (x \cdot y) \cdot (x \cdot y) \cdot (x \cdot y) \cdot (x \cdot y)$
%            $=(x \cdot x \cdot x \cdot x \cdot x) \cdot (y \cdot y \cdot y \cdot y \cdot y)$
%            $=x^5 y^5$
%
%          Note that the exponent on $xy$ can simply be applied to both $x$ and $y$.
%\end{example}
%
%
%      This demonstrates our third exponent rule,
%      the Product to a Power Rule:
%\begin{callout}
%\textbf{ \Large Product to a Power Rule} \\
%      When a product is raised to an exponent,
%      we can apply the exponent to each factor in the product.
%$$
%        \left(x\cdot y\right)^n = x^{n}\cdot y^{n}
% $$
%\end{callout}
%
%
%%%%%%%%%%%%%%%%%%
%      \section{Summary of the Rules of Exponents for Multiplication}
%  
%
%
%          If $a$ and $b$ are real numbers,
%          and $m$ and $n$ are positive integers,
%          then we have the following rules:
% 
%\textbf{Product Rule}
%$$
%            a^{m} \cdot a^{n} = a^{m+n}
%$$
%\textbf{Power to a Power Rule}
%   $$
%           (a^{m})^{n} = a^{m\cdot n}
%   $$
% \textbf{ Product to a Power Rule}
%   $$
%           (ab)^{m} = a^{m} \cdot b^{m}
%   $$  
%      Many examples will make use of more than one exponent rule.
%      In deciding which exponent rule to work with first,
%      it's important to remember that the order of operations still applies.
%\begin{example} Simplify the following expression.
%                $\left(3^7r^5\right)^4$\\
%    \begin{explanation}
%                Since we cannot simplify anything inside the parentheses, we'll begin simplifying this expression using the Product to a Power rule.
%                We'll apply the outer exponent of 4 to each factor inside the parentheses.
%                Then we'll use the Power to a Power Rule to finish the simplification process.
%      $$          
%                  \left(3^7r^5\right)^4 = \left(3^7\right)^4 \cdot \left(r^5\right)^4
%                  = 3^{7\cdot4} \cdot r^{5\cdot 4}
%                  = 3^{28}r^{20}
%         $$      
%                Note that $3^{28}$ is too large to actually compute, even with a calculator,
%                so we leave it written as $3^{28}$.
%\end{explanation}
%\end{example}
%\begin{example}
%Simplify the following expression.
%       $\left(t^3\right)^2\cdot \left(t^4\right)^5$\\
%\begin{explanation}
%                According to the order of operations,
%                we should first simplify any exponents before carrying out any multiplication.
%                Therefore, we'll begin simplifying this by applying the Power to a Power Rule and then finish using the Product Rule.
%           $$
%                  \left(t^3\right)^2\cdot \left(t^4\right)^5 = t^{3\cdot2}\cdot t^{4\cdot5}
%                  = t^6 \cdot t^{20}
%                  = t^{6+20}
%                  = t^{26}
%$$
%\end{explanation}
%\end{example}
% \begin{remark} 
%        We cannot simplify an expression like $x^2y^3$ using the Product Rule,
%        as the factors $x^2$ and $y^3$ do not have the same base.
%\end{remark}



\end{document}
