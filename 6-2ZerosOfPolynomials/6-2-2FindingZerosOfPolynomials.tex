\documentclass{ximera}

\input{../preamble}
\author{}
\license{Creative Commons Attribution-ShareAlike 4.0 International License}
\acknowledgement{}

\title{Finding Zeros of Polynomials}

\begin{document}
\begin{abstract}
  
\end{abstract}
\maketitle


%\typeout{************************************************}
%\typeout{Motivating Questions}
%\typeout{************************************************}

%\begin{motivatingQuestions}\begin{itemize}
%\item 
%\item 
%\item 
%\end{itemize}\end{motivatingQuestions}


%\typeout{************************************************}
%\typeout{Subsection Introduction}
%\typeout{************************************************}

\section{Introduction}


   This section covers a technique for factoring polynomials like $x^3+3x^2+2x+6$,
      which factors as $(x^2+2)(x+3)$.
      If there are four terms,
      the technique in this section might
      help you to factor the polynomial.

   Recall that to factor $3x+6$,
      we factor out the common factor $3$:
$
     \begin{array}{rl}
        3x+6 & =\attention{3}x+\attention{3}\cdot2 \\
        & =3(x+2)
  \end{array}
$   
      The ``$3$'' here could have been something more abstract,
      and it still would be valid to factor it out:
$
\begin{array}{rl}
 x(a+b)+2(a+b)&=x\attention{\overbrace{(a+b)}}+2\attention{\overbrace{(a+b)}}\\
        &=(a+b)(x+2)
\end{array}
$
      In this last example, we factored out the binomial factor $(a+b)$.
      Factoring out binomials is the essence of this section,
      so let's see that a few more times:
\begin{center}
$\begin{array}{rl}
        x(x+2)+3(x+2)&=x\attention{\overbrace{(x+2)}}+3\attention{\overbrace{(x+2)}}\\
        &=(x+2)(x+3)
\end{array}
$
\end{center}

\begin{center}
$\begin{array}{rl}
  z^2(2y+5)+3(2y+5)&=z^2\attention{\overbrace{(2y+5)}}+3\attention{\overbrace{(2y+5)}}\\
        &=(2y+5)(z^2+3)
\end{array}
$
\end{center}
      And even with an expression like $Q^2(Q-3)+Q-3$,
      if we re-write it in the right way using a $1$ and some parentheses,
      then it too can be factored:
\begin{center}
$\begin{array}{rl}
  Q^2(Q-3)+Q-3&=Q^2(Q-3)+1(Q-3)\\
        &=Q^2\attention{\overbrace{(Q-3)}}+1\attention{\overbrace{(Q-3)}}\\
        &=(Q-3)(Q^2+1)
\end{array}
$
\end{center}

 The truth is you are unlikely to come upon an expression like
      $x(x+2)+3(x+2)$, as in these examples.
      Why wouldn't someone have multiplied that out already?
      Or factored it all the way?
      So far in this section,
      we have only been looking at a stepping stone to a real factoring technique called
      \textbf{factoring by grouping}.

\begin{image}
\begin{tikzpicture}[grow                    = right,
                                    sibling distance        = 6em,
                                    level distance          = 10em,
                                    edge from parent/.style = {draw, -{Kite}},
                                    every node/.style       = {font=\footnotesize},
                                    sloped,
                                    treenode/.style = {shape=rectangle,
                                                        rounded corners,
                                                        draw,
                                                        align=center,
                                                        top color=white,
                                                        bottom color=secondcolor!20},
                                    root/.style     = {treenode,
                                                        font=\normalsize,
                                                        bottom color=firstcolor!30},
                                    env/.style      = {treenode,
                                                        font=\normalsize},
                                    dummy/.style    = {circle,
                                                        draw,
                                                        fill=thirdcolor!10!white}
                                    ]
                    \node [root] {Factor\\out GCF}
                        child[level distance = 5em, sibling distance = 7.5em,] {
                            node [dummy] {}
                            child[level distance = 10em, sibling distance = 3em] {
                                node [env] {$\begin{aligned}&\phantom{=}A^2-B^2\\&=(A-B)(A+B)\end{aligned}$}
                                edge from parent node [below] {\parbox{5em}{\centering Difference of Squares}}
                            }
                            % child[level distance = 22em, sibling distance = 5em] {
                            %     node [env] {$\begin{aligned}&\phantom{=}A^3-B^3\\&=(A-B)(A^2+AB+B^2)\end{aligned}$}
                            %     edge from parent node [above] {Difference of Cubes}
                            % }
                            % child[level distance = 11em, sibling distance = 4em] {
                            %     node [env] {$\begin{aligned}&\phantom{=}A^3+B^3\\&=(A+B)(A^2-AB+B^2)\end{aligned}$}
                            %     edge from parent node [above] {\parbox{4em}{\centering Sum of Cubes}}
                            % }
                            edge from parent node [below] {binomial}
                        }
                        child[level distance = 9em] {
                            node [env] {Try\\Grouping}
                            edge from parent node [] {\parbox{4em}{\centering four terms}}
                        }
                        child[level distance = 10em, sibling distance = 5em] {
                            node [dummy] {\parbox{5.6em}{\centering Are first and last terms perfect squares?}}
                            child[level distance = 10em, sibling distance = 10em] {
                                node [dummy] {}
                                child[level distance = 14em, sibling distance = 4em] {
                                    node [env] {try the AC method}
                                    edge from parent node [below] {leading coefficient $\neq1$}
                                }
                                child[level distance = 14em, sibling distance = 4em] {
                                    node [env] {find factors of $c$\\that add to $b$}
                                    edge from parent node [above] {leading coefficient $1$}
                                }
                                edge from parent node [below] {no}
                            }
                            child[level distance = 7em] {
                                node [dummy] {}
                                child[sibling distance = 6em] {
                                    node [env] {$\begin{aligned}&\phantom{=}A^2+2AB+B^2\\&=(A+B)^2\end{aligned}$}
                                    edge from parent node [below] {try}
                                }
                                child[sibling distance = 2em] {
                                    node [env] {$\begin{aligned}&\phantom{=}A^2-2AB+B^2\\&=(A-B)^2\end{aligned}$}
                                    edge from parent node [above] {try}
                                }
                                edge from parent node [above] {yes}
                            }
                            edge from parent node [above] {trinomial}
                        };
                \end{tikzpicture}
\end{image}

%\typeout{************************************************}
%\typeout{Summary}
%\typeout{************************************************}

%\begin{summary}\begin{itemize}
%\item 
%\item 
%\item
%\end{itemize}\end{summary}




\end{document}
