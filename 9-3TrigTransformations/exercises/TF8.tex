\documentclass{ximera}

\input{../../preamble.tex}

\author{Kenneth Berglund}
\acknowledgement{https://www.stitz-zeager.com/szct07042013.pdf}

\begin{document}
\licenseSZ
For each equation below, determine how many solutions lie in the interval $[0, 2\pi)$ and list them in increasing order, if applicable.

\begin{exercise}
The equation $\csc(\pi x) = 0$ has $\answer{0}$ solution(s) in the interval $[0, 2\pi)$. 

\end{exercise}

\begin{exercise}
The equation $\sin(5x) = 0$ has $\answer{10}$ solution(s) in the interval $[0, 2\pi)$.

\begin{exercise}
The solutions are $\answer{0}$,  $\answer{\frac{\pi}{5}}$, $\answer{\frac{2\pi}{5}}$, $\answer{\frac{3\pi}{5}}$, $\answer{\frac{4\pi}{5}}$, $\answer{\pi}$,  $\answer{\frac{6\pi}{5}}$, $\answer{\frac{7\pi}{5}}$, $\answer{\frac{8\pi}{5}}$, and $\answer{\frac{9\pi}{5}}$.
\end{exercise}
\end{exercise} 

\begin{exercise}
The equation $\sec(3x) = \sqrt{2}$ has $\answer{6}$ solution(s) in the interval $[0, 2\pi)$.

\begin{exercise}
The solutions are $\answer{\frac{\pi}{12}}$, $\answer{\frac{7\pi}{12}}$, $\answer{\frac{3\pi}{4}}$, $\answer{\frac{5\pi}{4}}$, $\answer{\frac{17\pi}{12}}$, and $\answer{\frac{23\pi}{12}}$.
\end{exercise}
\end{exercise} 

\begin{exercise}
The equation $\sin\left(\frac{x}{3}\right) = \frac{\sqrt{2}}{2}$ has $\answer{1}$ solution(s) in the interval $[0, 2\pi)$.

\begin{exercise}
The solution is $\answer{\frac{3\pi}{4}}$.
\end{exercise}
\end{exercise} 
\end{document}