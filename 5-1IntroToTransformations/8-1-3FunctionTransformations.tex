\documentclass[handout, noauthor, nooutcomes]{ximera}

\input{../preamble}
\author{}
\license{Creative Commons Attribution-ShareAlike 4.0 International License}
\acknowledgement{}

\title{Reflections of Functions}

\begin{document}
\begin{abstract}
  
\end{abstract}
\maketitle


%%\typeout{************************************************}
%%\typeout{Motivating Questions}
%%\typeout{************************************************}
%
%\begin{motivatingQuestions}\begin{itemize}
%\item 
%\item 
%\item 
%\end{itemize}\end{motivatingQuestions}


%\typeout{************************************************}
%\typeout{Reflections Across Axes}
%\typeout{************************************************}

%\section{Reflections Across Axes}
%Points $(x,y)$ and $(x,-y)$ are reflections of each other across the x-axis. Points $(x, y)$ and $(-x, y)$ are reflections of each other across the y-axis. In general, two points that are symmetric with respect to a line are reflections of each other across that line.
%\begin{image}
%\includegraphics[]{images/reflectiongraph}
%\end{image}
%\begin{callout}
%The following transformations result in reflections of the graph of $y = f(x)$
%\begin{itemize}
%\item Reflection across the x-axis
%\[
%y=-f(x)
%\]
%\item Reflection across the y-axis
%\[
%y=f(-x)
%\]
%\item Reflection through the origin
%\[
%y=-f(-x)
%\]
%\end{itemize}
%\end{callout}
%\begin{example}
%Find an equation for the reflection of $f(x) = \frac{5x - 9}{x^2+3}$ across each axis.
%\\
%\begin{explanation}
%Across the x-axis: $y = -f(x) = -\frac{5x-9}{x^2+3}=\frac{9-5x}{x^2+3}$
%\\
%Across the y-axis: $y = f(-x) = \frac{5(-x)-9}{(-x)^2+3}=\frac{-5x-9}{x^2+3}$
%\end{explanation}
%\end{example}

%\typeout{************************************************}
%\typeout{Putting it Together}
%\typeout{************************************************}

\section{Putting it Together}
Transformations may be performed one after another. If the transformations include stretches, shrinks, or reflections, the order in which the transformations are performed may make a difference. In those cases, be sure to pay particular attention to the order.

\begin{example}
\begin{enumerate}
\item The graph of $y=x^2$ undergoes the following transformations, in order. Find the equation of the graph that results.
\begin{itemize}
\item a horizontal shift 2 units to the right
\item a vertical stretch by a factor of 3
\item a vertical translation 5 units up
\end{itemize}
\item Apply the transformations above in the opposite order and find the equations of the graph that results.
\end{enumerate}
\begin{explanation}
\begin{enumerate}
\item Applying the transformations in order we have\\
$
\begin{array}{lc}
y = x^2& \text{Original function}\\
y = (x-2)^2& \text{Horizontal shift} \\
y = 3(x-2)^2& \text{Vertical stretch} \\
y = 3(x-2)^2+5& \text{Vertical translation}\\
y = 3x^2 - 12x + 17& \text{Expanded form}
\end{array}
$
\item Applying the transformations in the opposite order we have\\
$
\begin{array}{lc}
y = x^2& \text{Original function}\\
y = x^2 + 5 & \text{Vertical translation} \\
y = 3(x^2+5)& \text{Vertical stretch} \\
y = 3((x-2)^2+5) & \text{Horizontal translation}\\
y = 3x^2 - 12x + 27& \text{Expanded form}
\end{array}
$
\end{enumerate}
\end{explanation}
\end{example}

%%\typeout{************************************************}
%%\typeout{Summary}
%%\typeout{************************************************}
%
%\begin{summary}\begin{itemize}
%\item 
%\item 
%\item
%\end{itemize}\end{summary}




\end{document}
