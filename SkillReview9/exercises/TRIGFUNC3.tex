\documentclass{ximera}

\input{../../preamble.tex}

\author{Kenneth Berglund}
\acknowledgement{}

\begin{document}
Recall that the cotangent function is defined as the reciprocal of the tangent function: $\cot(x) = \frac{1}{\tan(x)} = \frac{\cos(x)}{\sin(x)}$. In this problem, we will find some properties of the cotangent function. 
\begin{exercise}
\begin{enumerate}
\item Recall that $\sin(x) = 0$ when $x$ is a multiple of $\pi$: $\ldots, -3\pi, -2\pi, -\pi, 0, \pi, 2\pi, 3\pi, \ldots$. Select the domain of the cotangent function.
\begin{multipleChoice}
\choice{$(- \infty, \infty)$}
\choice{$(- \infty, 0) \cup (0, \infty)$}
\choice[correct]{$\cdots \cup (-2\pi, -\pi) \cup (-\pi, 0) \cup (0, \pi) \cup (\pi, 2\pi) \cup \cdots$}
\choice{$\cdots \cup \left(-\frac{5\pi}{2}, -\frac{3\pi}{2}\right) \cup \left(-\frac{3\pi}{2}, --\frac{\pi}{2}\right) \cup \left(-\frac{\pi}{2}, \frac{\pi}{2}\right) \cup \left(\frac{\pi}{2}, \frac{3\pi}{2}\right)  \cup \cdots$}
\end{multipleChoice}


\item Recall that tangent is an odd function. Cotangent is 
\begin{multipleChoice}
\choice[correct]{odd.}
\choice{even.}
\choice{odd and even.}
\choice{neither odd nor even.}
\end{multipleChoice}


\item On the interval $\left(0, \frac{\pi}{2}\right)$, cotangent is 
\begin{multipleChoice}
\choice{increasing.}
\choice[correct]{decreasing.}
\choice{neither increasing nor decreasing.}
\end{multipleChoice}

\item Using knowledge of famous angles, $\cot\left(\frac{\pi}{3}\right) = \answer{1/\sqrt{3}}$.

\item Which of the following graphs is the graph of $\cot(x)$?
\begin{figure}[!h]
\begin{image}
\includegraphics[width=.4\linewidth]{TRIGFUNC3-2.png}
\hspace{20mm}
\includegraphics[width=.4\linewidth]{TRIGFUNC3-4.png}

\end{image}
\caption{A on the left and B on the right}
\end{figure}

\begin{figure}[!h]
\begin{image}
\includegraphics[width=.4\linewidth]{TRIGFUNC3-3.png}
\hspace{20mm}
\includegraphics[width=.4\linewidth]{TRIGFUNC3-1.png}
\end{image}
\caption{C on the left and D on the right}
\end{figure}

\begin{multipleChoice}
\choice{A}
\choice{B}
\choice{C}
\choice[correct]{D}
\end{multipleChoice}

\end{enumerate}
\end{exercise}

\end{document}