 \documentclass{ximera}

\input{../preamble}
\author{David Kish}
\license{Creative Commons Attribution-ShareAlike 4.0 International License}
\acknowledgement{}

\title{Application of Trig Identities}

\begin{document}
\begin{abstract}
  
\end{abstract}
\maketitle




%\typeout{************************************************}
%\typeout{Subsection Verifying Identities}
%\typeout{************************************************}

\section{Verifying Identities}
In this section, we will look at strategies to verify identities.
\begin{remark}
\centerline{\textbf{Strategies for Verifying Identities}} 

\begin{itemize}

\item  Try working on the more complicated side of the identity.

\item Use the Reciprocal and Quotient Identities to write functions on one side of the identity in terms of the functions on the other side of the identity.  Simplify the resulting complex fractions.

\item Add rational expressions with unlike denominators by obtaining common denominators.

\item  Use the Pythagorean Identities to `exchange' sines and cosines, secants and tangents, cosecants and cotangents, and simplify sums or differences of squares to one term. 

\item Multiply numerator \textbf{and} denominator by Pythagorean
Conjugates in order to take advantage of the Pythagorean Identities.

\item If you find yourself stuck working with one side of the identity, try starting with the other side of the identity and see if you can find a way to bridge the two parts of your work.


\end{itemize}
\end{remark}
\begin{example}
Verify the following:\\
\[
\tan{(\theta)}\cos{(\theta)}=\sin{(\theta)}
\]
\\
\begin{explanation}
Let's start with the more complicated side. We know that we want to end up with $\sin{(\theta)}$ in the end, so using the Quotient identity to replace $\tan{(\theta)}$ with $\frac{\sin{(\theta)}}{\cos{(\theta)}}$ is a reasonable place to start.
\[
\tan{(\theta)}\cos{(\theta)}=\left(\frac{\sin{(\theta)}}{\cos{(\theta)}}\right)\cos{(\theta)}
\]
We can now cancel our $\cos{(\theta)}$ terms.
\[
\tan{(\theta)}\cos{(\theta)}=\left(\frac{\sin{(\theta)}}{\cancel{\cos{(\theta)}}}\right)\cancel{\cos{(\theta)}}
\]
This, thankfully, leaves us with our original equation, so we have verified this identity.
\[
\tan{(\theta)}\cos{(\theta)}=\sin{(\theta)}
\]
\end{explanation}
\end{example}

\begin{example}
Verify the following:\\
\[
\frac{\sec{(x)}-cos{(x)}}{\sec{(x)}}=sin^2{(x)}
\]
\\
\begin{explanation}
Let's start with the more complicated side. Since $\sec{(x)}$ is the reciprocal of $\cos{(x)}$, rewriting this side completely in terms of $\cos{(x)}$ could help us verify this identity.
\[
\frac{\sec{(x)}-\cos{(x)}}{\sec{(x)}} = \frac{\frac{1}{\cos{(x)}}-\cos{(x)}}{\frac{1}{\cos{(x)}}}
\]
We can now simplify the fractional division by inverting and multiplying.
\[
\frac{\frac{1}{\cos{(x)}}-\cos{(x)}}{\frac{1}{\cos{(x)}}} = \left(\frac{1}{\cos{(x)}}-\cos{(x)}\right) \cdot \cos{(x)}
\]
We continue by distributing our $\cos{(x)}$ term and then simplify.
\[
\left(\frac{1}{\cos{(x)}}-\cos{(x)}\right) \cdot \cos{(x)} = \frac{\cos{(x)}}{\cos{(x)}}-\cos^2{(x)} = 1 - \cos^2{(x)}
\]
Using an alternate form of the Pythagorean Identity, we can make the following substitution.
\[
1 - \cos^2{(x)} = sin^2{(x)}
\]
We have now verified our original equation.
\[
\frac{\sec{(x)}-cos{(x)}}{\sec{(x)}}=sin^2{(x)}
\]
\end{explanation}
\end{example}
\begin{example}
Verify the following:\\
\[
\tan{(\theta)}+ \cot{(\theta)}=\csc{(\theta)}\sec{(\theta)}
\]
\\
\begin{explanation}
Let's start with the left side. By using the Quotient identities we can change both of our terms to be in the form of $\sin{(\theta})$ and  $\cos{(\theta})$ 
\[
\tan{(\theta)}+ \cot{(\theta)}=\frac{\sin{(\theta)}}{\cos{(\theta)}} + \frac{\cos{(\theta)}}{\sin{(\theta)}}
\]
We can find a common demoninator to begin combining our terms.
\[
\frac{\sin{(\theta)}}{\cos{(\theta)}} + \frac{\cos{(\theta)}}{\sin{(\theta)}} = \frac{\sin^2{(\theta)}}{\sin{(\theta)}\cos{(\theta)}} + \frac{\cos^2{(\theta)}}{\sin{(\theta)}\cos{(\theta)}}
\]
Now we combine our terms and simplify.
\[
 \frac{\sin^2{(\theta)}}{\sin{(\theta)}\cos{(\theta)}} + \frac{\cos^2{(\theta)}}{\sin{(\theta)}\cos{(\theta)}} = \frac{\sin^2{(\theta)} + \cos^2{(\theta)}}{\sin{(\theta)}\cos{(\theta)}} = \frac{1}{\sin{(\theta)}\cos{(\theta)}}
\]
We are now left with two terms being multiplied together. We will split them up to better show the next step.
\[
\frac{1}{\sin{(\theta)}\cos{(\theta)}} = \frac{1}{\sin{(\theta)}} \cdot \frac{1}{\cos{(\theta)}}
\]
We can use the Reciprocal identies for these two terms and we have found what we wanted.
\[
\frac{1}{\sin{(\theta)}} \cdot \frac{1}{\cos{(\theta)}} = \csc{(\theta)}\sec{(\theta)}
\]
The identity is verified.
\[
\tan{(\theta)}+ \cot{(\theta)}=\csc{(\theta)}\sec{(\theta)}
\]
\end{explanation}
\end{example}
\begin{example}
Verify the following:\\
\[
\sin{(x)}=\tan{(x)}+\cos{(x)}
\]
\\
\begin{explanation}
Let's start with the more complicated side.  By using the Quotient identities we can change both of our terms to be in the form of $\sin{(\theta})$ and  $\cos{(\theta})$ 
\[
\tan{(x)}+\cos{(x)}= \frac{\sin{(x)}}{\cos{(x)}}+\cos{(x)}
\]
We can find a common demoninator to begin combining our terms.
\[
\frac{\sin{(x)}}{\cos{(x)}}+\cos{(x)} = \frac{\sin{(x)}}{\cos{(x)}} + \frac{\cos^2{(x)}}{\cos{(x)}}
\]
We combine our terms.
\[
\frac{\sin{(x)}}{\cos{(x)}} + \frac{\cos^2{(x)}}{\cos{(x)}} =  \frac{\sin{(x)} + \cos^2{(x)}}{\cos{(x)}} 
\]
We use a Pythagorean identity.
\[
\frac{\sin{(x)} + \cos^2{(x)}}{\cos{(x)}} = \frac{\sin{(x)} + 1- \sin^2{(x)}}{\cos{(x)}}
\]
We are now at a difficult point. The equation does not seem to be simplifying and we are not making any progress. It is possible that this is \textbf{NOT} equal. In order, to prove that we have to go back and use a test value in our original equation. Let's try $\frac{\pi}{6}$.
\[
\sin{(\frac{\pi}{6})}=\tan{(\frac{\pi}{6})}+\cos{(\frac{\pi}{6})}
\]
Since this is a trig value, we can go ahead and evaluate it.
\[
\frac{1}{2}= \frac{\sqrt{3}}{3} + \frac{\sqrt{3}}{2}
\]
We again find a common denominator
\[
\frac{1}{2}= \frac{2\sqrt{3}}{6} + \frac{3\sqrt{3}}{6}
\]
With a little simplifying, we have the following equation.
\[
\frac{1}{2}= \frac{5\sqrt{3}}{6}
\]
$\frac{5\sqrt{3}}{6}$ is definitely larger than $\frac{1}{2}$ (verify with a calculator) and thus not equal.
\[
\frac{1}{2}\neq \frac{5\sqrt{3}}{6}
\]
We have proved that this is equation \textbf{NOT} equal. 
\[
\sin{(x)}\neq \tan{(x)}+\cos{(x)}
\]
\end{explanation}
\end{example}

%\typeout{************************************************}
%\typeout{Motivating Questions}
%\typeout{************************************************}

%\begin{motivatingQuestions}\begin{itemize}
%\item How to study, in a systematic way, ratios between the sides of a right triangle?
%\item What are the values of sine, cosine, and tangent, for the most frequent angles of $30^\circ$, $45^\circ$ and $60^\circ$? And why?
%\end{itemize}\end{motivatingQuestions}


%\typeout{************************************************}
%\typeout{Subsection Introduction}
%\typeout{************************************************}

%\section{Introduction}
%
%Recall that two triangles are called {\bf similar} if one of them can be obtained by rescaling and moving around the other. Here's an example:
%
%\begin{figure}[h]
%  \centering
%  \includegraphics[scale=.3]{/figures/9-1-1-intro.png}
%\end{figure}
%
%The dotted square symbol is a shorthand for ``$90^\circ$ degrees''. Triangles which have a $90^\circ$ angle are called {\bf right triangles}, and will be the main focus of our discussion. What do similar triangles actually have in common? Certainly angles, but not necessarily the lengths of the sides. However, the {\bf ratios} between any two sides of a triangle will remain the same, no matter how the triangle gets rescaled. Such ratios ultimately give us so much information about the given right triangle that they deserve special names: sine, cosine, and tangent.
%
%
%\section{Definitions and examples}
%
%\begin{callout}
%  {\bf Definition (right triangle trig):} Consider the following right triangle, with one angle $\theta$ (this is the lowercase greek letter ``theta'') indicated, and sides labeled $a$, $b$ and $c$.
%
%\begin{center}
%  \includegraphics[scale=.3]{/figures/9-1-1-defn.png}
%\end{center}
%  
%  Then:
%  \begin{itemize}
%  \item   The side labeled with $a$ is called the {\bf adjacent} side to $\theta$.
%  \item The side labeled with $b$ is called the {\bf opposite} side to $\theta$.
%  \item   The side labeled with $c$ is called the {\bf hypotenuse} of the triangle.
%  \end{itemize}
%  With this in place, we define the {\bf sine}, {\bf cosine}, and {\bf tangent} of $\theta$, by $$\sin\theta = \frac{b}{c} \left(= \frac{\rm opp.}{\rm hyp.}\right), \quad \cos\theta = \frac{a}{c}\left(=\frac{\rm adj.}{\rm hyp.}\right), ~~\mbox{and}~~ \tan\theta = \frac{b}{a} \left(= \frac{\rm opp.}{\rm adj.}\right).$$
%\end{callout}
%
%\begin{remark}
%  The hypotenuse of a right triangle is always the side opposite to the right angle. Also note that $\tan\theta = \sin\theta/\cos\theta$. You might have seen the mnemonic ``SOH CAH TOA'' before: for example, ``SOH'' means ``sine equals opposite over hypotenuse'', and so on.
%\end{remark}
%
%\begin{example}
%  For each of the following triangles with given angle $\theta$, identify the adjacent (adj.), opposite (opp.) and hypotenuse (hyp.), and compute $\sin\theta$, $\cos\theta$ and $\tan\theta$.
%
%  \begin{enumerate}[label=\alph*.]
%  \item  \begin{figure}[h]
%  \centering
%  \includegraphics[scale=.3]{/figures/9-1-1-triangle-5-12-13.png}
%\end{figure}
%
%
%    \begin{explanation}
%      We have ${\rm opp.} = 12$, ${\rm adj.} = 5$ and ${\rm hyp.} = 13$. This means that $$\sin\theta = \frac{12}{13}, \quad \cos\theta = \frac{5}{13},\quad \mbox{and}\quad \tan\theta = \frac{12}{5}.$$
%    \end{explanation}
%    
%  \item \begin{figure}[h]
%  \centering
%  \includegraphics[scale=.3]{/figures/9-1-1-triangle-6-8-10.png}
%\end{figure}
%
%    \begin{explanation}
%      We have ${\rm opp.} = 8$, ${\rm adj.} = 6$ and ${\rm hyp.} = 10$. This means that $$\sin\theta = \frac{8}{10}=\frac{4}{5}, \quad \cos\theta = \frac{6}{10}=\frac{3}{5},\quad \mbox{and}\quad \tan\theta = \frac{8}{10}=\frac{4}{3}.$$
%    \end{explanation}
%    
%  \item \begin{figure}[h]
%  \centering
%  \includegraphics[scale=.3]{/figures/9-1-1-triangle-18-24-30.png}
%\end{figure}
%
%
%        \begin{explanation}
%          We have ${\rm opp.} = 24$, ${\rm adj.} = 18$ and ${\rm hyp.} = 30$. This means that $$\sin\theta = \frac{24}{30}=\frac{4}{5}, \quad \cos\theta = \frac{18}{30}=\frac{3}{5},\quad \mbox{and}\quad \tan\theta = \frac{24}{18}=\frac{4}{3}.$$ Note that the values were the same values as in the previous item. This was expected, as the triangle there is similar to the triangle given here (the scaling factor is $3$).\end{explanation}    
%  \end{enumerate}
%\end{example}
%
%\begin{remark}
%  Note that in all of the above examples, the values of $\sin\theta$ and $\cos\theta$ were always less than $1$. This is always true, and a general consequence of the fact that the hypotenuse is always bigger than either of the other two sides.
%\end{remark}
%
%Often, one has information about the angles, but not about all the sides. Knowing $\sin\theta$, $\cos\theta$ and $\tan\theta$ helps us find out missing sides of a given right triangle. For that, the following fact is extremely important:
%
%\begin{callout}
%  {\bf Theorem (Fundamental Identity):} For any given angle $\theta$, we have that $$\sin^2\theta + \cos^2\theta =1.$$Here, $\sin^2\theta$ means $(\sin\theta)^2$, and similarly for $\cos^2\theta$.
%\end{callout}
%
%Why is this true? Consider again a right triangle like below:
%
%\begin{figure}[h]
%  \centering
%  \includegraphics[scale=.3]{/figures/9-1-1-identity.png}
%\end{figure}
%
%
%Then we know that $\sin\theta = b/c$ and $\cos\theta = a/c$. But the Pythagorean theorem also says that $a^2+b^2=c^2$. Putting all of this together, we have that $$\sin^2\theta + \cos^2\theta = \left(\frac{b}{c}\right)^2 + \left(\frac{a}{c}\right)^2 = \frac{b^2}{c^2}+\frac{a^2}{b^2} = \frac{a^2+b^2}{c^2} = \frac{c^2}{c^2} = 1,$$as required.
%
%Let's see how to apply this.
%
%\begin{example}For each of the following triangles, given the value of a trigonometric function at the indicated angle $\theta$, find the lengths of the missing sides.
%  \begin{enumerate}[label=\alph*.]
%  \item Given: $\sin\theta = \sqrt{2}/6$ on
%
%    \begin{figure}[h]
%      \centering
%      \includegraphics[scale=.3]{/figures/9-1-1-triangle-sin-2sqrt2-6.png}
%    \end{figure}
%
%    \begin{explanation}
% From the given information, we know that $$\frac{\sqrt{2}}{6} = \sin\theta = \frac{b}{2\sqrt{2}} \implies b = \frac{\sqrt{2}\times (2\sqrt{2})}{6} = \frac{2}{3}.$$Now we use the Pythagorean theorem: the relation $a^2 +(2/3)^2=(2\sqrt{2})^2$ gives us that $$a^2+\frac{4}{9} = 8 \implies a^2 = 8-\frac{4}{9} = \frac{68}{9} \implies a = \frac{2\sqrt{17}}{3}.$$Alternatively, to find the value of $a$, we can also use the fundamental identity $\sin^2\theta+\cos^2\theta=1$ to find $\cos\theta$ first --- which then yields $a$. Here's how this goes: $$ \left(\frac{\sqrt{2}}{6}\right)^2+\cos^2\theta=1 \implies \cos^2\theta = 1- \frac{2}{36} \implies \cos^2\theta = \frac{34}{36},$$and so $\cos\theta = \sqrt{34}/6$. Thus $$\frac{\sqrt{34}}{6} = \cos\theta = \frac{a}{2\sqrt{2}} \implies a = \frac{2\sqrt{2} \times \sqrt{34}}{6} = \frac{2\sqrt{17}}{3},$$as it should be. This is not something particular to this example: usually there is more than one strategy to solve this sort of problem. Which one is the best? You'll be the judge.
%    \end{explanation}
%    
%  \item Given: $\cos\theta = \sqrt{3}/7$ on
%
%    \begin{figure}[h]
%      \centering
%      \includegraphics[scale=.3]{/figures/9-1-1-triangle-cos-2sqrt46.png}
%    \end{figure}
%
%    \begin{explanation}
%      Since this time we were given $\cos\theta$, but also the opposite side to $\theta$, which does not appear on the expression for $\cos\theta$, we must rely more on the Pythagorean theorem instead. In any case, we know that $$\frac{\sqrt{3}}{7}= \cos\theta=\frac{a}{c}  \implies c = \frac{7a}{\sqrt{3}}. $$Now, the Pythagorean relation reads $a^2 + (2\sqrt{46})^2 = (7a/\sqrt{3})^2$, and so: $$a^2 + 184 = \frac{49a^2}{3} \implies 184 = \frac{49a^2}{3}-a^2$$Continuing to manipulate this, we see that $$184 = \frac{46a^2}{3} \implies a^2 = \frac{184 \times 3}{46} \implies a^2 = 12 \implies a = 2\sqrt{3}.$$It remains to find the value of $c$. So we go back to the beginning and compute $$c = \frac{7a}{\sqrt{3}}\implies c = \frac{7(2\sqrt{3})}{\sqrt{3}} \implies c=14.$$
%    \end{explanation}
%    
%  \end{enumerate}
%\end{example}
%
%
%\section{Values of trig functions for standard angles}
%
%We know that the sum of the inner angles of a triangle is always $180^\circ$. For right triangles, one of the angles is $90^\circ$, which means that the sum of the remaining two angles must also be $90^\circ$. Frequently we encounter triangles whose angles are $30^\circ$, $60^\circ$ and $90^\circ$, and also triangles whose angles are $45^\circ$, $45^\circ$ and $90^\circ$.
%
%[figure]
%
%These triangles have a special type of symmetry, which we'll exploit to find the values of sine, cosine, and tangent, for $30^\circ$, $45^\circ$ and $60^\circ$. Finding the values of these trig functions for arbitrary angles, by hand, is a very difficult task. We will see later some trigonometric identities that may help us find such values for other angles but, in general, using a calculator (paying close attention to whether it is set to right ``units'') is the way to go.
%
%\subsection{For $30^\circ$ and $60^\circ$}
%
%Consider an equilateral triangle of side length $\ell$. Equilateral means that all the sides have the same length. This implies that all the inner angles must be equal and, since they must add up to $180^\circ$, each of them equals $60^\circ$. But also draw a height $h$:
%
%\begin{figure}[h]
%  \centering
%  \includegraphics[scale=.3]{/figures/9-1-1-triangle-30.png}
%\end{figure}
%
%By the Pythagorean theorem, we know that $$\ell^2 = \left(\frac{\ell}{2}\right)^2 + h^2,$$and so we may compute: $$\ell^2 = \frac{\ell^2}{4} + h^2 \implies \frac{3\ell^2}{4} = h^2 \implies h = \frac{\ell\sqrt{3}}{2}.$$
%
%Now, relative to the $60^\circ$ angle, we recognize $${\rm opp.} = h = \frac{\ell\sqrt{3}}{2}, \quad {\rm adj.} = \frac{\ell}{2}, \quad\mbox{and}\quad {\rm hyp.} = \ell.$$
%
%This means that $$\sin(60^\circ) = \frac{h}{\ell} = \frac{\left(\frac{\ell\sqrt{3}}{2}\right)}{\ell} = \frac{\sqrt{3}}{2},$$as well as $$\cos(60^\circ) = \frac{\ell/2}{\ell} = \frac{1}{2}\quad\mbox{and}\quad \tan(60^\circ) = \frac{\sin(60^\circ)}{\cos(60^\circ)} = \frac{\sqrt{3}/2}{1/2} = \sqrt{3}.$$
%
%To find the values of $\sin(30^\circ)$, $\cos(30^\circ)$, and $\tan(30^\circ)$, we can use the same triangle, noting that the opposite side to $30^\circ$ is the adjacent side to $60^\circ$, and that the adjacent side to $30^\circ$ is the opposite side to $60^\circ$. Since the hypotenuse is always the side opposite to the right angle, we conclude that $$\sin(30^\circ) = \cos(60^\circ) = \frac{1}{2}, \quad \cos(30^\circ) = \sin(60^\circ) = \frac{\sqrt{3}}{2}$$and, finally, that $$\tan(30^\circ) = \frac{\sin(30^\circ)}{\cos(30^\circ)} = \frac{1/2}{\sqrt{3}/2} = \frac{1}{\sqrt{3}} = \frac{\sqrt{3}}{3}.$$
%
%\begin{remark}
%  This is a general phenomenon: two acute angles are called {\bf complementary} if they add up to $90^\circ$. In other words, the complementary angle to $\theta$ is always $90^\circ - \theta$, and $\sin\theta = \cos(90^\circ - \theta)$, as well as $\cos\theta = \sin(90^\circ - \theta)$. In particular, this justifies the name ``cosine'': it is the sine of the complement. We will discuss ``coterminal angles'' and ``cofunctions'' in more generality later.
%\end{remark}
%
%\subsection{For $45^\circ$}
%
%Consider a square of side length $\ell$, and draw a diagonal $d$.
%
%\begin{figure}[h]
%  \centering
%  \includegraphics[scale=.3]{/figures/9-1-1-triangle-45.png}
%\end{figure}
%
%By the Pythagorean theorem, $d^2 = \ell^2+\ell^2 = 2\ell^2$ implies that $d=\ell\sqrt{2}$. Relative to either of the $45^\circ$ angles, we have $${\rm opp.} = \ell, \quad {\rm adj.} =\ell, \quad\mbox{and}\quad {\rm hyp.}=d=\ell\sqrt{2}.$$
%Hence $$\sin(45^\circ) =\cos(45^\circ) =  \frac{\ell}{\ell\sqrt{2}} = \frac{1}{\sqrt{2}} = \frac{\sqrt{2}}{2} \quad\mbox{and}\quad \tan(45^\circ) = \frac{\sin(45^\circ)}{\cos(45^\circ)} = 1.$$
%
%\begin{remark}
%  It is convenient to write $\sqrt{2}/2$ instead of $1/\sqrt{2}$ (similarly for $\sqrt{3}/3$ versus $1/\sqrt{3}$), even though the latter is mathematically acceptable, because it makes it easier to estimate. Namely, knowing that $\sqrt{2} \approx 1.414$, we know that $\sqrt{2}/2 \approx 0.707$, but when looking at $1/\sqrt{2}$, what does it mean to divide $1$ by $1.414$? This is the general reason why rationalizing fractions is useful.
%\end{remark}
%
%%\typeout{************************************************}
%%\typeout{Summary}
%%\typeout{************************************************}
%
%\section{Standard values}
%
%We can summarize what we have discovered here in a table. Besides our standard angles of $30^\circ$, $45^\circ$, and $60^\circ$, we can also consider $0^\circ$ and $90^\circ$ as extreme cases. Let's do a quick thought experiment to understand this: if a right triangle had an angle of $0^\circ$, this triangle would in fact collapse to a line segment, and we would have ${\rm opp.} = 0$, while ${\rm hyp}. = {\rm adj}.$, suggesting we set $\sin(0^\circ) = 0$ and $\cos(0^\circ) = 1$. Since $0^\circ$ and $90^\circ$ are complementary, we're forced to set $\sin(90^\circ) = 1$ and $\cos(90^\circ) = 0$. But while $$\tan(0^\circ) = \frac{\sin(0^\circ)}{\cos(0^\circ)} = \frac{0}{1} = 0,$$computing $\tan(90^\circ)$ does not make sense, as we would have a division by $\cos(90^\circ) = 0$. We say that $\tan(90^ \circ)$ is {\bf undefined}, or that it {\bf does not exist} (``DNE'' for short, as usual). So, we have:
%
%$$
%\begin{array}{c||ccccc}
% & 0^\circ & 30^\circ & 45^\circ & 60^\circ & 90^\circ \\
%\hline\hline \\[-1em]  
%\sin \theta & 0 & \frac{1}{2} & \frac{\sqrt{2}}{2} & \frac{\sqrt{3}}{2} & 1 \\[1em]
% \cos \theta & 1 & \frac{\sqrt{3}}{2} & \frac{\sqrt{2}}{2} & \frac{1}{2} & 0 \\[1em]
%\tan\theta& 0 & \frac{\sqrt{3}}{3} & 1 & \sqrt{3} & {\rm DNE}
%\end{array}
%$$
%
%Those values should be committed to heart, but it's easier than what it seems. Here's how you can think about it:
%
%\begin{itemize}
%\item No need to memorize values for tangent: if you know $\sin \theta$ and $\cos \theta$, you can just compute $\tan\theta = \sin \theta/\cos\theta$.
%\item No need to memorize the values for cosine: recall that the cosine of an angle is the sine of the complement. So if you know values for sine, you're in business.
%\item How to memorize values for sine? The one thing you should remember here is that the values $0$, $1/2$, $\sqrt{2}/2$, $\sqrt{3}/2$ and $1$ will appear. What is their order? Simple: write them in increasing order, just like the angles from $0^\circ$ to $90^\circ$. So $$\sin(0^\circ) = 0, ~ \sin(30^\circ) = \frac{1}{2}, ~ \sin(45^\circ) = \frac{\sqrt{2}}{2}, ~ \sin(60^\circ) = \frac{\sqrt{3}}{2}, ~ \sin(90^\circ)=1.$$
%\end{itemize}
%
%\begin{summary}\begin{itemize}
%\item We have defined sine, cosine, and tangent, as ratios between sides of a right triangle. For each angle $\theta$, the fundamental identity $\sin^2\theta+\cos^2\theta=1$ holds. It can be used together with the Pythagorean Theorem to get information about all sides of a given triangle, when some of them might be missing, provided you have some information about the angles.
%\item We have established the standard values of sine, cosine, and tangent, for the most frequent angles of $30^\circ$, $45^\circ$, and $60^\circ$. Those values have been organized in a table. They are so frequent that knowing the values there by heart is useful, but exaggerated efforts into memorizing the table should not be wasted --- understanding how the values are deduced pays off more in the long run.
%\end{itemize}\end{summary}
%



\end{document}
