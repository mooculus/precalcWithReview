\documentclass{ximera}

\input{../../preamble.tex}

\author{Kenneth Berglund}
\acknowledgement{}

\begin{document}
\begin{exercise}
The function $f$ defined by  $f(x) = |x| + 3$ has $\answer{0}$ zero(s).
\end{exercise}

\begin{exercise}
The function $f$ defined by  $f(x) = |x| - 3$ has $\answer{2}$ zero(s).
\begin{exercise}
The zeros of $f$ are (from least to greatest) $\answer{-3}$ and $\answer{3}$. 
\end{exercise}
\end{exercise}

\begin{exercise}
The function $f$ defined by  $f(x) = |x - 3|$ has $\answer{1}$ zero(s).
\begin{exercise}
The zero of $f$ is $\answer{3}$. 
\end{exercise}
\end{exercise}

\begin{exercise}
The function $f$ defined by  $f(x) = |x - 3| - 3$ has $\answer{2}$ zero(s).
\begin{exercise}
The zeros of $f$ are (from least to greatest) $\answer{0}$ and $\answer{6}$. 
\end{exercise}
\end{exercise}

\begin{exercise}
The function $f$ defined by  $f(x) = |3x - 4|$ has $\answer{1}$ zero(s).
\begin{exercise}
The zero of $f$ is $\answer{\frac{4}{3}}$. 
\end{exercise}
\end{exercise}

\begin{exercise}
The function $f$ defined by  $f(x) = 2|3x + 1| - 4$ has $\answer{2}$ zero(s).
\begin{exercise}
The zeros of $f$ are (from least to greatest) $\answer{-1}$ and $\answer{\frac{1}{3}}$. 
\end{exercise}
\end{exercise}

\begin{exercise}
The function $f$ defined by  $f(x) = 2|3x + 1| + 4$ has $\answer{0}$ zero(s).
\end{exercise}
\end{document}