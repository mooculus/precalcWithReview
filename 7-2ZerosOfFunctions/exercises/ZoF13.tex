\documentclass{ximera}

\input{../../preamble.tex}

\author{Kenneth Berglund}
\acknowledgement{}

\begin{document}
\begin{exercise}
The function $f$ defined by  $f(x) = 2^{x - 3}$ has $\answer{0}$ zero(s).
\end{exercise}

\begin{exercise}
The function $f$ defined by  $f(x) = 2^x - 3$ has $\answer{1}$ zero(s).
\begin{exercise}
The zero of $f$ is $\log_2(\answer{3})$ 
\end{exercise}
\end{exercise}

\begin{exercise}
The function $f$ defined by  $f(x) = 2^{x - 3} - 3$ has $\answer{1}$ zero(s).
\begin{exercise}
The zero of $f$ is $\log_2(\answer{3}) + \answer{3}$. 
\end{exercise}
\end{exercise}

\begin{exercise}
The function $f$ defined by  $f(x) = e^{x^2 - 4} - 2$ has $\answer{2}$ zero(s).
\begin{exercise}
The zeros of $f$ are (from least to greatest) $\answer{-\sqrt{\ln(2)} + 4}$ and $\answer{\sqrt{\ln(2)} + 4}$. 
\end{exercise}
\end{exercise}

\begin{exercise}
The function $f$ defined by  $f(x) = e^{\log_2(x)} - 3$ has $\answer{1}$ zero(s).
\begin{exercise}
The zero of $f$ is $\answer{2^{\ln(3)}}$. 
\end{exercise}
\end{exercise}

\begin{exercise}
The function $f$ defined by  $f(x) = 7e^{x + 5} - 1$ has $\answer{1}$ zero(s).
\begin{exercise}
The zero of $f$ is $\answer{\ln(1/7) - 5}$. 
\end{exercise}
\end{exercise}
\end{document}