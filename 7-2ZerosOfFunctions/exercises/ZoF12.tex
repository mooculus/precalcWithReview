\documentclass{ximera}

\input{../../preamble.tex}

\author{Kenneth Berglund}
\acknowledgement{}

\begin{document}
\begin{exercise}
The function $f$ defined by  $f(x) = \ln(x + 1)$ has $\answer{1}$ zero(s).
\begin{exercise}
The zero of $f$ is $\answer{0}$. 
\end{exercise}
\end{exercise}

\begin{exercise}
The function $f$ defined by  $f(x) = \ln(-2x + 5)$ has $\answer{1}$ zero(s).
\begin{exercise}
The zero of $f$ is $\answer{2}$. 
\end{exercise}
\end{exercise}

\begin{exercise}
The function $f$ defined by  $f(x) = \ln(x^2 - 1)$ has $\answer{2}$ zero(s).
\begin{exercise}
The zeros of $f$ are (from least to greatest) $\answer{-\sqrt{2}}$ and $\answer{\sqrt{2}}$. 
\end{exercise}
\end{exercise}

\begin{exercise}
The function $f$ defined by  $f(x) = \ln(2\ln(x))$ has $\answer{1}$ zero(s).
\begin{exercise}
The zero of $f$ is $\answer{\sqrt{e}}$. 
\end{exercise}
\end{exercise}

\begin{exercise}
The function $f$ defined by  $f(x) = \ln(\ln(\ln(x)))$ has $\answer{1}$ zero(s).
\begin{exercise}
The zero of $f$ is $\answer{e^e}$. 
\end{exercise}
\end{exercise}

\begin{exercise}
The function $f$ defined by  $f(x) = 2\ln(x - 1) + 5$ has $\answer{1}$ zero(s).
\begin{exercise}
The zero of $f$ is $\answer{e^{-5/2} + 1}$. 
\end{exercise}
\end{exercise}

\end{document}
