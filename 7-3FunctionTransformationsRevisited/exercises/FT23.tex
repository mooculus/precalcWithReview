\documentclass{ximera}

\input{../../preamble.tex}

\author{Kenneth Berglund}
\license{Creative Commons Attribution-ShareAlike 4.0 International License}


\begin{document}

In this problem, we'll step through an analysis of the function $f$ defined by $f(x) = 2 \ast e^{\frac{x}{2} - 3} - 3$. The parent function of $f$ is defined by $f_0(x) = e^x$. We will follow the point $(1, e)$ on the graph of $e^x$. 
\begin{exercise}
What is the first transformation applied to $f_0$? 
\begin{multipleChoice}
\choice{Vertical stretch by a factor of 2}
\choice{Horizontal stretch by a factor of 2}
\choice[correct]{Shift right by 3 units}
\choice{Shift down by 3 units}
\end{multipleChoice}

\begin{exercise}
The point $(1, e)$ on the graph of $f_0$ transforms into $(\answer{4}, \answer{e})$ on the graph of $f_1(x) = f_0(x - 3)$. 

\begin{exercise}
What is the next transformation, applied to $f_1$?
\begin{multipleChoice}
\choice{Vertical stretch by a factor of 2}
\choice[correct]{Horizontal stretch by a factor of 2}
\choice{Shift down by 3 units}
\end{multipleChoice}

\begin{exercise}
The point $(4, e)$ on the graph of $f_1$ transforms into $(\answer{8}, \answer{e})$ on the graph of $f_2(x) = f_1\left(\frac{x}{2}\right)$. 

\begin{exercise}
What is the next transformation, applied to $f_2$?
\begin{multipleChoice}
\choice[correct]{Vertical stretch by a factor of 2}
\choice{Shift down by 3 units}
\end{multipleChoice}

\begin{exercise}
The point $(8, e)$ on the graph of $f_2$ transforms into $(\answer{8}, \answer{2e})$ on the graph of $f_3(x) = 2f_2\left(x\right)$. 

\begin{exercise}
The last transformation, applied to $f_3$, is a shift down by 3 units. The point $(8, 2e)$ on the graph of $f_3$ transforms into $(\answer{8}, \answer{2e - 3})$ on the graph of $f_4(x) = f_3(x) - 3$. 

\begin{exercise}
Consider the following graphs.

\includegraphics[width=1\linewidth]{FT23graph.png}

Which is the graph of $f$?
\begin{multipleChoice}
\choice[correct]{Red graph}
\choice{Blue graph}
\choice{Orange graph}

\end{multipleChoice}


\end{exercise}
\end{exercise}
\end{exercise}
\end{exercise}
\end{exercise}
\end{exercise}
\end{exercise}
\end{exercise}
\end{document}
