\documentclass{ximera}

\input{../../preamble.tex}

\author{David Kish}
\license{Creative Commons Attribution-ShareAlike 4.0 International License}

%\outcome{Calculating the rate of change.}
%\outcome{Discuss the meaning of antiderivatives of a position function.}

\begin{document}
\begin{exercise}
Given a function $f$, which of the following represents a vertical translation of 2 units upward, followed by a reflection across the y-axis?

\begin{multipleChoice}  
\choice[correct]{$y=f(-x)+2$}  
\choice{$y = f(2-x)$}  
\choice{$y=f(x)-2$}  
\choice{$y=2-f(x)$}
\choice{$y=-f(x-2)$}
\end{multipleChoice}  

\begin{exercise}
If the point $(-3, 7)$ is on the graph of $f$, the point $(\answer{3}, \answer{9})$ is on the graph of the transformed function.
\end{exercise}
\end{exercise}
\end{document}