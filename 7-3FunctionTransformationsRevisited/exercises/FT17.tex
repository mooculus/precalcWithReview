\documentclass{ximera}

\input{../../preamble.tex}

\author{Kenneth Berglund}
\license{Creative Commons Attribution-ShareAlike 4.0 International License}


\begin{document}
\begin{exercise}
Consider the function $f$ defined by $f(x) = 3(-x + 2)^2 - 5$.

Its parent function is $g(x) = \answer{x^2}$. 

\begin{exercise}
Select all the transformations that are applied to the graph of $g$ to obtain the graph of $f$.
\begin{selectAll}
\choice{vertical shift up}
\choice[correct]{vertical shift down}
\choice[correct]{vertical stretch}
\choice{vertical compression}
\choice[correct]{horizontal shift left}
\choice{horizontal shift right}
\choice{horizontal stretch}
\choice{horizontal compression}
\choice{reflection across $x$-axis}
\choice[correct]{reflection across $y$-axis}
\end{selectAll}

\begin{exercise}
The vertical shift is down by $\answer{5}$ units.

The vertical stretch is by a factor of $\answer{3}$.

The horizontal shift is left by $\answer{2}$ units.

\begin{exercise}
To produce the graph of $f$ starting with the graph of the parent function, in which order should you perform the following steps? Enter the numbers 1, 2, 3, and 4, accordingly. Use the order given in the textbook section.
%how to deal with orders that could be different
\begin{itemize}
\item Vertical shift down 5 units. $\answer{4}$
\item Vertical stretch by a factor of 3. $\answer{2}$
\item Horizontal shift left 2 units. $\answer{1}$
\item Reflection across $y$-axis. $\answer{3}$
\end{itemize}
Which of the following graphs is the graph of $f(x)$?

\includegraphics[width=1\linewidth]{FT17graph.png}

\begin{multipleChoice}
\choice[correct]{Red graph}
\choice{Blue graph}
\choice{Orange graph}
\choice{Black graph}
\end{multipleChoice}

\end{exercise}
\end{exercise}
\end{exercise}
\end{exercise}
\end{document}
