\documentclass{ximera}

\input{../preamble}
\author{Ivo Terek and Elizabeth Campolongo}
\license{Creative Commons Attribution-ShareAlike 4.0 International License}
\acknowledgement{}

\title{Average Rate of Change and Secant Lines}

\begin{document}
\begin{abstract}
\end{abstract}
\maketitle
%
%
%
\begin{motivatingQuestions}\begin{itemize}
\item What does a line passing through two points of a function represent?
\item How does this inform our understanding of the function?
\end{itemize}\end{motivatingQuestions}
%
%
%\typeout{************************************************}
%\typeout{Subsection Introduction}
%\typeout{************************************************}
%
\section{Introduction}
%
We begin by recalling the definitions of \textit{average rate of change} of a function and {\it secant line} to the graph of a function. 
\begin{definition}
For a function $f$ defined on an interval $[a,b]$, 
\begin{itemize}
\item the \dfn{average rate of change of $f$ on $[a,b]$} is the quantity%
\begin{equation*}
AV_{[a,b]} = \frac{f(b) - f(a)}{b-a}\text{.}
\end{equation*}

\item a \dfn{secant line} to the graph of $f$ is a line passing through two points $(a,f(a))$ and $(b,f(b))$, with $a \neq b$. \\
\\
%, in the graph of $y=f(x)$.\\
  {\bf Recall:} The slope of a secant line is the average rate of change of the function on the interval $[a,b]$.
\end{itemize}

This is illustrated in the figure below, where the green line (between the red points on the graph) is the secant line of $f$ from $(a,f(a))$ to $(b,f(b))$.

\begin{image}
\includegraphics{aroc-f-x-defn.pdf}
\end{image}

\end{definition}





Recall that given two points $(x_0,y_0)$ and $(x_1,y_1)$ in the plane, with $x_0 \neq x_1$, we can find the equation of the line passing through them by using the slope (``rise-over-run''): 
%
$$m= \frac{y_1-y_0}{x_1-x_0}.$$
%
Then the line equation is given by $y-y_0=m(x-x_0)$, simply because any given point $(x,y)$ in such line must realize the \emph{same} slope: 
%
$$m=\frac{y-y_0}{x-x_0}.$$
%
Of course, one may also use the point $(x_1,y_1)$ instead of $(x_0,y_0)$ and consider the equation $y-y_1=m(x-x_1)$, as it describes the same line.

\begin{image}[3in]
		  \begin{tikzpicture}
		    \coordinate (C) at (-4,1);
		    \coordinate (D) at (5,1);
		    \coordinate (E) at (5,7.5);
		     \tkzMarkRightAngle(C,D,E)
		    \draw[decoration={brace,mirror,raise=.2cm},decorate,thin] (-4,1)--(5,1);
		    \draw[decoration={brace,mirror,raise=.2cm},decorate,thin] (5.6,1.2)--(5.6,7.7);
		  %  \draw[decoration={brace,raise=.2cm},decorate,thin] (0,2)--(4,6.5);
		    \draw[very thick] (C)--(E)--cycle;
		    \draw[dotted] (C) edge (D);
		    \draw[dotted] (E) edge (D);
		    \node at (5.6,1.1) {$(x, y_1)$};
		   % \node[rotate=-90] at (4+.5,4) {A'};
		    \node[rotate=35] at (.8,4.8) {$y-y_1 = m(x-x_1)$};
		    \node at (5.1,7.8) {$(x,y)$};
		    \node at (-4.7, 1) {$(x_1,y_1)$};
		    \node at (6.4,4.35) {$y-y_1$};
		    \node at (.5,.4) {$x-x_1$};
		  \end{tikzpicture}
		\end{image}



With this in place, we'll focus on the situation where two such points lie in the graph of some function $y=f(x)$.

\begin{image}[2in]
		  \begin{tikzpicture}
		  \draw [thick, scale = 3, domain = -.5:2, smooth, variable =\x] plot ({\x},  {3/4*exp(\x)*cos(deg(\x)) + 1/2*exp(\x)*sin(deg(\x))});%{\x*(3-2*\x*\x)});
		    \coordinate (C) at (0,2.8);
		    \coordinate (D) at (5,1);
		    \coordinate (E) at (5,7.5);
		   % \draw[decoration={brace,mirror,raise=.2cm},decorate,thin] (0,2)--(4,2);
		   % \draw[decoration={brace,mirror,raise=.2cm},decorate,thin] (4,2)--(4,6.5);
		  %  \draw[decoration={brace,raise=.2cm},decorate,thin] (0,2)--(4,6.5);
		    \draw[blue] (C)--(E)--cycle;
		 %   \draw[dotted] (C) edge (D);
		%    \draw[dotted] (E) edge (D);
		      \node at (2,7) {$y= f(x)$};
		   \node at (.4,4) {$(a, f(a))$};
		   % \node[rotate=-90] at (4+.5,4) {A'};
		 %   \node[rotate=35] at (.8,4.8) {$f(x)-f(x_1) = m(x-x_1)$};
		    \node at (5.1,7.8) {$(b,f(b))$};
		%    \node at (-4.7, 1) {$(x_1,y_1)$};
		  \end{tikzpicture}
		\end{image}


\section{Definitions and examples}

\begin{callout}
  {\bf Definition:} Consider a function $y=f(x)$. A line passing through two points $(a,f(a))$ and $(b,f(b))$, with $a \neq b$, in the graph of $y=f(x)$, is called a \textbf{secant line} to the graph.\\
  {\bf Recall:} The slope of a secant line is the average rate of change of the function on the interval $[a,b]$.
\end{callout}

\begin{example}
  On the following situations, given a function $y=f(x)$ and two points in the graph, find the equation of the secant line they determine.
  \begin{enumerate}[label=\alph*.]
  \item $f(x) = x^2$, points $(1,f(1))$ and $(2,f(2))$.

    \begin{explanation}
      First, we have that $f(1) = 1^2=1$ and $f(2) = 2^2=4$, so the points given are actually $(1,1)$ and $(2,4)$. So $$m=\frac{4-1}{2-1}=  3$$means that the line equation we're looking for is $y-1=3(x-1)$, which may be rewritten simply as $y=3x-2$.

\begin{image}[2in]
		  \begin{tikzpicture}
		  \draw [thick, scale = 1, domain = -2:2.5, smooth, variable =\x] plot ({\x},  {\x*\x});%{\x*(3-2*\x*\x)});
		   \draw [blue, thick, scale = 1, domain = 1:2, smooth, variable =\x] plot ({\x},  {3*\x-2});
		 %   \draw[dotted] (C) edge (D);
		%    \draw[dotted] (E) edge (D);
		      \node at (4,5.6) {\large $y= x^2$};
		   \node [blue] at (.4,2) {$y = 3x-2$};
		   % \node[rotate=-90] at (4+.5,4) {A'};
		%    \node at (-4.7, 1) {$(x_1,y_1)$};
		  \end{tikzpicture}
		\end{image}

      
    \end{explanation}
  \item $f(x) = \sin x$, points $(\pi/4, f(\pi/4))$ and $(3\pi/4, f(3\pi/4))$.

    \begin{explanation}
      This time, we have that $f(\pi/4) = \sin(\pi/4) = \sqrt{2}/2$ and also that $f(3\pi/4) = \sin(3\pi/4) = \sqrt{2}/2$, so the points given were in fact $(\pi/4,\sqrt{2}/2)$ and $(3\pi/4,\sqrt{2}/2)$. Hence the slope of the secant line is $$m=\frac{\sqrt{2}/2 - \sqrt{2}/2}{3\pi/4 - \pi/4} = \frac{0}{\pi/2} = 0,$$so the line equation $y-\sqrt{2}/2 = 0(x-\pi/4)$ boils down to $y = \sqrt{2}/2$. Again, you should see $y=\sqrt{2}/2$ not as a single value of $y$, but as a line equation for which it just happens that $x$ does not appear --- thus describing a horizontal line.

\begin{image}[2in]
		  \begin{tikzpicture}
		  \draw [thick, scale = 2, domain = 0:6, smooth, variable = \x] plot ({\x}, {sin(deg(\x))});
		    \coordinate (C) at (.4,.8);
		    \coordinate (D) at (5,1);
		    \coordinate (E) at (6,.8);
		    \draw[blue, thick] (C)--(E)--cycle;
	            \node at (9.4,.6) {\LARGE $y= \sin(x)$};
		   \node at (0,1.5) {\large $y = \frac{\sqrt{2}}{2}$};
		   % \node[rotate=-90] at (4+.5,4) {A'};
		 %   \node[rotate=35] at (.8,4.8) {$f(x)-f(x_1) = m(x-x_1)$};
		 %   \node at (5.1,7.8) {$(a+h,f(a+h))$};
		%    \node at (-4.7, 1) {$(x_1,y_1)$};
		  \end{tikzpicture}
		\end{image}

    \end{explanation}
  \item $f(x)=2x+3$, points $(-1,f(-1))$ and $(3,f(3))$.

    \begin{explanation}
      Now, we have $f(-1) = 2(-1)+3=1$ and $f(3) = 2\cdot 3+3=9$, so the points given were $(-1,1)$ and $(3,9)$. The slope these points determine is $$m = \frac{9-1}{3-(-1)} = \frac{8}{4} = 2,$$so we obtain $y-1=2(x-(-1))$, which can be rewritten as $y=2x+3$. This is not a coincidence! The secant line to the graph of a line must be the line itself. This is because a line is determined by two points, and since both the original line and the secant line must share the two given points, they must be, in fact, equal.
      
%      \begin{image}[1in]
%		  \begin{tikzpicture}
%		  \draw [dotted, scale = .5, domain = -1.5:3.5, smooth, variable =\x] plot ({\x},  {2*\x+3});%{\x*(3-2*\x*\x)});
%		   \draw [blue, dotted, scale = .5, domain = -1.5:3.5, smooth, variable =\x] plot ({\x},  {2*\x+3});
%		 %   \draw[dotted] (C) edge (D);
%		%    \draw[dotted] (E) edge (D);
%		%      \node at (4,5.6) {\large $y= x^2$};
%		%   \node [blue] at (.4,2) {$y = 3x-2$};
%		   % \node[rotate=-90] at (4+.5,4) {A'};
%		%    \node at (-4.7, 1) {$(x_1,y_1)$};
%		  \end{tikzpicture}
%		\end{image}
%
%
%
%      [FIGURE, MIX COLORS?]
    \end{explanation}
  \end{enumerate}
\end{example}


\end{document}
