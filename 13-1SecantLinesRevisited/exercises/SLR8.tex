\documentclass{ximera}

\input{../../preamble.tex}

\author{Elizabeth Campolongo}
\acknowledgement{https://www.stitz-zeager.com/szca07042013.pdf}

\begin{document}
\begin{exercise}

For the following functions, 
%compute the slope of the secant line to the function from $x$ to $x+h$. Assume that the interval $[x,x+h]$ is in the domain of the function. 
find and simplify the difference quotients. [Hint: Remember our tricks for simplification, like multiplying by the conjugate, given in 12-3-3 Algebra of Secant lines.]

 $$k(x)= \sqrt{8x}$$.

\begin{enumerate}

\item $\av_{[4,1]} = \answer{\frac{2\sqrt{2}}{5}}$.

\item $\av_{[x,x+5]} =  \answer{\frac{2\sqrt{2}}{\sqrt{x+5} + \sqrt{x}}}$.

\item $\av_{[x,x+h]} = \frac{\answer{k(x+h)}-k(x)}{\answer{h}}   = \answer{\frac{2\sqrt{2}}{\sqrt{x+h} + \sqrt{x}}}$.  

\item What is the equation of the secant line to $k$ (in point-slope form) when
\begin{enumerate}
\item $x = 4$ and $h=5$? \\
$y-6\sqrt{2} = \answer{\frac{2\sqrt{2}}{5}}(x-\answer{9})$

\item $x = 4$ and $h = -3$? \\
$y - \answer{2\sqrt{2}} = \answer{\frac{2\sqrt{2}}{5}}(x-1)$

\item $x = 2$ and $h= 1$? \\
$y - \answer{4} = \answer{\frac{2\sqrt{2}}{\sqrt{3} + \sqrt{2}}}(x-2)$.
\end{enumerate}

	
\end{enumerate}
\end{exercise}
\end{document}