\documentclass{ximera}

\input{../../preamble.tex}

\author{Elizabeth Campolongo}
\acknowledgement{https://www.stitz-zeager.com/szca07042013.pdf}

\begin{document}
\begin{exercise}

For the following functions, 
%compute the slope of the secant line to the function from $x$ to $x+h$. Assume that the interval $[x,x+h]$ is in the domain of the function. 
find and simplify the difference quotients. [Hint: Remember our tricks for simplification, like multiplying by the conjugate, given in 12-3-3 Algebra of Secant lines.]

\begin{enumerate}
\item $f(x)= x^3$.

\begin{exercise}
$AV_{[1,4]} = \answer{21}$.

\begin{exercise}
$AV_{[x,x+2]} =  \answer{3x^2+6x+4}$.

\begin{exercise}
$AV_{[x,x+h]} = \frac{\answer{f(x+h)}-f(x)}{\answer{h}} = \answer{3x^2+3xh+h^2}$.  
\begin{exercise}
What is the equation of the secant line to $f$ (in point-slope form) when: 
\begin{enumerate}
\item  $x = 1$ and $h=3$?\\
$y-64= \answer{21}(x-\answer{4})$

\item $x = 2$ and $h=2$? \\
$y-8= \answer{28}(x-\answer{2})$

\item  $x = 1$ and $h=1$?\\
$y-1= \answer{7}(x-\answer{1})$

\item $x=-1$ and $h = \frac{1}{2}$? \\
$y\answer{+1}= \answer{\frac{19}{4}}(x+1)$
\end{enumerate}
\end{exercise}
\end{exercise}
\end{exercise}
\end{exercise}

\item $g(x)= 2x^2 -3x+1$.
\begin{exercise}
$AV_{[-1,1]} = \answer{-3}$.

\begin{exercise}
$AV_{[x,x+4]} =  \answer{4x+8-3}$.

\begin{exercise}
$AV_{[x,x+h]} = \frac{\answer{g(x+h)}-g(x)}{\answer{h}}  = \answer{4x + 2h - 3}$.  
\begin{exercise}
What is the equation of the secant line to $g$ (in point-slope form) when 
\begin{enumerate}
\item $x=-1$ and $h = 2$? \\
$y-6= \answer{-3}(x +\answer{1}).$

\item $x=-2$ and $h = 4$? \\
$y - 15 = \answer{-3}(x \answer{+2}).$

\item $x=\frac{1}{4}$ and $h = \frac{1}{3}$?\\
$y -\answer{\frac{3}{8}} = \answer{-\frac{4}{3}}(x-\frac{1}{4})$.
\end{enumerate}
\end{exercise}
\end{exercise}
\end{exercise}
\end{exercise}


\item $k(x)= \sqrt{8x}$.
\begin{exercise}
$AV_{[4,1]} = \answer{\frac{2\sqrt{2}}{5}}$.

\begin{exercise}
$AV_{[x,x+5]} =  \answer{\frac{2\sqrt{2}}{\sqrt{x+5} + \sqrt{x}}}$.

\begin{exercise}
$AV_{[x,x+h]} = \frac{\answer{k(x+h)}-k(x)}{\answer{h}}   = \answer{\frac{2\sqrt{2}}{\sqrt{x+h} + \sqrt{x}}}$.  
\begin{exercise}
What is the equation of the secant line to $k$ (in point-slope form) when
\begin{enumerate}
\item $x = 4$ and $h=5$? \\
$y-6\sqrt{2} = \answer{\frac{2\sqrt{2}}{5}}(x-\answer{9})$

\item $x = 4$ and $h = -3$? \\
$y - \answer{2\sqrt{2}} = \answer{\frac{2\sqrt{2}}{5}}(x-1)$

\item $x = 2$ and $h= 1$? \\
$y - \answer{4} = \answer{\frac{2\sqrt{2}}{\sqrt{3} + \sqrt{2}}}(x-2)$.
\end{enumerate}
\end{exercise}
\end{exercise}
\end{exercise}
\end{exercise}

\item $f(x)= \Big|\frac{2}{3}x-4\Big|$.
\begin{enumerate}
\item Simplify completely:
$AV_{[-4,9]} = \answer{-\frac{14}{39}}$
\item Observe that $f(x)$ is $\wordChoice{\choice[correct]{\text{always}}\choice{\text{sometimes}}\choice{\text{never}}}$ greater than or equal to zero.
\item Additionally,
\begin{multipleChoice}
\choice{$\frac{2}{3}x-4>0$ when $x<6$, so $\frac{2}{3}x-4<0$ when $x>6$.}
\choice{$\frac{2}{3}x-4>0$ when $x<-6$, so $\frac{2}{3}x-4<0$ when $x>-6$.}
\choice{$\frac{2}{3}x-4>0$ when $x>-6$, so $\frac{2}{3}x-4<0$ when $x<-6$.}
\choice[correct]{$\frac{2}{3}x-4>0$ when $x>6$, so $\frac{2}{3}x-4<0$ when $x<6$.}
\end{multipleChoice}

\end{enumerate}
\begin{exercise}
\begin{enumerate}
\item When $x<6$ and $h<0$,\\
$\frac{f(x+h)-f(x)}{h} = \answer{-\frac{2}{3}}$.  

\item When $x>6$ and $h>0$,\\
$\frac{f(x+h)-f(x)}{h} = \answer{\frac{2}{3}}$.  
\end{enumerate}
\begin{exercise}
What is the equation of the secant line to $f$ when $x = 2$ and $h=-1$ (in point-slope form)?\\
$y\answer{-\frac{8}{3}}= \answer{-\frac{2}{3}}(x-2)$
\end{exercise}
\end{exercise}


\item $f(x)= \frac{x-1}{x+2}$.

$\frac{f(x+h)-f(x)}{h} = \answer{\frac{3}{(x+2)(x+h+2)}}$.  
\begin{exercise}
What is the equation of the secant line to $f$ when $x = 0$ and $h=1$ (in point-slope form)?\\
$y+ \answer{1} = \answer{\frac{1}{2}}x$
\end{exercise}
	
\end{enumerate}
\end{exercise}
\end{document}