\documentclass{ximera}

\input{../../preamble.tex}

\author{Elizabeth Campolongo}

\begin{document}
\begin{exercise}
You have a bountiful garden full of tomatoes and peppers. To prepare for a repeat next year, you want to get a better idea of how long it took for them to grow to each stage--constructing sturdy tomato trellises is no joke!

You planted your first tomato plants in the garden when they were about 4 inches tall. 30 days later, they were two feet tall. At this point they should have a trellis and begin to grow out. Over the next 30 days before harvest time they grew out another 18 inches. 

Let $G(t)$ be a function that gives the growth of the tomato plant at time $t$ from planting. So, $G(0) = \answer{4}$ inches, $G(30) = \answer{24}$ inches, and $G(60) = \answer{42}$ inches.
%
\begin{exercise}
We want to find the average daily rate of growth of the tomato plants over each of the two stages.
\begin{enumerate}
\item What was the average rate of growth over the first 30 days after planting? \\
$AV_{[0,\answer{30}} = \frac{\answer{G(30)} - G(0)}{\answer{30}} = \answer{\frac{2}{3}}$ inches per day.

\item What was the average daily growth in the second 30 days after planting?\\
$AV_{[\answer{30},60]} = \frac{G(60)- \answer{G(30)}}{\answer{30}} = \answer{\frac{3}{5}}$ inches per day$.

\item Overall, the average daily growth of your tomato plants was \\
$AV_{[0,60]} = \frac{G(\answer{60})- G(\answer{0})}{\answer{60}} = \answer{\frac{19}{30}}$ inches per day$.

%First observe that 
%\begin{multipleChoice}
%\choice{$f(x)>0$ when $x<6$, so $f(x)<0$ when $x>6$.}
%\choice{$f(x)>0$ when $x<-6$, so $f(x)<0$ when $x>-6$.}
%\choice{$f(x)>0$ when $x>-6$, so $f(x)<0$ when $x<-6$.}
%\choice[correct]{$f(x)>0$ when $x>6$, so $f(x)<0$ when $x<6$.}
%\end{multipleChoice}

\end{enumerate}
\end{exercise}
\end{exercise}
\end{document}
