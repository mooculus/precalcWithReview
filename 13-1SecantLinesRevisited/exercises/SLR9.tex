\documentclass{ximera}

\input{../../preamble.tex}

\author{Elizabeth Campolongo}
\acknowledgement{https://www.stitz-zeager.com/szca07042013.pdf}

\begin{document}
\begin{exercise}

For the following functions, 
%compute the slope of the secant line to the function from $x$ to $x+h$. Assume that the interval $[x,x+h]$ is in the domain of the function. 
find and simplify the difference quotients. [Hint: Remember our tricks for simplification, like multiplying by the conjugate, given in 12-3-3 Algebra of Secant lines.]

$$f(x)= \Big|\frac{2}{3}x-4\Big|$$.

\begin{enumerate}

\item Simplify completely:
$\av_{[-4,9]} = \answer{-\frac{14}{39}}$
\item Observe that $f(x)$ is \wordChoice{\choice[correct]{\text{always}}\choice{\text{sometimes}}\choice{\text{never}}} greater than or equal to zero.
\item Additionally,
\begin{multipleChoice}
\choice{$\frac{2}{3}x-4>0$ when $x<6$, so $\frac{2}{3}x-4<0$ when $x>6$.}
\choice{$\frac{2}{3}x-4>0$ when $x<-6$, so $\frac{2}{3}x-4<0$ when $x>-6$.}
\choice{$\frac{2}{3}x-4>0$ when $x>-6$, so $\frac{2}{3}x-4<0$ when $x<-6$.}
\choice[correct]{$\frac{2}{3}x-4>0$ when $x>6$, so $\frac{2}{3}x-4<0$ when $x<6$.}
\end{multipleChoice}

\item When $x<6$ and $h<0$,\\
$\frac{f(x+h)-f(x)}{h} = \answer{-\frac{2}{3}}$.  

\item When $x>6$ and $h>0$,\\
$\frac{f(x+h)-f(x)}{h} = \answer{\frac{2}{3}}$.  

\item What is the equation of the secant line to $f$ when $x = 2$ and $h=-1$ (in point-slope form)?\\
$y\answer{-\frac{8}{3}}= \answer{-\frac{2}{3}}(x-2)$

\end{enumerate}
\end{exercise}
\end{document}