\documentclass[nooutcomes]{ximera}

\input{../preamble}
\author{Kenneth Berglund \& Ivo Terek}
\license{Creative Commons Attribution-ShareAlike 4.0 International License}
\acknowledgement{}


\title{Review of Inverse Functions}

\begin{document}
\licenseSZ
\begin{abstract}
  
\end{abstract}
\maketitle


%\typeout{************************************************}
%\typeout{Motivating Questions}
%\typeout{************************************************}

%\begin{motivatingQuestions}\begin{itemize}
%\item What are inverse functions?
%\end{itemize}\end{motivatingQuestions}


%\typeout{************************************************}
%\typeout{Introduction}
%\typeout{************************************************}
In Section 3-2-2, we briefly introduced the concept of \emph{inverse functions}. Recall that for a one-to-one function $f$, we can define the inverse function $f^{-1}$. If we think of $f$ as a process that takes some input $x$ and produces some output $f(x)$, then providing $f(x)$ as an input to $f^{-1}$ produces the original input $x$, and vice versa. Symbolically, we wrote that $f^{-1}(f(x)) = x$ and $f(f^{-1}(x)) = x$. 

We learned several important principles, which we summarize below.
\begin{itemize}
\item
A function $f$ has an inverse function if and only if there exists a function $g$ that undoes the work of $f$: that is, there is some function $g$ for which $g(f(x)) = x$ for each $x$ in the domain of $f$, and $f(g(y)) = y$ for each $y$ in the range of $f$. We call $g$ the inverse of $f$, and write $g = f^{-1}$.%
\item
A function $f$ has an inverse function if and only if the graph of $f$ passes the {\it Horizontal Line Test}.
\item
A function $f$ has an inverse function if and only if $f$ is a {\it one-to-one} function.
\item
When $f$ has an inverse, we know that writing ``$y = f(t)$'' and ``$t = f^{-1}(y)$''  are two different perspectives on the same statement.
\item If $(a, f(a))$ is a point on the graph of $f$, then $(f(a), a)$ is a point on the graph of $f^{-1}$. 
\item The graph of $f^{-1}$ is the graph of $f$ reflected across the line $y = x$.
\item The domain of $f$ is the range of $f^{-1}$ and the range of $f$ is the domain of $f^{-1}$.
\item If $f^{-1}$ is the inverse of $f$, then $f$ is the inverse of $f^{-1}$.

\end{itemize}

In this section, we'll explore inverse functions more in-depth. In particular, by practicing further how to algebraically determine whether the inverse for a given function $f$ exists, and how to find it.

\begin{example}
  Let $f$ be the function given by $f(x) = 4x^3+1$, for every real number $x$. Is $f$ one-to-one? If so, what is the inverse $f^{-1}$? What is its domain?

  \begin{explanation}
    Recall here that $f$ is one-to-one if $f(x_1) = f(x_2)$ always implies that $x_1=x_2$. What happens in this case? We start with the equality $$4x_1^3+1 = 4x_2^3+1,$$ and if we can arrive at $x_1=x_2$, then our function $f$ is one-to-one. Subtracting $1$ from both sides gives $4x_1^3=4x_2^3$. Dividing both sides by $4$, we have that $x_1^3 = x_2^3$. As the function ``taking the third power'' is one-to-one (namely, its inverse is ``taking cube roots''), it follows that $x_1=x_2$. This means that one can start with $y=4x^3+1$ and solve for $x$ as a function of $y$. When this is finished, one replaces $x$ and $y$ with $f^{-1}(x)$ and $x$, respectively, for the sake of conventions. We have that $$y=4x^3+1 \implies y-1 = 4x^3 \implies \frac{y-1}{4} = x^3 \implies \sqrt[3]{\frac{y-1}{4}} = x.$$Therefore, we conclude that $$f^{-1}(x) = \sqrt[3]{\frac{x-1}{4}}.$$Observe that the domain of $f^{-1}$ is precisely the set of all real numbers, as there is no positivity restriction for taking ``odd roots''. This turns out to be exactly the range of the original function $f$, suitably read in the $y$ variable: namely, the range of $f$ consists of all real numbers $y$, as the graph shows:
\begin{image}
\begin{tikzpicture}
    \begin{axis}
      \addplot[samples=200,domain=-5:5]{4*x^3+1};
    \end{axis}
\end{tikzpicture}
\end{image}
Observe that the graph also shows that $f$ passes the Horizontal Line Test, an useful sanity-check.
  \end{explanation}
\end{example}

\begin{callout}
  What happened with the domain of $f^{-1}$ on the previous example is a very general phenomenon: if a given function $f$ is one-to-one, so that $f^{-1}$ exists, the domain of $f^{-1}$ is exactly the range of the original function $f$.
\end{callout}

\begin{exploration}
   Study the function $f(x) = -x^5 + 10$. Is it one-to-one? If yes, find a formula for $f^{-1}$.
\end{exploration}


\begin{example} 
  Let $g$ be the rational function given by $g(x) = 2x/(1-x)$ for every real number $x$ not equal to $1$. Is $g$ one-to-one? If so, what is the inverse $g^{-1}$? What is its domain?

  \begin{explanation}
    The formula defining $g$ is different from the formula defining the function from the previous example: there, we had a polynomial, while here we have a rational function. The process to study it, however, is the same. To decide whether $g$ is one-to-one or not, we will start with $g(x_1)=g(x_2)$, and try to conclude that $x_1=x_2$. So, we start with $$\frac{2x_1}{1-x_1} = \frac{2x_2}{1-x_2}.$$By cross multiplying, we have that $$2x_1(1-x_2) = 2x_2(1-x_1).$$Distributing the products on both sides, we obtain $$2x_1-2x_1x_2 = 2x_2-2x_2x_1.$$Since $2x_2x_1=2x_1x_2$, adding this quantity to both sides gives us that $2x_1=2x_2$. Finally, dividing everything through by $2$, it follows that $x_1=x_2$. Therefore, $g$ is one-to-one. With this in place, we can attempt to find $g^{-1}$. Starting with $$y = \frac{2x}{1-x},$$the only thing it seems we might be able to try is to cross multiply terms, and distribute the products. So $$(1-x)y = 2x \implies x-xy = 2x.$$As our goal is to solve for $x$ in terms of $y$, let's move everything that contains $x$ in it to one side, and leave the rest which does not contain it on the other side. We obtain $y = 2x+xy$. Factoring $x$, it follows that $y = (2+y)x$. Finally, we conclude that $$x=\frac{y}{y+2}\implies g^{-1}(x) = \frac{x}{x+2},$$by replacing $x$ and $y$ with $g^{-1}(x)$ and $x$, respectively. We observe that the domain of $g^{-1}$ consists of all real numbers $x$ different from $-2$, in the same way that the range of $g$ consists of all real numbers $y$ different from $-2$ (to wit, if we had $-2 = 2x/(1-x)$, then cross multiplying and simplifying yields $-x=1-x$, leading to a terribly nonsensical $0=1$).
  \end{explanation}
\end{example}

\begin{exploration}
  Study the function $g(x) = 3x/(4-5x)$. Is it one-to-one? If yes, find a formula for $g^{-1}$.
\end{exploration}

\begin{example}
  Let $h$ be the function given by $h(x) = e^{\sqrt{x-3}}$ for every real number $x$ greater or equal to $3$. Is $h$ one-to-one? If so, what is the inverse $h^{-1}$? What is its domain?

  \begin{explanation}
    Let's start verifying whether $h$ is one-to-one or not. So, as before, we start with $h(x_1)=h(x_2)$ and try to obtain that $x_1=x_2$. In other words, we start with $e^{\sqrt{x_1-3}} = e^{\sqrt{x_2-3}}$. From arguing either that the exponential function itself is one-to-one, or applying $\ln$ on both sides of this equality, it follows that $\sqrt{x_1-3}=\sqrt{x_2-3}$. Raising both sides to the square, we have that $x_1-3=x_2-3$, and adding $3$ to everything yields $x_1=x_2$, as desired. Therefore, $h$ is one-to-one. To find a formula for $h^{-1}$, we start with $y= e^{\sqrt{x-3}}$ and solve for $x$ in terms of $y$. First, we apply $\ln$ on both sides of the equality, as to obtain $\ln y = \sqrt{x-3}$. Next, we take squares on both sides, so $(\ln y)^2 = x-3$. Finally, we add $3$ to both sides: $$x = (\ln y)^2+3 \implies h^{-1}(x) = (\ln x)^2+3.$$Note that $(\ln x)^2$ is not the same thing as $\ln(x^2)$ (to wit, the latter equals $2\ln x$, by basic properties of the logarithm function). The inverse $h^{-1}$ is defined for all $x>0$, due to the presence of $\ln x$ in its formula.
  \end{explanation}
\end{example}

\begin{exploration}
  Study the function $h(x) = \ln \sqrt{x+10}$. Is it one-to-one? If yes, find a formula for $h^{-1}$.
\end{exploration}

There's a general phenomenon which sometimes makes things simpler. To understand it, let's go back to the underpinning idea of what is the inverse function $f^{-1}$'s job: to undo what $f$ does. So, if we have two one-to-one functions $f$ and $g$ for which the composition $g\circ f$ makes sense, then $g\circ f$ should also be one-to-one. In other words, the composition of two ``reversible'' processes must also be ``reversible''. The next question, then, is what should be the inverse of the full composition $g\circ f$. Since $g\circ f$ first applies $f$, and then $g$, reversing it should be done on the opposite order of things, first applying $g^{-1}$, and then $f^{-1}$.

\begin{callout}
  {\bf Theorem:} The composition of two one-to-one functions $f$ and $g$, whenever it makes sense, is also one-to-one, and the relation $$(g\circ f)^{-1} = f^{-1}\circ g^{-1}$$ holds. More generally, the composition of any number of one-to-one functions is one-to-one, and the inverse of the composition is the composition of the inverses, \emph{but in the reverse order}.
\end{callout}

In practice, this result can be used to quickly find formulas for inverse function, provided one is able to express (or reverse engineer) the given function as a composition of simpler functions whose inverses are known. Properly applied, this also allows us to bypass the verification that the given function is one-to-one (again, if the simpler functions are more well-known to be one-to-one). 

Let's revisit the last example as an application of this:

\begin{example}
  Consider again the function given by $h(x) = e^{\sqrt{x-3}}$, for all $x\geq 3$, on Example 3. What does $h$ do to a number $x$? First, subtract $3$. Secondly, take the square root. Lastly, exponentiates it. Writing it mathematically, we have that $h = h_1\circ h_2\circ h_3$, where $$h_1(x) =e^x,\quad h_2(x) = \sqrt{x},\quad\mbox{and}\quad h_3(x)= x-3.$$All of those functions are one-to-one, with inverses given by $$h_1^{-1}(x) = \ln x,\quad h_2^{-1}(x) = x^2,\quad\mbox{and}\quad h_3^{-1}(x) = x+3.$$Hence \begin{align*}h^{-1}(x) &= (h_1\circ h_2\circ h_3)^{-1}(x) \\ &= h_3^{-1}\circ h_2^{-1} \circ h_1^{-1}(x) \\ &= h_3^{-1}\circ h_2^{-1}(\ln x) \\ &= h_3^{-1}((\ln x)^2) \\ &= (\ln x)^2 + 3, \end{align*}which agrees with what was obtained before.
\end{example}

\end{document}
