\documentclass[nooutcomes]{ximera}

\input{../preamble}
\author{David Kish}
\license{Creative Commons Attribution-ShareAlike 4.0 International License}
\acknowledgement{url of source material}

\title{Applications of Systems of Equations}

\begin{document}
\begin{abstract}
This sections looks at applications of systems of equations more closely.
\end{abstract}
\maketitle
\begin{example}
   Two Different Interest Rates
          Notah made some large purchases with his two credit cards one month and took on a total of
          $\$8{,}400$ in debt from the two cards.
          He didn't make any payments the first month,
          so the two credit card debts each started to accrue interest.
          That month, his Visa card charged $2\%$ interest and his Mastercard charged $2.5\%$ interest.
          Because of this, Notah's total debt grew by $\$178$.
          How much money did Notah charge to each card?\\
\begin{explanation}
          To start, we will define two variables based on our two unknowns.
          Let $v$ be the amount charged to the Visa card
          (in dollars)
          and let $m$ be the amount charged to the Mastercard
          (in dollars).
    
          To determine our equations, notice that we are given two different totals.
          We will use these to form our two equations.
          The total amount charged is $\$8{,}400$ so we have:
        \[ 
            (v\,\text{dollars})+(m\,\text{dollars})=\$8400
          \]
          Or without units:
          \[
            v+m=8400 
     \]
          The other total we were given is the total amount of interest,
          $\$178$, which is also in dollars.
          The Visa had $v$ dollars charged to it and accrues $2\%$ interest.
          So $0.02v$ is the dollar amount of interest that comes from using this card.
          Similarly, $0.025m$ is the dollar amount of interest from using the Mastercard.
          Together:
      \[
            0.02(v\,\text{dollars})+0.025(m\,\text{dollars})=\$178
         \]
          Or without units:
       \[
            0.02v+0.025m=178
  \]
          As a system, we write:
   \begin{center}
         $ 
            \begin{array}{cccc}
            v&+& m&=8400 \\
            0.02v&+& 0.025m&=178 
            \end{array}
$
\end{center}
          To solve this system by substitution,
          notice that it will be easier to solve for one of the variables in the first equation.
          We'll solve that equation for $v$:
\begin{center}
$
\begin{array}{rl}
            v+m&=8400\\
            v&=8400-m
\end{array}
$
\end{center}
          Now we will substitute $8400-m$ for $v$ in the second equation:
\begin{center}
$
          \begin{array}{rl}
            0.02v+0.025m&=178\\
            0.02(\substitute{8400-m})+0.025m&=178\\
            168-0.02m+0.025m&=178\\
            168+0.005m&=178\\
            0.005m&=10\\
            \divideunder{0.005m}{0.005}&=\divideunder{10}{0.005}\\
            m&=2000
      \end{array}
$
\end{center}
          Lastly, we can determine the value of $v$ by using the earlier equation where we isolated $v$:
\begin{center}
$
  \begin{array}{rl}
            v&=8400-m\\
            v&=8400-\substitute{2000}\\
            v&=6400\\
\end{array}
$
\end{center}        
          In summary, Notah charged $\$6400$ to the Visa and $\$2000$ to the Mastercard.
          We should check that these numbers work as solutions to our original system <em>and</em>
          that they make sense in context.
          (For instance, if one of these numbers were negative,
          or was something small like $\$0.50$,
          they wouldn't make sense as credit card debt.)
  \end{explanation}
\end{example}

\section{Mixture Problems}
      The next two examples are called
      \textbf{mixture problems},
      because they involve mixing two quantities together to form a combination and we want to find out how much of each quantity to mix.
 

 \begin{example}{Mixing Solutions with Two Different Concentrations}
          LaVonda is a meticulous bartender and she needs to serve $600$ milliliters of Rob Roy,
          an alcoholic cocktail that is $34\%$ alcohol by volume.
          The main ingredients are scotch that is $42\%$ alcohol and vermouth that is $18\%$ alcohol.
          How many milliliters of each ingredient should she mix together to make the concentration she needs?
\begin{explanation}
          The two unknowns are the quantities of each ingredient.
          Let $s$ be the amount of scotch (in
         mL) and let $v$ be the amount of vermouth
          (in
        mL).
    


          One quantity given to us in the problem is     $600$ mL.
          Since this is the total volume of the mixed drink, we must have:
          \[
            (s\,\text{mL})+(v\,\text{mL})=600\,\text{mL}
        \]
          Or without units:
         \[
            s+v=600
        \]
      
          To build the second equation,
          we have to think about the alcohol concentrations for the scotch,
          vermouth,
          and Rob Roy.
          It can be tricky to think about percentages like these correctly.
          One strategy is to focus on the \textit{amount} (in
        mL) of \textit{alcohol} being mixed.
          If we have $s$ milliliters of scotch that is $42\%$ alcohol,
          then $0.42s$ is the actual \textit{amount} (in
       mL) of alcohol in that scotch.
          Similarly, $0.18v$ is the amount of alcohol in the vermouth.
          And the final cocktail is
          $600$ mL
          of liquid that is $34\%$ alcohol,
          so it has $0.34(600)=204$ milliliters of alcohol.
          All this means:
          \[
            0.42(s\,\text{mL})+0.18(v\,\text{mL})=204\,\text{mL}
          \]
          Or without units:
       \[
            0.42s+0.18v=204
         \]

      
          So our system is:
\begin{center}
$
            \begin{array}{cccc}
            s&+& v&=600 \\
            0.42s&+&0.18v&=204
            \end{array}
      $
\end{center}

          To solve this system, we'll solve for $s$ in the first equation:
$
         \begin{array}{rl}
            s+v&=600 \\
            s&=600-v
       \end{array}
$
          And then substitute $s$ in the second equation with $600-v$:
\begin{center}
$
      \begin{array}{rl}
            0.42s+0.18v&=204 \\
            0.42(\substitute{600-v})+0.18v&=204\\
            252-0.42v+0.18v&=204\\
            252-0.24v&=204\\
            -0.24v&=-48\\
            \divideunder{-0.24v}{-0.24}&=\divideunder{-48}{-0.24}\\
            v&=200
       \end{array}
$
\end{center}
          As a last step,
          we will determine $s$ using the equation where we had isolated $s$:
\begin{center}          
$
\begin{array}{rl}
            s&=600-v\\
            s&=600-\substitute{200}\\
            s&=400
       \end{array}
$
\end{center}
          In summary, LaVonda needs to combine
$400$ mL
          of scotch with
       $200$ mL
          of vermouth to create
   $600$ mL
          of Rob Roy that is $34\%$ alcohol by volume.
\end{explanation}
\end{example}

    
      As a check for the previous example,
      we can use estimation
      to see that our solution is reasonable.
      Since LaVonda is making a $34\%$ solution,
      she would need to use more of the $42\%$ concentration than the $18\%$ concentration,
      because $34\%$ is closer to $42\%$ than to $18\%$.
      This agrees with our answer because we found that she needed
  $400$ mL
      of the $42\%$ solution and
      
        $200$ mL
      of the $18\%$ solution.
      This is an added check that we have found reasonable answers.

\begin{example}{Mixing a Coffee Blend}
          Desi owns a coffee shop and they want to mix two different types of coffee beans to make a blend that sells for $\$12.50$ per pound.
          They have some coffee beans from Columbia that sell for $\$9.00$ per pound and some coffee beans from Honduras that sell for $\$14.00$ per pound.
          How many pounds of each should they mix to make $30$ pounds of the blend?
\begin{explanation}
          Before we begin, it may be helpful to try to estimate the solution.
          Let's compare the three prices.
          Since $\$12.50$ is between the prices of $\$9.00$ and $\$14.00$,
          this mixture is possible.
          Now we need to estimate the amount of each type needed.
          The price of the blend ($\$12.50$ per pound) is closer to the higher priced beans ($\$14.00$ per pound) than the lower priced beans
          ($\$9.00$ per pound).
          So we will need to use more of that type.
          Keeping in mind that we need a total of $30$ pounds,
          we roughly estimate $20$ pounds of the $\$14.00$ Honduran beans and $10$ pounds of the $\$9.00$ Columbian beans.
          How good is our estimate?
          Next we will solve this exercise exactly.

          To set up our system of equations we define variables,
          letting $C$ be the amount of Columbian coffee beans
          (in pounds)
          and $H$ be the amount of Honduran coffee beans
          (in pounds).

          The equations in our system will come from the total amount of beans and the total cost.
          The equation for the total amount of beans can be written as:
      \[    
            (C\,\text{lb})+(H\,\text{lb})=30\,\text{lb}
   \]     
          Or without units:
         \[
            C+H=30
\]
          To build the second equation,
          we have to think about the cost of all these beans.
          If we have $C$ pounds of Columbian beans that cost $\$9.00$ per pound,
          then $9C$ is the cost of those beans in dollars.
          Similarly, $14H$ is the cost of the Honduran beans.
          And the total cost is for $30$ pounds of beans priced at $\$12.50$ per pound,
          totaling $12.5(30)=37.5$ dollars.
          All this means:
    \[      
            \left(9\,\tfrac{\text{dollars}}{\text{lb}}\right)(C\,\text{lb})+\left(14\,\tfrac{\text{dollars}}{\text{lb}}\right)(H\,\text{lb})=\left(12.50\,\tfrac{\text{dollars}}{\text{lb}}\right)(30\,\text{lb})
 \]
          Or without units and carrying out the multiplication on the right:
     \[
            9C+14H=37.5
\]
          Now our system is:

\begin{center}
$
            \begin{array}{cccc}
            C&+& H&=30 \\
            9C&+& 14H&=37.50
            \end{array}
$
 \end{center}

          To solve the system, we'll solve the first equation for $C$:
\begin{center}
$
\begin{array}{rl}
            C+H&=30\\
            C&=30-H
\end{array}
$
\end{center}
          Next, we'll substitute $C$ in the second equation with $30-H$:
\begin{center}
$
    \begin{array}{rl}
            9C+14H&=375 \\
            9(\substitute{30-H})+14H&=375 \\
            270-9H+14H&=375 \\
            270+5H&=375 \\
            5H&=105 \\
            H&=21
 \end{array}
$
\end{center}
          Since $H=21$, we can conclude that $C=9$.

          In summary, Desi needs to mix $21$ pounds of the Honduran coffee beans with $9$ pounds of the Columbian coffee beans to create this blend.
          Our estimate at the beginning was pretty close,
          so we feel this answer is reasonable.
\end{explanation}
\end{example}



%
%
%%\typeout{************************************************}
%%\typeout{Motivating Questions}
%%\typeout{************************************************}
%
%\begin{motivatingQuestions}
%%Often start a section. 
%\item Question 1
%\item Question 2
%\end{motivatingQuestions}
%
%
%%\typeout{************************************************}
%%\typeout{Introduction}
%%\typeout{************************************************}
%
%
%
%%\typeout{************************************************}
%%\typeout{section}
%%\typeout{************************************************}
%
%\section{Subsection Title}
%Start every file with a section.
%
%%\typeout{************************************************}
%%\typeout{Problem Types in the Text}
%%\typeout{************************************************}
%
%\section{Problem Types in the Text}
%
%\begin{exploration}
%This is a question where the answer is not proviced in the text.  The idea is that students will work on together in lecture.  It often motivates the upcoming content.
%	\begin{enumerate}[label=\alph*.]
%	\item Problem 1
%	\item Problem 2
%	\end{enumerate}
%\end{exploration}
%
%
%\begin{problem}
%Use a Problem when students are supposed to enter an answer.  These should be straightforward things or the answer should be in a hint or explanation.  And the answer should be given in the printed text.  
%$y=\answer{10x}$
%	\begin{hint}
%	Hint here
%	\end{hint}
%	\begin{explanation}
%	One approach to pattern recognition is to look for a relationship in each row. Here, the $y$-value in each row is always 10 more than the $x$-value. So the pattern is described by the equation $y=10x$
%	\end{explanation}
%\end{problem}
%
%
%\begin{example}
%A standard example with solution in the text.
%	\begin{explanation}
%	Every example should have an explanation.
%	\end{explanation}
%\end{example}
%
%
%%\typeout{************************************************}
%%\typeout{Other Environments in Text}
%%\typeout{************************************************}
%
%\section{Other Environmentsin the Text}
%
%\begin{remark}
%Something you want to call attention to in the text.
%\end{remark}
%
%
%\begin{definition}
%Define a word or words.  Be sure to use the dfn command around your \dfn{vocab words}.
%\end{definition}
%
%
%\begin{callout}
%Something that you want to standout that is not a remark.  Basically just puts it in a blue box.
%\end{callout}
%
%\begin{summary}\begin{itemize}
%%Often ends a section
%\item First point
%\item Second point
%\end{itemize}\end{summary}
%
%
%%\typeout{************************************************}
%%\typeout{Tables and Graphs}
%%\typeout{************************************************}
%
%\section{Tables and Graphs}
%
%How to make a table:
%
%
%\[
%\begin{array}{cc}
%t&V\\
%\hline
%0&0.0\\
%1&0.5\\
%2&1.0\\
%3&1.5\\
%4&2.0\\
%5&2.5
%\end{array}
%\]
%
%
%Side-by-side tables (or images or whatever):
%
%
%\[
%\begin{array}{cccccc}
%%\(
%{\begin{array}{cc}
%t&r(t)\\
%\hline
%0&12\\
%3&10\\
%6&8\\
%9&6
%\end{array}}&&&&&
%%\)
%%\end{center}
%%\begin{center}
%%\(
%{\begin{array}{cc}
%t&s(t)\\
%\hline
%0&12\\
%3&9\\
%6&6.75\\
%9&5.0625
%\end{array}}\\
%\end{array}
%\]
%
%
%How to make an image:
%\begin{image}
%\includegraphics{ColumbusChicago.png}
%\end{image}
%
%
%Draw graphs in tikzi when possible.  Here are two.
%
%\begin{image}
%\begin{tikzpicture}
%    \begin{axis}
%        \addplot[samples=200,domain=0.01:8]{ln(x)};
%    \end{axis}
%\end{tikzpicture}
%\end{image}
%
%\begin{image}
%\begin{tikzpicture}
%    \begin{axis}[xlabel={},ylabel={},width=0.75\linewidth,
%                xmin=-5,xmax=5,
%                ymin=-5,ymax=5,
%                xtick={-4,4},
%                ytick={-4,4},
%                clip=false]
%        \addplot[soliddot] coordinates {(0,0)} node[pin=240:{Carl's house}] {};
%        \addplot[soliddot] coordinates {(2, 3)} node[pin=-30:{restaurant}] {};
%        \addplot[soliddot] coordinates {(-3, 2)} node[pin=100:{pet shop}] {};
%        \addplot[soliddot] coordinates {(-2, -4)} node[pin=150:{gas station}] {};
%        \addplot[soliddot] coordinates {(3, -3)} node[pin=120:{bar}] {};
%        \addplot[mark=none] coordinates {(5, 0)} node[above left] {east};
%        \addplot[mark=none] coordinates {(-5, 0)} node[above right] {west};
%        \addplot[mark=none] coordinates {(0, 5)} node[below right] {north};
%        \addplot[mark=none] coordinates {(0, -5)} node[above right] {south};
%    \end{axis}
%\end{tikzpicture}
%\end{image}
%
%
%You can also add Desmos interactives.  Create them in a Desmos account (I think we have an OSU one.  We should look into that!).  Save them.  Then pull the graph number out of the url.
%\begin{center}  
%\desmos{lxllnpdi6w}{800}{600}  
%\end{center}
%
%%\typeout{************************************************}
%%\typeout{Online Features}
%%\typeout{************************************************}
%
%\section{Online Features}
%
%To add a url, use the link command.
%For more about formatting in Ximera see \link[this url]{https://ximera.osu.edu/intro/gettingStarted/graphicsAndVideos/graphicsAndVideos}.
%
%
%You can also embed \link[YouTube]{https://www.youtube.com/} videos.
%\begin{center}
%\youtube{0aQpLSu2fMs}
%\end{center}
%
%
%
%
%
%\newpage
%
%%\typeout{************************************************}
%%\typeout{Overviews}
%%\typeout{************************************************}
%
%\section{Overviews}
%
%Each Unit has an overview with the organization and learning objectives
%
%\begin{overview}
%\item Generally a folder %(author if relevant)
%	\begin{enumerate}
%	\item some stuff covered in these sections
%		\textit{a subtopic} 
%		\textit{another subtopic} 
%	\item More stuff	
%	\end{enumerate}	
%\item Another Folder 
%	\begin{enumerate}	
%	\item Stuff 
%	\end{enumerate} 
%\end{overview}
%
%
%\begin{objectives}
%\item Learning Objectives Category (Course level learning objective)
%	\begin{itemize}
%	\item more specific goal
%	\item another one 
%	\end{itemize}
%\item Another Category
%	\begin{itemize}
%	\item Linear 
%	\item Parabolas 
%	\item Polynomials 
%	\end{itemize}
%\end{objectives}
%
%
%
%
%\newpage
%
%%\typeout{************************************************}
%%\typeout{Homeworks}
%%\typeout{************************************************}
%
%\section{Homework}
%Each homework problem should be it's own file.  Then the homework is put together using an exerciselist file.  See a Unit 1 folder for an example.  Be sure to keep all the same conventions, just changing the actual problem.  
%
%Some Ximera problem types are available \link[in the footnote]{https://ximera.osu.edu/intro/gettingStarted/questionAndAnswerTypes/questionAndAnswerTypes}.  We can add more here as we come across them.

\end{document}
