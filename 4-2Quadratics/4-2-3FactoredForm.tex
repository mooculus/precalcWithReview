\documentclass{ximera}

\input{../preamble.tex}
\author{David Kish and Elizabeth Miller}
\license{Creative Commons Attribution-ShareAlike 4.0 International License}


\title{Factored Form}

\begin{document}
\begin{abstract}
We explore different methods for solving quadratic functions.
\end{abstract}
\maketitle
We have previously looked at different forms of quadratic functions. We've looked at standard form and vertex form, where characteristics like $y$-intercept and vertex can be found easily by looking at the function. Another useful way to look at quadratic functions is to have them written out as a product of linear factors. This can help us to quickly determine the $x$-intercepts of a quadratic function and to get a good idea of the position and shape of the graph. Not all quadratics can be written in factored form, so we will begin by addressing those.
\begin{remark}
\textbf{Irreducible quadratic factors} are quadratic factors that when set equal to zero only have complex roots.  As a result they cannot be reduced into factors containing only real numbers, hence the name irreducible. 
\end{remark}
As seen in the graphs below, the graphs of the functions do not cross the $x$-axis, so they do not have $x$-intercepts. The first graph, $y=x^2+x+1$ is entirely above the $x$-axis and the second graph, $y=-x^2+x-1$ is entirely below the $x$-axis. Since neither of them cross the $x$-axis, they have no $x$-intercepts and are irreducible.
%\begin{image}
%\includegraphics[width=2in]{irreducible.PNG} \hfill \includegraphics[width=2in]{irreducible2.PNG}
%\end{image}

%%%%%%%%%
\section{Factored Form (or Root Form)}
\begin{callout} \textbf{\large{Factored (Root) Form of a Quadratic Function}}\\
          A quadratic function whose graph has $x$-intercepts (called roots) at the points $(r,0)$ and $(s,0)$ can be written as:\\
\[
f(x)=a(x-r)(x-s)
\]

\end{callout}

This form is called \dfn{Root Form} because the roots of the quadratic can be easily read off from this form. It is also sometimes called \dfn{Factored Form} because the quadratic is factored into a product of linear terms.  We often call quadratics written as $(ax+b)(cx+d)$ a \dfn{factored} quadratic.  This is not quite the same as Factored Form (Root Form) though because the leading constant is not pulled out to the front.  To minimize this confusion, we will typically but not exclusively use the name Root Form.
%\begin{example}
%\begin{explanation}

%\end{explanation}
%\end{example}
%%%%%%%%%
\section{Factoring from Standard Form when $a=1$}
When $a=1$, putting a quadratic in Root Form is the same as factoring a quadratic.  In general, \dfn{factoring} refers to writing as a product of linear terms, but does not necessarily imply that the $a$ term is pulled out front like it is in Root Form.  

\begin{example}
Factor the following quadratic into a product of linear factors:
\[
x^2+3x+2
\]

\begin{explanation}
For us to begin factoring this quadratic, we have to look at the $b$ and $c$ terms. We are looking for $2$ numbers that muliply to $2$ (or $c$) and add up to $3$ (or $b$). By going through the factors of $2$ we can see that the only numbers that satisfy these conditions are $2$ and $1$.
\[
2+1=3
\]
\[
2\cdot 1=2
\]
This means that we can factor the quadratic the following way:
\[
x^2+3x+2=(x+2)(x+1)
\]
The quadratic is now written as a product of linear factors and because $a=1$, these are also our $x$-intercepts (or roots) for our function.
\end{explanation}
\end{example}

%%%%%%%%%
\section{Factoring from Standard Form when $a>1$}
\begin{example}
Rewrite the following quadratic in Root Form.
\[
f(x)= 3x^2-4x-7
\]
\begin{explanation}
First we will start by pulling out a $3$ from every term.
\[
3\left(x^2-\frac{4}{3}x-\frac{7}{3}\right)
\]
We now have to find factors that add up to $-\frac{4}{3}$ and multiply to $-\frac{7}{3}$. In this particular case, we can see that the difference between the numerators ($7$\&$4$) is $3$, and since $\frac{3}{3}=1$ our job will be a little easier. This leads us to the following factors:
\[
\frac{3}{3}+\frac{-7}{3}=-\frac{4}{3}
\]
\[
\frac{3}{3}\cdot \frac{-7}{3}= -\frac{7}{3}
\]
So, our factored form is as follows:
\[
f(x) = 3\left(x+\frac{3}{3}\right)\left(x-\frac{7}{3}\right)
\]
%We now have the equation written as a product of linear components, and if we want to find the $x$ intercepts we can set each of the components equal to $0$.
%\begin{align*}
%x+\frac{3}{3}&=0\\
%x&=-\frac{3}{3}\\
%x&=-1
%\end{align*}
%\begin{align*}
%x-\frac{7}{3}&=0\\
%x&=\frac{7}{3}
%\end{align*}
%So our solutions or $x$ intercepts are $x=-1$ and $x=\frac{7}{3}$
\end{explanation}

Note that in the previous example, it is not necessary to pull at the $3$ as the first step.  Instead, we could pull out the $3$ as the last step and still have the root form.

\begin{explanation}
This way, we start with
\[
f(x)= 3x^2-4x-7
\]
We now have to find numbers $m_1,m_2,b_1, \text{ and } b_2$ such that:

\begin{align*}
(m_1x+b_1)(m_2x+b_2)&=m_1m_2x^2+m_1b_2x+m_2b_1x+b_1b_2\\
				&=m_1m_2x^2+(m_1b_2+m_2b_1)x+b_1b_2\\
				&= 3x^2-4x-7
\end{align*}

This means that we need

\begin{align*}
m_1m_2&=3\\
m_1b_2+m_2b_1&=-4\\
b_1b_2&=-7
\end{align*}
because the only way two quadratics in standard form can be equal is if they have the same coefficients for each term.

Through a little trial and error, we find that:
\begin{align*}
m_1&=1\\
b_1 &=1\\
m_2&=3\\
b_2&=7
\end{align*}
will work.

We now have the equation written as a product of linear components
\[
f(x)= 3x^2-4x-7=(x+1)(3x-7)
\]
Now, to write our answer in Root Form, we just need to factor out both $m_1$ and $m_2$.  Since in this example, $m_1=1$, we don't actually have to do any thing for that one.
\[
f(x)= 3x^2-4x-7=(x+1)(3x-7)=(x+1)\left[3\left(x-\frac{7}{3}\right)\right]=3(x+1)\left(x-\frac{7}{3}\right)
\]
\end{explanation}
Now, we have our quadratic in Root Form and can read off our roots as $x=1$ and $x=\frac{-7}{3}$.
\end{example}

%%%%%%%%%%
\section{Quadratic Formula}
When there appears to be no easy way to factor a quadratic, our best option is to use the Quadratic Formula. Let's try the previous example with the Quadratic Formula
\begin{callout}
\[
x=\frac{-b\pm \sqrt{b^2-4ac}}{2a}
\]
when $ax^2+bx+c =0$
\end{callout}
\begin{example}
Find the solutions to the following quadratic equation:
\[
3x^2-4x-7=0
\]
\begin{explanation}
\begin{align*}
x&=\frac{-(-4)\pm \sqrt{(-4)^2-4(3)(-7)}}{2(3)}\\
&=\frac{4 \pm \sqrt{16-(-84)}}{6}\\
&=\frac{4 \pm \sqrt{100}}{6}\\
&=\frac{4 \pm 10}{6}\\
x&=\frac{4+10}{6}&x=\frac{4-10}{6}\\
x&=\frac{14}{6}&x=\frac{-6}{6}\\
x&=-\frac{7}{3}&x=-1\\
\end{align*}
We still get $x=-1$ and $x=\frac{7}{3}$ as our roots. This can be a very useful tool especially with more complicated quadratic equations.

Now that we know the roots, we can use the $a$-value we see in standard form, $a=3$ and the two roots $r=1$ and $s=\frac{-7}{3}$ to plug into our Root Form formula $a(x-r)(x-s)$, so again we get as a final answer that
\[
f(x)=3(x+1)\left(x-\frac{7}{3}\right)
\]
\end{explanation}
\end{example}
%%%%%%%%%
\section{Factoring when missing a term}
We've talked about factoring $ax^2+bx+c$ when all terms are present, but what do we do when one of the terms is missing? Since there are three terms we have three different cases to address.

The first case also happens to be the easiest to solve. How do you a solve a quadratic that is missing the $ax^2$ term? This is a bit of a trick question, because without an $x^2$ term, we are no longer dealing with a quadratic. 
\begin{example}
Solve the following equation
\[
2x-9=0
\]

\begin{explanation}
We can see here that we are only dealing with a linear terms and there are no quadratic ($x^2$) terms. This means we do not have to factor and we can solve for $x$ directly.
\begin{align*}
2x-9&=0\\
2x&=9\\
x&=\frac{9}{2}\\
\end{align*}
\end{explanation}
\end{example}
The second case is when the middle $bx$ term is missing. 
\begin{example}
Factor the quadratic $f(x)=x^2-9$ into linear components.\\

\begin{explanation}
This quadratic is a special case called ``difference of squares.'' There is no ``middle'' term and the remaining two terms are both perfect squares, so we can use a shortcut when factoring.
\begin{callout}
\textbf{Difference of Squares}\\
When $a$,$b$ are non zero.
\[
a^2-b^2=(a+b)(a-b)
\]
\end{callout}
In our case, we can see that $x^2$ is a perfect square and $9$ is also a perfect square because $9=3^2$. This means that our original quadratic will be factored like this:
\[
x^2-9=(x+3)(x-3)
\]
We can also think of it in the same way as factoring other quadratics. Since there is no middle term, we can look at factors of $-9$ that add up to $0$. $3$ and $-3$ add up to $0$ and multiply out to $-9$. The difference of squares is just a useful pattern that helps to speed up our factoring process.
\end{explanation}
\end{example}
The last case is when there is no constant or $c$ term. 
\begin{example}
Factor the quadratic $f(x) = x^2+2x$ into linear components.\\

\begin{explanation}
Since there is a common $x$ factor in both terms we can pull out that factor and we are left with a product of linear components. 
\begin{align*}
f(x) &= x^2+2x\\
&= x\cdot x+2x\\
&=x(x+2)
\end{align*}
Again, our factoring is already simplified. We do not have go through the whole process of factoring. If we have a quadratic function with only the $ax^2$ and $bx$ term, then we will always be able to pull out at least an $x$ term when factoring.
\end{explanation}
\end{example}
%\section{Factoring by Grouping}
%\begin{example}
%Factor the following quadratic into a product of linear factors:
%\[
%3x^2+5x-2
%\]
%\begin{explanation}
%For this method, we first have to start with multiplying the $a$ and $c$ term.
%\end{explanation}
%\end{example}

 


\end{document}
