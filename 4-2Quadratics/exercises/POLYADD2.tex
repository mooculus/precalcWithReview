\documentclass{ximera}

\input{../../preamble.tex}

\author{Kenneth Berglund}




\begin{document}
In this exercise, we will rewrite the following quadratic in vertex form $a(x - h)^2 + k$, using the method of completing the square:
\[
f(x)=5x^2+15x-2.
\]
\begin{exercise}

Factoring the leading coefficient out of the $x^2$ and $x$ terms yields
$$
\answer{5}(x^2 + \answer{3}x) + \answer{-2}.
$$

\begin{exercise}
Notice that we can't rewrite $x^2 +3x$ as a square of a linear polynomial. We need to add a constant term in order to make $x^2 + 3x$ into a square. 

We want to find numbers a number $h$ such that $(x - h)^2 = x^2 + 3x + h^2$. 

$$
h = \answer{-3/2} \text{ and } h^2 = \answer{9/4}
$$
\begin{hint}
Try multiplying out $(x - h)^2$ and matching the coefficients of each term with the coefficients of the corresponding term in $x^2 + 3x + h^2$. 
\end{hint}


\begin{exercise}
Modify the quadratic to introduce the needed $h^2$ term.
$$
5(x^2 + 3x + \answer{9/4} - \answer{9/4}) - 2
$$ 

Use positive numbers in your answer.

\begin{hint}
Remember that if you add a term to an expression, to preserve equality, you must subtract the same term.
\end{hint}

\begin{exercise}
Now, regroup and distribute the 5 so you end up with an expression in the form below.
$$
5(x^2+3x+\frac{9}{4}) + \answer{-\frac{53}{4}}
$$ 

\begin{exercise}
Now rewrite $x^2 + 3x + \frac{9}{4}$ as a square of a linear polynomial to finish completing the square!
$$
5(x - \answer{-3/2})^2 - \answer{\frac{53}{4}}
$$ 
\begin{hint}
What square is equal to $x^2 + 3x + \frac{9}{4}$? We found it before!
\end{hint}

\begin{exercise}
As a bonus, here's what the manipulation would look like when done all at once:
\begin{align*}
5x^2 + 15x -2 & = 5(x^2 + 3x) -2 \\
& = 5(x^2 + 3x + \answer{9/4} - \answer{9/4}) - 2  \\
& = 5(x^2+3x+\answer{9/4})-\answer{45/4} - \frac{8}{4}\\
& = 5(x^2+3x+\answer{9/4})-\answer{53/4}\\
& = 5(x + \frac{3}{2})^2 - \answer{53/4}
\end{align*}
\end{exercise}
\end{exercise}
\end{exercise}
\end{exercise}
\end{exercise}
\end{exercise}



\end{document}