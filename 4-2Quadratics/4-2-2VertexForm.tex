\documentclass{ximera}


\graphicspath{
  {./}
  {1-1QuantitativeReasoning/}
  {1-2RelationsAndGraphs/}
  {1-3ChangingInTandem/}
  {2-1LinearEquations/}
  {2-2LinearModeling/}
  {2-3ExponentialModeling/}
  {3-1WhatIsAFunction/}
  {3-2FunctionProperties/}
  {3-3AverageRatesOfChange/}
  {4-1BuildingNewFunctions/}
  {4-2Polynomials/}
  {5-1RationalFunctions/}
   {5-2ExponentialFunctions/}
  {6-1Domain/}
  {6-2Range/}
  {6-3CompositionOfFunctions/}
  {7-1ZerosOfFunctions/}
  {7-XZerosOfPolynomials/}
  {7-2ZerosOfFamousFunctions/}
  {8-0Review/}
  {8-1FunctionTransformations/}
  {8-2SolvingInequalities/}
  {8-3FunctionTransformationsProject/}
  {9-1RightTriangleTrig/}
  {9-2TheUnitCircle/}
  {9-3TrigIdentities/}
  {10-1UnitCircleToFunctionGraph/}
  {10-2TrigFunctions/}
  {10-3SomeApplicationsOfTrig/}
  {11-1InverseFunctionsRevisited/}
  {11-2Logarithms/}
  {11-3InverseTrig/}
  {12-1SystemsOfEquations/}
  {12-2NonlinearSystems/}
  {12-3ApplicationsOfSystems/}
  {13-1SecantLinesRevisited/}
  {13-2Functions-TheBigPicture/}
  {14-1DisplacementVsDistance/}
  {1-1QuantitativeReasoning/exercises/}
  {1-2RelationsAndGraphs/exercises/}
  {../1-3ChangingInTandem/exercises/}
  {../2-1LinearEquations/exercises/}
  {../2-2LinearModeling/exercises/}
  {../2-3ExponentialModeling/exercises/}
  {../3-1WhatIsAFunction/exercises/}
  {../3-2FunctionProperties/exercises/}
  {../3-3AverageRatesOfChange/exercises/}
  {../5-2ExponentialFunctions/exercises/}
  {../4-1BuildingNewFunctions/exercises/}
  {../4-2Polynomials/exercises/}
  {../5-1RationalFunctions/exercises/}
  {../6-1Domain/exercises/}
  {../6-2Range/exercises/}
  {../6-3CompositionOfFunctions/exercises/}
  {../7-1ZerosOfFunctions/exercises/}
  {../7-XZerosOfPolynomials/exercises/}
  {../7-2ZerosOfFamousFunctions/exercises/}
  {../8-1FunctionTransformations/exercises/}
  {../12-1SystemsOfEquations/exercises/}
  {../8-3FunctionTransformationsProject/exercises/}
  {../8-0Review/exercises/}
  {../8-2SolvingInequalities/exercises/}
  {../8-3FunctionTransformationsProject/exercises/}
  {../9-1RightTriangleTrig/exercises/}
  {../9-2TheUnitCircle/exercises/}
  {../9-3TrigIdentities/exercises/}
  {../10-1UnitCircleToFunctionGraph/exercises/}
  {../10-2TrigFunctions/exercises/}
  {../10-3SomeApplicationsOfTrig/exercises/}
  {../11-1InverseFunctionsRevisited/exercises/}
  {../11-2Logarithms/exercises/}
  {../11-3InverseTrig/exercises/}
  {../12-1SystemsOfEquations/exercises/}
  {../12-2NonlinearSystems/exercises/}
  {../12-3ApplicationsOfSystems/exercises/}
  {../13-1SecantLinesRevisited/exercises/}
  {../13-2Functions-TheBigPicture/exercises/}
  {../14-1DisplacementVsDistance/exercises/}
}

\DeclareGraphicsExtensions{.pdf,.png,.jpg,.eps}

\newcommand{\mooculus}{\textsf{\textbf{MOOC}\textnormal{\textsf{ULUS}}}}

\usepackage[makeroom]{cancel} %% for strike outs

\ifxake
\else
\usepackage[most]{tcolorbox}
\fi


%\typeout{************************************************}
%\typeout{New Environments}
%\typeout{************************************************}

%% to fix for web can be removed when deployed offically with ximera2
\let\image\relax\let\endimage\relax
\NewEnviron{image}{% 
  \begin{center}\BODY\end{center}% center
}



\NewEnviron{folder}{
      \addcontentsline{toc}{section}{\textbf{\BODY}}
}

\ifxake
\let\summary\relax
\let\endsummary\relax
\newtheorem*{summary}{Summary}
\newtheorem*{callout}{Callout}
\newtheorem*{overview}{Overview}
\newtheorem*{objectives}{Objectives}
\newtheorem*{motivatingQuestions}{Motivating Questions}
\newtheorem*{MM}{Metacognitive Moment}
      
%% NEEDED FOR XIMERA 2
%\ximerizedEnvironment{summary}
%\ximerizedEnvironment{callout}
%\ximerizedEnvironment{overview} 
%\ximerizedEnvironment{objectives}
%\ximerizedEnvironment{motivatingQuestions}
%\ximerizedEnvironment{MM}
\else
%% CALLOUT
\NewEnviron{callout}{
  \begin{tcolorbox}[colback=blue!5, breakable,pad at break*=1mm]
      \BODY
  \end{tcolorbox}
}
%% MOTIVATING QUESTIONS
\NewEnviron{motivatingQuestions}{
  \begin{tcolorbox}[ breakable,pad at break*=1mm]
    \textbf{\Large Motivating Questions}\hfill
    %\begin{itemize}[label=\textbullet]
      \BODY
    %\end{itemize}
  \end{tcolorbox}
}
%% OBJECTIVES
\NewEnviron{objectives}{  
    \vspace{.5in}
      %\begin{tcolorbox}[colback=orange!5, breakable,pad at break*=1mm]
    \textbf{\Large Learning Objectives}
    \begin{itemize}[label=\textbullet]
      \BODY
    \end{itemize}
    %\end{tcolorbox}
}
%% DEFINITION
\let\definition\relax
\let\enddefinition\relax
\NewEnviron{definition}{
  \begin{tcolorbox}[ breakable,pad at break*=1mm]
    \noindent\textbf{Definition}~
      \BODY
  \end{tcolorbox}
}
%% OVERVIEW
\let\overview\relax
\let\overview\relax
\NewEnviron{overview}{
  \begin{tcolorbox}[ breakable,pad at break*=1mm]
    \textbf{\Large Overview}
    %\begin{itemize}[label=\textbullet] %% breaks Xake
      \BODY
    %\end{itemize}
  \end{tcolorbox}
}
%% SUMMARY
\let\summary\relax
\let\endsummary\relax
\NewEnviron{summary}{
  \begin{tcolorbox}[ breakable,pad at break*=1mm]
    \textbf{\Large Summary}
    %\begin{itemize}[label=\textbullet] %% breaks Xake
      \BODY
    %\end{itemize}
  \end{tcolorbox}
}
%% REMARK
\let\remark\relax
\let\endremark\relax
\NewEnviron{remark}{
  \begin{tcolorbox}[colback=green!5, breakable,pad at break*=1mm]
    \noindent\textbf{Remark}~
      \BODY
  \end{tcolorbox}
}
%% EXPLANATION
\let\explanation\relax
\let\endexplanation\relax
\NewEnviron{explanation}{
    \normalfont
    \noindent\textbf{Explanation}~
      \BODY
}
%% EXPLORATION
\let\exploration\relax
\let\endexploration\relax
\NewEnviron{exploration}{
  \begin{tcolorbox}[colback=yellow!10, breakable,pad at break*=1mm]
    \noindent\textbf{Exploration}~
      \BODY
  \end{tcolorbox}
}
%% METACOGNITIVE MOMENTS
\let\MM\relax
\let\endMM\relax
\NewEnviron{MM}{
  \begin{tcolorbox}[colback=pink!15, breakable,pad at break*=1mm]
    \noindent\textbf{Metacognitive Moment}~
      \BODY
  \end{tcolorbox}
}


\fi





%Notes on what envirnoment to use:  Example with Explanation in text; if they are supposed to answer- Problem; no answer - Exploration


%\typeout{************************************************}
%% Header and footers
%\typeout{************************************************}

\newcommand{\licenseAcknowledgement}{Licensed under Creative Commons 4.0}
\newcommand{\licenseAPC}{\renewcommand{\licenseAcknowledgement}{\textbf{Acknowledgements:} Active Prelude to Calculus (https://activecalculus.org/prelude) }}
\newcommand{\licenseSZ}{\renewcommand{\licenseAcknowledgement}{\textbf{Acknowledgements:} Stitz Zeager Open Source Mathematics (https://www.stitz-zeager.com/) }}
\newcommand{\licenseAPCSZ}{\renewcommand{\licenseAcknowledgement}{\textbf{Acknowledgements:} Active Prelude to Calculus (https://activecalculus.org/prelude) and Stitz Zeager Open Source Mathematics (https://www.stitz-zeager.com/) }}
\newcommand{\licenseORCCA}{\renewcommand{\licenseAcknowledgement}{\textbf{Acknowledgements:}Original source material, products with readable and accessible
math content, and other information freely available at pcc.edu/orcca.}}
\newcommand{\licenseY}{\renewcommand{\licenseAcknowledgement}{\textbf{Acknowledgements:} Yoshiwara Books (https://yoshiwarabooks.org/)}}
\newcommand{\licenseOS}{\renewcommand{\licenseAcknowledgement}{\textbf{Acknowledgements:} OpenStax College Algebra (https://openstax.org/details/books/college-algebra)}}
\newcommand{\licenseAPCSZCSCC}{\renewcommand{\licenseAcknowledgement}{\textbf{Acknowledgements:} Active Prelude to Calculus (https://activecalculus.org/prelude), Stitz Zeager Open Source Mathematics (https://www.stitz-zeager.com/), CSCC PreCalculus and Calculus texts (https://ximera.osu.edu/csccmathematics)}}

\ifxake\else %% do nothing on the website
\usepackage{fancyhdr}
\pagestyle{fancy}
\fancyhf{}
\fancyhead[R]{\sectionmark}
\fancyfoot[L]{\thepage}
\fancyfoot[C]{\licenseAcknowledgement}
\renewcommand{\headrulewidth}{0pt}
\renewcommand{\footrulewidth}{0pt}
\fi

%%%%%%%%%%%%%%%%



%\typeout{************************************************}
%\typeout{Table of Contents}
%\typeout{************************************************}


%% Edit this to change the font style
\newcommand{\sectionHeadStyle}{\sffamily\bfseries}


\makeatletter

%% part uses arabic numerals
\renewcommand*\thepart{\arabic{part}}


\ifxake\else
\renewcommand\chapterstyle{%
  \def\maketitle{%
    \addtocounter{titlenumber}{1}%
    \pagestyle{fancy}
    \phantomsection
    \addcontentsline{toc}{section}{\textbf{\thepart.\thetitlenumber\hspace{1em}\@title}}%
                    {\flushleft\small\sectionHeadStyle\@pretitle\par\vspace{-1.5em}}%
                    {\flushleft\LARGE\sectionHeadStyle\thepart.\thetitlenumber\hspace{1em}\@title \par }%
                    {\setcounter{problem}{0}\setcounter{sectiontitlenumber}{0}}%
                    \par}}





\renewcommand\sectionstyle{%
  \def\maketitle{%
    \addtocounter{sectiontitlenumber}{1}
    \pagestyle{fancy}
    \phantomsection
    \addcontentsline{toc}{subsection}{\thepart.\thetitlenumber.\thesectiontitlenumber\hspace{1em}\@title}%
    {\flushleft\small\sectionHeadStyle\@pretitle\par\vspace{-1.5em}}%
    {\flushleft\Large\sectionHeadStyle\thepart.\thetitlenumber.\thesectiontitlenumber\hspace{1em}\@title \par}%
    %{\setcounter{subsectiontitlenumber}{0}}%
    \par}}



\renewcommand\section{\@startsection{paragraph}{10}{\z@}%
                                     {-3.25ex\@plus -1ex \@minus -.2ex}%
                                     {1.5ex \@plus .2ex}%
                                     {\normalfont\large\sectionHeadStyle}}
\renewcommand\subsection{\@startsection{subparagraph}{10}{\z@}%
                                    {3.25ex \@plus1ex \@minus.2ex}%
                                    {-1em}%
                                    {\normalfont\normalsize\sectionHeadStyle}}

\fi

%% redefine Part
\renewcommand\part{%
   {\setcounter{titlenumber}{0}}
  \if@openright
    \cleardoublepage
  \else
    \clearpage
  \fi
  \thispagestyle{plain}%
  \if@twocolumn
    \onecolumn
    \@tempswatrue
  \else
    \@tempswafalse
  \fi
  \null\vfil
  \secdef\@part\@spart}

\def\@part[#1]#2{%
    \ifnum \c@secnumdepth >-2\relax
      \refstepcounter{part}%
      \addcontentsline{toc}{part}{\thepart\hspace{1em}#1}%
    \else
      \addcontentsline{toc}{part}{#1}%
    \fi
    \markboth{}{}%
    {\centering
     \interlinepenalty \@M
     \normalfont
     \ifnum \c@secnumdepth >-2\relax
       \huge\sffamily\bfseries \partname\nobreakspace\thepart
       \par
       \vskip 20\p@
     \fi
     \Huge \bfseries #2\par}%
    \@endpart}
\def\@spart#1{%
    {\centering
     \interlinepenalty \@M
     \normalfont
     \Huge \bfseries #1\par}%
    \@endpart}
\def\@endpart{\vfil\newpage
              \if@twoside
               \if@openright
                \null
                \thispagestyle{empty}%
                \newpage
               \fi
              \fi
              \if@tempswa
                \twocolumn
                \fi}



\makeatother





%\typeout{************************************************}
%\typeout{Stuff from Ximera}
%\typeout{************************************************}



\usepackage{array}  %% This is for typesetting long division
\setlength{\extrarowheight}{+.1cm}
\newdimen\digitwidth
\settowidth\digitwidth{9}
\def\divrule#1#2{
\noalign{\moveright#1\digitwidth
\vbox{\hrule width#2\digitwidth}}}





\newcommand{\RR}{\mathbb R}
\newcommand{\R}{\mathbb R}
\newcommand{\N}{\mathbb N}
\newcommand{\Z}{\mathbb Z}

\newcommand{\sagemath}{\textsf{SageMath}}


\def\d{\,d}
%\renewcommand{\d}{\mathop{}\!d}
\newcommand{\dd}[2][]{\frac{\d #1}{\d #2}}
\newcommand{\pp}[2][]{\frac{\partial #1}{\partial #2}}
\renewcommand{\l}{\ell}
\newcommand{\ddx}{\frac{d}{\d x}}



%\newcommand{\unit}{\,\mathrm}
\newcommand{\unit}{\mathop{}\!\mathrm}
\newcommand{\eval}[1]{\bigg[ #1 \bigg]}
\newcommand{\seq}[1]{\left( #1 \right)}
\renewcommand{\epsilon}{\varepsilon}
\renewcommand{\phi}{\varphi}


\renewcommand{\iff}{\Leftrightarrow}

\DeclareMathOperator{\arccot}{arccot}
\DeclareMathOperator{\arcsec}{arcsec}
\DeclareMathOperator{\arccsc}{arccsc}
\DeclareMathOperator{\sign}{sign}


%\DeclareMathOperator{\divergence}{divergence}
%\DeclareMathOperator{\curl}[1]{\grad\cross #1}
\newcommand{\lto}{\mathop{\longrightarrow\,}\limits}

\renewcommand{\bar}{\overline}

\colorlet{textColor}{black}
\colorlet{background}{white}
\colorlet{penColor}{blue!50!black} % Color of a curve in a plot
\colorlet{penColor2}{red!50!black}% Color of a curve in a plot
\colorlet{penColor3}{red!50!blue} % Color of a curve in a plot
\colorlet{penColor4}{green!50!black} % Color of a curve in a plot
\colorlet{penColor5}{orange!80!black} % Color of a curve in a plot
\colorlet{penColor6}{yellow!70!black} % Color of a curve in a plot
\colorlet{fill1}{penColor!20} % Color of fill in a plot
\colorlet{fill2}{penColor2!20} % Color of fill in a plot
\colorlet{fillp}{fill1} % Color of positive area
\colorlet{filln}{penColor2!20} % Color of negative area
\colorlet{fill3}{penColor3!20} % Fill
\colorlet{fill4}{penColor4!20} % Fill
\colorlet{fill5}{penColor5!20} % Fill
\colorlet{gridColor}{gray!50} % Color of grid in a plot

\newcommand{\surfaceColor}{violet}
\newcommand{\surfaceColorTwo}{redyellow}
\newcommand{\sliceColor}{greenyellow}




\pgfmathdeclarefunction{gauss}{2}{% gives gaussian
  \pgfmathparse{1/(#2*sqrt(2*pi))*exp(-((x-#1)^2)/(2*#2^2))}%
}





%\typeout{************************************************}
%\typeout{ORCCA Preamble.Tex}
%\typeout{************************************************}


%% \usepackage{geometry}
%% \geometry{letterpaper,total={408pt,9.0in}}
%% Custom Page Layout Adjustments (use latex.geometry)
%% \usepackage{amsmath,amssymb}
%% \usepackage{pgfplots}
\usepackage{pifont}                                         %needed for symbols, s.a. airplane symbol
\usetikzlibrary{positioning,fit,backgrounds}                %needed for nested diagrams
\usetikzlibrary{calc,trees,positioning,arrows,fit,shapes}   %needed for set diagrams
\usetikzlibrary{decorations.text}                           %needed for text following a curve
\usetikzlibrary{arrows,arrows.meta}                         %needed for open/closed intervals
\usetikzlibrary{positioning,3d,shapes.geometric}            %needed for 3d number sets tower

%% NEEDED FOR XIMERA 1
%\usetkzobj{all}       %NO LONGER VALID
%%%%%%%%%%%%%%

\usepackage{tikz-3dplot}
\usepackage{tkz-euclide}                     %needed for triangle diagrams
\usepgfplotslibrary{fillbetween}                            %shade regions of a plot
\usetikzlibrary{shadows}                                    %function diagrams
\usetikzlibrary{positioning}                                %function diagrams
\usetikzlibrary{shapes}                                     %function diagrams
%%% global colors from https://www.pcc.edu/web-services/style-guide/basics/color/ %%%
\definecolor{ruby}{HTML}{9E0C0F}
\definecolor{turquoise}{HTML}{008099}
\definecolor{emerald}{HTML}{1c8464}
\definecolor{amber}{HTML}{c7502a}
\definecolor{amethyst}{HTML}{70485b}
\definecolor{sapphire}{HTML}{263c53}
\colorlet{firstcolor}{sapphire}
\colorlet{secondcolor}{turquoise}
\colorlet{thirdcolor}{emerald}
\colorlet{fourthcolor}{amber}
\colorlet{fifthcolor}{amethyst}
\colorlet{sixthcolor}{ruby}
\colorlet{highlightcolor}{green!50!black}
\colorlet{graphbackground}{white}
\colorlet{wood}{brown!60!white}
%%% curve, dot, and graph custom styles %%%
\pgfplotsset{firstcurve/.style      = {color=firstcolor,  mark=none, line width=1pt, {Kite}-{Kite}, solid}}
\pgfplotsset{secondcurve/.style     = {color=secondcolor, mark=none, line width=1pt, {Kite}-{Kite}, solid}}
\pgfplotsset{thirdcurve/.style      = {color=thirdcolor,  mark=none, line width=1pt, {Kite}-{Kite}, solid}}
\pgfplotsset{fourthcurve/.style     = {color=fourthcolor, mark=none, line width=1pt, {Kite}-{Kite}, solid}}
\pgfplotsset{fifthcurve/.style      = {color=fifthcolor,  mark=none, line width=1pt, {Kite}-{Kite}, solid}}
\pgfplotsset{highlightcurve/.style  = {color=highlightcolor,  mark=none, line width=5pt, -, opacity=0.3}}   % thick, opaque curve for highlighting
\pgfplotsset{asymptote/.style       = {color=gray, mark=none, line width=1pt, <->, dashed}}
\pgfplotsset{symmetryaxis/.style    = {color=gray, mark=none, line width=1pt, <->, dashed}}
\pgfplotsset{guideline/.style       = {color=gray, mark=none, line width=1pt, -}}
\tikzset{guideline/.style           = {color=gray, mark=none, line width=1pt, -}}
\pgfplotsset{altitude/.style        = {dashed, color=gray, thick, mark=none, -}}
\tikzset{altitude/.style            = {dashed, color=gray, thick, mark=none, -}}
\pgfplotsset{radius/.style          = {dashed, thick, mark=none, -}}
\tikzset{radius/.style              = {dashed, thick, mark=none, -}}
\pgfplotsset{rightangle/.style      = {color=gray, mark=none, -}}
\tikzset{rightangle/.style          = {color=gray, mark=none, -}}
\pgfplotsset{closedboundary/.style  = {color=black, mark=none, line width=1pt, {Kite}-{Kite},solid}}
\tikzset{closedboundary/.style      = {color=black, mark=none, line width=1pt, {Kite}-{Kite},solid}}
\pgfplotsset{openboundary/.style    = {color=black, mark=none, line width=1pt, {Kite}-{Kite},dashed}}
\tikzset{openboundary/.style        = {color=black, mark=none, line width=1pt, {Kite}-{Kite},dashed}}
\tikzset{verticallinetest/.style    = {color=gray, mark=none, line width=1pt, <->,dashed}}
\pgfplotsset{soliddot/.style        = {color=firstcolor,  mark=*, only marks}}
\pgfplotsset{hollowdot/.style       = {color=firstcolor,  mark=*, only marks, fill=graphbackground}}
\pgfplotsset{blankgraph/.style      = {xmin=-10, xmax=10,
                                        ymin=-10, ymax=10,
                                        axis line style={-, draw opacity=0 },
                                        axis lines=box,
                                        major tick length=0mm,
                                        xtick={-10,-9,...,10},
                                        ytick={-10,-9,...,10},
                                        grid=major,
                                        grid style={solid,gray!20},
                                        xticklabels={,,},
                                        yticklabels={,,},
                                        minor xtick=,
                                        minor ytick=,
                                        xlabel={},ylabel={},
                                        width=0.75\textwidth,
                                      }
            }
\pgfplotsset{numberline/.style      = {xmin=-10,xmax=10,
                                        minor xtick={-11,-10,...,11},
                                        xtick={-10,-5,...,10},
                                        every tick/.append style={thick},
                                        axis y line=none,
                                        y=15pt,
                                        axis lines=middle,
                                        enlarge x limits,
                                        grid=none,
                                        clip=false,
                                        axis background/.style={},
                                        after end axis/.code={
                                          \path (axis cs:0,0)
                                          node [anchor=north,yshift=-0.075cm] {\footnotesize 0};
                                        },
                                        every axis x label/.style={at={(current axis.right of origin)},anchor=north},
                                      }
            }
\pgfplotsset{openinterval/.style={color=firstcolor,mark=none,ultra thick,{Parenthesis}-{Parenthesis}}}
\pgfplotsset{openclosedinterval/.style={color=firstcolor,mark=none,ultra thick,{Parenthesis}-{Bracket}}}
\pgfplotsset{closedinterval/.style={color=firstcolor,mark=none,ultra thick,{Bracket}-{Bracket}}}
\pgfplotsset{closedopeninterval/.style={color=firstcolor,mark=none,ultra thick,{Bracket}-{Parenthesis}}}
\pgfplotsset{infiniteopeninterval/.style={color=firstcolor,mark=none,ultra thick,{Kite}-{Parenthesis}}}
\pgfplotsset{openinfiniteinterval/.style={color=firstcolor,mark=none,ultra thick,{Parenthesis}-{Kite}}}
\pgfplotsset{infiniteclosedinterval/.style={color=firstcolor,mark=none,ultra thick,{Kite}-{Bracket}}}
\pgfplotsset{closedinfiniteinterval/.style={color=firstcolor,mark=none,ultra thick,{Bracket}-{Kite}}}
\pgfplotsset{infiniteinterval/.style={color=firstcolor,mark=none,ultra thick,{Kite}-{Kite}}}
\pgfplotsset{interval/.style= {ultra thick, -}}
%%% cycle list of plot styles for graphs with multiple plots %%%
\pgfplotscreateplotcyclelist{pccstylelist}{%
  firstcurve\\%
  secondcurve\\%
  thirdcurve\\%
  fourthcurve\\%
  fifthcurve\\%
}
%%% default plot settings %%%
\pgfplotsset{every axis/.append style={
  axis x line=middle,    % put the x axis in the middle
  axis y line=middle,    % put the y axis in the middle
  axis line style={<->}, % arrows on the axis
  scaled ticks=false,
  tick label style={/pgf/number format/fixed},
  xlabel={$x$},          % default put x on x-axis
  ylabel={$y$},          % default put y on y-axis
  xmin = -7,xmax = 7,    % most graphs have this window
  ymin = -7,ymax = 7,    % most graphs have this window
  domain = -7:7,
  xtick = {-6,-4,...,6}, % label these ticks
  ytick = {-6,-4,...,6}, % label these ticks
  yticklabel style={inner sep=0.333ex},
  minor xtick = {-7,-6,...,7}, % include these ticks, some without label
  minor ytick = {-7,-6,...,7}, % include these ticks, some without label
  scale only axis,       % don't consider axis and tick labels for width and height calculation
  cycle list name=pccstylelist,
  tick label style={font=\footnotesize},
  legend cell align=left,
  grid = both,
  grid style = {solid,gray!20},
  axis background/.style={fill=graphbackground},
}}
\pgfplotsset{framed/.style={axis background/.style ={draw=gray}}}
%\pgfplotsset{framed/.style={axis background/.style ={draw=gray,fill=graphbackground,rounded corners=3ex}}}
%%% other tikz (not pgfplots) settings %%%
%\tikzset{axisnode/.style={font=\scriptsize,text=black}}
\tikzset{>=stealth}
%%% for nested diagram in types of numbers section %%%
\newcommand\drawnestedsets[4]{
  \def\position{#1}             % initial position
  \def\nbsets{#2}               % number of sets
  \def\listofnestedsets{#3}     % list of sets
  \def\reversedlistofcolors{#4} % reversed list of colors
  % position and draw labels of sets
  \coordinate (circle-0) at (#1);
  \coordinate (set-0) at (#1);
  \foreach \set [count=\c] in \listofnestedsets {
    \pgfmathtruncatemacro{\cminusone}{\c - 1}
    % label of current set (below previous nested set)
    \node[below=3pt of circle-\cminusone,inner sep=0]
    (set-\c) {\set};
    % current set (fit current label and previous set)
    \node[circle,inner sep=0,fit=(circle-\cminusone)(set-\c)]
    (circle-\c) {};
  }
  % draw and fill sets in reverse order
  \begin{scope}[on background layer]
    \foreach \col[count=\c] in \reversedlistofcolors {
      \pgfmathtruncatemacro{\invc}{\nbsets-\c}
      \pgfmathtruncatemacro{\invcplusone}{\invc+1}
      \node[circle,draw,fill=\col,inner sep=0,
      fit=(circle-\invc)(set-\invcplusone)] {};
    }
  \end{scope}
  }
\ifdefined\tikzset
\tikzset{ampersand replacement = \amp}
\fi
\newcommand{\abs}[1]{\left\lvert#1\right\rvert}
%\newcommand{\point}[2]{\left(#1,#2\right)}
\newcommand{\highlight}[1]{\definecolor{sapphire}{RGB}{59,90,125} {\color{sapphire}{{#1}}}}
\newcommand{\firsthighlight}[1]{\definecolor{sapphire}{RGB}{59,90,125} {\color{sapphire}{{#1}}}}
\newcommand{\secondhighlight}[1]{\definecolor{emerald}{RGB}{20,97,75} {\color{emerald}{{#1}}}}
\newcommand{\unhighlight}[1]{{\color{black}{{#1}}}}
\newcommand{\lowlight}[1]{{\color{lightgray}{#1}}}
\newcommand{\attention}[1]{\mathord{\overset{\downarrow}{#1}}}
\newcommand{\nextoperation}[1]{\mathord{\boxed{#1}}}
\newcommand{\substitute}[1]{{\color{blue}{{#1}}}}
\newcommand{\pinover}[2]{\overset{\overset{\mathrm{\ #2\ }}{|}}{\strut #1 \strut}}
\newcommand{\addright}[1]{{\color{blue}{{{}+#1}}}}
\newcommand{\addleft}[1]{{\color{blue}{{#1+{}}}}}
\newcommand{\subtractright}[1]{{\color{blue}{{{}-#1}}}}
\newcommand{\multiplyright}[2][\cdot]{{\color{blue}{{{}#1#2}}}}
\newcommand{\multiplyleft}[2][\cdot]{{\color{blue}{{#2#1{}}}}}
\newcommand{\divideunder}[2]{\frac{#1}{{\color{blue}{{#2}}}}}
\newcommand{\divideright}[1]{{\color{blue}{{{}\div#1}}}}
\newcommand{\negate}[1]{{\color{blue}{{-}}}\left(#1\right)}
\newcommand{\cancelhighlight}[1]{\definecolor{sapphire}{RGB}{59,90,125}{\color{sapphire}{{\cancel{#1}}}}}
\newcommand{\secondcancelhighlight}[1]{\definecolor{emerald}{RGB}{20,97,75}{\color{emerald}{{\bcancel{#1}}}}}
\newcommand{\thirdcancelhighlight}[1]{\definecolor{amethyst}{HTML}{70485b}{\color{amethyst}{{\xcancel{#1}}}}}
\newcommand{\lt}{<} %% Bart: WHY?
\newcommand{\gt}{>} %% Bart: WHY?
\newcommand{\amp}{&} %% Bart: WHY?


%%% These commands break Xake
%% \newcommand{\apple}{\text{🍎}}
%% \newcommand{\banana}{\text{🍌}}
%% \newcommand{\pear}{\text{🍐}}
%% \newcommand{\cat}{\text{🐱}}
%% \newcommand{\dog}{\text{🐶}}

\newcommand{\apple}{PICTURE OF APPLE}
\newcommand{\banana}{PICTURE OF BANANA}
\newcommand{\pear}{PICTURE OF PEAR}
\newcommand{\cat}{PICTURE OF CAT}
\newcommand{\dog}{PICTURE OF DOG}


%%%%% INDEX STUFF
\newcommand{\dfn}[1]{\textbf{#1}\index{#1}}
\usepackage{imakeidx}
\makeindex[intoc]
\makeatletter
\gdef\ttl@savemark{\sectionmark{}}
\makeatother












 % for drawing cube in Optimization problem
\usetikzlibrary{quotes,arrows.meta}
\tikzset{
  annotated cuboid/.pic={
    \tikzset{%
      every edge quotes/.append style={midway, auto},
      /cuboid/.cd,
      #1
    }
    \draw [every edge/.append style={pic actions, densely dashed, opacity=.5}, pic actions]
    (0,0,0) coordinate (o) -- ++(-\cubescale*\cubex,0,0) coordinate (a) -- ++(0,-\cubescale*\cubey,0) coordinate (b) edge coordinate [pos=1] (g) ++(0,0,-\cubescale*\cubez)  -- ++(\cubescale*\cubex,0,0) coordinate (c) -- cycle
    (o) -- ++(0,0,-\cubescale*\cubez) coordinate (d) -- ++(0,-\cubescale*\cubey,0) coordinate (e) edge (g) -- (c) -- cycle
    (o) -- (a) -- ++(0,0,-\cubescale*\cubez) coordinate (f) edge (g) -- (d) -- cycle;
    \path [every edge/.append style={pic actions, |-|}]
    (b) +(0,-5pt) coordinate (b1) edge ["x"'] (b1 -| c)
    (b) +(-5pt,0) coordinate (b2) edge ["y"] (b2 |- a)
    (c) +(3.5pt,-3.5pt) coordinate (c2) edge ["x"'] ([xshift=3.5pt,yshift=-3.5pt]e)
    ;
  },
  /cuboid/.search also={/tikz},
  /cuboid/.cd,
  width/.store in=\cubex,
  height/.store in=\cubey,
  depth/.store in=\cubez,
  units/.store in=\cubeunits,
  scale/.store in=\cubescale,
  width=10,
  height=10,
  depth=10,
  units=cm,
  scale=.1,
}

\author{David Kish, Ivo Terek}
\license{Creative Commons Attribution-ShareAlike 4.0 International License}


\title{Vertex Form}

\begin{document}
\licenseY
\begin{abstract}
We explore the vertex form of a quadratic.
\end{abstract}
\maketitle


\section{The Vertex Form of a Quadratic}

      We have learned the standard form of a quadratic function's formula, which is $f(x)=ax^2+bx+c$.
      But quadratic functions also have different forms, similar to linear functions. Here, we will learn another form for quadratic functions called the vertex form.
\begin{callout} \textbf{\large{Vertex Form of a Quadratic Function}}
          A quadratic function whose graph has vertex at the point $(h,k)$ is given by
$$
f(x)=a(x-h)^2+k
$$
\end{callout}
    Using graphing technology, consider the graphs of $f(x)=x^2-6x+7$ and $g(x)=(x-3)^2-2$ on the same axes.

    We see only one parabola because these are two different forms of the same function.
    Indeed, if we convert $g(x)$ into standard form:
	\begin{align*}
		g(x)&=(x-3)^2-2\\
			&=(x^2-6x+9)-2\\
			&=x^2-6x+7
	\end{align*}
    it is clear that $f$ and $g$ are the same function.
 

    
    \begin{image}
    \begin{tikzpicture}
        \begin{axis}[]
            \addplot [firstcurve, domain=0:6] {x^2-6*x+7};
            \addplot [soliddot, color=fourthcolor] coordinates {(3,-2)} node[below] {$(3,-2)$};
        \end{axis}
    \end{tikzpicture}
    \end{image}
    $$ \text{Graph of } f(x)=x^2-6x+7 
    $$ 
    \begin{image}
        \begin{tikzpicture}
            \begin{axis}[]
                \addplot [secondcurve, domain=0:6] {(x-3)^2-2};
                \addplot [soliddot, color=fourthcolor] coordinates {(3,-2)} node[below] {$(3,-2)$};
            \end{axis}
        \end{tikzpicture}
    \end{image}
    $$\text{Graph of } g(x)=(x-3)^2-2 $$
    We see that the graphs of the two parabolas are completely identical.


    The formula given for $g$ is said to be in vertex form
    because it allows us to read the vertex without doing any calculations.
    The vertex of the parabola is $(3,-2)$.
    We can see those numbers in $g(x)=(x-3)^2-2$.
    The $x$-value is the solution to $(x-3)=0$,
    and the $y$-value is the constant
    added at the end.



      
\begin{example}
      Here are the graphs of three more functions with formulas in vertex form.
      Compare each function with the vertex of its graph.

   
       

           \begin{image}
\begin{tikzpicture}
                    \begin{axis}[]
                        \addplot [domain=-0.449:4.449] {(x-2)^2+1};
                        \addplot [firstcurve,soliddot] coordinates {(2,1)} node[right, firstcolor] {$(2,1)$};
                        \end{axis}
                \end{tikzpicture}
\end{image}
$$
r(x)=(x-2)^2+1
$$
           \begin{image}
\begin{tikzpicture}
                    \begin{axis}[]
                        \addplot[domain=-7:5.325] {-1/4*(x+1)^2+3};
                        \addplot [firstcurve,soliddot] coordinates {(-1,3)} node[above left, firstcolor] {$(-1,3)$};
                    \end{axis}
                \end{tikzpicture}
\end{image}
$$
s(x)=-\frac{1}{4}(x+1)^2+3
$$      
\begin{image}
\begin{tikzpicture}
                    \begin{axis}[]
                        \addplot [domain=-4.62:-1.38] {4*(x+3)^2-3.5};
                        \addplot [firstcurve,soliddot] coordinates {(-3,-3.5)} node[below, firstcolor] {$(-3,-3.5)$};
                    \end{axis}
                \end{tikzpicture}
\end{image}
$$
t(x)=4(x+3)^2-3.5
$$
\end{example}

      Notice that the $x$-coordinate of the vertex has the opposite sign as the value in the function formula.
      On the other hand,
      the $y$-coordinate of the vertex has the same sign as the value in the function formula.
      Let's look at an example to understand why.
      We will evaluate $r(2)$.

        $r(\substitute{2})=(\substitute{2}-2)^2+1$
        $=1$

      The $x$-value is the solution to $(x-2)=0$,
      which is positive $2$.
      When we substitute $\substitute{2}$ for $x$ we get the value $y=1$.
      Note that these coordinates are the coordinates of the vertex at $(2,1)$.
      Now we will see how to rewrite quadratics given to us in standard form in vertex form.

%\section{Completing the square}



%Consider the following squares of binomials.
%
%
%\[
% \begin{array}{|l|c|c|c|}
% \hline
% \text{Square of binomial } (x+p)^2 & p & 2p & p^2\\
% \hline
% (x+\substitute{5})^2 = x^2 + 10x + 25 & \substitute{5} & 2(\substitute{5}) = 10 & \substitute{5}^2 = 25 \\
% (x\substitute{-3})^2 = x^2 - 6x + 9 & \substitute{-3} & 2(\substitute{-3}) = -6 & (\substitute{-3})^2 = 9 \\
% (x\substitute{-12})^2 = x^2 -24 x + 144 & \substitute{-12} & 2(\substitute{-12}) = -24 & (\substitute{-12})^2 = 144 \\
% \hline
%\end{array}
%\]
%
%In each case, the square of the binomial is a quadratic trinomial,
%$$
%    (x+p)^2 = x^2 + 2px + p^2.
%$$
%Note that the coefficient of the linear term, 
%$$
%    2p,
%$$
%is twice the constant in the binomial, and the constant term of the trinomial, 
%$$
%    p^2,
%$$
%is the square of the constant in the binomial.
%   
%\begin{question}  
%Question: What is the linear term of $(x+6)^2$? 
%\begin{multipleChoice}  
%\choice{$x^2$}  
%\choice[correct]{$12x$}  
%\choice{$6x$}  
%\choice{$36$}  
%\end{multipleChoice}  
%\end{question}
%
%
%   
%   
%
%We would like to reverse the process and write a quadratic expression as the square of a binomial. For example, what constant term can we add to
%$$
%    x^2 - 16x
%$$
%to produce a perfect square trinomial? Compare the expression to the formula above:
%\begin{align*}
%    x^2 + 2px + p^2 &= (x+p)^2 \\
%    x^2 - 16x + \ \rule{0.75cm}{0.15mm} &= (x + \ \rule{0.75cm}{0.15mm})^2.
%\end{align*}   
%We see that   
%$$
%    2p = -16,
%$$
%so
%$$
%    p = \frac{1}{2}(-16) = \substitute{-8},
%$$
%and 
%$$
%    p^2 = (-8)^2 = \substitute{64}.
%$$   
%We substitute these values for $p^2$ and $p$ into the equation to find
%$$
%    x^2 - 16 x + \answer{64} = (x + \answer{- 8})^2.
%$$
%Notice that in the resulting trinomial, the constant term is equal to the square of one-half the coefficient of $x$. In other words, we can find the constant term by taking one-half the coefficient of $x$ and then squaring the result. Adding a constant term obtained in this way is called completing the square.   


\section{Converting to Vertex Form by Completing the Square}
In order to convert a quadratic from standard form to vertex form, we will use a technique called completing the square. Consider a generic $f(x) = ax^2+bx+c$, with $a$, $b$ and $c$ real numbers, and assume that $a \neq 0$. We assume this because if $a$ were zero, we would have a linear function instead of a quadratic one, which is not the focus here. The quadratic formula $$  x = \frac{-b \pm \sqrt{b^2-4ac}}{2a},$$used to find the values of $x$ for which $f(x) = 0$ or, in other words, the possible $x$-intercepts, is usually a source of grief for people studying quadratic functions for the first time. Let's try to remedy this by understanding where such a formula comes from. The main strategy here is a little algebraic device called ``completing the squares'', which is also useful for finding the coordinates $(h,k)$ of the vertex of the parabola given by the graph of $f(x)$.

Very briefly, the procedure consists in noting that $(x-h)^2 = x^2-2hx+h^2$ and paying close attention to the $-2hx$ term. Comparing this with the linear term you were given will tell you what $h$ should be. Add and subtract whatever you need to in the quadratic function you were given, to produce $(x-h)^2$ (or, more generally, the multiple $a(x-h)^2$), and whatever constant term is left outside the factored $(\cdots)^2$ will be the desired $k$. We'll see several examples below.

\begin{example}
  For each of the following quadratic functions, rewrite it in the form $f(x) = a(x-h)^2+k$, for suitable numbers $h$ and $k$. Such a point $(h,k)$ is automatically the vertex of the parabola in question.
  \begin{enumerate}[label=\alph*.]
  \item $f(x) = x^2-4x+7$. \\[.5em]
    \begin{explanation}
      Comparing $x^2-4x$ with $x^2-2hx$ suggests that $h=2$ does the trick. Since $h^2=4$, we add and subtract $4$ in the expression for the given $f(x)$, to get $$  f(x) = x^2-4x+7 =(x^2-4x+4)-4+7 = (x-2)^2+3.   $$Hence, the vertex of the parabola $y=x^2-4x+7$ is the point $(2,3)$.
    \end{explanation}
  \item $f(x) = 2x^2 + 6x+12$. \\[.5em]
    \begin{explanation}
      This time, look at $2x^2+6x = 2(x^2+3x)$, and compare $x^2+3x$ with $x^2-2hx$. It seems like $h=-3/2$ is what we need. Note that $h^2 = 9/4$. Because of the $2$ we had to factor out, we'll add and subtract $2 \cdot (9/4) = 9/2$ to complete the square. So \begin{align*}f(x) &= 2x^2+6x+12 = 2x^2+6x + \frac{9}{2} - \frac{9}{2} + 12 \\ &= 2\left(x^2+3x+\frac{9}{4}\right) -\frac{9}{2} + 12 = 2\left(x+\frac{3}{2}\right)^2 + \frac{15}{2}\\ &=  2\left(x-\left(-\frac{3}{2}\right)\right)^2 + \frac{15}{2}.\end{align*}Thus $(h,k) = (-3/2, 15/2)$ is the vertex of this parabola.
    \end{explanation}
  \item $f(x) = 3x^2 + 12x + 14$. \\[.5em]
    \begin{explanation}
      As before, we start focusing on $3x^2+12x = 3(x^2+4x)$. Compare $x^2+4x$ with $x^2-2hx$ to see that we need $h = -2$ here. Since $h^2 = 4$, let's add and subtract $3 \cdot 4 = 12$ from the original expression, to obtain
      \begin{align*}
        f(x) &= 3x^2+12x+14 = (3x^2+12x+12)-12+14 \\ &= 3(x^2+4x+4) + 2 = 3(x+2)^2+2 \\ &= 3(x-(-2))^2+2.
      \end{align*}
So, the vertex of this parabola has coordinates $(h,k) = (-2,2)$.
    \end{explanation}
  \end{enumerate}
\end{example}

Very generally, consider $f(x) = ax^2+bx+c$, with $a$, $b$ and $c$ real numbers, with $a \neq 0$. Let's repeat everything we have done in the previous example, with $a$, $b$ and $c$ instead of concrete numbers. Here are the steps we can follow:

\begin{itemize}
\item First, look only at $ax^2 + bx = a(x^2 + bx/a)$.
\item Compare $x^2+bx/a$ with $x^2-2hx$. The $h$ we need here is $h = -b/2a$. Note that $h^2 = b^2/4a^2$.
\item Because of the $a$ we had to factor out in the beginning, let's add and subtract $a \cdot (b^2/4a^2) = b^2/4a$ from the original expression.
\item Compute \begin{align*}  ax^2+bx+c &= \left(ax^2+bx + \frac{b^2}{4a}\right) - \frac{b^2}{4a}+c \\ &= a\left(x^2+\frac{bx}{a} + \frac{b^2}{4a^2}\right) + \frac{-b^2+4ac}{4a} \\ &= a\left(x+\frac{b}{2a}\right)^2+\frac{-b^2+4ac}{4a} \\ &= a\left(x - \left(-\frac{b}{2a}\right)\right)^2 + \frac{-(b^2-4ac)}{4a}. \end{align*}
\item Hence, the vertex of the parabola described by $y=ax^2+bx+c$ is given by $$(h,k) = \left(-\frac{b}{2a}, -\frac{b^2-4ac}{4a}\right).$$
\end{itemize}

And with those computations in place, we are in fact very close to understanding where the quadratic formula came from. Solving $ax^2+bx+c=0$ is, by the above, the same as solving $$a\left(x+\frac{b}{2a}\right)^2 - \frac{b^2-4ac}{4a} = 0.$$Reorganize as $$a\left(x+\frac{b}{2a}\right)^2 = \frac{b^2-4ac}{4a},$$and divide by $a$ to get $$\left(x+\frac{b}{2a}\right)^2 = \frac{b^2-4ac}{4a^2}.$$Assuming that $b^2-4ac \geq 0$, we may take square roots on both sides: $$\left|x+\frac{b}{2a} \right| = \frac{\sqrt{b^2-4ac}}{2|a|}.$$
Eliminating the absolute values, we have $$x+\frac{b}{2a} = \frac{\pm\sqrt{b^2-4ac}}{2a}.$$Now solve for $x$, by putting everything on the right side under a common denominator: $$x = \frac{-b\pm \sqrt{b^2-4ac}}{2a}.$$Mystery solved. For each choice of sign $\pm$, we get one $x$-intercept. Now, we also observe that the average of such solutions does give the $x$-coordinate of the vertex, as you might expect: $$\frac{1}{2}\left(\frac{-b-\sqrt{b^2-4ac}}{2a} + \frac{-b+\sqrt{b^2-4ac}}{2a}\right) = \frac{1}{2} \left(\frac{-2b}{2a}\right) = -\frac{b}{2a}.$$

The $y$-coordinate of the vertex will be, naturally, $k = f(-b/2a)$. This can also be used as a shortcut to write quadratic functions in vertex-form.

\begin{example}
  Write $f(x) = x^2 - 8x+15$ in vertex-form $f(x) = a(x-h)^2+k$ without completing the squares explicitly.\\[.5em]
  \begin{explanation}
    We can factor the quadratic as $f(x) = (x-3)(x-5)$. This means that the $x$-coordinate of the vertex is $h = (3+5)/2 = 4$, and so $$k =f(4) = 4^2-8\cdot 4 + 15 = -1.$$Hence $x^2-8x+15 = (x-4)^2-1$. As a quick sanity-check (particular to \emph{this} example), note that factoring this last result as a difference of squares (because $1^2=1$) does give $(x-3)(x-5)$.
  \end{explanation}
\end{example}
%Now we will use completing the square to solve quadratic equations. First, we will solve equations in which the coefficient of the squared term is 1. Consider the function
%$$
%    f(x) = x^2 - 6x - 7,
%$$
%and follow the steps to find the solutions.
%
%\textbf{Step 1}
%
%Begin by separating the constant term from the other terms of the equation, to get
%$$
%    f(x) = (x^2 - 6x\ \rule{1cm}{0.15mm}) + (-7\ \rule{1cm}{0.15mm}).
%$$
%
%\textbf{Step 2}
%
%Now complete the square in the parentheses on the left. Because
%$$
%    p = \frac{1}{2}(-6) = -3
%$$
%and
%$$
%    p^2 = (-3)^2 = 9,
%$$
%we add and subtract $9$ in our equation to get
%$$
%    f(x) = (x^2-6x+\substitute{9}) + (-7 - \substitute{9}).
%$$
%Notice that this doesn't change the value of the function, since we're simply adding $9 - 9 = 0$ and re-grouping.
%
%\textbf{Step 3}
%
%The expression in the left set of parentheses is now the square of a binomial, namely $(x-3)^2$. We write the left side in its square form and simplify the right side, which gives us the final vertex form,
%$$
%    f(x) = (x-3)^2 - 16.
%$$
%As before, you can verify that this function is equivalent to the original by expanding the binomial.
%
%\begin{image}
%\begin{tikzpicture}
%    \begin{axis}[width=0.75\linewidth,
%            			xmin=-2.25,xmax=7.25,
%           				ymin=-2.25,ymax=20.25,
%            			minor ytick=,minor xtick=,
%	                	xtick={-2,...,7}, ytick={-20,-18,...,2},
%            			clip=false,ymax=2,ymin=-20]
%        \addplot [firstcurve, domain=-1.25:7.25] {x^2-6*x-7} node[pos=0.3, above right]{$\Large{y=x^2-6x-7}$};
%        \addplot [soliddot, color=fourthcolor] coordinates {(3,-16)} node[below] {$(3,-16)$};
%    \end{axis}
%\end{tikzpicture}
%\end{image}
%$$ \text{Graph of } f(x)=x^2 - 6x - 7 $$ 
%
%\begin{image}
%    \begin{tikzpicture}
%        \begin{axis}[width=0.75\linewidth,
%            			xmin=-2.25,xmax=7.25,
%           				ymin=-2.25,ymax=20.25,
%            			minor ytick=,minor xtick=,
%	                	xtick={-2,...,7}, ytick={-20,-18,...,2},
%            			clip=false,ymax=2,ymin=-20]
%            \addplot [secondcurve, domain=-1.25:7.25] {(x-3)^2-16} node[pos=0.3, above right]{$\Large{y=(x-3)^2-16}$};
%        \addplot [soliddot, color=fourthcolor] coordinates {(3,-16)} node[below] {$(3,-16)$};
%        \end{axis}
%    \end{tikzpicture}
%\end{image}
%$$ \text{Graph of } g(x)=(x-3)^2-16 $$
%
%\begin{example}
%Convert $f(x) = x^2 - 4x - 3$ to vertex form by completing the square.
%
%\begin{explanation}
%\begin{enumerate}[label=\arabic*.]
%    \item Separate the constant term:
%    $$
%        f(x) = (x^2 - 4x\ \rule{1cm}{0.15mm}) + (-3\ \rule{1cm}{0.15mm}).
%    $$
%    \item Complete the square in the left parentheses. The coefficient of $x$ is $-4$, so
%    $$
%        p = \frac{1}{2}(-4) = -2,
%    $$
%    and
%    $$
%        p^2 = (-2)^2 = 4.
%    $$
%    Add and subtract $4$ to the equation:
%    $$
%        f(x) = (x^2 - 4x + \substitute{4}) + (-3-\substitute{4})
%    $$
%    \item Factor the left into a binomial square and simplify the right:
%    $$
%        f(x) = (x-2)^2 - 7.
%    $$
%\end{enumerate}
%\end{explanation}
%\end{example}
%
%\begin{callout}
%Take a moment to think about how to tell if $x^2 + bx + c$ is the square of a binomial.
%\end{callout}
%   
%\subsection{The General Case}
%   
%What we've done so far only works if the coefficient of $x^2$ is $1$. If we want to put a quadratic with leading coefficient different from $1$ into vertex form, we do the same steps, but in addition factor by the leading coefficient as follows.
%
%\begin{example} 
%Put $f(x) = 2x^2 - 6x - 5$ into vertex form.
%\begin{explanation}
%\begin{enumerate}
%    \item[1(a).] As before, start by separating the constant term from the $x$ and $x^2$ terms: 
%    $$
%    f(x) = (2x^2 - 6x\ \rule{1cm}{0.15mm}) - (5 \ \rule{1cm}{0.15mm})
%    $$
%    \item[1(b).] Because the coefficient of $x^2$ is $2$, we must factor the left set of parentheses by $2$.
%    $$
%    f(x) = \substitute{2} (x^2 - 3x\ \rule{1cm}{0.15mm}) - (5 \ \rule{1cm}{0.15mm})
%    $$
%    \item[2.] Complete the square in the left parentheses. The coefficient of $x$ is $-3$, so
%    $$
%        p = \frac{1}{2}(-3) = -\frac{3}{2},
%    $$
%    and
%    $$
%        p^2 = \left(-\frac{3}{2}\right)^2 = \frac{9}{4}.
%    $$
%    We add $\frac{9}{4}$ to the equation as before, but now we must subtract by $2 \cdot \frac{9}{4}$ instead of just $\frac{9}{4}$ to account for the extra $2$ coming from the coefficient of $x^2$:
%    $$
%        f(x) = \substitute{2}\left(x^2 - 3x +\substitute{\frac{9}{4}}\right) + \left(-\frac{5}{2} -\substitute{2 \cdot \frac{9}{4}}\right).
%    $$
%    \item[3.] Factor the left parentheses into a binomial square and simplify the constants in the right parentheses:
%    $$
%        f(x) =2 \left(x-\frac{3}{2}\right)^2 - \frac{19}{4}.
%    $$
%\end{enumerate}
%\end{explanation}
%\end{example}
%
%
%\begin{summary}
%\begin{enumerate}
%    \item[1.] \begin{enumerate}
%        \item[(a)] Write the function in standard form and separate the constant term from the $x$ and $x^2$ terms.
%
%        \item[(b)] Factor the $x$ and $x^2$ terms by the coefficient of the quadratic term.
%    \end{enumerate}
%    \item[2.] Complete the square for the $x$ and $x^2$ terms:
%    \begin{enumerate}
%        \item[(a)] Multiply the coefficient of the first-degree term by one-half, then square the result.
%        \item[(b)] Add the value obtained in (a) inside the parentheses containing $x$ and $x^2$, then subtract this value times the original coefficient of $x^2$ from the constant term.
%    \end{enumerate}
%    \item[3.] Write the first set of parentheses as the square of a binomial. Simplify the constant term.
%\end{enumerate}
%\end{summary}

% %\section{Exponent Rules}
%   %%%%%%%%%%%%%%%%
%    \section{Product Rule}
%        If we write out $3^5\cdot 3^2$ without using exponents,
%        we'd have:
%$$
% 3^5 \cdot 3^2 = \left(3 \cdot 3\cdot 3\cdot 3\cdot 3\right) \cdot \left(3 \cdot 3\right)
%$$
%        If we then count how many $3$s are being multiplied together,
%        we find we have $5+2=7$, a total of seven $3$s.
%        So $3^5\cdot 3^2$ simplifies like this:
%       $$
%          3^5\cdot 3^2 = 3^{5+2} = 3^7
%$$
%\begin{example}          
%Simplify $x^2\cdot x^3$. \\
%\begin{explanation}       
%          To simplify $x^2\cdot x^3$,
%          we write this out in its expanded form,
%          as a product of $x$'s, we have
%$$
%            x^2\cdot x^3 =(x\cdot x)(x \cdot x \cdot x)
%            =x\cdot x\cdot x \cdot x \cdot x
%            =x^5
%   $$   
%          Note that we obtained the exponent of $5$ by adding $2$ and $3$.
%  \end{explanation}
%\end{example}
%
%      This example demonstrates our first exponent rule,
%      the Product Rule:\\
%\begin{callout}
%\textbf{ \Large Product Rule of Exponents} \\
%      When multiplying two expressions that have the same base,
%      we can simplify the product by adding the exponents.
%        $$
%x^m \cdot x^n = x^{m+n}
%$$
%\end{callout}
%   Recall that $x=x^1$. It helps to remember this when multiplying certain expressions together.
%\begin{example}        
%  Multiply $x(x^3+2)$ by using the distributive property.\\
%\begin{explanation}
%          According to the distributive property,
%        $x(x^3+2)=x\cdot x^3 + x\cdot2$
%          How can we simplify that term $x\cdot x^3$?
%          It's really the same as $x^1\cdot x^3$,
%          so according to the Product Rule, it is $x^4$.
%          So we have:
%        $$
%            x(x^3+2)=x\cdot x^3 + x\cdot2
%            =x^4+2x
%$$
%\end{explanation}
%\end{example}         
%
%%%%%%%%%%%%%%%%%
% \section{Power to a Power Rule}
%
%        If we write out $\left(3^5\right)^2$ without using exponents,
%        we'd have $3^5$ multiplied by itself:
%   $$
%         \left(3^5\right)^2 = \left(3^5\right)\cdot \left(3^5\right)
%         = \left(3\cdot 3\cdot 3\cdot 3 \cdot 3 \right) \cdot \left(3 \cdot 3\cdot 3\cdot 3\cdot 3\right)
%      $$ 
%        If we again count how many $3$s are being multiplied,
%        we have a total of two groups each with five $3$s.
%        So we'd have $2\cdot 5=10$ instances of a $3$.
%        So $\left(3^5\right)^2$ simplifies like this:
%   
%          $\left(3^5\right)^2 = 3^{2\cdot 5}$
%          $= 3^{10}$
%
%
%\begin{example}
%          Simplify $\left(x^2\right)^3$.\\
%\begin{explanation}
%          To simplify $\left(x^2\right)^3$,
%          we write this out in its expanded form,
%          as a product of $x$'s, we have
%            $\left(x^2\right)^3 =\left(x^2\right) \cdot \left(x^2\right)\cdot\left(x^2\right)$
%            $=(x \cdot x)\cdot (x \cdot x)\cdot (x \cdot x)$
%            $=x^6$
%     \end{explanation}
%          Note that we obtained the exponent of $6$ by multiplying $2$ and $3$.
%\end{example}
%      This demonstrates our second exponent rule,
%      the Power to a Power Rule:
%\begin{callout}
%\textbf{ \Large Power to a Power Rule} \\
%      when a base is raised to an exponent and that expression is raised to another exponent,
%      we multiply the exponents.
%   $$
%      \left(x^m\right)^n = x^{m \cdot n}
%   $$
%\end{callout}
%
%%%%%%%%%%%%%%%%%%%
%      \section{ Product to a Power Rule}
%        The third exponent rule deals with having multiplication inside a set of parentheses and an exponent outside the parentheses.
%        If we write out $\left(3t\right)^5$ without using an exponent,
%        we'd have $3t$ multiplied by itself five times:
%$$
%      (3t)^5= (3t)(3t)(3t)(3t)(3t)
%$$
%        Keeping in mind that there is multiplication between every $3$ and $t$,
%        and multiplication between all of the parentheses pairs,
%        we can reorder and regroup the factors:
%          $\left(3t\right)^5 = (3\cdot t)\cdot (3\cdot t)\cdot (3\cdot t)\cdot (3\cdot t)\cdot (3\cdot t)$
%          $= \left(3\cdot 3\cdot 3\cdot 3\cdot 3 \right) \cdot \left(t \cdot t \cdot t \cdot t \cdot t\right)$
%          $= 3^5 t^5$
%        We could leave it written this way if $3^5$ feels especially large.
%        But if you are able to evaluate $3^5=243$,
%        then perhaps a better final version of this expression is $243t^5$.
% 
%        We essentially applied the outer exponent to each factor inside the parentheses.
%        It is important to see how the exponent $5$ applied to \textbf{both} the $3$ \textbf{and} the $t$,
%        not just to the $t$.
%   
%
% \begin{example}   
%          Simplify $(xy)^5$.
%    
%          To simplify $(xy)^5$,
%          we write this out in its expanded form,
%          as a product of $x$'s and $y$'s, we have
%
%            $(xy)^5 =(x \cdot y) \cdot (x \cdot y) \cdot (x \cdot y) \cdot (x \cdot y) \cdot (x \cdot y)$
%            $=(x \cdot x \cdot x \cdot x \cdot x) \cdot (y \cdot y \cdot y \cdot y \cdot y)$
%            $=x^5 y^5$
%
%          Note that the exponent on $xy$ can simply be applied to both $x$ and $y$.
%\end{example}
%
%
%      This demonstrates our third exponent rule,
%      the Product to a Power Rule:
%\begin{callout}
%\textbf{ \Large Product to a Power Rule} \\
%      When a product is raised to an exponent,
%      we can apply the exponent to each factor in the product.
%$$
%        \left(x\cdot y\right)^n = x^{n}\cdot y^{n}
% $$
%\end{callout}
%
%
%%%%%%%%%%%%%%%%%%
%      \section{Summary of the Rules of Exponents for Multiplication}
%  
%
%
%          If $a$ and $b$ are real numbers,
%          and $m$ and $n$ are positive integers,
%          then we have the following rules:
% 
%\textbf{Product Rule}
%$$
%            a^{m} \cdot a^{n} = a^{m+n}
%$$
%\textbf{Power to a Power Rule}
%   $$
%           (a^{m})^{n} = a^{m\cdot n}
%   $$
% \textbf{ Product to a Power Rule}
%   $$
%           (ab)^{m} = a^{m} \cdot b^{m}
%   $$  
%      Many examples will make use of more than one exponent rule.
%      In deciding which exponent rule to work with first,
%      it's important to remember that the order of operations still applies.
%\begin{example} Simplify the following expression.
%                $\left(3^7r^5\right)^4$\\
%    \begin{explanation}
%                Since we cannot simplify anything inside the parentheses, we'll begin simplifying this expression using the Product to a Power rule.
%                We'll apply the outer exponent of 4 to each factor inside the parentheses.
%                Then we'll use the Power to a Power Rule to finish the simplification process.
%      $$          
%                  \left(3^7r^5\right)^4 = \left(3^7\right)^4 \cdot \left(r^5\right)^4
%                  = 3^{7\cdot4} \cdot r^{5\cdot 4}
%                  = 3^{28}r^{20}
%         $$      
%                Note that $3^{28}$ is too large to actually compute, even with a calculator,
%                so we leave it written as $3^{28}$.
%\end{explanation}
%\end{example}
%\begin{example}
%Simplify the following expression.
%       $\left(t^3\right)^2\cdot \left(t^4\right)^5$\\
%\begin{explanation}
%                According to the order of operations,
%                we should first simplify any exponents before carrying out any multiplication.
%                Therefore, we'll begin simplifying this by applying the Power to a Power Rule and then finish using the Product Rule.
%           $$
%                  \left(t^3\right)^2\cdot \left(t^4\right)^5 = t^{3\cdot2}\cdot t^{4\cdot5}
%                  = t^6 \cdot t^{20}
%                  = t^{6+20}
%                  = t^{26}
%$$
%\end{explanation}
%\end{example}
% \begin{remark} 
%        We cannot simplify an expression like $x^2y^3$ using the Product Rule,
%        as the factors $x^2$ and $y^3$ do not have the same base.
%\end{remark}

\begin{summary}
\begin{itemize}
	\item We can always rewrite quadratic equations into vertex form, from which we can read off the vertex of the associated parabola
	\item For an arbitrary quadratic function $f(x) = ax^2+bx+c$, we found formulas for the coordinates $(h,k)$ of its vertex by completing the squares. We have also concluded that the $x$-coordinate $h$ of the vertex is, in fact, the average of the $x$-intercepts of the parabola described as the graph of $y=f(x)$ and, in particular, we have seen how to deduce the famous quadratic formula with this general strategy.
\end{itemize}
\end{summary}

\end{document}
