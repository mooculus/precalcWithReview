\documentclass{ximera}


\graphicspath{
  {./}
  {1-1QuantitativeReasoning/}
  {1-2RelationsAndGraphs/}
  {1-3ChangingInTandem/}
  {2-1LinearEquations/}
  {2-2LinearModeling/}
  {2-3ExponentialModeling/}
  {3-1WhatIsAFunction/}
  {3-2FunctionProperties/}
  {3-3AverageRatesOfChange/}
  {4-1BuildingNewFunctions/}
  {4-2Polynomials/}
  {5-1RationalFunctions/}
   {5-2ExponentialFunctions/}
  {6-1Domain/}
  {6-2Range/}
  {6-3CompositionOfFunctions/}
  {7-1ZerosOfFunctions/}
  {7-XZerosOfPolynomials/}
  {7-2ZerosOfFamousFunctions/}
  {8-0Review/}
  {8-1FunctionTransformations/}
  {8-2SolvingInequalities/}
  {8-3FunctionTransformationsProject/}
  {9-1RightTriangleTrig/}
  {9-2TheUnitCircle/}
  {9-3TrigIdentities/}
  {10-1UnitCircleToFunctionGraph/}
  {10-2TrigFunctions/}
  {10-3SomeApplicationsOfTrig/}
  {11-1InverseFunctionsRevisited/}
  {11-2Logarithms/}
  {11-3InverseTrig/}
  {12-1SystemsOfEquations/}
  {12-2NonlinearSystems/}
  {12-3ApplicationsOfSystems/}
  {13-1SecantLinesRevisited/}
  {13-2Functions-TheBigPicture/}
  {14-1DisplacementVsDistance/}
  {1-1QuantitativeReasoning/exercises/}
  {1-2RelationsAndGraphs/exercises/}
  {../1-3ChangingInTandem/exercises/}
  {../2-1LinearEquations/exercises/}
  {../2-2LinearModeling/exercises/}
  {../2-3ExponentialModeling/exercises/}
  {../3-1WhatIsAFunction/exercises/}
  {../3-2FunctionProperties/exercises/}
  {../3-3AverageRatesOfChange/exercises/}
  {../5-2ExponentialFunctions/exercises/}
  {../4-1BuildingNewFunctions/exercises/}
  {../4-2Polynomials/exercises/}
  {../5-1RationalFunctions/exercises/}
  {../6-1Domain/exercises/}
  {../6-2Range/exercises/}
  {../6-3CompositionOfFunctions/exercises/}
  {../7-1ZerosOfFunctions/exercises/}
  {../7-XZerosOfPolynomials/exercises/}
  {../7-2ZerosOfFamousFunctions/exercises/}
  {../8-1FunctionTransformations/exercises/}
  {../12-1SystemsOfEquations/exercises/}
  {../8-3FunctionTransformationsProject/exercises/}
  {../8-0Review/exercises/}
  {../8-2SolvingInequalities/exercises/}
  {../8-3FunctionTransformationsProject/exercises/}
  {../9-1RightTriangleTrig/exercises/}
  {../9-2TheUnitCircle/exercises/}
  {../9-3TrigIdentities/exercises/}
  {../10-1UnitCircleToFunctionGraph/exercises/}
  {../10-2TrigFunctions/exercises/}
  {../10-3SomeApplicationsOfTrig/exercises/}
  {../11-1InverseFunctionsRevisited/exercises/}
  {../11-2Logarithms/exercises/}
  {../11-3InverseTrig/exercises/}
  {../12-1SystemsOfEquations/exercises/}
  {../12-2NonlinearSystems/exercises/}
  {../12-3ApplicationsOfSystems/exercises/}
  {../13-1SecantLinesRevisited/exercises/}
  {../13-2Functions-TheBigPicture/exercises/}
  {../14-1DisplacementVsDistance/exercises/}
}

\DeclareGraphicsExtensions{.pdf,.png,.jpg,.eps}

\newcommand{\mooculus}{\textsf{\textbf{MOOC}\textnormal{\textsf{ULUS}}}}

\usepackage[makeroom]{cancel} %% for strike outs

\ifxake
\else
\usepackage[most]{tcolorbox}
\fi


%\typeout{************************************************}
%\typeout{New Environments}
%\typeout{************************************************}

%% to fix for web can be removed when deployed offically with ximera2
\let\image\relax\let\endimage\relax
\NewEnviron{image}{% 
  \begin{center}\BODY\end{center}% center
}



\NewEnviron{folder}{
      \addcontentsline{toc}{section}{\textbf{\BODY}}
}

\ifxake
\let\summary\relax
\let\endsummary\relax
\newtheorem*{summary}{Summary}
\newtheorem*{callout}{Callout}
\newtheorem*{overview}{Overview}
\newtheorem*{objectives}{Objectives}
\newtheorem*{motivatingQuestions}{Motivating Questions}
\newtheorem*{MM}{Metacognitive Moment}
      
%% NEEDED FOR XIMERA 2
%\ximerizedEnvironment{summary}
%\ximerizedEnvironment{callout}
%\ximerizedEnvironment{overview} 
%\ximerizedEnvironment{objectives}
%\ximerizedEnvironment{motivatingQuestions}
%\ximerizedEnvironment{MM}
\else
%% CALLOUT
\NewEnviron{callout}{
  \begin{tcolorbox}[colback=blue!5, breakable,pad at break*=1mm]
      \BODY
  \end{tcolorbox}
}
%% MOTIVATING QUESTIONS
\NewEnviron{motivatingQuestions}{
  \begin{tcolorbox}[ breakable,pad at break*=1mm]
    \textbf{\Large Motivating Questions}\hfill
    %\begin{itemize}[label=\textbullet]
      \BODY
    %\end{itemize}
  \end{tcolorbox}
}
%% OBJECTIVES
\NewEnviron{objectives}{  
    \vspace{.5in}
      %\begin{tcolorbox}[colback=orange!5, breakable,pad at break*=1mm]
    \textbf{\Large Learning Objectives}
    \begin{itemize}[label=\textbullet]
      \BODY
    \end{itemize}
    %\end{tcolorbox}
}
%% DEFINITION
\let\definition\relax
\let\enddefinition\relax
\NewEnviron{definition}{
  \begin{tcolorbox}[ breakable,pad at break*=1mm]
    \noindent\textbf{Definition}~
      \BODY
  \end{tcolorbox}
}
%% OVERVIEW
\let\overview\relax
\let\overview\relax
\NewEnviron{overview}{
  \begin{tcolorbox}[ breakable,pad at break*=1mm]
    \textbf{\Large Overview}
    %\begin{itemize}[label=\textbullet] %% breaks Xake
      \BODY
    %\end{itemize}
  \end{tcolorbox}
}
%% SUMMARY
\let\summary\relax
\let\endsummary\relax
\NewEnviron{summary}{
  \begin{tcolorbox}[ breakable,pad at break*=1mm]
    \textbf{\Large Summary}
    %\begin{itemize}[label=\textbullet] %% breaks Xake
      \BODY
    %\end{itemize}
  \end{tcolorbox}
}
%% REMARK
\let\remark\relax
\let\endremark\relax
\NewEnviron{remark}{
  \begin{tcolorbox}[colback=green!5, breakable,pad at break*=1mm]
    \noindent\textbf{Remark}~
      \BODY
  \end{tcolorbox}
}
%% EXPLANATION
\let\explanation\relax
\let\endexplanation\relax
\NewEnviron{explanation}{
    \normalfont
    \noindent\textbf{Explanation}~
      \BODY
}
%% EXPLORATION
\let\exploration\relax
\let\endexploration\relax
\NewEnviron{exploration}{
  \begin{tcolorbox}[colback=yellow!10, breakable,pad at break*=1mm]
    \noindent\textbf{Exploration}~
      \BODY
  \end{tcolorbox}
}
%% METACOGNITIVE MOMENTS
\let\MM\relax
\let\endMM\relax
\NewEnviron{MM}{
  \begin{tcolorbox}[colback=pink!15, breakable,pad at break*=1mm]
    \noindent\textbf{Metacognitive Moment}~
      \BODY
  \end{tcolorbox}
}


\fi





%Notes on what envirnoment to use:  Example with Explanation in text; if they are supposed to answer- Problem; no answer - Exploration


%\typeout{************************************************}
%% Header and footers
%\typeout{************************************************}

\newcommand{\licenseAcknowledgement}{Licensed under Creative Commons 4.0}
\newcommand{\licenseAPC}{\renewcommand{\licenseAcknowledgement}{\textbf{Acknowledgements:} Active Prelude to Calculus (https://activecalculus.org/prelude) }}
\newcommand{\licenseSZ}{\renewcommand{\licenseAcknowledgement}{\textbf{Acknowledgements:} Stitz Zeager Open Source Mathematics (https://www.stitz-zeager.com/) }}
\newcommand{\licenseAPCSZ}{\renewcommand{\licenseAcknowledgement}{\textbf{Acknowledgements:} Active Prelude to Calculus (https://activecalculus.org/prelude) and Stitz Zeager Open Source Mathematics (https://www.stitz-zeager.com/) }}
\newcommand{\licenseORCCA}{\renewcommand{\licenseAcknowledgement}{\textbf{Acknowledgements:}Original source material, products with readable and accessible
math content, and other information freely available at pcc.edu/orcca.}}
\newcommand{\licenseY}{\renewcommand{\licenseAcknowledgement}{\textbf{Acknowledgements:} Yoshiwara Books (https://yoshiwarabooks.org/)}}
\newcommand{\licenseOS}{\renewcommand{\licenseAcknowledgement}{\textbf{Acknowledgements:} OpenStax College Algebra (https://openstax.org/details/books/college-algebra)}}
\newcommand{\licenseAPCSZCSCC}{\renewcommand{\licenseAcknowledgement}{\textbf{Acknowledgements:} Active Prelude to Calculus (https://activecalculus.org/prelude), Stitz Zeager Open Source Mathematics (https://www.stitz-zeager.com/), CSCC PreCalculus and Calculus texts (https://ximera.osu.edu/csccmathematics)}}

\ifxake\else %% do nothing on the website
\usepackage{fancyhdr}
\pagestyle{fancy}
\fancyhf{}
\fancyhead[R]{\sectionmark}
\fancyfoot[L]{\thepage}
\fancyfoot[C]{\licenseAcknowledgement}
\renewcommand{\headrulewidth}{0pt}
\renewcommand{\footrulewidth}{0pt}
\fi

%%%%%%%%%%%%%%%%



%\typeout{************************************************}
%\typeout{Table of Contents}
%\typeout{************************************************}


%% Edit this to change the font style
\newcommand{\sectionHeadStyle}{\sffamily\bfseries}


\makeatletter

%% part uses arabic numerals
\renewcommand*\thepart{\arabic{part}}


\ifxake\else
\renewcommand\chapterstyle{%
  \def\maketitle{%
    \addtocounter{titlenumber}{1}%
    \pagestyle{fancy}
    \phantomsection
    \addcontentsline{toc}{section}{\textbf{\thepart.\thetitlenumber\hspace{1em}\@title}}%
                    {\flushleft\small\sectionHeadStyle\@pretitle\par\vspace{-1.5em}}%
                    {\flushleft\LARGE\sectionHeadStyle\thepart.\thetitlenumber\hspace{1em}\@title \par }%
                    {\setcounter{problem}{0}\setcounter{sectiontitlenumber}{0}}%
                    \par}}





\renewcommand\sectionstyle{%
  \def\maketitle{%
    \addtocounter{sectiontitlenumber}{1}
    \pagestyle{fancy}
    \phantomsection
    \addcontentsline{toc}{subsection}{\thepart.\thetitlenumber.\thesectiontitlenumber\hspace{1em}\@title}%
    {\flushleft\small\sectionHeadStyle\@pretitle\par\vspace{-1.5em}}%
    {\flushleft\Large\sectionHeadStyle\thepart.\thetitlenumber.\thesectiontitlenumber\hspace{1em}\@title \par}%
    %{\setcounter{subsectiontitlenumber}{0}}%
    \par}}



\renewcommand\section{\@startsection{paragraph}{10}{\z@}%
                                     {-3.25ex\@plus -1ex \@minus -.2ex}%
                                     {1.5ex \@plus .2ex}%
                                     {\normalfont\large\sectionHeadStyle}}
\renewcommand\subsection{\@startsection{subparagraph}{10}{\z@}%
                                    {3.25ex \@plus1ex \@minus.2ex}%
                                    {-1em}%
                                    {\normalfont\normalsize\sectionHeadStyle}}

\fi

%% redefine Part
\renewcommand\part{%
   {\setcounter{titlenumber}{0}}
  \if@openright
    \cleardoublepage
  \else
    \clearpage
  \fi
  \thispagestyle{plain}%
  \if@twocolumn
    \onecolumn
    \@tempswatrue
  \else
    \@tempswafalse
  \fi
  \null\vfil
  \secdef\@part\@spart}

\def\@part[#1]#2{%
    \ifnum \c@secnumdepth >-2\relax
      \refstepcounter{part}%
      \addcontentsline{toc}{part}{\thepart\hspace{1em}#1}%
    \else
      \addcontentsline{toc}{part}{#1}%
    \fi
    \markboth{}{}%
    {\centering
     \interlinepenalty \@M
     \normalfont
     \ifnum \c@secnumdepth >-2\relax
       \huge\sffamily\bfseries \partname\nobreakspace\thepart
       \par
       \vskip 20\p@
     \fi
     \Huge \bfseries #2\par}%
    \@endpart}
\def\@spart#1{%
    {\centering
     \interlinepenalty \@M
     \normalfont
     \Huge \bfseries #1\par}%
    \@endpart}
\def\@endpart{\vfil\newpage
              \if@twoside
               \if@openright
                \null
                \thispagestyle{empty}%
                \newpage
               \fi
              \fi
              \if@tempswa
                \twocolumn
                \fi}



\makeatother





%\typeout{************************************************}
%\typeout{Stuff from Ximera}
%\typeout{************************************************}



\usepackage{array}  %% This is for typesetting long division
\setlength{\extrarowheight}{+.1cm}
\newdimen\digitwidth
\settowidth\digitwidth{9}
\def\divrule#1#2{
\noalign{\moveright#1\digitwidth
\vbox{\hrule width#2\digitwidth}}}





\newcommand{\RR}{\mathbb R}
\newcommand{\R}{\mathbb R}
\newcommand{\N}{\mathbb N}
\newcommand{\Z}{\mathbb Z}

\newcommand{\sagemath}{\textsf{SageMath}}


\def\d{\,d}
%\renewcommand{\d}{\mathop{}\!d}
\newcommand{\dd}[2][]{\frac{\d #1}{\d #2}}
\newcommand{\pp}[2][]{\frac{\partial #1}{\partial #2}}
\renewcommand{\l}{\ell}
\newcommand{\ddx}{\frac{d}{\d x}}



%\newcommand{\unit}{\,\mathrm}
\newcommand{\unit}{\mathop{}\!\mathrm}
\newcommand{\eval}[1]{\bigg[ #1 \bigg]}
\newcommand{\seq}[1]{\left( #1 \right)}
\renewcommand{\epsilon}{\varepsilon}
\renewcommand{\phi}{\varphi}


\renewcommand{\iff}{\Leftrightarrow}

\DeclareMathOperator{\arccot}{arccot}
\DeclareMathOperator{\arcsec}{arcsec}
\DeclareMathOperator{\arccsc}{arccsc}
\DeclareMathOperator{\sign}{sign}


%\DeclareMathOperator{\divergence}{divergence}
%\DeclareMathOperator{\curl}[1]{\grad\cross #1}
\newcommand{\lto}{\mathop{\longrightarrow\,}\limits}

\renewcommand{\bar}{\overline}

\colorlet{textColor}{black}
\colorlet{background}{white}
\colorlet{penColor}{blue!50!black} % Color of a curve in a plot
\colorlet{penColor2}{red!50!black}% Color of a curve in a plot
\colorlet{penColor3}{red!50!blue} % Color of a curve in a plot
\colorlet{penColor4}{green!50!black} % Color of a curve in a plot
\colorlet{penColor5}{orange!80!black} % Color of a curve in a plot
\colorlet{penColor6}{yellow!70!black} % Color of a curve in a plot
\colorlet{fill1}{penColor!20} % Color of fill in a plot
\colorlet{fill2}{penColor2!20} % Color of fill in a plot
\colorlet{fillp}{fill1} % Color of positive area
\colorlet{filln}{penColor2!20} % Color of negative area
\colorlet{fill3}{penColor3!20} % Fill
\colorlet{fill4}{penColor4!20} % Fill
\colorlet{fill5}{penColor5!20} % Fill
\colorlet{gridColor}{gray!50} % Color of grid in a plot

\newcommand{\surfaceColor}{violet}
\newcommand{\surfaceColorTwo}{redyellow}
\newcommand{\sliceColor}{greenyellow}




\pgfmathdeclarefunction{gauss}{2}{% gives gaussian
  \pgfmathparse{1/(#2*sqrt(2*pi))*exp(-((x-#1)^2)/(2*#2^2))}%
}





%\typeout{************************************************}
%\typeout{ORCCA Preamble.Tex}
%\typeout{************************************************}


%% \usepackage{geometry}
%% \geometry{letterpaper,total={408pt,9.0in}}
%% Custom Page Layout Adjustments (use latex.geometry)
%% \usepackage{amsmath,amssymb}
%% \usepackage{pgfplots}
\usepackage{pifont}                                         %needed for symbols, s.a. airplane symbol
\usetikzlibrary{positioning,fit,backgrounds}                %needed for nested diagrams
\usetikzlibrary{calc,trees,positioning,arrows,fit,shapes}   %needed for set diagrams
\usetikzlibrary{decorations.text}                           %needed for text following a curve
\usetikzlibrary{arrows,arrows.meta}                         %needed for open/closed intervals
\usetikzlibrary{positioning,3d,shapes.geometric}            %needed for 3d number sets tower

%% NEEDED FOR XIMERA 1
%\usetkzobj{all}       %NO LONGER VALID
%%%%%%%%%%%%%%

\usepackage{tikz-3dplot}
\usepackage{tkz-euclide}                     %needed for triangle diagrams
\usepgfplotslibrary{fillbetween}                            %shade regions of a plot
\usetikzlibrary{shadows}                                    %function diagrams
\usetikzlibrary{positioning}                                %function diagrams
\usetikzlibrary{shapes}                                     %function diagrams
%%% global colors from https://www.pcc.edu/web-services/style-guide/basics/color/ %%%
\definecolor{ruby}{HTML}{9E0C0F}
\definecolor{turquoise}{HTML}{008099}
\definecolor{emerald}{HTML}{1c8464}
\definecolor{amber}{HTML}{c7502a}
\definecolor{amethyst}{HTML}{70485b}
\definecolor{sapphire}{HTML}{263c53}
\colorlet{firstcolor}{sapphire}
\colorlet{secondcolor}{turquoise}
\colorlet{thirdcolor}{emerald}
\colorlet{fourthcolor}{amber}
\colorlet{fifthcolor}{amethyst}
\colorlet{sixthcolor}{ruby}
\colorlet{highlightcolor}{green!50!black}
\colorlet{graphbackground}{white}
\colorlet{wood}{brown!60!white}
%%% curve, dot, and graph custom styles %%%
\pgfplotsset{firstcurve/.style      = {color=firstcolor,  mark=none, line width=1pt, {Kite}-{Kite}, solid}}
\pgfplotsset{secondcurve/.style     = {color=secondcolor, mark=none, line width=1pt, {Kite}-{Kite}, solid}}
\pgfplotsset{thirdcurve/.style      = {color=thirdcolor,  mark=none, line width=1pt, {Kite}-{Kite}, solid}}
\pgfplotsset{fourthcurve/.style     = {color=fourthcolor, mark=none, line width=1pt, {Kite}-{Kite}, solid}}
\pgfplotsset{fifthcurve/.style      = {color=fifthcolor,  mark=none, line width=1pt, {Kite}-{Kite}, solid}}
\pgfplotsset{highlightcurve/.style  = {color=highlightcolor,  mark=none, line width=5pt, -, opacity=0.3}}   % thick, opaque curve for highlighting
\pgfplotsset{asymptote/.style       = {color=gray, mark=none, line width=1pt, <->, dashed}}
\pgfplotsset{symmetryaxis/.style    = {color=gray, mark=none, line width=1pt, <->, dashed}}
\pgfplotsset{guideline/.style       = {color=gray, mark=none, line width=1pt, -}}
\tikzset{guideline/.style           = {color=gray, mark=none, line width=1pt, -}}
\pgfplotsset{altitude/.style        = {dashed, color=gray, thick, mark=none, -}}
\tikzset{altitude/.style            = {dashed, color=gray, thick, mark=none, -}}
\pgfplotsset{radius/.style          = {dashed, thick, mark=none, -}}
\tikzset{radius/.style              = {dashed, thick, mark=none, -}}
\pgfplotsset{rightangle/.style      = {color=gray, mark=none, -}}
\tikzset{rightangle/.style          = {color=gray, mark=none, -}}
\pgfplotsset{closedboundary/.style  = {color=black, mark=none, line width=1pt, {Kite}-{Kite},solid}}
\tikzset{closedboundary/.style      = {color=black, mark=none, line width=1pt, {Kite}-{Kite},solid}}
\pgfplotsset{openboundary/.style    = {color=black, mark=none, line width=1pt, {Kite}-{Kite},dashed}}
\tikzset{openboundary/.style        = {color=black, mark=none, line width=1pt, {Kite}-{Kite},dashed}}
\tikzset{verticallinetest/.style    = {color=gray, mark=none, line width=1pt, <->,dashed}}
\pgfplotsset{soliddot/.style        = {color=firstcolor,  mark=*, only marks}}
\pgfplotsset{hollowdot/.style       = {color=firstcolor,  mark=*, only marks, fill=graphbackground}}
\pgfplotsset{blankgraph/.style      = {xmin=-10, xmax=10,
                                        ymin=-10, ymax=10,
                                        axis line style={-, draw opacity=0 },
                                        axis lines=box,
                                        major tick length=0mm,
                                        xtick={-10,-9,...,10},
                                        ytick={-10,-9,...,10},
                                        grid=major,
                                        grid style={solid,gray!20},
                                        xticklabels={,,},
                                        yticklabels={,,},
                                        minor xtick=,
                                        minor ytick=,
                                        xlabel={},ylabel={},
                                        width=0.75\textwidth,
                                      }
            }
\pgfplotsset{numberline/.style      = {xmin=-10,xmax=10,
                                        minor xtick={-11,-10,...,11},
                                        xtick={-10,-5,...,10},
                                        every tick/.append style={thick},
                                        axis y line=none,
                                        y=15pt,
                                        axis lines=middle,
                                        enlarge x limits,
                                        grid=none,
                                        clip=false,
                                        axis background/.style={},
                                        after end axis/.code={
                                          \path (axis cs:0,0)
                                          node [anchor=north,yshift=-0.075cm] {\footnotesize 0};
                                        },
                                        every axis x label/.style={at={(current axis.right of origin)},anchor=north},
                                      }
            }
\pgfplotsset{openinterval/.style={color=firstcolor,mark=none,ultra thick,{Parenthesis}-{Parenthesis}}}
\pgfplotsset{openclosedinterval/.style={color=firstcolor,mark=none,ultra thick,{Parenthesis}-{Bracket}}}
\pgfplotsset{closedinterval/.style={color=firstcolor,mark=none,ultra thick,{Bracket}-{Bracket}}}
\pgfplotsset{closedopeninterval/.style={color=firstcolor,mark=none,ultra thick,{Bracket}-{Parenthesis}}}
\pgfplotsset{infiniteopeninterval/.style={color=firstcolor,mark=none,ultra thick,{Kite}-{Parenthesis}}}
\pgfplotsset{openinfiniteinterval/.style={color=firstcolor,mark=none,ultra thick,{Parenthesis}-{Kite}}}
\pgfplotsset{infiniteclosedinterval/.style={color=firstcolor,mark=none,ultra thick,{Kite}-{Bracket}}}
\pgfplotsset{closedinfiniteinterval/.style={color=firstcolor,mark=none,ultra thick,{Bracket}-{Kite}}}
\pgfplotsset{infiniteinterval/.style={color=firstcolor,mark=none,ultra thick,{Kite}-{Kite}}}
\pgfplotsset{interval/.style= {ultra thick, -}}
%%% cycle list of plot styles for graphs with multiple plots %%%
\pgfplotscreateplotcyclelist{pccstylelist}{%
  firstcurve\\%
  secondcurve\\%
  thirdcurve\\%
  fourthcurve\\%
  fifthcurve\\%
}
%%% default plot settings %%%
\pgfplotsset{every axis/.append style={
  axis x line=middle,    % put the x axis in the middle
  axis y line=middle,    % put the y axis in the middle
  axis line style={<->}, % arrows on the axis
  scaled ticks=false,
  tick label style={/pgf/number format/fixed},
  xlabel={$x$},          % default put x on x-axis
  ylabel={$y$},          % default put y on y-axis
  xmin = -7,xmax = 7,    % most graphs have this window
  ymin = -7,ymax = 7,    % most graphs have this window
  domain = -7:7,
  xtick = {-6,-4,...,6}, % label these ticks
  ytick = {-6,-4,...,6}, % label these ticks
  yticklabel style={inner sep=0.333ex},
  minor xtick = {-7,-6,...,7}, % include these ticks, some without label
  minor ytick = {-7,-6,...,7}, % include these ticks, some without label
  scale only axis,       % don't consider axis and tick labels for width and height calculation
  cycle list name=pccstylelist,
  tick label style={font=\footnotesize},
  legend cell align=left,
  grid = both,
  grid style = {solid,gray!20},
  axis background/.style={fill=graphbackground},
}}
\pgfplotsset{framed/.style={axis background/.style ={draw=gray}}}
%\pgfplotsset{framed/.style={axis background/.style ={draw=gray,fill=graphbackground,rounded corners=3ex}}}
%%% other tikz (not pgfplots) settings %%%
%\tikzset{axisnode/.style={font=\scriptsize,text=black}}
\tikzset{>=stealth}
%%% for nested diagram in types of numbers section %%%
\newcommand\drawnestedsets[4]{
  \def\position{#1}             % initial position
  \def\nbsets{#2}               % number of sets
  \def\listofnestedsets{#3}     % list of sets
  \def\reversedlistofcolors{#4} % reversed list of colors
  % position and draw labels of sets
  \coordinate (circle-0) at (#1);
  \coordinate (set-0) at (#1);
  \foreach \set [count=\c] in \listofnestedsets {
    \pgfmathtruncatemacro{\cminusone}{\c - 1}
    % label of current set (below previous nested set)
    \node[below=3pt of circle-\cminusone,inner sep=0]
    (set-\c) {\set};
    % current set (fit current label and previous set)
    \node[circle,inner sep=0,fit=(circle-\cminusone)(set-\c)]
    (circle-\c) {};
  }
  % draw and fill sets in reverse order
  \begin{scope}[on background layer]
    \foreach \col[count=\c] in \reversedlistofcolors {
      \pgfmathtruncatemacro{\invc}{\nbsets-\c}
      \pgfmathtruncatemacro{\invcplusone}{\invc+1}
      \node[circle,draw,fill=\col,inner sep=0,
      fit=(circle-\invc)(set-\invcplusone)] {};
    }
  \end{scope}
  }
\ifdefined\tikzset
\tikzset{ampersand replacement = \amp}
\fi
\newcommand{\abs}[1]{\left\lvert#1\right\rvert}
%\newcommand{\point}[2]{\left(#1,#2\right)}
\newcommand{\highlight}[1]{\definecolor{sapphire}{RGB}{59,90,125} {\color{sapphire}{{#1}}}}
\newcommand{\firsthighlight}[1]{\definecolor{sapphire}{RGB}{59,90,125} {\color{sapphire}{{#1}}}}
\newcommand{\secondhighlight}[1]{\definecolor{emerald}{RGB}{20,97,75} {\color{emerald}{{#1}}}}
\newcommand{\unhighlight}[1]{{\color{black}{{#1}}}}
\newcommand{\lowlight}[1]{{\color{lightgray}{#1}}}
\newcommand{\attention}[1]{\mathord{\overset{\downarrow}{#1}}}
\newcommand{\nextoperation}[1]{\mathord{\boxed{#1}}}
\newcommand{\substitute}[1]{{\color{blue}{{#1}}}}
\newcommand{\pinover}[2]{\overset{\overset{\mathrm{\ #2\ }}{|}}{\strut #1 \strut}}
\newcommand{\addright}[1]{{\color{blue}{{{}+#1}}}}
\newcommand{\addleft}[1]{{\color{blue}{{#1+{}}}}}
\newcommand{\subtractright}[1]{{\color{blue}{{{}-#1}}}}
\newcommand{\multiplyright}[2][\cdot]{{\color{blue}{{{}#1#2}}}}
\newcommand{\multiplyleft}[2][\cdot]{{\color{blue}{{#2#1{}}}}}
\newcommand{\divideunder}[2]{\frac{#1}{{\color{blue}{{#2}}}}}
\newcommand{\divideright}[1]{{\color{blue}{{{}\div#1}}}}
\newcommand{\negate}[1]{{\color{blue}{{-}}}\left(#1\right)}
\newcommand{\cancelhighlight}[1]{\definecolor{sapphire}{RGB}{59,90,125}{\color{sapphire}{{\cancel{#1}}}}}
\newcommand{\secondcancelhighlight}[1]{\definecolor{emerald}{RGB}{20,97,75}{\color{emerald}{{\bcancel{#1}}}}}
\newcommand{\thirdcancelhighlight}[1]{\definecolor{amethyst}{HTML}{70485b}{\color{amethyst}{{\xcancel{#1}}}}}
\newcommand{\lt}{<} %% Bart: WHY?
\newcommand{\gt}{>} %% Bart: WHY?
\newcommand{\amp}{&} %% Bart: WHY?


%%% These commands break Xake
%% \newcommand{\apple}{\text{🍎}}
%% \newcommand{\banana}{\text{🍌}}
%% \newcommand{\pear}{\text{🍐}}
%% \newcommand{\cat}{\text{🐱}}
%% \newcommand{\dog}{\text{🐶}}

\newcommand{\apple}{PICTURE OF APPLE}
\newcommand{\banana}{PICTURE OF BANANA}
\newcommand{\pear}{PICTURE OF PEAR}
\newcommand{\cat}{PICTURE OF CAT}
\newcommand{\dog}{PICTURE OF DOG}


%%%%% INDEX STUFF
\newcommand{\dfn}[1]{\textbf{#1}\index{#1}}
\usepackage{imakeidx}
\makeindex[intoc]
\makeatletter
\gdef\ttl@savemark{\sectionmark{}}
\makeatother












 % for drawing cube in Optimization problem
\usetikzlibrary{quotes,arrows.meta}
\tikzset{
  annotated cuboid/.pic={
    \tikzset{%
      every edge quotes/.append style={midway, auto},
      /cuboid/.cd,
      #1
    }
    \draw [every edge/.append style={pic actions, densely dashed, opacity=.5}, pic actions]
    (0,0,0) coordinate (o) -- ++(-\cubescale*\cubex,0,0) coordinate (a) -- ++(0,-\cubescale*\cubey,0) coordinate (b) edge coordinate [pos=1] (g) ++(0,0,-\cubescale*\cubez)  -- ++(\cubescale*\cubex,0,0) coordinate (c) -- cycle
    (o) -- ++(0,0,-\cubescale*\cubez) coordinate (d) -- ++(0,-\cubescale*\cubey,0) coordinate (e) edge (g) -- (c) -- cycle
    (o) -- (a) -- ++(0,0,-\cubescale*\cubez) coordinate (f) edge (g) -- (d) -- cycle;
    \path [every edge/.append style={pic actions, |-|}]
    (b) +(0,-5pt) coordinate (b1) edge ["x"'] (b1 -| c)
    (b) +(-5pt,0) coordinate (b2) edge ["y"] (b2 |- a)
    (c) +(3.5pt,-3.5pt) coordinate (c2) edge ["x"'] ([xshift=3.5pt,yshift=-3.5pt]e)
    ;
  },
  /cuboid/.search also={/tikz},
  /cuboid/.cd,
  width/.store in=\cubex,
  height/.store in=\cubey,
  depth/.store in=\cubez,
  units/.store in=\cubeunits,
  scale/.store in=\cubescale,
  width=10,
  height=10,
  depth=10,
  units=cm,
  scale=.1,
}

\author{David Kish}
\license{Creative Commons Attribution-ShareAlike 4.0 International License}


\title{Definition of Quadratics}

\begin{document}
\begin{abstract}
We explore quadratic functions.
\end{abstract}
\maketitle
\section{Quadratic Graphs}
\begin{example}
      Hannah fired a toy rocket from the ground,
      which launched into the air with an initial speed of $64$ feet per second.
      The height of the rocket can be modeled by the equation $y=-16t^2+64t$,
      where $t$ is how many seconds had passed since the launch.
      To see the shape of the graph made by this equation,
      we make a table of values, plot the points, and connect them with a curve.\\
$$
\begin{array}{ccc}
                t & -16t^2+64t & Point\\
\hline
                 0 &  -16(0)^2+64(0) =0 & (0,0) \\
        1& -16(1)^2+64(1) =48&(1,48) \\
          2 & -16(2)^2+64(2) =64 &(2,64)\\
3&-16(3)^2+64(3)=48 &(3,48)\\
            4 & -16(4)^2+64(4)=0  &(4,0)
    \end{array}
$$
\begin{image}
\begin{tikzpicture}
                    \begin{axis}[xmin=-2,xmax=6,ymin=-15,ymax=75,
                                ytick=,
                                minor ytick={-10,0,...,70},
                                xlabel={$t$}]
                        \addplot+[domain=0:4,-] {-16*x^2+64*x};
                        \addplot[soliddot] coordinates {(0,0)} node[above left]{$(0,0)$};
                        \addplot[soliddot] coordinates {(1,48)} node[above left]{$(1,48)$};
                        \addplot[soliddot] coordinates {(2,64)} node[above]{$(2,64)$};
                        \addplot[soliddot] coordinates {(3,48)} node[above right]{$(3,48)$};
                        \addplot[soliddot] coordinates {(4,0)} node[above right]{$(4,0)$};
                    \end{axis}
                \end{tikzpicture}
\end{image}
\end{example}


 A curve with the shape that we see in the above figure
      is called a \textbf{parabola}.
      Notice the symmetry in the table,
      how the $y$-values in rows $t=0$ and $t=4$ match as well as rows $t=1$ and $t=3$? 
      Also notice the symmetry in the shape of the graph,
      how its left side is a mirror image of its right side.

  The graph of a quadratic equation $y=ax^2+bx+c$ is a parabola which opens upward or downward
        according to the sign of the leading coefficient $a$.
        If the leading coefficient is positive,
        the parabola opens upward.
        If the leading coefficient is negative,
        the parabola opens downward.

You can play with changing the value of $a$ on the graph using Desmos and see how that changes the parabola.  

\begin{center}  
\desmos{nmlghfrws9}{800}{600}  
\end{center}

 The \textbf{vertex}
      of a parabola is the highest or lowest point on the graph,
      depending uponon whether the graph opens downward or upward.
      In Example 1, the vertex is $(2,64)$.
      This tells us that Hannah's rocket reached its maximum height of $64$ feet after $2$ seconds.
      If the parabola opens downward, as in the rocket example,
      then the $y$-value of the vertex is the \textbf{maximum} $y$-value.
      If the parabola opens upward then the $y$-value of the vertex is the \textbf{minimum} $y$-value.
      The \textbf{axis of symmetry}
      is a vertical line that passes through the vertex, cutting the parabola into two symmetric halves.
      We write the axis of symmetry as an equation of a vertical line so it always starts with ``$x=$''.
      In Example 1, the equation for the axis of symmetry is $x=2$.
  
\begin{definition}[Intercepts]
	Suppose $f$ is a function and set $y=f(x)$.

	An \dfn{$x$-intercept} is a point $(a,0)$ such that $f(a) = 0$. That is, it's a 
	point where the graph of the function intersects the $x$-axis. 

	The \dfn{$y$-intercept} is a point $(0,b)$ such that $f(0) = b$. That is, it's a 
	point in which the graph of the function intersects the $y$-axis. Unlike 
	$x$-intercepts, a function can only have one $y$-intercept.
\end{definition}
      It is possible for a quadratic graph to have zero, one, or two $x$-intercepts.

      In Example 1, first note that this is a function $y=f(t)$. We will have $y$-intercepts but rather than having $x$-intercepts, these will be $t$-intercepts due to our use of the variable $t$ and not $x$.
      The point $(\firsthighlight{0},\secondhighlight{0})$ is the starting point of the rocket,
      and it is where the graph crosses the $y$-axis,
      so it is the $y$-intercept.
      The $y$-value of $\secondhighlight{0}$ means the rocket was on the ground
      when the $t$-value was $\firsthighlight{0}$, which was when the rocket launched.
      The point $(0,0)$ on the path of the rocket is also a $t$-intercept.
      The $t$-value of $0$ indicates the time when the rocket was launched from the ground.
      There is another $t$-intercept at the point $(4,0)$,
      which means the rocket came back to hit the ground after $4$ seconds.
   





      
 \begin{example}
          Use technology to graph and make a table of the quadratic function $f$ defined by
          $f(x)=2x^2+4x-3$ and find each of the key points or features.
\begin{enumerate}
                \item Find the vertex.
                \item Find the vertical intercept  (i.e. the $y$-intercept).
                       \item Find the horizontal or (i.e. the $x$-intercept(s)).
               \item  Find $f(-2)$.
      \item Solve $f(x)=3$ using the graph.
           \item Solve $f(x)\le 3$ using the graph.
        \end{enumerate}
\begin{explanation}
           The specifics of how to use any one particular technology tool vary.
            Whether you use an app, a physical calculator,
            or something else, a table and graph should look like:\\
  $$
  \begin{array}{cc}
x & f(x) \\
\hline
-2 & -3 \\
-1 & -5 \\
 0 & -3 \\
 1 &  3 \\
 2 & 13 \\
  \end{array}
  $$

\begin{image}
\begin{tikzpicture}
                \begin{axis}
                    \addplot+[domain=-3.449:1.449] {2*x^2+4*x-3};
                \end{axis}
            \end{tikzpicture}
\end{image}
            Additional features of your technology tool can enhance the graph to help answer these questions.
            You may be able to make the graph appear like:
\begin{image}
      \begin{tikzpicture}
                \begin{axis}[clip=false]
                    \addplot [domain=-3.449:1.449, firstcurve] {2*x^2+4*x-3} node[below right] {$y=f(x)$};
                    \addplot [domain=-7:7, secondcurve] {3} node[above left] {$y=3$};
                    \addplot [soliddot, color=thirdcolor] coordinates {(-1,-5)} node[below left] {$(-1,-5)$};
                    \addplot [soliddot, color=fourthcolor] coordinates {(1,3)} node[above right] {$(1,3)$};
                    \addplot [soliddot, color=fourthcolor] coordinates {(0,-3)} node[below right] {$(0,-3)$};
                    \addplot [soliddot, color=fourthcolor] coordinates {(-3,3)} node[above left] {$(-3,3)$};
                    \addplot [soliddot, color=fifthcolor] coordinates {(-2.581,0)} node[above left] {$(-2.6,0)$};
                    \addplot [soliddot, color=fifthcolor] coordinates {(0.581,0)} node[above right] {$(0.6,0)$};
                    \addplot [soliddot, color=fourthcolor] coordinates {(-2,-3)} node[below left] {$(-2,-3)$};
                \end{axis}
            \end{tikzpicture}
   \end{image}
\begin{enumerate}
                \item The vertex is $(-1,-5)$.
                \item The vertical intercept is $(0,-3)$.
                \item The horizontal intercepts are approximately $(-2.6,0)$ and $(0.6,0)$.
              \item  When $x=-2$, $y=-3$, so $f(-2)=-3$.
 	 \item The solutions to $f(x)=3$ are the $x$-values where $y=3$. We graph the horizontal line $y=3$ and find the $x$-values where the graphs intersect. The solution set is $\{-3,1\}$.
       	\item  The solutions are all of the $x$-values where the function's graph is below (or touching) the line $y=3$. The interval is $[-3,1]$.
\end{enumerate}
\end{explanation}
\end{example}

%A polynomal is a particular type of algebraic expression
%\begin{enumerate}
%\item       A company's sales, $s$
%            (in millions of dollars),
%            can be modeled by $2.2t+5.8$,
%            where $t$ stands for the number of years since $2010$.
%\item     The height of an object from the ground, $h$
%            (in feet),
%            launched upward from the top of a building can be modeled by $-16t^2+32t+300$,
%            where $t$ represents the amount of time
%            (in seconds)
%            since the launch.
% \item The volume of an open-top box with a square base, $V$
%            (in cubic inches),
%            can be calculated by $30s^2-\frac{1}{2}s^2$,
%            where $s$ stands for the length of the square base,
%            and the box sides have to be cut from a certain square piece of metal.
%\end{enumerate}
%\section{Polynomial Vocabulary}
%       A polynomial is an expression with one or more
%          terms summed together.
%          A term of a polynomial must either be a plain number
%          or the product of a number and one or more variables raised to natural number powers.
%          The expression $0$ is also considered a polynomial,
%          with zero terms.
%\begin{example}
%Here are some examples of polynomials
%\begin{enumerate}
%   \item  Here are three polynomials: 
%$x^2-5x+2$, $t^3-1$, $7y$.
%        \item The expression $3x^4y^3+7xy^2-12xy$ is an example of a polynomial in more than one variable.
%    \item   The polynomial $x^2-5x+3$ has three terms:
%              $x^2$, $-5x$, and $3$.
%\item The polynomial $ 3x^4+7xy^2-12xy$ also has three terms.
%\item The polynomial $t^3-1$ has two terms.
%\end{enumerate}
%\end{example}
%\begin{definition}
%The coefficient
%          (or numerical coefficient)
%          of a term in a polynomial is the numerical factor in the term.
%\end{definition}
%\begin{example}
%\begin{enumerate}
%   \item  The coefficient of the term
%              $\frac{4}{3}x^6$ is $\frac{4}{3}$.
%\item   The coefficient of the second term of the polynomial $x^2-5x+3$ is $-5$.
%       \item The coefficient of the term
%              $\frac{y^7}{4}$ is $\frac{1}{4}$, because we can rewrite $\frac{y^7}{4}$ as $\frac{1}{4}y^7$.
%\end{enumerate}
%\end{example}
% A term in a polynomial with no variable factor is called a
%          constant term.
%\begin{example}
%   The constant term of the polynomial $x^2-5x+3$ is $3$.
%\end{example}
%\begin{definition}
%      The degree of a term is one way to measure how large it is.
%          When a term only has one variable,
%          its degree is the exponent on that variable.
%          When a term has more than one variable,
%          its degree is the sum of the exponents on the variables.
%          A nonzero constant term has degree $0$.
%\end{definition}
%\begin{example}
%  \begin{enumerate}
%  \item The degree of $5x^2$ is $2$.
%  \item The degree of $-\frac{4}{7}y^5$ is $5$.
%  \item     The degree of $-4x^2y^3$ is $5$.
%  \item  The degree of $17$ is $0$. Constant terms always have $0$ degree.
%  \end{enumerate}
%\end{example}
%
%\begin{definition}
%  The \dfn{degree} of a nonzero polynomial
%  is the greatest degree that appears amongst its terms
%\end{definition}
%\begin{remark}
%   To help us recognize a polynomial's degree,
%        the standard convention at this level is to write a polynomial's terms in order from highest degree to lowest degree.
%        When a polynomial is written in this order,
%        it is written in standard form.
%        For example,
%        it is standard practice to write $7-4x-x^2$ as
%        $-x^2-4x+7$ since $-x^2$ is the leading term.
%        By writing the polynomial in standard form,
%        we can look at the first term to determine both the polynomial's degree and leading term.
%\end{remark}
%
%%%%%%%%%%%
%\section{Adding and Subtracting Polynomials}
%        Bayani started a company that makes one product: one-gallon ketchup jugs for industrial kitchens.
%        The company's production expenses only come from two things:
%        supplies and labor.
%        The cost of supplies, $S$
%        (in thousands of dollars),
%        can be modeled by $S=0.05x^2+2x+30$,
%        where $x$ is number of thousands of jugs of ketchup produced.
%        The labor cost for his employees, $L$
%        (in thousands of dollars),
%        can be modeled by $0.1x^2+4x$,
%        where $x$ again represents the number of jugs they produce
%        (in thousands of jugs).
%        Find a model for the company's total production costs.
%
%%%%%%%%%%%
%\section{Evaluating Polynomial Expressions}
%
%      Recall that evaluating expressions involves replacing the variable(s) in an expression with specific numbers and calculating the result.
%      Here, we will look at evaluating polynomial expressions.
%
%\begin{example}
%Evaluate the expression 
%\begin{center}
%$-12y^3+4y^2-9y+2$ for $y=-5$ 
%\end{center}
%\begin{explanation}
%  We will replace $y$ with $-5$ and simplify the result:
%\begin{center}
%$
%\begin{array}{cl}
%   12y^3+4y^2-9y+2 & = -12(\substitute{-5})^3+4(\substitute{-5})^2-9(\substitute{-5})+2\\
%      & = -12(-125)+4(25)+45+2 \\
%        & = 1647  \\
%\end{array}
%$
%\end{center}
%\end{explanation}
%\end{example}

% %\section{Exponent Rules}
%   %%%%%%%%%%%%%%%%
%    \section{Product Rule}
%        If we write out $3^5\cdot 3^2$ without using exponents,
%        we'd have:
%$$
% 3^5 \cdot 3^2 = \left(3 \cdot 3\cdot 3\cdot 3\cdot 3\right) \cdot \left(3 \cdot 3\right)
%$$
%        If we then count how many $3$s are being multiplied together,
%        we find we have $5+2=7$, a total of seven $3$s.
%        So $3^5\cdot 3^2$ simplifies like this:
%       $$
%          3^5\cdot 3^2 = 3^{5+2} = 3^7
%$$
%\begin{example}          
%Simplify $x^2\cdot x^3$. \\
%\begin{explanation}       
%          To simplify $x^2\cdot x^3$,
%          we write this out in its expanded form,
%          as a product of $x$'s, we have
%$$
%            x^2\cdot x^3 =(x\cdot x)(x \cdot x \cdot x)
%            =x\cdot x\cdot x \cdot x \cdot x
%            =x^5
%   $$   
%          Note that we obtained the exponent of $5$ by adding $2$ and $3$.
%  \end{explanation}
%\end{example}
%
%      This example demonstrates our first exponent rule,
%      the Product Rule:\\
%\begin{callout}
%\textbf{ \Large Product Rule of Exponents} \\
%      When multiplying two expressions that have the same base,
%      we can simplify the product by adding the exponents.
%        $$
%x^m \cdot x^n = x^{m+n}
%$$
%\end{callout}
%   Recall that $x=x^1$. It helps to remember this when multiplying certain expressions together.
%\begin{example}        
%  Multiply $x(x^3+2)$ by using the distributive property.\\
%\begin{explanation}
%          According to the distributive property,
%        $x(x^3+2)=x\cdot x^3 + x\cdot2$
%          How can we simplify that term $x\cdot x^3$?
%          It's really the same as $x^1\cdot x^3$,
%          so according to the Product Rule, it is $x^4$.
%          So we have:
%        $$
%            x(x^3+2)=x\cdot x^3 + x\cdot2
%            =x^4+2x
%$$
%\end{explanation}
%\end{example}         
%
%%%%%%%%%%%%%%%%%
% \section{Power to a Power Rule}
%
%        If we write out $\left(3^5\right)^2$ without using exponents,
%        we'd have $3^5$ multiplied by itself:
%   $$
%         \left(3^5\right)^2 = \left(3^5\right)\cdot \left(3^5\right)
%         = \left(3\cdot 3\cdot 3\cdot 3 \cdot 3 \right) \cdot \left(3 \cdot 3\cdot 3\cdot 3\cdot 3\right)
%      $$ 
%        If we again count how many $3$s are being multiplied,
%        we have a total of two groups each with five $3$s.
%        So we'd have $2\cdot 5=10$ instances of a $3$.
%        So $\left(3^5\right)^2$ simplifies like this:
%   
%          $\left(3^5\right)^2 = 3^{2\cdot 5}$
%          $= 3^{10}$
%
%
%\begin{example}
%          Simplify $\left(x^2\right)^3$.\\
%\begin{explanation}
%          To simplify $\left(x^2\right)^3$,
%          we write this out in its expanded form,
%          as a product of $x$'s, we have
%            $\left(x^2\right)^3 =\left(x^2\right) \cdot \left(x^2\right)\cdot\left(x^2\right)$
%            $=(x \cdot x)\cdot (x \cdot x)\cdot (x \cdot x)$
%            $=x^6$
%     \end{explanation}
%          Note that we obtained the exponent of $6$ by multiplying $2$ and $3$.
%\end{example}
%      This demonstrates our second exponent rule,
%      the Power to a Power Rule:
%\begin{callout}
%\textbf{ \Large Power to a Power Rule} \\
%      when a base is raised to an exponent and that expression is raised to another exponent,
%      we multiply the exponents.
%   $$
%      \left(x^m\right)^n = x^{m \cdot n}
%   $$
%\end{callout}
%
%%%%%%%%%%%%%%%%%%%
%      \section{ Product to a Power Rule}
%        The third exponent rule deals with having multiplication inside a set of parentheses and an exponent outside the parentheses.
%        If we write out $\left(3t\right)^5$ without using an exponent,
%        we'd have $3t$ multiplied by itself five times:
%$$
%      (3t)^5= (3t)(3t)(3t)(3t)(3t)
%$$
%        Keeping in mind that there is multiplication between every $3$ and $t$,
%        and multiplication between all of the parentheses pairs,
%        we can reorder and regroup the factors:
%          $\left(3t\right)^5 = (3\cdot t)\cdot (3\cdot t)\cdot (3\cdot t)\cdot (3\cdot t)\cdot (3\cdot t)$
%          $= \left(3\cdot 3\cdot 3\cdot 3\cdot 3 \right) \cdot \left(t \cdot t \cdot t \cdot t \cdot t\right)$
%          $= 3^5 t^5$
%        We could leave it written this way if $3^5$ feels especially large.
%        But if you are able to evaluate $3^5=243$,
%        then perhaps a better final version of this expression is $243t^5$.
% 
%        We essentially applied the outer exponent to each factor inside the parentheses.
%        It is important to see how the exponent $5$ applied to \textbf{both} the $3$ \textbf{and} the $t$,
%        not just to the $t$.
%   
%
% \begin{example}   
%          Simplify $(xy)^5$.
%    
%          To simplify $(xy)^5$,
%          we write this out in its expanded form,
%          as a product of $x$'s and $y$'s, we have
%
%            $(xy)^5 =(x \cdot y) \cdot (x \cdot y) \cdot (x \cdot y) \cdot (x \cdot y) \cdot (x \cdot y)$
%            $=(x \cdot x \cdot x \cdot x \cdot x) \cdot (y \cdot y \cdot y \cdot y \cdot y)$
%            $=x^5 y^5$
%
%          Note that the exponent on $xy$ can simply be applied to both $x$ and $y$.
%\end{example}
%
%
%      This demonstrates our third exponent rule,
%      the Product to a Power Rule:
%\begin{callout}
%\textbf{ \Large Product to a Power Rule} \\
%      When a product is raised to an exponent,
%      we can apply the exponent to each factor in the product.
%$$
%        \left(x\cdot y\right)^n = x^{n}\cdot y^{n}
% $$
%\end{callout}
%
%
%%%%%%%%%%%%%%%%%%
%      \section{Summary of the Rules of Exponents for Multiplication}
%  
%
%
%          If $a$ and $b$ are real numbers,
%          and $m$ and $n$ are positive integers,
%          then we have the following rules:
% 
%\textbf{Product Rule}
%$$
%            a^{m} \cdot a^{n} = a^{m+n}
%$$
%\textbf{Power to a Power Rule}
%   $$
%           (a^{m})^{n} = a^{m\cdot n}
%   $$
% \textbf{ Product to a Power Rule}
%   $$
%           (ab)^{m} = a^{m} \cdot b^{m}
%   $$  
%      Many examples will make use of more than one exponent rule.
%      In deciding which exponent rule to work with first,
%      it's important to remember that the order of operations still applies.
%\begin{example} Simplify the following expression.
%                $\left(3^7r^5\right)^4$\\
%    \begin{explanation}
%                Since we cannot simplify anything inside the parentheses, we'll begin simplifying this expression using the Product to a Power rule.
%                We'll apply the outer exponent of 4 to each factor inside the parentheses.
%                Then we'll use the Power to a Power Rule to finish the simplification process.
%      $$          
%                  \left(3^7r^5\right)^4 = \left(3^7\right)^4 \cdot \left(r^5\right)^4
%                  = 3^{7\cdot4} \cdot r^{5\cdot 4}
%                  = 3^{28}r^{20}
%         $$      
%                Note that $3^{28}$ is too large to actually compute, even with a calculator,
%                so we leave it written as $3^{28}$.
%\end{explanation}
%\end{example}
%\begin{example}
%Simplify the following expression.
%       $\left(t^3\right)^2\cdot \left(t^4\right)^5$\\
%\begin{explanation}
%                According to the order of operations,
%                we should first simplify any exponents before carrying out any multiplication.
%                Therefore, we'll begin simplifying this by applying the Power to a Power Rule and then finish using the Product Rule.
%           $$
%                  \left(t^3\right)^2\cdot \left(t^4\right)^5 = t^{3\cdot2}\cdot t^{4\cdot5}
%                  = t^6 \cdot t^{20}
%                  = t^{6+20}
%                  = t^{26}
%$$
%\end{explanation}
%\end{example}
% \begin{remark} 
%        We cannot simplify an expression like $x^2y^3$ using the Product Rule,
%        as the factors $x^2$ and $y^3$ do not have the same base.
%\end{remark}



\end{document}
