\documentclass[nooutcomes]{ximera}
\input{../preamble}


\author{Elizabeth Miller}
\license{Creative Commons Attribution 4.0 International License}
\acknowledgement{https://spot.pcc.edu/math/orcca/ed2/html/orcca.html}



\title{Linear Equations: Slope}

\begin{document}

\begin{abstract}
  We explore the slope of lines.
\end{abstract}
\maketitle






%\typeout{************************************************}
%\typeout{Definition of Slope}
%\typeout{************************************************}

%\section{Slope}
We observed that a constant rate of change between points produces a linear relationship, whose graph is a straight line. Such a constant rate of change has a special name, \textbf{slope}, and we'll explore slope in more depth here.


\begin{definition}
 When $x$ and $y$ are two variables where the rate of change between any two points is always the same, we call this common rate of change the \dfn{slope}. Since having a constant rate of change means the graph will be a straight line, its also called the \dfn{slope of the line}.
\end{definition}


Considering the definition for \textbf{rate of change}, this means that when $x$  and $y$  are two variables where the rate of change between any two points is always the same, then you can calculate slope, $m$, by finding two distinct data points $(x_1,y_1)$ and $(x_2,y_2)$ ,  and calculating 

$$ m=\frac{\text{change in } y}{\text{change in } x} = \frac{\Delta y}{\Delta x} = \frac{y_2-y_1}{x_2-x_1}$$

A slope is a rate of change. So if there are units for the horizontal and vertical variables, then there will be units for the slope. The slope will be measured in $$\frac{\text{vertical units}}{\text{horizontal units}}$$.  

\begin{definition}
If the slope is constant and nonzero, we say that there is a \dfn{linear relationship} between $x$ and $y$.  When the slope is 0,  we say that $y$  is \dfn{constant} with respect to $x$. 
\end{definition}

Here are some linear scenarios with different slopes. As you read each scenario, note how a slope is more meaningful with units.
\begin{itemize} 
\item If a tree grows 2.5 feet every year, its rate of change in height is the same from year to year. So the height and time have a linear relationship where the slope is 2.5 ft⁄yr.
\item If a company loses 2 million dollars every year, its rate of change in reserve funds is the same from year to year. So the company's reserve funds and time have a linear relationship where the slope is -2  million dollars per year.
\item If Sakura is an adult who has stopped growing, her rate of change in height is the same from year to year—it's zero. So the slope is 0 in⁄yr. Sakura's height is constant with respect to time.
\end{itemize}


\begin{remark}
A useful phrase for remembering the definition of slope is ``rise over run.'' Here, ``rise'' refers to ``change in $y$'', $\Delta y$, and ``run'' refers to ``change in $x$", $\Delta x$. Be careful though. As we have learned, the horizontal direction comes first in mathematics, followed by the vertical direction. The phrase ``rise over run'' reverses this. (It's a bit awkward to say, but the phrase ``run under rise'' puts the horizontal change first.)
\end{remark}


\begin{example}
On Dec. 31, Yara had only \$50 in her savings account. For the the new year, she resolved to deposit \$20 into her savings account each week, without withdrawing any money from the account. 

Yara keeps her resolution, and her account balance increases steadily by \$20 each week. That's a constant rate of change, so her account balance and time have a linear relationship with slope of 20 $\frac{\text{dollars}}{\text{week}}$.  

\begin{explanation}

We can model the balance, $y$,  in dollars, in Yara's savings account $x$  weeks after she started making deposits with an equation. Since Yara started with \$50  and adds \$20  each week, then $x$  weeks after she started making deposits, $$y=50+20x$$ where $y$  is a dollar amount. Notice that the slope, $$20 \frac{\text{dollars}}{\text{week}}$$,  serves as the multiplier for $x$ weeks.

We can also consider Yara's savings using a table

\begin{center}
\(
\begin{array}{||cccc||}
\hline
&x &y&\\
&\text{(weeks since Dec 31)}&\text{(savings account balance in dollars)}&\\
\hline 
\hline
&0&50&\\
+1\rightarrow&1&70&\leftarrow +20\\
+1\rightarrow&2&90&\leftarrow +20\\
+2\rightarrow&4&130&\leftarrow +40\\
+3\rightarrow&7&190&\leftarrow +60\\
+5\rightarrow&12&290&\leftarrow +100\\
\hline 
\hline
\end{array}
\)
\end{center}

%\includegraphics{2-2table8.jpg}


In first few rows of the table, we see that when the number of weeks $x$ increases by 1, the balance $y$  increases by 20.   The row-to-row rate of change is $$\frac{\text{20 dollars}}{\text{1 week}} = 20 \frac{\text{dollars}}{\text{week}},$$   the slope. In any table for a linear relationship, whenever $x$  increases by 1  unit, $y$  will increase by the slope.

In further rows, notice that as row-to-row change in $x$ increases, row-to-row change in $y$  increases proportionally to preserve the constant rate of change. Looking at the change in the last two rows of the table, we see $x$  increases by 5  and $y$  increases by 100,  which gives a rate of change of  $$\frac{\text{100 dollars}}{\text{5 week}} = 20 \frac{\text{dollars}}{\text{week}},$$  the value of the slope again.

We can see this constant rate of change on the graph by drawing in \textbf{slope triangles} between points on the graph, showing the change in $x$ as a horizontal distance and the change in $y$ as a vertical distance.

\begin{image}
  \begin{tikzpicture}
        \begin{axis}[xlabel={weeks},
                    ylabel={savings (in dollars)},
                    xmin=-1,xmax=15,
                    ymin=-20,ymax=320,
                    xtick={2,4,...,14},
                    minor xtick={1,2,...,15},
                    ytick={100,200,300},
                    minor ytick={20,40,...,320},
                    ]
            \addplot+[domain = 0:13,->]{50+20*x};
            \addplot[soliddot] coordinates {(0,50) (1,70) (2,90) (4,130) (7,190) (12,290)};
            \addplot[guideline,->] coordinates {(0,50) (1,50) (1,65)};
            \addplot[guideline,->] coordinates {(1,70) (2,70) (2,85)};
            \addplot[guideline,->] coordinates {(2,90) (4,90) (4,125)};
            \addplot[guideline,->] coordinates {(4,130) (7,130) (7,185)};
            \addplot[guideline,-] coordinates {(7,190) (12,190)} node[pos=0.5,below] {$5$ weeks};
            \addplot[guideline,->] coordinates {(12,190) (12,285)} node[pos=0.5,below,rotate=90] {$\$100$};
        \end{axis}
\end{tikzpicture}
\end{image}

\end{explanation}

\end{example}


%\typeout{************************************************}
%\typeout{The Relationship Between Slope and Increase/Decrease}
%\typeout{************************************************}
\section{The Relationship Between Slope and Increase/Decrease}  

In a linear relationship, as the $x$-value increases (in other words as you read its graph from left to right):
\begin{itemize}
\item if the $y$-values increase (in other words, the line goes upward), its slope is positive. \\
\begin{image}
\begin{tikzpicture}[scale=0.60]
    \begin{axis}[]
        \addplot[firstcurve,domain=-6.25:7]{4/5*x-2};
    \end{axis}
\end{tikzpicture}
\end{image}
\item if the $y$-values decrease (in other words, the line goes downward), its slope is negative. \\
\begin{image}
\begin{tikzpicture}[scale=0.60]
    \begin{axis}[]
        \addplot[firstcurve,domain=-4.2:5.6]{-10/7*x+1};
    \end{axis}
\end{tikzpicture}
\end{image}
\item if the $y$-values don't change (in other words, the line is flat, or horizontal), its slope is 0. \\
\begin{image}
\begin{tikzpicture}[scale=0.60]
    \begin{axis}[]
        \addplot[firstcurve,domain=-7:7]{2.35};
    \end{axis}
\end{tikzpicture}
\end{image}
\end{itemize}
%\end{remark}
%\end{tcolorbox}


%\typeout{************************************************}
%\typeout{Finding the Slope by Two Given Points}
%\typeout{************************************************}
\section{Finding the Slope by Two Given Points}
Whenever you know two points on a line, you can find the slope of the line directly from the definition of slope.

\begin{example}
Your neighbor planted a sapling from Portland Nursery in his front yard. Ever since, for several years now, it has been growing at a constant rate. By the end of the third year, the tree was 15 ft tall; by the end of the sixth year, the tree was 27 ft tall. What's the tree's rate of growth (i.e. the slope)?

\begin{explanation}
We could sketch a graph for this scenario, and include a slope triangle. If we did that, it would look like:

\begin{image}
\begin{tikzpicture}
    \begin{axis}[xlabel={years since planting},
                ylabel={height (ft)},
                xmin=-1,xmax=9,
                ymin=-3,ymax=30,
                minor ytick=,
                minor xtick=,
                xtick={0,1,...,8},
                ytick={0,3,...,27},
                clip=false]
        \addplot+[domain = 1:6.5]{4*x+3};
          \addplot[soliddot] coordinates {(3, 15)} node[above left] {$(3,15)$};
          \addplot[soliddot] coordinates {(6, 27)} node[above left] {$(6,27)$};
          \addplot[guideline] coordinates {(3,15) (6,15)} node[pos=0.5,below] {$3$ years};
          \addplot[guideline,->] coordinates {(6,15) (6,27)} node[pos=0.5,below,rotate=90] {$12$ ft};
    \end{axis}
\end{tikzpicture}
\end{image}

We don't actually need the picture, though, to find the slope.  From the definition of slope, we have that 

$$ m=\frac{\text{change in } y}{\text{change in } x} = \frac{\Delta y}{\Delta x} = \frac{y_2-y_1}{x_2-x_1}$$

We know that after 3 yr, the height is 15 ft. As an ordered pair, that information gives us the point (3,15) which we can label as $(x_1,y_1)$  Similarly, the background information tells us to consider (6,27), which we label as $(x_2,y_2)$.  Here, $x_1$ and $y_1$ represent the first point's $x$-value and $y$-value, and $x_2$ and $y_2$ represent the second point's $x$-value and 
$y$-value.

Substiuting in our values for $x_1=3$, $y_1=15$, $x_2=6$, and $y_2=27$ into our definition of slope, we have

$$ m=\frac{\text{change in } y}{\text{change in } x} = \frac{\Delta y}{\Delta x} = \frac{y_2-y_1}{x_2-x_1}=\frac{\Delta y}{\Delta x} = \frac{27-15}{6-3}=\frac{12\text{ft}}{3\text{yr}}=\answer[given]{4} \frac{\text{ft}}{\text{yr}}$$
\end{explanation}

\end{example}



\end{document}
