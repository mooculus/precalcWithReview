\documentclass{ximera}

\input{../../preamble.tex}

\author{Bobby Ramsey}
\license{Creative Commons Attribution-ShareAlike 4.0 International License}

%\outcome{Calculating the rate of change.}
%\outcome{Discuss the meaning of antiderivatives of a position function.}

\begin{document}
\begin{exercise}
The graph below shows the temperature $T$, in degrees Celsius, of an object at time $t$, in minutes. 

\begin{image}
\begin{tikzpicture}
    \begin{axis}
    		[
    		xlabel={time, in minutes},
    		ylabel={temperature, in degrees C},
    		width=0.75\linewidth,
                xmin=-0.25,xmax=6.25,
                ymin=0,ymax=100,
                minor ytick=,
                minor xtick=,
		xtick={0,1,...,6},
                ytick={0,10,...,100},
                clip=false,
                ]
	
	\addplot[color=blue, ultra thick, smooth, samples=200, domain=0:6.25]{21 + 64*(0.5)^x};
						
        \addplot[soliddot] coordinates {(0,85)} node[color = blue, above right] {$(0,85)$};
        \addplot[soliddot] coordinates {(3,29)} node[color = blue, above right] {$(3,29)$};
        \addplot[soliddot] coordinates {(6,22)} node[color = blue, above right] {$(6,22)$};
    \end{axis}
\end{tikzpicture}
\end{image}


\begin{enumerate}
\item Based on the graph above, is this object heating up or cooling down?
	\begin{multipleChoice}
		\choice{Heating Up}
		\choice[correct]{Cooling Down}
	\end{multipleChoice}

\item What is the rate of change in this data between the point corresponding to $t=0$ minutes, and the point corresponding to $t=3$ minutes?
$\answer{-56/3}$ degrees Celsius/minute.
	\begin{hint}
		Recall that the rate of change between two data points is given by $\dfrac{\Delta T}{\Delta t}$.
	\end{hint}

\item What is the rate of change in this data between the point corresponding to $t=3$ minutes, and the point corresponding to $t=6$ minutes?
	$\answer{-7/3}$ degrees Celsius/minute.

\item Based on the your answers above, does this data always have the same rate of change?
	\begin{multipleChoice}
		\choice{Yes}
		\choice[correct]{No}
	\end{multipleChoice}
\end{enumerate}

%\begin{multipleChoice}
%\choice{constant function.}
%\choice{positive function. }
%\choice{negative function.}
%\choice{position function.}
%\choice[correct]{velocity function.}
%\end{multipleChoice}


\end{exercise}
\end{document}