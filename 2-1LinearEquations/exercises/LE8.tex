\documentclass{ximera}

\input{../../preamble.tex}

\author{Bobby Ramsey}
\license{Creative Commons Attribution-ShareAlike 4.0 International License}

%\outcome{Calculating the rate of change.}
%\outcome{Discuss the meaning of antiderivatives of a position function.}

\begin{document}
\begin{exercise}
Perpendicular lines

\begin{definition}
Two lines in the plane are \dfn{perpendicular} if the intersect at a right-angle of 90 $^\circ$. Any vertical line is perpendicular to any horizontal line.
Two non-vertical lines are perpendicular if and only if their slopes multiply to -1. That is, if two lines are perpendicular and the slope of the first line $m_1$ and the slope of the second line $m_2$, then $m_1 m_2= -1$.
\end{definition}

\begin{enumerate}
	\item Suppose a line has equation $y = 3x + 4$. An equation of the line perpendicular to this line, with $y$-intercept at $(0, -2)$ is given in slope-intercept form by
	$y = \answer{-1/3}x + \answer{-2}$.

	\item Suppose a line has equation $x = -2$. An equation of the line perpendicular to this, which passes through the point $(4, 2)$ has equation $y = \answer{2}$.

	\item Suppose a line has equation $5x + 2y = -4$. An equation of the line perpendicular to this, which passes through the point $(2, -3)$ is given in point-slope form by $y - \answer{-3} = \answer{2/5}\left( x - \answer{2} \right)$.

\end{enumerate}


\end{exercise}
\end{document}