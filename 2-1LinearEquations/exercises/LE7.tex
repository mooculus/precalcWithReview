\documentclass{ximera}

\input{../../preamble.tex}

\author{Bobby Ramsey}
\license{Creative Commons Attribution-ShareAlike 4.0 International License}

%\outcome{Calculating the rate of change.}
%\outcome{Discuss the meaning of antiderivatives of a position function.}

\begin{document}
\begin{exercise}
Parallel lines

\begin{callout}
Remember that two lines in the plane are parallel if they never intersect. This means that they are each traveling in the same direction. Any two vertical lines are parallel.
Any two horizontal lines are parallel. Two non-vertical lines are parallel if and only if they have the same slope.
\end{callout}

\begin{enumerate}
	\item Suppose a line has equation $y = 3x + 4$. An equation of the line parallel to this line, with $y$-intercept at $(0, -2)$ is given in slope-intercept form by
	$y = \answer{3}x + \answer{-2}$.

	\item Suppose a line has equation $x = -2$. An equation of the line parallel to this, which passes through the point $(4, 2)$ has equation $x = \answer{4}$.

	\item Suppose a line has equation $5x + 2y = -4$. An equation of the line parallel to this, which passes through the point $(2, -3)$ is given in point-slope form by $y - \answer{-3} = \answer{-5/2}\left( x - \answer{2} \right)$.

\end{enumerate}


\end{exercise}
\end{document}