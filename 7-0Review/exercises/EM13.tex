\documentclass{ximera}

\input{../../preamble.tex}

\author{Ivo Terek}
\license{Creative Commons Attribution-ShareAlike 4.0 International License}

%\outcome{Calculating the rate of change.}
%\outcome{Discuss the meaning of antiderivatives of a position function.}

\begin{document}
\begin{exercise}
 Let's say the cost of renting a scooter is as follows: $\$1$ to unlock the scooter, and then $\$0.20$ for each minute you have travelled. Fill the following table with prices in terms of time:

  $$
\begin{array}{cc}
\text{minutes}& \text{Price}\\
\hline
0 & \$1 \\
5& \$\answer{2}\\
10& \$\answer{3}\\
15& \$\answer{4}\\
20 & \$\answer{5}
\end{array}
$$

\begin{exercise}
  What does seem more adequate to model this situation?
  \begin{multipleChoice}
    \choice[correct]{A linear function}
    \choice{An exponential function}
  \end{multipleChoice}
  \begin{exercise}
    Find a linear formula for the fare $f$ paid in terms of the amount $m$ of miles travelled. Answer: $f(m) = \answer{0.2}m+\answer{1}$.
  \end{exercise}
\end{exercise}

\end{exercise}
\end{document}