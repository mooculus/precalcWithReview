\documentclass{ximera}

\input{../../preamble.tex}

\author{Ivo Terek}
\license{Creative Commons Attribution-ShareAlike 4.0 International License}

%\outcome{Calculating the rate of change.}
%\outcome{Discuss the meaning of antiderivatives of a position function.}

\begin{document}
\begin{exercise}

  Consider the rational function $$    f(x) = \frac{(x-1)^2(x-2)^5(x-3)^3(x-4)^2}{(x-1)^3(x-2)^4(x-3)^3(x-4)^6}.  $$Select the correct options below, regarding the graph of $y=f(x)$.
  \begin{multipleChoice}
    \choice[correct]{The line $x=1$ is a vertical asymptote.}
    \choice{There is a hole in the graph with $x$-coordinate equal to $1$.}
    \choice{The line $x=2$ is a vertical asymptote.}
    \choice[correct]{There is a hole in the graph with $x$-coordinate equal to $2$.}
    \choice{The line $x=3$ is a vertical asymptote.}
    \choice[correct]{There is a hole in the graph with $x$-coordinate equal to $3$.}
    \choice[correct]{The line $x=4$ is a vertical asymptote.}
    \choice{There is a hole in the graph with $x$-coordinate equal to $4$.}
  \end{multipleChoice}
\end{exercise}
\end{document}
