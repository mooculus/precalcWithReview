\documentclass{ximera}

\input{../preamble}
\author{Ivo Terek}
\license{Creative Commons Attribution-ShareAlike 4.0 International License}
\acknowledgement{}

\title{The Famous Function $f(x)=1/x$}

\begin{document}
\begin{abstract}
  
\end{abstract}
\maketitle


%\typeout{************************************************}
%\typeout{Motivating Questions}
%\typeout{************************************************}

\begin{motivatingQuestions}
\item What is a possible explanation, in terms of functions, for the fact that one cannot divide by zero? 
\item Are $\sin$, $\cos$ and $\tan$ really the only relevant trigonometric functions? Are there others? If so, how to understand them?
\end{motivatingQuestions}


%\typeout{************************************************}
%\typeout{Subsection Introduction}
%\typeout{************************************************}

\section{Introduction}

We know that if $a$ and $b$ are two real numbers, then $a/b$ makes sense, as long as $b$ is not equal to zero. Let's look at what happens when we make divisions by numbers very close to zero, but not equal to zero. Take $a=1$ for simplicity.

\begin{align*}
  \frac{1}{0.1} &= 10 \\ \frac{1}{0.01} &= 100 \\ \frac{1}{0.001} &= 1000 \\ \frac{1}{0.0001} &= 10000
\end{align*}

This pattern makes us want to say that $1/0$ equals to $+\infty$ (whatever $+\infty$ means, at this point), but this doesn't work. To understand why, let's consider divisions by numbers very close to zero, but this time negative. 

\begin{align*}
  \frac{1}{-0.1} &= -10 \\ \frac{1}{-0.01} &= -100 \\ \frac{1}{-0.001} &= -1000 \\ \frac{1}{-0.0001} &= -10000
\end{align*}

The same reasoning as before would tempt us to say that $1/0$ equals $-\infty$. And this raises the question of whether $\infty$ or $-\infty$ is the better choice. While on an instinctive psychological level we could think that $+\infty$ is better than $-\infty$, there's really no way to decide\footnote{Algebraically, the explanation is simple: if one could make sense of $1/0$ and say that equals some number $c$, then this would give $1 = 0 \cdot c$, so $1 = 0$ --- which is a complete collapse of the number system we have to deal with in our daily lives. But this doesn't give intuition for what is going on.} --- and this turns out to be related to the concept of \emph{limit}, which you'll learn in Calculus.

\section{Graph and asymptotics}

To continue our discussion in a more precise way, let's consider the function $f$, defined for all real numbers \emph{except for zero}, given by $f(x) = 1/x$. This is a very famous function, particularly useful as the building block for \emph{rational functions}, which we'll discuss soon. Note that essentially what we have just done in the introduction was to consider the values \[   f(0.1), f(0.01), f(0.001) \mbox{ and } f(0.0001),\]as well as \[f(-0.1), f(-0.01), f(-0.001), \mbox{ and } f(-0.0001).  \]
To get a good idea of the behavior a function has, our main strategy so far has been to just consider its graph. Naturally, plugging a handful of values won't cut it. Let's see what happens when we go to the other extreme and make divisions by very large numbers:

\begin{align*}
  \frac{1}{10} &= 0.1 \\ \frac{1}{100} &= 0.01 \\ \frac{1}{1000} &= 0.001 \\ \frac{1}{10000} &= 0.0001
\end{align*}

And from the negative side:

\begin{align*}
  \frac{1}{-10} &= -0.1 \\ \frac{1}{-100} &= -0.01 \\ \frac{1}{-1000} &= -0.001 \\ \frac{1}{-10000} &= -0.0001
\end{align*}


Here's what the graph looks like.

% \begin{image}
% \begin{tikzpicture}
%                     \begin{axis}[]
%                         \addplot+[domain=-6.29:-0.001] {1/x}; \addplot+[domain=0.001:6.29]{1/x}
%                     \end{axis}
%                 \end{tikzpicture}
% \end{image}

% [IS THIS A BUILT-IN MACRO? PROBLEM WITH 0 OUTSIDE DOMAIN]



\begin{tikzpicture}[line cap=round,line join=round,>=triangle 45,x=1.0cm,y=1.0cm]
\draw[->,color=black] (-4.93,0) -- (5.72,0);
\foreach \x in {-4.5,-4,-3.5,-3,-2.5,-2,-1.5,-1,-0.5,0.5,1,1.5,2,2.5,3,3.5,4,4.5,5,5.5}
\draw[shift={(\x,0)},color=black] (0pt,2pt) -- (0pt,-2pt) node[below] {\footnotesize $\x$};
\draw[->,color=black] (0,-4.1) -- (0,4.18);
\foreach \y in {-4,-3.5,-3,-2.5,-2,-1.5,-1,-0.5,0.5,1,1.5,2,2.5,3,3.5,4}
\draw[shift={(0,\y)},color=black] (2pt,0pt) -- (-2pt,0pt) node[left] {\footnotesize $\y$};
\draw[color=black] (0pt,-10pt) node[right] {\footnotesize $0$};
\clip(-4.93,-4.1) rectangle (5.72,4.18);
\draw[smooth,samples=100,domain=-4.930790040223839:5.722239807143025] plot(\x,{1/(\x)});
\end{tikzpicture}

[MAKE BETTER GRAPH, WILL FIX TIKZ AFTER DRAFTS ARE DONE]

Here's what we can immediately see from the graph, confirming our intuition from the several divisions previously done:

\begin{callout}
  {\bf Asymptotics of $1/x$.}
  \begin{itemize}
  \item If $x \to +\infty$, then $1/x \to 0$ (reads ``when $x$ tends to $+\infty$, $1/x$ tends to $0$'').
  \item If $x \to 0^+$, then $1/x \to +\infty$ (reads ``when $x$ tends to zero from the right, $1/x$ tends to $+\infty$'').
  \item If $x \to 0^-$, then $1/x \to -\infty$ (reads ``when $x$ tends to zero from the left, $1/x$ tends to $-\infty$'').
  \item If $x \to -\infty$, then $1/x \to 0$ (reads ``when $x$ tends to $-\infty$, $1/x$ tends to $0$'').
  \end{itemize}
\end{callout}

We say that the line $x=0$ is a \emph{vertical asymptote} for $f(x) = 1/x$, while the line $y=0$ is a \emph{horizontal asymptote}. We will discuss asymptotes of rational functions in general in the next unit. Next, as far as symmetries go, we can see that the graph is symmetric about the line $y=x$:

[ADD GRAPH AGAIN WITH ANTI-DIAGONAL INCLUDED]

This indicates that $f(x) = 1/x$ is an odd function. You can also see this algebraically via \[  f(-x) = \frac{1}{-x} = -\frac{1}{x} = -f(x).  \]
By the way, the graph of $f(x) = 1/x$ is called a \emph{hyperbola}.

\section{Application: Inverses of trigonometric functions}

It turns out that applying $f$ to some famous functions, such as the usual trigonometric functions $\sin$, $\cos$ and $\tan$, usually produces new interesting functions which can help to model several situations in a perhaps simpler way.

\begin{definition}
  The \emph{cosecant}, \emph{secant} and \emph{cotangent} functions are defined, respectively, by \[   \csc x = \frac{1}{\sin x}, \quad \sec x = \frac{1}{\cos x} \quad\mbox{and}\quad \cot x = \frac{1}{\tan x}.  \]
\end{definition}

\begin{exploration}
  \begin{enumerate}[label=\alph*.]
  \item For which values of $x$ do we have that $\sin x = 0$? Draw the graph of $\sin$. 
  \item For which values of $x$ is $\csc$ undefined? Recall that one cannot divide by zero.
  \item Repeat items (a) and (b) replacing $\sin$ and $\csc$ with $\cos$ and $\sec$, respectively. What about $\tan$ and $\cot$?
  \end{enumerate}
  [THIS IS PROBABLY NOT VERY ADEQUATE. CHANGE LATER]
\end{exploration}

\begin{callout}
  {\bf Warning:} Do not confuse inverses of trigonometric functions, as discussed above, with inverse trigonometric functions such as $\arcsin$, $\arccos$ and $\arctan$ (as in ``inverse functions'', as discussed in Section 3-2).
\end{callout}


%\typeout{************************************************}
%\typeout{Summary}
%\typeout{************************************************}

\begin{summary}
\item The function $f(x) = 1/x$ is defined for all non-zero values of $x$. It is an odd function, and its asymptotics can be understood by its graph, called a \emph{hyperbola}.
\item We can compose $f(x) = 1/x$ with functions we frequently encounter, to produce new functions which may prove useful when modeling certain problems and real life situations. For instance, doing this to trigonometric functions, one obtains \[   \csc x = \frac{1}{\sin x}, \quad \sec x = \frac{1}{\cos x} \quad\mbox{and} \quad \cot x = \frac{1}{\tan x}.  \]They are called, respectively, the \emph{cosecant}, \emph{secant} and \emph{cotangent} functions.
\end{summary}




\end{document}
