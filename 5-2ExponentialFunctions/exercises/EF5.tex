\documentclass{ximera}

\input{../../preamble.tex}

\author{Ivo Terek}
\license{Creative Commons Attribution-ShareAlike 4.0 International License}

%\outcome{Calculating the rate of change.}
%\outcome{Discuss the meaning of antiderivatives of a position function.}

\begin{document}
\begin{exercise}

In a laboratory, there is a culture with a certain number $P_0$ of bacteria. Let's say that $t$ hours after the initial observation, the population of bacteria is $P(t)$, and that it grows at a rate of $22\%$ per hour. Given that one hour after the initial observation the population is of $P(1) = 120$ bacteria, what is the amount of bacteria we have in the culture after four hours? 

If $P(t)=ab^t$, then $a=\answer{\frac{120}{1.22}}$ and $b=\answer{1.22}$. 

Find the exact value written as a calculator-ready response.  $P(4)=\frac{120}{1.22}(1.22)^4$

Use a calcululator to round your answer to the nearest integer. $P(4)\cong \answer{218}$ \calcHW

\end{exercise}
\end{document}
