\documentclass[nooutcomes]{ximera}
\input{../preamble.tex}


\author{Bobby Ramsey}
\license{Creative Commons Attribution 4.0 International License}
\acknowledgement{https://spot.pcc.edu/math/orcca/ed2/html/section-domain-and-range.html}
\acknowledgement{https://spot.pcc.edu/math/orcca/ed2/html/section-comparison-symbols-and-notation-for-intervals.html}

\title{Range}
% Learning Objectives for this section
%\begin{itemize}
%	\item Definition of the Range
%	\item Ranges of Famous Functions 
%	\item Spotting Values not in the Range
%\end{itemize}


\begin{document}
\licenseORCCA
\begin{abstract}
 	We examine the outputs that are actually achieved.
\end{abstract}
\maketitle


%\typeout{************************************************}
%\typeout{Review Questions}
%\typeout{************************************************}

%\section{Review Materials}
%    \begin{itemize}[label=\textbullet]
%	\item \link[Combining Like Terms]{https://spot.pcc.edu/math/orcca/ed2/html/section-combining-like-terms.html}
%	\item \link[Algebraic Properties and Simplifying Expressions]{https://spot.pcc.edu/math/orcca/ed2/html/section-algebraic-properties-and-simplifying-expressions.html}
%   \end{itemize}
\begin{motivatingQuestions}\begin{itemize}
	%Often start a section. 
	\item If $f$ is a function from a set $A$ to a set $B$, does every item in $B$ actually get related to something from $A$?
\end{itemize}\end{motivatingQuestions}

\section{Introduction}

	In the last section, for a function from $A$ to $B$,  we called the set $A$ the \emph{domain} and we called the set $B$ the \emph{codomain}.
	
	Let $f$ be the function defined by $f(x) = x^2$. We can consider this a function from the set of all real numbers $(-\infty, \infty)$ to the set of all 
	real numbers $(-\infty, \infty)$. In this case, the domain is $(-\infty, \infty)$ and the codomain is also $(-\infty, \infty)$.
	We know that for any real number $x$, the value of $x^2$ is never negative. That means there is no input to $f$ that ever gives a negative output.
		
	Let $g$ be the function from the set of capital letters to the set of natural numbers, which assigns each letter to its placement in the alphabet. 
	This means $g(A)=1$ since `A' is the first letter of the alphabet. Similarly $g(B) = 2$ and $g(Z)=26$. In this case the domain is the set of capital
	letters $\{ A, B, C, \ldots, Z\}$ and the codomain is the set of natural numbers $\{1, 2, 3, 4, \ldots \}$. For the function $g$
	there are only 26 capital letters in the alphabet, so no number past greater than 26 is ever an output of $g$.  
	
	For both the function $f$ and $g$ just given, not every number in the codomain is actually achieved as the output of the function. 
	There is a difference between the codomain, which measures the ``possible outputs'' and the actual outputs that are achieved.	
	
	
	\begin{exploration}
		\begin{enumerate}[label=\alph*.]
			\item Suppose the quadratic function $f$ is given by $f(x) = x^2$. Are there any values that are never achieved as an output?
			\item Explain the difference in finding the domain of a function and finding the range of the function, if you are given the graph of the function. What if you're given a formula for the function instead?
		\end{enumerate}
	\end{exploration}


\section{The Range of a Function}
	

	
	
	% From https://spot.pcc.edu/math/orcca/ed2/html/section-domain-and-range.html
	\begin{definition}
		Let $f$ be a function from $A$ to $B$. The \dfn{range} of $f$ is the collection of the outputs of $f$.
	\end{definition}
	This means the \emph{range} consists of the outputs that are actually achieved. Not everything that is ``possible'', but only those outputs that actually
	come out of the function. For each $b$ in the range of the function $f$, there is actually an $a$ in the domain with $f(a) = b$.

	\begin{example}
		Let $f$ be the function whose values are completely given in the following table. Find the range of $f$.
		$$ \begin{array}{c|c}
			  x & f(x)\\\hline
			  0 & 3\\
			  1 & -2\\
			  2 & 4\\
			  3 & 3\\
			  4 & 0
		\end{array}$$

		\begin{explanation}

			The only outputs of this function are $-2$, $0$, $3$, and $4$. 
			(Even though the output $3$ is identified twice, it only gets counted once here.) 
			There are no intervals, just the separated points. The range of $f$ is $\{ -2, 0, 3, 4 \}$.
		\end{explanation}	
	\end{example}
	
	
	\begin{example}

		Let $s$ be the function given by $s(t) = \dfrac{1}{t}$. Is $-\frac{1}{2}$ in the range of $s$? What about $\sqrt{5}$ or $0$?

		\begin{explanation}
		
			If we want to see if $-\frac{1}{2}$ is in the range of $s$, that means we need to check whether there is a number $a$ in the domain
			of $s$ with $s(a) = -\frac{1}{2}$. 
			\begin{align*}
				s(a) &= -\frac{1}{2}\\
				\frac{1}{a} &= -\frac{1}{2}\\
				2a \left( \frac{1}{a} \right) &= 2a \left( - \frac{1}{2} \right)\\
				\frac{2a}{a} &= -\frac{2a}{2}\\
				2 &= -a \\			
				-2 &= a
			\end{align*}
			That means $s(-2) = -\frac{1}{2}$. We found an input that gives the output value $-\frac{1}{2}$, which means $-\frac{1}{2}$ is in 
			the range of $s$.

			Let's try the same calculation for $\sqrt{5}$.
			\begin{align*}
				s(b) &= \sqrt{5}\\
				\frac{1}{b} &= \sqrt{5}\\
				b \left( \frac{1}{b} \right) &= b \left( \sqrt{5} \right)\\
				1 &= b\sqrt{5}\\
				\frac{1}{\sqrt{5}} &= b
			\end{align*}
			That means $s\left(\frac{1}{\sqrt{5}} \right) = \sqrt{5}$, so that $\sqrt{5}$ is in the range of $s$.
				
			Now for $0$:
			\begin{align*}
				s(c) &= 0\\
				\frac{1}{c} &= 0\\
				c \left( \frac{1}{c} \right) &= c \left( 0 \right)\\
				1 &= 0
			\end{align*}
			This statement is false for all choices of $c$. That means there is no value of $c$ with $s(c)=0$. In particular, $0$ is not in the range of $s$.	
		\end{explanation}
	\end{example}

	\begin{example}
		The entire graph of a function $g$ is given in the graph below. Find the range of $g$.
		\begin{image}
			\begin{tikzpicture}
				\begin{axis}[
		    			width=0.75\linewidth,
                			xmin=-6.25,xmax=6.25,
               				ymin=-6.25,ymax=6.25,
                			minor ytick=,minor xtick=,
		                	xtick={-6,...,6}, ytick={-6,...,6},
                			clip=false,
                			]
	
					\addplot[domain=-6:0, color=penColor]{(1/6)*x^2} node[above right, pos=0.25]{\Large{$y=g(x)$}};
					\addplot[domain=0:2, color=penColor]{2*x-4};
					\addplot[domain=2:4, color=penColor]{4};
					\addplot[hollowdot] coordinates {(0,0)};
					\addplot[hollowdot] coordinates {(-6,6)};
        				\addplot[hollowdot] coordinates {(0,-4)};
        				\addplot[soliddot] coordinates {(2,0)};
 					\addplot[hollowdot] coordinates {(2,4)};
        				\addplot[soliddot] coordinates {(4,4)};
        				\addplot[soliddot] coordinates {(6,-1)};
    				\end{axis}
			\end{tikzpicture}
		\end{image}
		
		\begin{explanation}
			
			\begin{center}
				\desmos{fsdehrzpe6}{800}{600}
			\end{center}
				
			Notice that as $x$ changes between $-6$ to $0$, the graph takes all outputs from $6$ to $0$ (not including the endpoints). As $x$
			changes from $0$ to $2$, all the numbers from $-4$ to $0$ show up as outputs (including $0$). Together, this means that every
			number in the interval $(-4, 6)$ is in the range. The last two pieces of the graph have outputs $4$ and $-1$, which are already included
			in this interval.
			
			The range is $(-4, 6)$.
			
		\end{explanation}
	\end{example}


	\begin{example}
		Let $f$ be the function given by $f(x) = 3x-5$. Find the range of $f$.
		
		\begin{explanation}
			We know the domain of $f$ is $(-\infty, \infty)$, so any real number is a valid input. We also know that $f$ is a linear polynomial function
			(a polynomial of degree 1) so we know what its graph is a straight line with slope $m = 3$.
			\begin{image}
				\begin{tikzpicture}
					\begin{axis}[
			    			width=0.75\linewidth,
	                			xmin=-6.25,xmax=6.25,
	               				ymin=-6.25,ymax=6.25,
	                			minor ytick=,minor xtick=,
			                	xtick={-6,...,6}, ytick={-6,...,6},
	                			clip=false,
	                			]
		
						\addplot+[domain=-0.25:3.75, color=penColor]{3*x-5} node[below right, pos=0.75]{\Large{$y=f(x)$}};
	    				\end{axis}
				\end{tikzpicture}
			\end{image}
			Since this graph goes higher and higher as $x$ travels to the right, and lower as $x$ travels to the left, we believe that every number is 
			eventually an output of this function. 
			
			Let's make a calculation to be sure that's true by taking an arbitrary real number and show that it's actually an output of $f$.
			If $a$ and $b$ are real numbers with $f(a)=b$, then
			\begin{align*}
				f(a) &= b\\
				3a-5 &= b\\
				3a &= b + 5\\
				a &= \dfrac{b+5}{3}.
			\end{align*}
			
			That means that if $b$ is any arbitrary real number, then $f\left( \frac{b+5}{3} \right) = b$ so that $b$ is achieved as an output of the
			function, so $b$ is in the range. In other words, this gives us a formula to identify an input that gives $b$ as an output.
			This means the range of $f$ is $(-\infty, \infty)$.

		\end{explanation}
	\end{example}
	

\section{The Range of Famous Functions}
	
	\begin{callout}
		\begin{enumerate}	
			\item The Absolute Value function - The average value of a number is never negative. The Absolute Value function has range $[0, \infty)$.
	
			\item Polynomial functions -  This depends on the degree of the polynomial.
				\begin{enumerate}
					\item Odd degree - The range is $(-\infty, \infty)$.
					\item Even degree - We can only be precise with monomials (polynomials with only one term) like $5x^2$ or $-6x^8$. 
						\begin{enumerate}
							\item If the monomial has positive coefficient, the range is $[0, \infty)$.
							\item If the monomial has negative coefficient, the range is $(-\infty, 0]$.
						\end{enumerate}
				\end{enumerate}
					
			\item The Square Root function - Even-index radicals never have negative outputs. Their range is $[0, \infty)$.		
	
			\item Exponential functions - Exponential functions $b^x$, for $b >0$ with $b \neq 1$,  have range $(0, \infty)$. Notice that $0$ is
					never an output for these kinds of functions.
	
			\item Logarithms - Logarithms have range $(-\infty, \infty)$.
				
			\item The Sine function - The sine function $\sin(x)$ has range $[-1,1]$.
		\end{enumerate}
	\end{callout}

\section{Spotting Values not in the Range}
	Finding the range of a function is quite a bit more involved than finding the domain. Here are some guidelines if you are given a formula for the function instead of its graph.
	\begin{callout}
		\begin{enumerate}
			\item The output of an even-index radical is never negative.
			\item The output of $x^n$ is never negative if $n$ is an even natural number.
			\item The output of an exponential function $b^x$ is always positive.
		\end{enumerate}
	\end{callout}
	
	\begin{example}
		Find the range of the following functions 
		\begin{enumerate}
			\item $\displaystyle f(x) = 2 - 4 \sqrt{x}$.
			\item $\displaystyle g(x) = e^x + \frac{1}{2}$.
			\item $\displaystyle h(x) = 3 + 5\sin(x)$.
		\end{enumerate}

		\begin{explanation}

			You'll notice in these calculations, that finding the range of a function is a great deal more complicated than finding the domain, unless we have an accurate graph of the
			function, as above.
						
			For each of these calculations, we will follow the same two steps. 
			The idea is that the first step shows that the range is ``no more than'' the interval we build, and the second step shows that the range is ``no less than'' that interval. The
			only possibility left to us is that the range is exactly the interval we have constructed.

			More specifically, in the first step will find a bound on the range. We'll determine if any numbers are too big or too small to be a valid output of the function. 
			This will give us an interval that the range will have to be inside.
			In the second step, we'll see that everything inside that interval is actually attained by the function, by constructing an input value that gets assigned to that output.
			
			
			\begin{enumerate}
				\item Let's start by looking at the graph of $f$
					\begin{image}
						\begin{tikzpicture}
							\begin{axis}[
					    			width=0.75\linewidth,
			                			xmin=-0.25,xmax=9.25,
			               				ymin=-12.25,ymax=3.25,
			                			minor ytick=,minor xtick=,
					                	xtick={0,...,9}, ytick={-12,...,3},
			                			clip=false,
			                			]
				
								\addplot+[->,domain=0:9, samples=200, color=penColor]{2-4*(x)^(0.5)} node[above right, pos=0.5]{\Large{$y=f(x)$}};
			        				\addplot[soliddot] coordinates {(0,2)};
			    				\end{axis}
						\end{tikzpicture}
					\end{image}
					Notice that the graph of $f$ looks similar to what we know for $\sqrt{x}$, but upside down, stretched, and moved up. From the 
					graph, we would say that the range should be $(-\infty, 2]$. Let's see how we can verify that with a few calculations.
					
					We know that the range of $\sqrt{x}$ is $[0, \infty)$ so for any $x$ in the domain of $f$,
					\begin{align*}
						\sqrt{x} &\geq 0\\
						-4 \sqrt{x} &\leq 0 \\
						2 - 4\sqrt{x} &\leq 2 \\
						f(x) &\leq 2
					\end{align*}
					The outputs of $f$ are never larger than $2$, so the only numbers in the range are less than or equal to $2$. That is, the range 
					must be inside the interval $(-\infty, 2]$.
		
					To verify that the range is exactly $(-\infty, 2]$, suppose $b \leq 2$, then:
					\begin{align*}
						f(a) &= b\\
						2 - 4\sqrt{a} &= b\\
						-4\sqrt{a} & = b-2\\
						\sqrt{a} &= \frac{b-2}{-4}\\
							\sqrt{a}&= \frac{-(b-2)}{4}\\
							\sqrt{a}&= \frac{-b+2}{4}\\
							\left(\sqrt{a}\right)^2&= \left(\frac{-b+2}{4}\right)^2\\
						a &= \left( \frac{-b+2}{4} \right)^2.
					\end{align*}
					
					For this value of $a$, we have $f\left( a \right) = b$. That means $b$ is in the range of $f$. That means every number in the interval $(-\infty, 2]$
					is in the range of $f$. 
					
					The range of $f$ is $(-\infty, 2]$.
	
				\item The graph of $g$ looks like this.
					\begin{image}
						\begin{tikzpicture}
							\begin{axis}[
					    			width=0.75\linewidth,
			                			xmin=-5.25,xmax=5.25,
			               				ymin=-1.25,ymax=9.25,
			                			minor ytick=,minor xtick=,
					                	xtick={-5,...,5}, ytick={-1,...,9},
			                			clip=false,
			                			]
				
								\addplot+[domain=-5.25:2.15, samples=200, color=penColor]{e^x+0.5} node[right, pos=0.5]{\Large{$y=g(x)$}};
			    				\end{axis}
						\end{tikzpicture}
					\end{image}
					This graph looks exactly like the graph of our famous function $e^x$, just moved vertically upward by $\frac{1}{2}$. 
					The range of $e^x$ is $(0, \infty)$, so our range should be shifted by $\frac{1}{2}$ to be $\left( \frac{1}{2}, \infty \right)$. 

					To verify this with a calculation, the fact that $e^x$ has range $(0,\infty)$ means for any value of $x$,
					\begin{align*}
						e^x &> 0\\
						e^x + \frac{1}{2} &> \frac{1}{2}
					\end{align*}
					The outputs of $g$ are greater than $\frac{1}{2}$, so the range of $g$ is in the interval	$\left( \frac{1}{2}, \infty\right)$.
					
					To verify that the range is exactly $\left( \frac{1}{2}, \infty\right)$, suppose $b > \frac{1}{2}$, then:
					\begin{align*}
						g(a) &= b\\
						e^a + \frac{1}{2} &= b\\
						e^a & = b-\frac{1}{2}
					\end{align*}
					Since the range of $e^x$ is $(0, \infty)$, and $b-\frac{1}{2} > 0$, there is a value for $a$ with $e^a = b-\frac{1}{2}$.
					
					(For instance, if $b=1$, that would mean $b-\frac{1}{2} = \frac{1}{2}$, so we would be looking to see if there is a value of $a$  with $e^a = \frac{1}{2}$. This is 
					illustrated in the graph below.)
					\begin{image}
						\begin{tikzpicture}
							\begin{axis}[
					    			width=0.75\linewidth,
			                			xmin=-2.25,xmax=2.25,
			               				ymin=-0.25,ymax=6.25,
			                			minor ytick=,minor xtick=,
					                	xtick={-2,...,2}, ytick={0,...,6},
			                			clip=false,
			                			]
				
								\addplot+[domain=-2.25:1.8, color=penColor]{e^x} node[below right, pos=0.75]{\Large{$y=e^x$}};
			    					\addplot[soliddot] coordinates {(-0.6931,0.5)};
			    					\draw[color=penColor2] (axis cs:-0.6931, 0.5) -- (axis cs: 0, 0.5) node[pos=1, right]{$\frac{1}{2}$};
			    					\draw[color=penColor2] (axis cs:-0.6931, 0.5) -- (axis cs: -0.6931, 0) node[pos=1, below]{$a$};
			    				\end{axis}
						\end{tikzpicture}
					\end{image}
					
					For this value of $a$, we have $g\left( a \right) = b$, meaning that $b$ is in the range of $g$.
					The range of $g$ is $\left( \frac{1}{2}, \infty\right)$.

				\item The graph of $h$ is given below, from which it appears the range of $h$ is $[-2, 8]$.
				
					\begin{image}
						\begin{tikzpicture}
							\begin{axis}[
						            xmin=-6.75,xmax=6.75,ymin=-2.5,ymax=8.5,
						            axis lines=center,
						            xtick={-6.28, -4.71, -3.14, -1.57, 0, 1.57, 3.142, 4.71, 6.28},
						            xticklabels={$-2\pi$,$-3\pi/2$,$-\pi$, $-\pi/2$, $0$, $\pi/2$, $\pi$, $3\pi/2$, $2\pi$},
						            ytick={-2,...,8},
						            %ticks=none,
						            width=6in,
						            height=3in,
						            unit vector ratio*=1 1 1,
						            xlabel=$x$, ylabel=$y$,
			                			minor ytick=,minor xtick=,
						            every axis y label/.style={at=(current axis.above origin),anchor=south},
						            every axis x label/.style={at=(current axis.right of origin),anchor=west},
						          ]        
						          \addplot[very thick, penColor, samples=300,smooth, domain=(-6.75:6.75)] {3+5*sin(deg(x))}
						          node[color=penColor2, pos=0.2, right] {\Large{$y=h(x)$}};
						        \end{axis}
						\end{tikzpicture}
					\end{image}
				
				The range of our famous function $sine$ is $[-1, 1]$. That means for any $x$
					\begin{align*}
						-1 &\leq \sin(x) \leq 1\\
						-5 &\leq 5\sin(x) \leq 5\\
						3+(-5) &\leq 3+5\sin(x) \leq 3+5\\
						-2 &\leq 3+5\sin(x) \leq 8
					\end{align*}
					The outputs of $h$ are in the interval $[-2, 8]$. To verify that the range is exactly $[-2, 8]$, suppose 
					$b$ is a number with $-2 \leq b \leq 8$. Then:
					\begin{align*}
						h(a) &= b\\
						3 + 5\sin(a) &= b\\
						5\sin(a)& = b-3\\
						\sin(a) &= \frac{b-3}{5}.
					\end{align*}
					Because $-2 \leq b \leq 8$, we know $-1 \leq \frac{b-3}{5} \leq 1$. That means there is a number $a$ with $\sin(a) = \frac{b-3}{5}$, since the range of 
					$\sin$ is $[-1,1]$.
					
					(For instance, if $b=3$ then $\frac{b-3}{5} = 0$ so we would be looking for a value of $a$ with $\sin(a) = 0$. We know that $\sin(0) = 0$, so this means we'd take $a=0$.)
					For this value of $a$, we have $h(a) = b$, so $b$ is in the range of $h$. The range of $h$ is then $[-2, 8]$.
			\end{enumerate}
		\end{explanation}
	\end{example}



\end{document}
