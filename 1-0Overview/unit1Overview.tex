\documentclass{ximera}

\input{../preamble}
\author{Elizabeth Miller}
\license{Creative Commons Attribution-ShareAlike 4.0 International License}


\title{Unit 1: Variables and CoVariation}

\begin{document}
\begin{abstract}
\end{abstract}
\maketitle

\begin{overview}
\item Units, Percents, and Estimations %(David) 
	\begin{enumerate}
	\item Estimations 
	
	\textit{Reminders of roles of addition, subtraction, multiplication, division and how to choose each one} 
	
	\item Formalize the canceling units as fractions 
	
	\textit{Quick review of properties of fractions, adding, subtracting, multiplying, dividing fractions} 
	
	\item Formula for finding percent (portion = amount*percent) 
	
	\textit{Percent increase and then decrease = what is the whole?} 
	\end{enumerate}
	
\item Relations and Graphs %(Elizabeth) 
	\begin{enumerate}
	
	\item Find points, plot points, intercepts, end behavior 

	\item Define, recognize, and graph a relation

	\item Reference sheet of famous functions and questions about what those graphs show 
	\end{enumerate} 

\item Covariation %(Elizabeth) 
	\begin{enumerate}
	\item Graph famous functions in Desmos, %sliders of a point moving along the curve, as you increase or decrease (x, price, time, etc.) how does the y-value change 
	
	%\item How does your estimate change as you rounded up or rounded down? 
	
	\item Novel word problems  %When did he turn around and head back towards his house? Type problems   %How does the cost change as you change the number of units you produce? 
	
	\item Filling up a vase with water of different shapes 
	\end{enumerate} 
\end{overview}


\begin{objectives}
\item Deep understanding in multiple representations (graphs, tables, equations, words, applications, data)  
	\begin{itemize}
	\item Switching back and forth between these representations AND being okay with the idea that mathematics is represented in multiple ways. 
	\item functions that are not given by formulas or numbers like: i.e. items at your local Target store to current price at your Target store.  
	\item Relating the representations.  For example, what do the “x and y” in the equation tell us about the graph and vice versa (I.e., the equation tells us the relationship between the coordinates of each point and the points on the graph are the “club” of points that make the equation true.  Also, what do we know about points not on the graph with respect to the equation?) 
	\item Number sense to Symbol (expression?) sense to Function sense. 
	\end{itemize}
\item Covariation (how does one variable change as you change another) 
\item Familiarity with Famous Functions (tables and graphs ONLY, give expressions maybe but not expected to work with algebraically) 
	\begin{itemize}
	\item Linear 
	\item Parabolas 
	\item Polynomials 
	\item Rational Functions 
	\item Exponential 
	\item Logarithms 
	\item Trig 
	\item Inverse Trig 
	\item Roots 
	\item Absolute value 
	\item Logistic growth 
	\item Functions represented only by a Graph 
	\item Piecewise Functions
	\end{itemize}
\item Graphs
	\begin{itemize}
	\item Introduce Desmos
	\item Plotting Points, understanding how to read graphs 
	\end{itemize}
\item Excursions in Mathematics
	\begin{itemize}
	\item Rough Estimations (“is this reasonable” type questions, theme throughout course)  
	\item Units
	\item Percents
	\end{itemize}
\item Computations
	\begin{itemize}
	\item Non-Rules:  Making sense vs. “sounding nice” (like $(a+b)^2 \neq a^2+b^2$), being familiar with common errors and avoiding them (connect concepts to procedures) 
	\end{itemize}
\end{objectives}



\end{document}
