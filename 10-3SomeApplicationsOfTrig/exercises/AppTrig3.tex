\documentclass{ximera}

\input{../../preamble.tex}

\author{Elizabeth Campolongo}

\begin{document}
\begin{exercise}

Solve the equation:
$$\csc (\theta) = -\sqrt{2}.$$

Give exact answers as in the examples. [\textit{Hint}: Draw the associated triangle.]

\begin{enumerate}
\item This time we will start with the unit circle.
\begin{image}
				\begin{tikzpicture}
					\begin{axis}[
				            xmin=-1.1,xmax=1.1,ymin=-1.1,ymax=1.1,
				            axis lines=center,
				            width=4in,
				            xtick={-1,1},
				            ytick={-1,1},
				            clip=false,
				            unit vector ratio*=1 1 1,
				            xlabel=$x$, ylabel=$y$,
				            every axis y label/.style={at=(current axis.above origin),anchor=south},
				            every axis x label/.style={at=(current axis.right of origin),anchor=west},
				          ]        
				          \addplot [dashed, smooth, domain=(0:360)] ({cos(x)},{sin(x)}); %% unit circle
				
				          \addplot+[thick,penColor2,->] plot coordinates {(0,0) (-.707,-.707)}; %% 225 degrees
				
%				          \addplot+[thick,penColor,->] plot coordinates {(0,0) (1,0)}; %% 40 degrees
				          
				          \addplot [textColor,smooth, domain=(0:226)] ({.10*cos(x)},{.10*sin(x)});
				          \addplot [penColor3,smooth, domain=(180:226)] ({.15*cos(x)},{.15*sin(x)});
				          \node at (axis cs:.07,.15) [anchor=west] {$\theta_1$};
				          \node at (axis cs:-.3,-.1) [anchor=west, penColor3] {$\theta_2$};
				         \end{axis}
				\end{tikzpicture}
				\end{image}
Here we see two angles indicated on the unit circle: $\theta_1$ and $\theta_2$. Which one is the reference angle? 
 \begin{multipleChoice}
    \choice{$\theta_1$}
    \choice[correct]{$\theta_2$}
  \end{multipleChoice}

\item So, what is the reference angle, $\theta_R$? $\answer{\frac{\pi}{4}}$

\item What is the period of $\csc\theta$? $\answer{2\pi}$%$\answer{(0,\pi)}$ $\cup$ $\answer{(\pi,2\pi)}$
\end{enumerate}
 \begin{exercise}
Now let's return to the unit circle. 
%Now consider the unit circle. Draw the angles corresponding to $\theta_R$ in one period $(0,\pi) \cup (\pi,2\pi)$ on the unit circle. 
\begin{image}
				\begin{tikzpicture}
					\begin{axis}[
				            xmin=-1.1,xmax=1.1,ymin=-1.1,ymax=1.1,
				            axis lines=center,
				            width=4in,
				            xtick={-1,1},
				            ytick={-1,1},
				            clip=false,
				            unit vector ratio*=1 1 1,
				            xlabel=$x$, ylabel=$y$,
				            every axis y label/.style={at=(current axis.above origin),anchor=south},
				            every axis x label/.style={at=(current axis.right of origin),anchor=west},
				          ]        
				          \addplot [dashed, smooth, domain=(0:360)] ({cos(x)},{sin(x)}); %% unit circle
				
						          
				          \addplot+[thick,penColor2,->] plot coordinates {(0,0) (.707,.707)}; %% 45 degrees
				          				
				          \addplot+[thick,penColor,->] plot coordinates {(0,0) (-.707,.707)}; %% 135 degrees
				            \addplot+[thick,penColor3,->] plot coordinates {(0,0) (-.707,-.707)}; %% 225 degrees
				              \addplot+[thick,penColor4,->] plot coordinates {(0,0) (.707,-.707)}; %% 315 degrees
				          
%				          \addplot [penColor2,smooth, samples=500, domain=(0:1350)] ({(.05+.15*x/1350)*cos(x)},{(.05+.15*x/1350)*sin(x)});
%				          \addplot [penColor3,smooth, domain=(0:41)] ({.2*cos(x)},{.2*sin(x)});
				          \node at (axis cs:.1,.08) [anchor=west, penColor2] {$\theta_A$}; %\pi/4
				            \node at (axis cs:-.15,.2) [anchor=west] {$\theta_B$}; %3\pi/4
				          \node at (axis cs:-.25,-.07) [anchor=west, penColor3] {$\theta_C$}; %5\pi/4
				          \node at (axis cs:.1,.-.07) [anchor=west, penColor4] {$\theta_D$}; %7\pi/4
					\addplot+[penColor3, soliddot] coordinates{(0,-1)} node[above right]{$(0,-1)$};

				         \end{axis}
				\end{tikzpicture}
				\end{image}
Above we have four angles indicated on the unit circle, each measuring counterclockwise from the $x$-axis.
\begin{enumerate}
\item $\theta_A = \answer{\frac{\pi}{4}}$.

\item $\theta_B = \answer{\frac{3\pi}{4}}$.

\item $\theta_C = \answer{\frac{5\pi}{4}}$.

\item $\theta_D = \answer{\frac{7\pi}{4}}$.

\item Which angles are the solutions to $\csc \theta = - \sqrt{2}$? 
\begin{multipleChoice}
\choice{$\theta_A$ and $\theta_C$}
\choice[correct]{$\theta_C$ and $\theta_D$}
\choice{$\theta_B$ and $\theta_D$}
\choice{$\theta_A$ and $\theta_B$}
\end{multipleChoice}
\end{enumerate}

%Choose the associated angles .... Draw the angles corresponding to $\theta_R$ in one period $[0,2\pi)$ on the unit circle. 
 \begin{exercise}
Now that we've done each of these steps, what are the general solutions to the equation? $\answer{\frac{5\pi}{4} + 2\pi k}$ and $\answer{\frac{7\pi}{4} + 2\pi k}$ for any integer $k$.
	

\end{exercise}
\end{exercise}
\end{exercise}
\end{document}