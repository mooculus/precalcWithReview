\documentclass{ximera}

\input{../../preamble.tex}

\author{Bobby Ramsey}
\license{Creative Commons Attribution-ShareAlike 4.0 International License}


\begin{document}
 
	Suppose an object is dropped from a height of 490 meters, and strikes the ground 10 seconds later. Let $h(t)$ denote the height of the object at time $t$, with $h$
	measured in meters, and $t$ measured in seconds with $t=0$ corresponding to the instant the object was released.
	\begin{exercise}
		\begin{center}
			The domain of $h$ is $[\answer{0},\answer{10}]$.
		\end{center}
		\begin{exercise}
			\begin{center}
				The range of $h$ is $[\answer{0},\answer{490}]$.
			\end{center}
			\begin{exercise}
				\begin{center}
					The average rate of change of $h$ between $t=0$ and $t=10$ is $\answer{-49}$ m/s.
				\end{center}
			\end{exercise}
		\end{exercise}
	\end{exercise}


\end{document}