\documentclass{ximera}

\input{../../preamble.tex}

\author{Elizabeth Miller}
\license{Creative Commons Attribution-ShareAlike 4.0 International License}
\acknowledgement{https://activecalculus.org/prelude/sec-changing-in-tandem.html}

%\outcome{Understand the relationship between position, velocity and acceleration.}
%\outcome{Discuss the meaning of antiderivatives of a position function.}

\begin{document}
\begin{exercise}
The graph below shows the fuel consumption (in miles per gallon, mpg) of a car driving at various speeds (in miles per hour, mph).

\begin{image}
\includegraphics[width=.6\textwidth]{ChangingInTandemFigure1.png}
\end{image}

\begin{enumerate}
\item How much gas is used on a 400 mile trip at 80 mph?  \\
amount of gas = $\answer{400/24}$ gallons
%how do I add an interval of fudge factor?
\begin{hint}
When the car is going 80 mpr, it appears from the graph that the fuel consumption is approximately 24 mpg.
\end{hint}

\item How much gas is saved by traveling 60 mph instead of 70 mph on a 600 mile trip?  \\
saved gas = $\answer{600/30-600/27}$ gallons
%how do I add an interval of fudge factor?

\item According to this graph, what is the most fuel efficient speed to travel? \\
most fuel efficient speed=$\answer{55}$ mph
%how do I add an interval of fudge factor?

\end{enumerate}

%\begin{multipleChoice}
%\choice{constant function.}
%\choice{positive function. }
%\choice{negative function.}
%\choice{position function.}
%\choice[correct]{velocity function.}
%\end{multipleChoice}


\end{exercise}
\end{document}