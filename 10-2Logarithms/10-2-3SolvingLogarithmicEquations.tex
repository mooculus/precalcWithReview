\documentclass[nooutcomes]{ximera}

\input{../preamble}
\author{Kenneth Berglund}
\license{Creative Commons Attribution-ShareAlike 4.0 International License}
\acknowledgement{}

\title{Solving Logarithmic Equations}

\begin{document}
\begin{abstract}
  
\end{abstract}
\maketitle

%\typeout{************************************************}
%\typeout{Motivating Questions}
%\typeout{************************************************}

%\begin{motivatingQuestions}\begin{itemize}
%\item What are inverse functions?
%\end{itemize}\end{motivatingQuestions}


%\typeout{************************************************}
%\typeout{Introduction}
%\typeout{************************************************}
\section{Using Inverses to Solve Equations}
Now that we have an understanding of the properties of logarithms, we're prepared to solve equations involving logarithms and exponential functions. Before we do that, however, let's discuss a method of solving equations that you're already familiar with. 

Consider the equation $$x + 2 = 7.$$ You may have already found that the solution is $x = 5$, but let's think about the process of finding the solution. 

Our general plan when solving equations is to isolate the variable we're solving for. In this case, we'd like to isolate $x$ by itself on one side of the equation. However, $x$ is not by itself: it's contained in a sum! Naturally, to undo the addition of 2, we subtract 2 from both sides and obtain $x = 5$. The key here is that it was stuck in some operation, and in order to ``access'' the $x$, we had to undo that operation. 

We can also view this process in the context of functions. Let $f$ be a function defined by $f(x) = x + 2$. Then, our equation becomes $f(x) = 7$. In the language of functions, ``undoing'' $f$ corresponds to applying the inverse function $f^{-1}$. In this case, $f^{-1}(x) = x - 2$. By applying $f^{-1}$ to both sides of our original equation, we find that 
\begin{align*}
f(x) & = 7 \\
f^{-1}(f(x)) & = f^{-1}(7) \\
x &= 7 - 2 \\
x & = 5.
\end{align*}

This may seem like an awfully strange way to subtract 2, but it has the benefit of being usable for any invertible function. 

For example, say we want to solve the equation $\frac{x + 1}{x} = 4$. If we define a function $g$ by $g(x) = \frac{x + 1}{x}$, our equation becomes $g(x) = 4$. We can find that the inverse is defined by $g^{-1}(x) = \frac{1}{x - 1}$. Therefore, 
\begin{align*}
g(x) & = 4 \\
g^{-1}(g(x)) & = g^{-1}(4) \\
x & = \frac{1}{4 - 1} \\
x & = \frac{1}{3}
\end{align*}
yields the solution to the equation. 

Since we had to do quite a bit of work to find the equation for $g^{-1}$ in the above scenario, this method may not be useful in that context. However, there are many functions for which we already know the inverse! For example, the inverse function of $h(x) = x^3$ is $h^{-1}(x) = \sqrt[3]{x}$. Therefore, if we want to solve $h(x) = 343$, we can apply $h^{-1}$ on both sides to find that 
\begin{align*}
h^{-1}(h(x)) & = h^{-1}(343) \\
x & = 7.
\end{align*}

Another important example of inverse functions that we know instantly comes from logarithms! If $f(x) = b^x$, then we know from our previous discussion that $f^{-1}(x) = \log_b(x)$. This is the definition of the logarithm, and looking at solving equations from the point of view of applying inverses is key to solving logarithmic and exponential equations. 

For example, if we want to solve the equation $\log(2t - 5) = 7$, we can define $f(x) = \log(x)$, so $f^{-1}(x) = 10^x$. This means our equation is $f(2t - 5) = 7$. Therefore, 
\begin{align*}
f(2t - 5) & = 7 \\
f^{-1}(f(2t - 5)) & = f^{-1}(7) \\
2t - 5 & = 10^7 \\
2t &= 1000000 + 5 \\
t &= \frac{1000005}{2}
\end{align*}
yields the solution to the equation. 

\section{Exponential Equations}
\begin{example}
Solve the equation $-4^{x -1} + 6 = 3$. 
\end{example}
\begin{explanation}
Notice that the variable we're solving for in this equation is located in the exponent of the exponential expression $4^{x - 1}$. Whenever this occurs, we call the equation an \emph{exponential equation}. 

If we define a function by $f(x) = 4^{x}$, then our equation becomes $-f(x - 1) + 6 = 2$. In order to solve this equation, we must use the inverse function: $f^{-1}(x) = \log_4(x)$. However, before we can apply this to both sides of the equation, we need to isolate $f(x - 1)$ like so:
\begin{align*}
-f(x - 1) + 6 & = 3 \\
-f(x - 1) &= -3 \\
f(x - 1) & = 3.
\end{align*}

Now we can take $f^{-1}$ of both sides of the equation and obtain
\begin{align*}
f^{-1}(f(x - 1)) &  = f^{-1}(3) \\
x - 1 & = \log_4(3) \\
x & = \log_4(3) + 1.
\end{align*}

Therefore, the solution to the equation $-4^x + 6 = 3$ is $\log_4(3) + 1$. Your first instinct might be that this doesn't seem like a solution, since there's still a logarithm in our expression! However, there is no nicer way to write the number $\log_4(3)$. If you were to plug this into a calculator, you would get a decimal approximation to the value of $\log_4(3)$, but the decimal approximation loses some information, so the exact value of the solution is $\log_4(3) + 1$. 

The process of writing out a function $f(x) = 4^x$ and then taking inverses may seem unnecessary, and indeed, there's no need to actually be so explicit when doing your own calculations. For example, the work
\begin{align*}
-4^{x - 1} + 6 & = 3 \\
-4^{x - 1} & = -3 \\
4^{x - 1} & = 3 \\
\log_4(4^{x - 1}) & = \log_4(3) \\
x - 1 & = \log_4(3) \\
x & = \log_4(3) + 1
\end{align*}
would be perfectly sufficient, and is usually how work for this kind of problem would be written. However, it must be emphasized that solving exponential equations involves more than just the basic operations of addition, subtraction, multiplication, and division. We now need to involve the process of taking logarithms of both sides of the equation. 

\end{explanation}

\begin{example}
Solve the equation $3^x = 5^{2 - x}$. 
\end{example}

\begin{explanation}
At first glance, this problem seems fundamentally different from the previous example. Instead of dealing with an exponential function with one base, we're dealing with two different bases: 3 and 5. 

However, recall from the previous section that any positive real number can be written as a power of any number we want. In this case, 3 can be written as $3 = 5^{\log_5(3)}$. Therefore, our equation becomes
$$
5^{x\log_5(3)} = 5^{2 - x}.
$$

If $f(x) = 5^x$, then our equation has become $f(x \log_5(3)) = f(2 - x)$. To solve this, we can take $f^{-1} = \log_5$ of both sides of the equation and do some more algebra to isolate $x$. 
\begin{align*}
\log_5(5^{x \log_5(3)}) & = \log_5(5^{2 -x }) \\
x\log_5(3) & = 2 - x \\
x\log_5(3) + x & = 2 \\
x(\log_5(3) + 1) & = 2 \\
x & = \frac{2}{\log_5(3) + 1},
\end{align*}
which is our solution.

A more conventional way to solve this equation is to take $\log_5$ of both sides at the very beginning:
\begin{align*}
\log_5(3^x) & = \log_5(5^{2 - x}) \\
\log_5(3^x) & = 2 - x,
\end{align*}
using the fact that logs and exponentials are inverses. Now we can apply the power property of logarithms to simplify the left-hand side:
$$
x\log_5(3) = 2 - x.
$$
From here, we can isolate all the $x$s on one side and factor:
\begin{align*}
x\log_5(3) & = 2 - x \\
x\log_5(3) + x & = 2 \\
x(\log_5(3) + 1) & = 2 \\
x & = \frac{2}{\log_5(3) + 1}.
\end{align*}

\end{explanation}

\begin{callout}
In fact, we can take $\log_b$ for any positive $b \ne 1$
\end{callout}

\section{Logarithmic Equations}
\begin{example}
Solve the equation $5\log_2(x + 3) = -2$. 
\end{example}
\begin{explanation}
To start off, divide both sides by 5 to isolate the $\log_2(x + 3)$ on the left-hand side.
$$
\log_2(x + 3) = -\frac{2}{5}
$$
Next, we apply the inverse of $\log_2$ to both sides of the equation, obtaining
\begin{align*}
2^{\log_2(x + 3)} & = 2^{-2/5} \\
x + 3 & = \frac{1}{\sqrt[5]{4}} \\
x & = \frac{1}{\sqrt[5]{4}} - 3.
\end{align*}

Since logarithms are not always defined (their domain is only positive real numbers), we should check that plugging in our solution for $x$ does not result in any part of our original equation being undefined. In our case, this amounts to checking that $\log_2(x + 3)$ is defined, that is, that $x + 3$ is positive. Since $x + 3 = \frac{1}{\sqrt[5]{4}} > 0$, our solution is $\frac{1}{\sqrt[5]{4}} - 3$.
\end{explanation}

\begin{example}
Solve the equation $\log_6(x) = 1 - \log_6(x - 1)$.
\end{example}

\begin{explanation}
As in Example 2, there appear to be too many functions going on here. However, if we add $\log_6(x - 1)$ to both sides of the equation and use the product property of logarithms, we obtain:
\begin{align*}
\log_6(x) + \log_6(x - 1) & = 1 \\
\log_6(x(x - 1)) & = 1. 
\end{align*}

Next, we can apply the inverse function of $\log_6$, which is given by $f(x) = 6^x$. Doing so, we see that 
\begin{align*}
6^{\log_6(x(x - 1))} & = 6^1 \\
x(x - 1) & = 6 \\
x^2 - x - 6 & = 0.
\end{align*}
This results in a quadratic equation! This is something we know how to solve. By using our preferred method, we find that $x = 3$ or $x = -2$. 

We're not done yet, however! We need to check that these $x$-values don't cause any logarithms in our original equation to be undefined. Note that $\log_6(-2)$ is undefined, since $-2$ is negative, so $x = -2$ is not a solution to our equation. Since $\log_6(3)$ and $\log_6(3 - 1)$ are both defined, $x = 3$ is our only solution.
\end{explanation}

\begin{summary}

\begin{itemize}
\item When solving exponential equations, our strategy is to isolate a single exponential on one side of the equation, then apply a logarithm to both sides to undo the exponential. 

\item When solving logarithmic equations, our strategy is to isolate a single logarithm on one side of the equation, then apply an exponential function to both sides to undo the logarithm.

\item Since the domain of logarithms is only $(0, \infty)$, we need to check that our solutions do not make our original logarithmic equations undefined.  
\end{itemize}\end{summary}

\end{document}
