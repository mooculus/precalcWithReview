\documentclass{ximera}

\input{../../preamble.tex}

\author{Kenneth Berglund}
\acknowledgement{https://www.stitz-zeager.com/szct07042013.pdf}

\begin{document}
\licenseSZ
In this exercise, we will find the solutions to the trigonometric equation $\cos(3x) = \frac{1}{2}$ that lie in the interval $[0, 2\pi)$.

To start, make the substitution $u = 3x$ and find solutions to the equation $\cos(u) = \frac{1}{2}$ for $u$ in the interval $[0, 2\pi)$. 

The solutions to $\cos(u) = \frac{1}{2}$ lying in the interval $[0, 2\pi)$ are $u = \frac{\pi}{3}$ and $u = \frac{5\pi}{3}$.

\begin{exercise}
The period of the cosine function is $2\pi$, so all real solutions to $\cos(u) = \frac{1}{2}$ are of the form $u = \frac{\pi}{3} + \answer{2\pi}k$ or $u = \frac{5\pi}{3} + \answer{2\pi}k$ for some integer $k$. 

\begin{exercise}
Undoing our substitution $u = 3x$, we find that all real solutions to $\cos(3x) = \frac{1}{2}$ are of the form $x = \answer{\frac{\pi}{9}} + \answer{\frac{2\pi}{3}}k$ or $x = \answer{\frac{5\pi}{9}} + \answer{\frac{2\pi}{3}}k$ for some integer $k$.

\begin{exercise}
There are $\answer{6}$ solutions to $\cos(3x) = \frac{1}{2}$ that lie in the interval $[0, 2\pi)$.

\begin{exercise}
The solutions to $\cos(3x) = \frac{1}{2}$ that lie in the interval $[0, 2\pi)$ are (in increasing order) $\answer{\frac{\pi}{9}}$, $\answer{\frac{5\pi}{9}}$, $\answer{\frac{7\pi}{9}}$, $\answer{\frac{11\pi}{9}}$, $\answer{\frac{13\pi}{9}}$, and $\answer{\frac{17\pi}{9}}$.


\end{exercise}
\end{exercise}
\end{exercise}
\end{exercise}
\end{document}