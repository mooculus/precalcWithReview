\documentclass{ximera}

\input{../../preamble.tex}

\author{Kenneth Berglund}
\acknowledgement{https://www.stitz-zeager.com/szct07042013.pdf}

\begin{document}
\licenseSZ
In this exercise, we will find the solutions to the trigonometric equation $\tan(2x - \pi) = 1$ that lie in the interval $[0, 2\pi)$.
\begin{exercise}
To start, make the substitution $u = 2x - \pi$ and find solutions to the equation $\tan(u) = 1$ for $u$ in the interval $[0, 2\pi)$. 

The solutions to $\tan(u) = 1$ lying in the interval $[0, 2\pi)$ are $u = \frac{\pi}{4}$ and $u = \frac{5\pi}{4}$.

\begin{exercise}
The period of the tangent function is $\pi$, so all real solutions to $\tan(u) = 1$ are of the form $u = \frac{\pi}{4} + \answer{\pi}k$ for some integer $k$. 

\begin{exercise}
Undoing our substitution $u = 2x - \pi$, we find that all real solutions to $\tan(2x-\pi) = 1$ are of the form $x = \answer{\frac{5\pi}{8}} + \answer{\frac{\pi}{2}}k$ for some integer $k$.

\begin{exercise}
There are $\answer{4}$ solutions to $\tan(2x - \pi) = 1$ that lie in the interval $[0, 2\pi)$.

\begin{exercise}
The solutions to $\tan(2x - \pi) = 1$ that lie in the interval $[0, 2\pi)$ are (in increasing order) $\answer{\frac{\pi}{8}}$, $\answer{\frac{5\pi}{8}}$, $\answer{\frac{9\pi}{8}}$, and $\answer{\frac{13\pi}{8}}$.

\end{exercise}
\end{exercise}
\end{exercise}
\end{exercise}
\end{exercise}
\end{document}