\documentclass{ximera}

\input{../../preamble.tex}

\author{Elizabeth Miller}
\license{Creative Commons Attribution-ShareAlike 4.0 International License}
\acknowledgement{https://activecalculus.org/prelude/sec-changing-in-tandem.html}

%\outcome{Understand the relationship between position, velocity and acceleration.}
%\outcome{Discuss the meaning of antiderivatives of a position function.}

\begin{document}
\begin{exercise}


\end{exercise}
\end{document}



%\typeout{************************************************}
%\typeout{Exercises 1.3.4 Exercises}
%\typeout{************************************************}
%
\begin{exercises-subsection}{Exercises}{}{Exercises}{}{}{ez-changing-aroc}
\begin{divisionexercise}{1}{}{}{ez-changing-aroc-soda}%
\hypertarget{p-236}{}%
A cold can of soda is removed from a refrigerator.  Its temperature $F$ in degrees Fahrenheit is measured at $5$-minute intervals, as recorded in the following table.%
\begin{table}
\centering
\begin{tabular}{lllllllll}
$t$ (minutes)&$0$&$5$&$10$&$15$&$20$&$25$&$30$&$35$\tabularnewline\hrulethin
$F$ (Fahrenheit temp)&$37.00$&$44.74$&$50.77$&$55.47$&$59.12$&$61.97$&$64.19$&$65.92$
\end{tabular}
\caption{Data for the soda's temperature as a function of time.\label{T-ez-changing-aroc-soda-table}}
\end{table}
\hypertarget{p-237}{}%
\leavevmode%
\begin{enumerate}[label=\alph*.]
\item\hypertarget{li-105}{}\hypertarget{p-238}{}%
Determine $AV_{[0,5]}$, $AV_{[5,10]}$, and $AV_{[10,15]}$, including appropriate units.  Choose one of these quantities and write a careful sentence to explain its meaning.  Your sentence might look something like ``On the interval $\ldots$, the temperature of the soda is $\ldots$ on average by $\ldots$''.%
\item\hypertarget{li-106}{}\hypertarget{p-239}{}%
On which interval is there more total change in the soda's temperature:  $[10,20]$ or $[25,35]$?%
\item\hypertarget{li-107}{}\hypertarget{p-240}{}%
What can you observe about when the soda's temperature appears to be changing most rapidly?%
\item\hypertarget{li-108}{}\hypertarget{p-241}{}%
Estimate the soda's temperature when $t = 37$ minutes.  Write at least one sentence to explain your thinking.%
\end{enumerate}
%
\end{divisionexercise}%
\begin{divisionexercise}{2}{}{}{ez-changing-aroc-position-graph}%
\hypertarget{p-244}{}%
The position of a car driving along a straight road at time $t$ in minutes is given by the function $y = s(t)$ that is pictured in \hyperref[F-aroc-ez-position-graph]{Figure~\ref{F-aroc-ez-position-graph}}. The car's position function has units measured in thousands of feet. For instance, the point $(2,4)$ on the graph indicates that after 2 minutes, the car has traveled 4000 feet.%
\begin{figure}
\centering
\includegraphics[width=0.4\linewidth]{images/aroc-ez-position-graph}
\caption{The graph of $y = s(t)$, the position of the car (measured in thousands of feet from its starting location) at time $t$ in minutes.\label{F-aroc-ez-position-graph}}
\end{figure}
\hypertarget{p-245}{}%
\leavevmode%
\begin{enumerate}[label=\alph*.]
\item\hypertarget{li-109}{}\hypertarget{p-246}{}%
In everyday language, describe the behavior of the car over the provided time interval. In particular, carefully discuss what is happening on each of the time intervals $[0,1]$, $[1,2]$, $[2,3]$, $[3,4]$, and $[4,5]$, plus provide commentary overall on what the car is doing on the interval $[0,12]$.%
\item\hypertarget{li-110}{}\hypertarget{p-247}{}%
Compute the average rate of change of $s$ on the intervals $[3,4]$, $[4,6]$, and $[5,8]$.  Label your results using the notation ``$AV_{[a,b]}$'' appropriately, and include units each quantity.%
\item\hypertarget{li-111}{}\hypertarget{p-248}{}%
On the graph of $s$, sketch the three lines whose slope corresponds to the values of $AV_{[3,4]}$, $AV_{[4,6]}$, and $AV_{[5,8]}$ that you computed in (b).%
\item\hypertarget{li-112}{}\hypertarget{p-249}{}%
Is there a time interval on which the car's average velocity is $5000$ feet per minute?  Why or why not?%
\item\hypertarget{li-113}{}\hypertarget{p-250}{}%
Is there ever a time interval when the car is going in reverse?  Why or why not?%
\end{enumerate}
%
\end{divisionexercise}%
\begin{divisionexercise}{3}{}{}{ez-changing-aroc-3}%
\hypertarget{p-253}{}%
Consider an inverted conical tank (point down) whose top has a radius of $3$ feet and that is $2$ feet deep.  The tank is initially empty and then is filled at a constant rate of $0.75$ cubic feet per minute.  Let $V=f(t)$ denote the volume of water (in cubic feet) at time $t$ in minutes, and let $h= g(t)$ denote the depth of the water (in feet) at time $t$.  It turns out that the formula for the function $g$ is $g(t) = \left( \frac{t}{\pi} \right)^{1/3}$.%
\par
\hypertarget{p-254}{}%
\leavevmode%
\begin{enumerate}[label=\alph*.]
\item\hypertarget{li-114}{}\hypertarget{p-255}{}%
In everyday language, describe how you expect the height function $h = g(t)$ to behave as time increases.%
\item\hypertarget{li-115}{}\hypertarget{p-256}{}%
For the height function $h = g(t) = \left( \frac{t}{\pi} \right)^{1/3}$, compute $AV_{[0,2]}$, $AV_{[2,4]}$, and $AV_{[4,6]}$.  Include units on your results.%
\item\hypertarget{li-116}{}\hypertarget{p-257}{}%
Again working with the height function, can you determine an interval $[a,b]$ on which $AV_{[a,b]} = 2$ feet per minute?  If yes, state the interval; if not, explain why there is no such interval.%
\item\hypertarget{li-117}{}\hypertarget{p-258}{}%
Now consider the volume function, $V = f(t)$.  Even though we don't have a formula for $f$, is it possible to determine the average rate of change of the volume function on the intervals $[0,2]$, $[2,4]$, and $[4,6]$?  Why or why not?%
\end{enumerate}
%
\end{divisionexercise}%
\end{exercises-subsection}
\end{sectionptx}
