\documentclass[nooutcomes]{ximera}

\input{../preamble}
\author{Kenneth Berglund}
\license{Creative Commons Attribution-ShareAlike 4.0 International License}
\acknowledgement{}

\title{Definition of Logarithms}

\begin{document}
\begin{abstract}
  
\end{abstract}
\maketitle


%\typeout{************************************************}
%\typeout{Motivating Questions}
%\typeout{************************************************}

%\begin{motivatingQuestions}\begin{itemize}
%\item What are inverse functions?
%\end{itemize}\end{motivatingQuestions}


%\typeout{************************************************}
%\typeout{Introduction}
%\typeout{************************************************}
In Section 3-2-2, we briefly introduced the concept of \emph{inverse functions}. Recall that for a one-to-one function $f$, we can define the inverse function $f^{-1}$. If we think of $f$ as a process that takes some input $x$ and produces some output $f(x)$, then providing $f(x)$ as an input to $f^{-1}$ produces the original input $x$, and vice versa. Symbolically, we wrote that $f^{-1}(f(x)) = x$ and $f(f^{-1}(x)) = x$. 

For a more down-to-earth example, consider the function $b$, which has as its domain the set of all students in the class. The rule for $b$ is that if $x$ is a student, then $b(x)$ is the student's birthdate. The inverse function $b^{-1}$, then, takes a birthdate and outputs the student who was born on that date. Notice that the domain of $b^{-1}$ is not the set of all days of the year! It is only able to be defined for dates that are someone's birthday, that is, dates that are in the range of $b$. This highlights an important property of functions and their inverses: the range of $f$ is the domain of $f^{-1}$!



\begin{summary}\begin{itemize}
\item a
\end{itemize}\end{summary}

\end{document}
