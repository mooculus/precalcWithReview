\documentclass{ximera}

\input{../../preamble.tex}

\author{Bobby Ramsey}
\license{Creative Commons Attribution-ShareAlike 4.0 International License}

\begin{document}

	A right circular cone has a \textbf{fixed slant height} of  7 m.  Call $h$ the height of the cone and $r$ the radius, as in the figure below.
	\begin{image}
		\begin{tikzpicture}[scale=0.5]
			%\draw[penColor,very thick] (0,4) ellipse (4 and 1);
			\draw[very thick,penColor!20!background] (2,2) arc (0:180:2 and .5);% top half of ellipse
			\draw[very thick,penColor] (-2,2) arc (180:360:2 and .5);% bottom half of ellipse
			%\draw[penColor, very thick] (0,4) -- (4,4);
			\draw[dashed,penColor!50!background, very thick] (0,2) -- (2,2);
			  \draw[decoration={brace,mirror,raise=0.07cm},decorate,thin] (2,2)--(0,6.25);
			\draw[ penColor,very thick] (-2,2) -- (0,6.25);
			\draw[ penColor,very thick] (2,2) -- (0,6.25);
			\draw[ dashed,penColor2,very thick] (0,2) -- (0,6.25);
			\node[above, penColor] at (1,2.05) {$r$ };
			\node[right, penColor2] at (0,4.1) {$h$ };
			\node[right, penColor2] at (1.1,4.25) {$7$ m };
		\end{tikzpicture}
	\end{image}

\begin{exercise}	
	$h$ is a function of $r$. The formula for $h(r)$ is given by:
	$$ h(r) = \answer{\sqrt{49-r^2}} $$.
	\begin{hint}
		Notice that $h$ and $r$  form the legs of a right triangle with the slant height of the cone as its hypotenuse. 
		Think about the Pythagorean Theorem $a^2 + b^2 = c^2$.
	\end{hint}
	\begin{exercise}
		The domain of $h$ is: $\left[ \answer{-7} , \answer{7} \right]$.
		\begin{hint}
			You know that $49-r^2$ can not be negative. Try plotting the parabola $y=49-x^2$ and seeing where the graph is above the $x$-axis.
		\end{hint}
	\end{exercise}
	
\end{exercise}
\end{document}