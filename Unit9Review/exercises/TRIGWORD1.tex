\documentclass{ximera}

\input{../../preamble.tex}

\author{David Kish}
\license{Creative Commons Attribution-ShareAlike 4.0 International License}



\begin{document}
\begin{exercise}
The height of the tide in a small beach town is measured along a seawall. Water levels oscillate between $6$ feet at low tide and $12$ feet at high tide. On a particular day, low tide occurred at 6 AM and high tide occurred at noon. Approximately every 12 hours, the cycle repeats. Find an equation to model the water levels where $x$ represents the time in hours and $y$ represents the height of the tide in feet. Hint: Use midnight (12 AM) as $x=0$
\begin{enumerate}
\item Which periodic function makes the most sense for this model?
\wordChoice{\choice{$\sin$}\choice[correct]{$\cos$}\choice{$\tan$}}
\item What is the Amplitude of the tide?\\
$\answer{3}$
\item What is the Period of the tide?\\
$\answer{12}$ hours.
\item Find the $b$ value for this model where is $b$ is the horizontal scale coefficient as in this equation $f(x)=a\sin(bx)+d$. \\
Hint: $P=\frac{2\pi}{b}$\\
$\answer{\frac{\pi}{6}}$
\item What is the vertical shift for this model?\\
$\answer{9}$
\item Write an equation that models the water levels.\\
$y=\answer{3\cos(\frac{\pi}{6}x)+9}$
\end{enumerate}
\end{exercise}

\end{document}
