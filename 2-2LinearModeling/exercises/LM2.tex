\documentclass{ximera}

\input{../../preamble.tex}

\author{Kenneth Berglund}
\acknowledgement{https://www.stitz-zeager.com/szca07042013.pdf}

\begin{document}
\begin{exercise}
\licenseSZ

Water freezes at $0^\circ$ Celsius and $32^\circ$ Fahrenheit and it boils at $100^\circ$C and $212^\circ$F.

Write your answers as improper fractions if necessary. 

\begin{enumerate}
\item A formula that expresses temperature in the Fahrenheit scale (represented by the variable $f$) in terms of
degrees Celsius (which we represent by the variable $c$) is $f = \answer{(9/5)c + 32}$. 

\item Using the above function, $20^\circ$C is $\answer{68}^\circ$ Fahrenheit.

\item A formula that expresses temperature in the Celsius scale (represented by the variable $c$) in terms of
degrees Fahrenheit (which we represent by the variable $f$) is $c = \answer{(5/9)f - 160/9}$. 

\item Using the above formula, $110^\circ$F is $\answer{130/3}^\circ$ Celsius.

\item The temperature at which $f = c$ is $\answer{-40}^\circ$.

	
\end{enumerate}

\end{exercise}
\end{document}