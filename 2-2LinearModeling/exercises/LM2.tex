\documentclass{ximera}

\input{../../preamble.tex}

\author{Kenneth Berglund}
\acknowledgement{https://www.stitz-zeager.com/szca07042013.pdf}

\begin{document}
\begin{exercise}

Water freezes at $0^\circ$ Celsius and $32^\circ$ Fahrenheit and it boils at $100^\circ$C and $212^\circ$F.

Write your answers as improper fractions if necessary. 

\begin{enumerate}
\item A linear function $F$ that expresses temperature in the Fahrenheit scale in terms of
degrees Celsius (which we represent by the variable $x$) is $F(x) = \answer{(9/5)x + 32}$. 

\item Using the above function, $20^\circ$C is $\answer{68}^\circ$ Fahrenheit.

\item A linear function $C$ that expresses temperature in the Celsius scale in terms of
degrees Fahrenheit (which we represent by the variable $x$) is $C(x) = \answer{(5/9)x - 160/9}$. 

\item Using the above function, $110^\circ$F is $\answer{130/3}^\circ$ Celsius.

\item The temperature $x$ at which $F(x) = C(x)$ is $\answer{-40}^\circ$.

	
\end{enumerate}

\end{exercise}
\end{document}