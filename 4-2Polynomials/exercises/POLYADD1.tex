\documentclass{ximera}

\input{../../preamble.tex}

\author{Kenneth Berglund}




\begin{document}
In this exercise, we will rewrite the following quadratic in vertex form $a(x - h)^2 + k$, using the method of completing the square:
\[
f(x)=4x^2-24x+9.
\]
\begin{exercise}

Factoring the leading coefficient out of the $x^2$ and $x$ terms yields
$$
\answer{4}(x^2 - \answer{6}x) + \answer{9}.
$$

\begin{exercise}
Notice that we can't rewrite $x^2 - 6x$ as a square of a linear polynomial. We need to add a constant term in order to make $x^2 - 6x$ into a square. 

We want to find numbers $m$ and $n$ so $(x + m)^2 = x^2 - 6x + n$. That is, we want to find a number $n$ that we can add onto $x^2 - 6x$ so it is the square of the linear polynomial $x + m$. 

$$
m = \answer{-3} \text{ and } n = \answer{9}
$$
\begin{hint}
Try multiplying out $(x + m)^2$ and matching the coefficients of each term with the coefficients of the corresponding term in $x^2 - 6x + n$. 
\end{hint}


\begin{exercise}
Modify the quadratic so that the square appears, in the disguise of $x^2 - 6x + 9$.
$$
4(x^2 - 6x + \answer{9} - \answer{9}) + 9
$$ 

Use positive numbers in your answer.

\begin{hint}
Remember that if you add a term to an expression, to preserve equality, you must subtract the same term.
\end{hint}

\begin{exercise}
Now rewrite $x^2 - 6x + 9$ as a square of a linear polynomial!
$$
4((x + \answer{-3})^2 - 9) + 9
$$ 
\begin{hint}
What square is equal to $x^2 - 6x + 9$? We found it before!
\end{hint}

\begin{exercise}
Finally, distribute the 4 to each term in the parentheses and simplify to finish writing the quadratic in vertex form.
$$
4(x -3)^2 - \answer{27}
$$ 

\begin{exercise}
As a bonus, here's what the manipulation would look like when done all at once:
\begin{align*}
4x^2 - 24x + 9 & = 4(x^2 - 6x) + 9 \\
& = 4(x^2 - 6x + 9 - 9) + 9 \\
& = 4((x - 3)^2 - 9) + 9 \\
& = 4(x - 3)^2 -36 + 9\\
& = 4(x - 3)^2 -27
\end{align*}
\end{exercise}
\end{exercise}
\end{exercise}
\end{exercise}
\end{exercise}
\end{exercise}



\end{document}