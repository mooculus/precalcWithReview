\documentclass{ximera}

\input{../../preamble.tex}

\author{Kenneth Berglund}




\begin{document}
For each quadratic $ax^2 + bx + c$, compute the discriminant $b^2 - 4ac$ and find the number of roots of each function. 

\begin{exercise}
\begin{enumerate}
\item The discriminant of $f(x) = 2x^2 - 3x + 5$ is $\answer{-31}$. $f$ has $\answer{0}$ root(s).

\item The discriminant of $g(x) = x^2 - 10x + 25$ is $\answer{0}$. $g$ has $\answer{1}$ root(s).

\item The discriminant of $h(x) = \frac{1}{2}x^2 + 9x - 1$ is $\answer{79}$. $h$ has $\answer{2}$ root(s). 

\item The discriminant of $p(x) = 5x^2 - 2x + 1$ is $\answer{-16}$. $p$ has $\answer{0}$ root(s). 

\item The discriminant of $q(x) = x^2 - 2x + \frac{1}{2}$ is $\answer{2}$. $q$ has $\answer{2}$ root(s). 

\item The discriminant of $r(x) = 5x^2 + 20x + 20$ is $\answer{0}$. $r$ has $\answer{1}$ root(s).
\end{enumerate}


\begin{exercise}
Let $f(x)$ be a quadratic function. 
\begin{enumerate}
\item If the discriminant of $f$ is negative, $f$ has $\answer{0}$ root(s).

\item If the discriminant of $f$ is zero, $f$ has $\answer{1}$ root(s).

\item If the discriminant of $f$ is positive, $f$ has $\answer{2}$ root(s).
\end{enumerate}
\end{exercise}
\end{exercise}


\end{document}