\documentclass{ximera}

\input{../preamble}
\author{David Kish}
\license{Creative Commons Attribution-ShareAlike 4.0 International License}


\title{Definition of Polynomials}

\begin{document}
\begin{abstract}
We explore polynomial functions.
\end{abstract}
\maketitle
A polynomal is a particular type of algebraic expression
\begin{enumerate}
\item       A company's sales, $s$
            (in millions of dollars),
            can be modeled by $2.2t+5.8$,
            where $t$ stands for the number of years since $2010$.
\item     The height of an object from the ground, $h$
            (in feet),
            launched upward from the top of a building can be modeled by $-16t^2+32t+300$,
            where $t$ represents the amount of time
            (in seconds)
            since the launch.
 \item The volume of an open-top box with a square base, $V$
            (in cubic inches),
            can be calculated by $30s^2-\frac{1}{2}s^2$,
            where $s$ stands for the length of the square base,
            and the box sides have to be cut from a certain square piece of metal.
\end{enumerate}
\section{Polynomial Vocabulary}
       A polynomial is an expression with one or more
          terms summed together.
          A term of a polynomial must either be a plain number
          or the product of a number and one or more variables raised to natural number powers.
          The expression $0$ is also considered a polynomial,
          with zero terms.
\begin{example}
Here are some examples of polynomials
\begin{enumerate}
   \item  Here are three polynomials: 
$x^2-5x+2$, $t^3-1$, $7y$.
        \item The expression $3x^4y^3+7xy^2-12xy$ is an example of a polynomial in more than one variable.
    \item   The polynomial $x^2-5x+3$ has three terms:
              $x^2$, $-5x$, and $3$.
\item The polynomial $ 3x^4+7xy^2-12xy$ also has three terms.
\item The polynomial $t^3-1$ has two terms.
\end{enumerate}
\end{example}
\begin{definition}
The coefficient
          (or numerical coefficient)
          of a term in a polynomial is the numerical factor in the term.
\end{definition}
\begin{example}
\begin{enumerate}
   \item  The coefficient of the term
              $\frac{4}{3}x^6$ is $\frac{4}{3}$.
\item   The coefficient of the second term of the polynomial $x^2-5x+3$ is $-5$.
       \item The coefficient of the term
              $\frac{y^7}{4}$ is $\frac{1}{4}$, because we can rewrite $\frac{y^7}{4}$ as $\frac{1}{4}y^7$.
\end{enumerate}
\end{example}
 A term in a polynomial with no variable factor is called a
          constant term.
\begin{example}
   The constant term of the polynomial $x^2-5x+3$ is $3$.
\end{example}
\begin{definition}
      The degree of a term is one way to measure how large it is.
          When a term only has one variable,
          its degree is the exponent on that variable.
          When a term has more than one variable,
          its degree is the sum of the exponents on the variables.
          A nonzero constant term has degree $0$.
\end{definition}
\begin{example}
 \item The degree of $5x^2$ is $2$.
\item The degree of $-\frac{4}{7}y^5$ is $5$.
\item     The degree of $-4x^2y^3$ is $5$.
\item  The degree of $17$ is $0$. Constant terms always have $0$ degree.
\end{example}
\begin{definition}
   The \dfn{degree} of a nonzero polynomial
          is the greatest degree that appears amongst its terms
\end{definition}
\begin{remark}
   To help us recognize a polynomial's degree,
        the standard convention at this level is to write a polynomial's terms in order from highest degree to lowest degree.
        When a polynomial is written in this order,
        it is written in standard form.
        For example,
        it is standard practice to write $7-4x-x^2$ as
        $-x^2-4x+7$ since $-x^2$ is the leading term.
        By writing the polynomial in standard form,
        we can look at the first term to determine both the polynomial's degree and leading term.
\end{remark}

%%%%%%%%%%
\section{Adding and Subtracting Polynomials}
        Bayani started a company that makes one product: one-gallon ketchup jugs for industrial kitchens.
        The company's production expenses only come from two things:
        supplies and labor.
        The cost of supplies, $S$
        (in thousands of dollars),
        can be modeled by $S=0.05x^2+2x+30$,
        where $x$ is number of thousands of jugs of ketchup produced.
        The labor cost for his employees, $L$
        (in thousands of dollars),
        can be modeled by $0.1x^2+4x$,
        where $x$ again represents the number of jugs they produce
        (in thousands of jugs).
        Find a model for the company's total production costs.

%%%%%%%%%%
\section{Evaluating Polynomial Expressions}

      Recall that evaluating expressions involves replacing the variable(s) in an expression with specific numbers and calculating the result.
      Here, we will look at evaluating polynomial expressions.

\begin{example}
Evaluate the expression 
\begin{center}
$-12y^3+4y^2-9y+2$ for $y=-5$ 
\end{center}
\begin{explanation}
  We will replace $y$ with $-5$ and simplify the result:
\begin{center}
$
\begin{array}{cl}
   12y^3+4y^2-9y+2 & = -12(\substitute{-5})^3+4(\substitute{-5})^2-9(\substitute{-5})+2\\
      & = -12(-125)+4(25)+45+2 \\
        & = 1647  \\
\end{array}
$
\end{center}
\end{explanation}
\end{example}

% %\section{Exponent Rules}
%   %%%%%%%%%%%%%%%%
%    \section{Product Rule}
%        If we write out $3^5\cdot 3^2$ without using exponents,
%        we'd have:
%\[
% 3^5 \cdot 3^2 = \left(3 \cdot 3\cdot 3\cdot 3\cdot 3\right) \cdot \left(3 \cdot 3\right)
%\]
%        If we then count how many $3$s are being multiplied together,
%        we find we have $5+2=7$, a total of seven $3$s.
%        So $3^5\cdot 3^2$ simplifies like this:
%       \[
%          3^5\cdot 3^2 = 3^{5+2} = 3^7
%\]
%\begin{example}          
%Simplify $x^2\cdot x^3$. \\
%\begin{explanation}       
%          To simplify $x^2\cdot x^3$,
%          we write this out in its expanded form,
%          as a product of $x$'s, we have
%\[
%            x^2\cdot x^3 =(x\cdot x)(x \cdot x \cdot x)
%            =x\cdot x\cdot x \cdot x \cdot x
%            =x^5
%   \]   
%          Note that we obtained the exponent of $5$ by adding $2$ and $3$.
%  \end{explanation}
%\end{example}
%
%      This example demonstrates our first exponent rule,
%      the Product Rule:\\
%\begin{callout}
%\textbf{ \Large Product Rule of Exponents} \\
%      When multiplying two expressions that have the same base,
%      we can simplify the product by adding the exponents.
%        \[
%x^m \cdot x^n = x^{m+n}
%\]
%\end{callout}
%   Recall that $x=x^1$. It helps to remember this when multiplying certain expressions together.
%\begin{example}        
%  Multiply $x(x^3+2)$ by using the distributive property.\\
%\begin{explanation}
%          According to the distributive property,
%        $x(x^3+2)=x\cdot x^3 + x\cdot2$
%          How can we simplify that term $x\cdot x^3$?
%          It's really the same as $x^1\cdot x^3$,
%          so according to the Product Rule, it is $x^4$.
%          So we have:
%        \[
%            x(x^3+2)=x\cdot x^3 + x\cdot2
%            =x^4+2x
%\]
%\end{explanation}
%\end{example}         
%
%%%%%%%%%%%%%%%%%
% \section{Power to a Power Rule}
%
%        If we write out $\left(3^5\right)^2$ without using exponents,
%        we'd have $3^5$ multiplied by itself:
%   \[
%         \left(3^5\right)^2 = \left(3^5\right)\cdot \left(3^5\right)
%         = \left(3\cdot 3\cdot 3\cdot 3 \cdot 3 \right) \cdot \left(3 \cdot 3\cdot 3\cdot 3\cdot 3\right)
%      \] 
%        If we again count how many $3$s are being multiplied,
%        we have a total of two groups each with five $3$s.
%        So we'd have $2\cdot 5=10$ instances of a $3$.
%        So $\left(3^5\right)^2$ simplifies like this:
%   
%          $\left(3^5\right)^2 = 3^{2\cdot 5}$
%          $= 3^{10}$
%
%
%\begin{example}
%          Simplify $\left(x^2\right)^3$.\\
%\begin{explanation}
%          To simplify $\left(x^2\right)^3$,
%          we write this out in its expanded form,
%          as a product of $x$'s, we have
%            $\left(x^2\right)^3 =\left(x^2\right) \cdot \left(x^2\right)\cdot\left(x^2\right)$
%            $=(x \cdot x)\cdot (x \cdot x)\cdot (x \cdot x)$
%            $=x^6$
%     \end{explanation}
%          Note that we obtained the exponent of $6$ by multiplying $2$ and $3$.
%\end{example}
%      This demonstrates our second exponent rule,
%      the Power to a Power Rule:
%\begin{callout}
%\textbf{ \Large Power to a Power Rule} \\
%      when a base is raised to an exponent and that expression is raised to another exponent,
%      we multiply the exponents.
%   \[
%      \left(x^m\right)^n = x^{m \cdot n}
%   \]
%\end{callout}
%
%%%%%%%%%%%%%%%%%%%
%      \section{ Product to a Power Rule}
%        The third exponent rule deals with having multiplication inside a set of parentheses and an exponent outside the parentheses.
%        If we write out $\left(3t\right)^5$ without using an exponent,
%        we'd have $3t$ multiplied by itself five times:
%\[
%      (3t)^5= (3t)(3t)(3t)(3t)(3t)
%\]
%        Keeping in mind that there is multiplication between every $3$ and $t$,
%        and multiplication between all of the parentheses pairs,
%        we can reorder and regroup the factors:
%          $\left(3t\right)^5 = (3\cdot t)\cdot (3\cdot t)\cdot (3\cdot t)\cdot (3\cdot t)\cdot (3\cdot t)$
%          $= \left(3\cdot 3\cdot 3\cdot 3\cdot 3 \right) \cdot \left(t \cdot t \cdot t \cdot t \cdot t\right)$
%          $= 3^5 t^5$
%        We could leave it written this way if $3^5$ feels especially large.
%        But if you are able to evaluate $3^5=243$,
%        then perhaps a better final version of this expression is $243t^5$.
% 
%        We essentially applied the outer exponent to each factor inside the parentheses.
%        It is important to see how the exponent $5$ applied to \textbf{both} the $3$ \textbf{and} the $t$,
%        not just to the $t$.
%   
%
% \begin{example}   
%          Simplify $(xy)^5$.
%    
%          To simplify $(xy)^5$,
%          we write this out in its expanded form,
%          as a product of $x$'s and $y$'s, we have
%
%            $(xy)^5 =(x \cdot y) \cdot (x \cdot y) \cdot (x \cdot y) \cdot (x \cdot y) \cdot (x \cdot y)$
%            $=(x \cdot x \cdot x \cdot x \cdot x) \cdot (y \cdot y \cdot y \cdot y \cdot y)$
%            $=x^5 y^5$
%
%          Note that the exponent on $xy$ can simply be applied to both $x$ and $y$.
%\end{example}
%
%
%      This demonstrates our third exponent rule,
%      the Product to a Power Rule:
%\begin{callout}
%\textbf{ \Large Product to a Power Rule} \\
%      When a product is raised to an exponent,
%      we can apply the exponent to each factor in the product.
%\[
%        \left(x\cdot y\right)^n = x^{n}\cdot y^{n}
% \]
%\end{callout}
%
%
%%%%%%%%%%%%%%%%%%
%      \section{Summary of the Rules of Exponents for Multiplication}
%  
%
%
%          If $a$ and $b$ are real numbers,
%          and $m$ and $n$ are positive integers,
%          then we have the following rules:
% 
%\textbf{Product Rule}
%\[
%            a^{m} \cdot a^{n} = a^{m+n}
%\]
%\textbf{Power to a Power Rule}
%   \[
%           (a^{m})^{n} = a^{m\cdot n}
%   \]
% \textbf{ Product to a Power Rule}
%   \[
%           (ab)^{m} = a^{m} \cdot b^{m}
%   \]  
%      Many examples will make use of more than one exponent rule.
%      In deciding which exponent rule to work with first,
%      it's important to remember that the order of operations still applies.
%\begin{example} Simplify the following expression.
%                $\left(3^7r^5\right)^4$\\
%    \begin{explanation}
%                Since we cannot simplify anything inside the parentheses, we'll begin simplifying this expression using the Product to a Power rule.
%                We'll apply the outer exponent of 4 to each factor inside the parentheses.
%                Then we'll use the Power to a Power Rule to finish the simplification process.
%      \[          
%                  \left(3^7r^5\right)^4 = \left(3^7\right)^4 \cdot \left(r^5\right)^4
%                  = 3^{7\cdot4} \cdot r^{5\cdot 4}
%                  = 3^{28}r^{20}
%         \]      
%                Note that $3^{28}$ is too large to actually compute, even with a calculator,
%                so we leave it written as $3^{28}$.
%\end{explanation}
%\end{example}
%\begin{example}
%Simplify the following expression.
%       $\left(t^3\right)^2\cdot \left(t^4\right)^5$\\
%\begin{explanation}
%                According to the order of operations,
%                we should first simplify any exponents before carrying out any multiplication.
%                Therefore, we'll begin simplifying this by applying the Power to a Power Rule and then finish using the Product Rule.
%           \[
%                  \left(t^3\right)^2\cdot \left(t^4\right)^5 = t^{3\cdot2}\cdot t^{4\cdot5}
%                  = t^6 \cdot t^{20}
%                  = t^{6+20}
%                  = t^{26}
%\]
%\end{explanation}
%\end{example}
% \begin{remark} 
%        We cannot simplify an expression like $x^2y^3$ using the Product Rule,
%        as the factors $x^2$ and $y^3$ do not have the same base.
%\end{remark}



\end{document}
