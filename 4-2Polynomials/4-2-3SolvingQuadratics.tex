\documentclass{ximera}

\input{../preamble.tex}
\author{David Kish}
\license{Creative Commons Attribution-ShareAlike 4.0 International License}


\title{Factored Form}

\begin{document}
\begin{abstract}
We explore different methods for solving quadratic functions.
\end{abstract}
\maketitle
We have previously looked at different forms of quadratic functions. We've looked at standard form and vertex form, where characteristics like $y$-intercept and vertex can be found easily by looking at the function. Another useful way to look at quadratic functions is to have them written out as a product of linear factors. This can help us to quickly determine the $x$-intercepts of a quadratic function and to get a good idea of the position and shape of the graph. Not all quadratics can be written in factored form, so we will begin by addressing those.
\begin{remark}
\textbf{Irreducible quadratic factors} are quadratic factors that when set equal to zero only have complex roots.  As a result they cannot be reduced into factors containing only real numbers, hence the name irreducible. 
\end{remark}
As seen in the graphs below, the graphs of the functions do not cross the $x$-axis, so they do not have $x$-intercepts. The first graph, $y=x^2+x+1$ is entirely above the $x$-axis and the second graph, $y=-x^2+x-1$ is entirely below the $x$-axis. Since neither of them cross the $x$-axis, they have no $x$-intercepts and are irreducible.
\begin{image}
\includegraphics[width=2in]{irreducible.PNG} \hfill \includegraphics[width=2in]{irreducible2.PNG}
\end{image}
\section{Intercept Form}
\begin{callout} \textbf{\large{Intercept Form of a Quadratic Function}}\\
          A quadratic function whose graph has x intercepts at the points $(r,0)$ and $(s,0)$ can be written as:\\
\[
f(x)=a(x-r)(x-s)
\]
\end{callout}

\begin{example}
\begin{explanation}

\end{explanation}
\end{example}
\section{Factoring from Standard Form when $a=1$}
When $a=1$, out factored form is the same as our Intercept form.

\begin{example}
Factor the following quadratic into a product of linear factors:
\[
x^2+3x+2
\]

\begin{explanation}
For us to begin factoring this quadratic, we have to look at the $b$ and $c$ terms. We are looking for $2$ numbers that muliply to $2$ (or $c$) and add up to $3$ (or $b$). By going through the factors of $2$ we can see that the only numbers that satisfy these conditions are $2$ and $1$.
\[
2+1=3
\]
\[
2\cdot 1=2
\]
This means that we can factor the quadratic the following way:
\[
x^2+3x+2=(x+2)(x+1)
\]
The quadratic is now written as a product of linear factors and because $a=1$, these are also our $x$-intercepts for our function.
\end{explanation}
\end{example}

\section{Factoring from Standard Form when $a>1$}
\section{Quadratic Formula}
\begin{callout}
\[
x=\frac{-b\pm \sqrt{b^2-4ac}}{2a}
\]
when $ax^2+bx+c =0$
\end{callout}
\section{Factoring when missing a term}
\begin{example}
Factor the quadratic $f(x)=x^2-9$ into linear components.
\begin{explanation}
This quadratic is a special case called ``difference of squares.'' There is no ``middle'' term and the remaining two terms are both perfect squares, so we can use a shortcut when factoring.
\begin{callout}
\textbf{Difference of Squares}\\
When $a$,$b$ are non zero.
\[
a^2-b^2=(a+b)(a-b)
\]
\end{callout}
In our case, we can see that $x^2$ is a perfect square and $9$ is also a perfect square because $9=3^2$. This means that our original quadratic will be factored like this:
\[
x^2-9=(x+3)(x-3)
\]
We can also think of it in the same way as factoring other quadratics. Since there is no middle term, we can look at factors of $-9$ that add up to $0$. $3$ and $-3$ add up to $0$ and multiply out to $-9$. The difference of squares is just a useful pattern that helps to speed up our factoring process.
\end{explanation}
\end{example}
\section{Factoring by Grouping}
\begin{example}
Factor the following quadratic into a product of linear factors:
\[
3x^2+5x-2
\]
\begin{explanation}
For this method, we first have to start with multiplying the $a$ and $c$ term.
\end{explanation}
\end{example}

 


\end{document}
