\documentclass{ximera}

\input{../preamble.tex}
\author{David Kish}
\license{Creative Commons Attribution-ShareAlike 4.0 International License}


\title{Factored Form}

\begin{document}
\begin{abstract}
We explore different methods for solving quadratic functions.
\end{abstract}
\maketitle
\section{Intercept Form}
\begin{callout} \textbf{\large{Intercept Form of a Quadratic Function}}\\
          A quadratic function whose graph has x intercepts at the points $(r,0)$ and $(s,0)$ can be written as:\\
\[
f(x)=a(x-r)(x-s)
\]
\end{callout}

\begin{example}
\begin{explanation}

\end{explanation}
\end{example}
\section{Factoring from Standard Form when $a=1$}
\section{Factoring from Standard Form when $a>1$}
\section{Special Cases of Factoring}
\begin{example}
Factor the quadratic $f(x)=x^2-9$ into linear components.
\begin{explanation}
This quadratic is a special case called ``difference of squares.'' There is no ``middle'' term and the remaining two terms are both perfect squares, so we can use a shortcut when factoring.
\begin{callout}
\textbf{Difference of Squares}\\
When $a$,$b$ are non zero.
\[
a^2-b^2=(a+b)(a-b)
\]
\end{callout}
In our case, we can see that $x^2$ is a perfect square and $9$ is also a perfect square because $9=3^2$. This means that our original quadratic will be factored like this:
\[
x^2-9=(x+3)(x-3)
\]
We can also think of it in the same way as factoring other quadratics. Since there is no middle term, we can look at factors of $-9$ that add up to $0$. $3$ and $-3$ add up to $0$ and multiply out to $-9$. The difference of squares is just a useful pattern that helps to speed up our factoring process.
\end{explanation}
\end{example}
\section{Quadratic Formula}
\begin{callout}
\[
x=\frac{-b\pm \sqrt{b^2-4ac}}{2a}
\]
when $ax^2+bx+c =0$
\end{callout}

 


\end{document}
