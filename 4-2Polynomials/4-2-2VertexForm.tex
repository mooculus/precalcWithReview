\documentclass{ximera}

\input{../preamble.tex}
\author{David Kish}
\license{Creative Commons Attribution-ShareAlike 4.0 International License}


\title{Vertex Form}

\begin{document}
\begin{abstract}
We explore the vertex form of a quadratic.
\end{abstract}
\maketitle


\section{The Vertex Form of a Quadratic}

      We have learned the standard form of a quadratic function's formula, which is $f(x)=ax^2+bx+c$.
      we will learn another form called the vertex form.
\begin{callout} \textbf{\large{Vertex Form of a Quadratic Function}}
          A quadratic function whose graph has vertex at the point $(h,k)$ is given by
$$
f(x)=a(x-h)^2+k
$$
\end{callout}
      Using graphing technology, consider the graphs of $f(x)=x^2-6x+7$ and $g(x)=(x-3)^2-2$ on the same axes.

        We see only one parabola because these are two different forms of the same function.
        Indeed, if we convert $g(x)$ into standard form:
	\begin{align*}
		g(x)&=(x-3)^2-2\\
			&=(x^2-6x+9)-2\\
			&=x^2-6x+7
	\end{align*}
        it is clear that $f$ and $g$ are the same function.
 

       Graph of $f(x)=x^2-6x+7$ and $g(x)=(x-3)^2-2$
       the graphs of the two parabolas overlap each other completely

       \begin{image}
\begin{tikzpicture}
                \begin{axis}[]
                    \addplot [firstcurve, domain=0:6] {x^2-6*x+7};
                    \addplot [secondcurve, domain=0:6] {(x-3)^2-2+0.1};
                    \addplot [soliddot, color=fourthcolor] coordinates {(3,-2)} node[below] {$(3,-2)$};
                \end{axis}
            \end{tikzpicture}
       \end{image}



      The formula given for $g$ is said to be in vertex form
      because it allows us to read the vertex without doing any calculations.
      The vertex of the parabola is $(3,-2)$.
      We can see those numbers in $g(x)=(x-3)^2-2$.
      The $x$-value is the solution to $(x-3)=0$,
      and the $y$-value is the constant
      added at the end.



      
\begin{example}
      Here are the graphs of three more functions with formulas in vertex form.
      Compare each function with the vertex of its graph.

   
       

           \begin{image}
\begin{tikzpicture}
                    \begin{axis}[]
                        \addplot [domain=-0.449:4.449] {(x-2)^2+1};
                        \addplot [firstcurve,soliddot] coordinates {(2,1)} node[right, firstcolor] {$(2,1)$};
                        \end{axis}
                \end{tikzpicture}
\end{image}
$$
r(x)=(x-2)^2+1
$$
           \begin{image}
\begin{tikzpicture}
                    \begin{axis}[]
                        \addplot[domain=-7:5.325] {-1/4*(x+1)^2+3};
                        \addplot [firstcurve,soliddot] coordinates {(-1,3)} node[above left, firstcolor] {$(-1,3)$};
                    \end{axis}
                \end{tikzpicture}
\end{image}
$$
s(x)=-\frac{1}{4}(x+1)^2+3
$$      
\begin{image}
\begin{tikzpicture}
                    \begin{axis}[]
                        \addplot [domain=-4.62:-1.38] {4*(x+3)^2-3.5};
                        \addplot [firstcurve,soliddot] coordinates {(-3,-3.5)} node[below, firstcolor] {$(-3,-3.5)$};
                    \end{axis}
                \end{tikzpicture}
\end{image}
$$
t(x)=4(x+3)^2-3.5
$$
\end{example}

      Notice that the $x$-coordinate of the vertex has the opposite sign as the value in the function formula.
      On the other hand,
      the $y$-coordinate of the vertex has the same sign as the value in the function formula.
      Let's look at an example to understand why.
      We will evaluate $r(2)$.

        $r(\substitute{2})=(\substitute{2}-2)^2+1$
        $=1$

      The $x$-value is the solution to $(x-2)=0$,
      which is positive $2$.
      When we substitute $\substitute{2}$ for $x$ we get the value $y=1$.
      Note that these coordinates create the vertex at $(2,1)$.
      Now we can define the vertex form of a quadratic function.


   

% %\section{Exponent Rules}
%   %%%%%%%%%%%%%%%%
%    \section{Product Rule}
%        If we write out $3^5\cdot 3^2$ without using exponents,
%        we'd have:
%$$
% 3^5 \cdot 3^2 = \left(3 \cdot 3\cdot 3\cdot 3\cdot 3\right) \cdot \left(3 \cdot 3\right)
%$$
%        If we then count how many $3$s are being multiplied together,
%        we find we have $5+2=7$, a total of seven $3$s.
%        So $3^5\cdot 3^2$ simplifies like this:
%       $$
%          3^5\cdot 3^2 = 3^{5+2} = 3^7
%$$
%\begin{example}          
%Simplify $x^2\cdot x^3$. \\
%\begin{explanation}       
%          To simplify $x^2\cdot x^3$,
%          we write this out in its expanded form,
%          as a product of $x$'s, we have
%$$
%            x^2\cdot x^3 =(x\cdot x)(x \cdot x \cdot x)
%            =x\cdot x\cdot x \cdot x \cdot x
%            =x^5
%   $$   
%          Note that we obtained the exponent of $5$ by adding $2$ and $3$.
%  \end{explanation}
%\end{example}
%
%      This example demonstrates our first exponent rule,
%      the Product Rule:\\
%\begin{callout}
%\textbf{ \Large Product Rule of Exponents} \\
%      When multiplying two expressions that have the same base,
%      we can simplify the product by adding the exponents.
%        $$
%x^m \cdot x^n = x^{m+n}
%$$
%\end{callout}
%   Recall that $x=x^1$. It helps to remember this when multiplying certain expressions together.
%\begin{example}        
%  Multiply $x(x^3+2)$ by using the distributive property.\\
%\begin{explanation}
%          According to the distributive property,
%        $x(x^3+2)=x\cdot x^3 + x\cdot2$
%          How can we simplify that term $x\cdot x^3$?
%          It's really the same as $x^1\cdot x^3$,
%          so according to the Product Rule, it is $x^4$.
%          So we have:
%        $$
%            x(x^3+2)=x\cdot x^3 + x\cdot2
%            =x^4+2x
%$$
%\end{explanation}
%\end{example}         
%
%%%%%%%%%%%%%%%%%
% \section{Power to a Power Rule}
%
%        If we write out $\left(3^5\right)^2$ without using exponents,
%        we'd have $3^5$ multiplied by itself:
%   $$
%         \left(3^5\right)^2 = \left(3^5\right)\cdot \left(3^5\right)
%         = \left(3\cdot 3\cdot 3\cdot 3 \cdot 3 \right) \cdot \left(3 \cdot 3\cdot 3\cdot 3\cdot 3\right)
%      $$ 
%        If we again count how many $3$s are being multiplied,
%        we have a total of two groups each with five $3$s.
%        So we'd have $2\cdot 5=10$ instances of a $3$.
%        So $\left(3^5\right)^2$ simplifies like this:
%   
%          $\left(3^5\right)^2 = 3^{2\cdot 5}$
%          $= 3^{10}$
%
%
%\begin{example}
%          Simplify $\left(x^2\right)^3$.\\
%\begin{explanation}
%          To simplify $\left(x^2\right)^3$,
%          we write this out in its expanded form,
%          as a product of $x$'s, we have
%            $\left(x^2\right)^3 =\left(x^2\right) \cdot \left(x^2\right)\cdot\left(x^2\right)$
%            $=(x \cdot x)\cdot (x \cdot x)\cdot (x \cdot x)$
%            $=x^6$
%     \end{explanation}
%          Note that we obtained the exponent of $6$ by multiplying $2$ and $3$.
%\end{example}
%      This demonstrates our second exponent rule,
%      the Power to a Power Rule:
%\begin{callout}
%\textbf{ \Large Power to a Power Rule} \\
%      when a base is raised to an exponent and that expression is raised to another exponent,
%      we multiply the exponents.
%   $$
%      \left(x^m\right)^n = x^{m \cdot n}
%   $$
%\end{callout}
%
%%%%%%%%%%%%%%%%%%%
%      \section{ Product to a Power Rule}
%        The third exponent rule deals with having multiplication inside a set of parentheses and an exponent outside the parentheses.
%        If we write out $\left(3t\right)^5$ without using an exponent,
%        we'd have $3t$ multiplied by itself five times:
%$$
%      (3t)^5= (3t)(3t)(3t)(3t)(3t)
%$$
%        Keeping in mind that there is multiplication between every $3$ and $t$,
%        and multiplication between all of the parentheses pairs,
%        we can reorder and regroup the factors:
%          $\left(3t\right)^5 = (3\cdot t)\cdot (3\cdot t)\cdot (3\cdot t)\cdot (3\cdot t)\cdot (3\cdot t)$
%          $= \left(3\cdot 3\cdot 3\cdot 3\cdot 3 \right) \cdot \left(t \cdot t \cdot t \cdot t \cdot t\right)$
%          $= 3^5 t^5$
%        We could leave it written this way if $3^5$ feels especially large.
%        But if you are able to evaluate $3^5=243$,
%        then perhaps a better final version of this expression is $243t^5$.
% 
%        We essentially applied the outer exponent to each factor inside the parentheses.
%        It is important to see how the exponent $5$ applied to \textbf{both} the $3$ \textbf{and} the $t$,
%        not just to the $t$.
%   
%
% \begin{example}   
%          Simplify $(xy)^5$.
%    
%          To simplify $(xy)^5$,
%          we write this out in its expanded form,
%          as a product of $x$'s and $y$'s, we have
%
%            $(xy)^5 =(x \cdot y) \cdot (x \cdot y) \cdot (x \cdot y) \cdot (x \cdot y) \cdot (x \cdot y)$
%            $=(x \cdot x \cdot x \cdot x \cdot x) \cdot (y \cdot y \cdot y \cdot y \cdot y)$
%            $=x^5 y^5$
%
%          Note that the exponent on $xy$ can simply be applied to both $x$ and $y$.
%\end{example}
%
%
%      This demonstrates our third exponent rule,
%      the Product to a Power Rule:
%\begin{callout}
%\textbf{ \Large Product to a Power Rule} \\
%      When a product is raised to an exponent,
%      we can apply the exponent to each factor in the product.
%$$
%        \left(x\cdot y\right)^n = x^{n}\cdot y^{n}
% $$
%\end{callout}
%
%
%%%%%%%%%%%%%%%%%%
%      \section{Summary of the Rules of Exponents for Multiplication}
%  
%
%
%          If $a$ and $b$ are real numbers,
%          and $m$ and $n$ are positive integers,
%          then we have the following rules:
% 
%\textbf{Product Rule}
%$$
%            a^{m} \cdot a^{n} = a^{m+n}
%$$
%\textbf{Power to a Power Rule}
%   $$
%           (a^{m})^{n} = a^{m\cdot n}
%   $$
% \textbf{ Product to a Power Rule}
%   $$
%           (ab)^{m} = a^{m} \cdot b^{m}
%   $$  
%      Many examples will make use of more than one exponent rule.
%      In deciding which exponent rule to work with first,
%      it's important to remember that the order of operations still applies.
%\begin{example} Simplify the following expression.
%                $\left(3^7r^5\right)^4$\\
%    \begin{explanation}
%                Since we cannot simplify anything inside the parentheses, we'll begin simplifying this expression using the Product to a Power rule.
%                We'll apply the outer exponent of 4 to each factor inside the parentheses.
%                Then we'll use the Power to a Power Rule to finish the simplification process.
%      $$          
%                  \left(3^7r^5\right)^4 = \left(3^7\right)^4 \cdot \left(r^5\right)^4
%                  = 3^{7\cdot4} \cdot r^{5\cdot 4}
%                  = 3^{28}r^{20}
%         $$      
%                Note that $3^{28}$ is too large to actually compute, even with a calculator,
%                so we leave it written as $3^{28}$.
%\end{explanation}
%\end{example}
%\begin{example}
%Simplify the following expression.
%       $\left(t^3\right)^2\cdot \left(t^4\right)^5$\\
%\begin{explanation}
%                According to the order of operations,
%                we should first simplify any exponents before carrying out any multiplication.
%                Therefore, we'll begin simplifying this by applying the Power to a Power Rule and then finish using the Product Rule.
%           $$
%                  \left(t^3\right)^2\cdot \left(t^4\right)^5 = t^{3\cdot2}\cdot t^{4\cdot5}
%                  = t^6 \cdot t^{20}
%                  = t^{6+20}
%                  = t^{26}
%$$
%\end{explanation}
%\end{example}
% \begin{remark} 
%        We cannot simplify an expression like $x^2y^3$ using the Product Rule,
%        as the factors $x^2$ and $y^3$ do not have the same base.
%\end{remark}



\end{document}
