\documentclass{ximera}
\input{preamble}
\input{orccaImagePreamble}
\usepackage{hyperref}
\usepackage{lipsum}
\usepackage{lmodern}
\usepackage{tcolorbox}
\usepackage{cancel}


\author{David Kish, Debi Stout}
% Source: 1116 Materials



\title{Estimates, Units, and Percents}

\begin{document}

\begin{abstract}
  Some abstract
\end{abstract}
\maketitle

%\typeout{************************************************}
%\typeout{Rough Estimages}
%\typeout{************************************************}

\section{Rough Estimates}

\begin{tcolorbox}%[colback=blue!5]
\section{Tips for Doing Rough Estimates} 
\begin{itemize}
\item Estimations are \textbf{NOT} guesses.  They can sometimes be educated guesses, but estimations, at least in this course, will always still require some sort of calculation(s).
\item Rough estimations are meant to be estimations which can be calculated mentally, meaning without calculators, or even pen and paper.  You will still be expected to record/write down your work and thought process for your instructor, but it should be work that you are able to do mentally.
\item Everyone has different skill and comfort levels with their mental calculations -- you may need to round values a lot to make them something you are comfortable doing in your head.  This is completely fine, but the important thing is that in your solutions/write-ups you explain what numbers you rounded and why.
\end{itemize}
\end{tcolorbox}
\begin{problem}[In Class Activity]Pizza Party: 
Now let’s give this a try but working with your group to determine what you would buy for a pizza party in the following scenario: 
You and your roommate are going to have some people over later and so you go to the grocery store to grab some snacks.  Everyone has agreed to pitch in \$10 to pay for pizza and snacks for the night, and you think about 12 people are coming over.  Here are the prices of various snacks from the grocery store: 
\begin{itemize}
    \item Bag of tortilla chips - \$3.99 
    \item Salsa - \$3.79 
    \item Bag of “normal” chips - \$2.99 
    \item Dozen cookies from bakery - \$4.99 
    \item Veggie Tray - \$14.99 
    \item 12 pack of soda - \$5.79 
\end{itemize}
You want to get at least one of each of these items, but you’re also going to order some pizza and breadsticks. 
Here are the prices from your pizza place: 
\begin{itemize}
    \item Cheese: 
    \begin{itemize}
        \item Medium - \$9.99 
        \item Large - \$11.99 
    \end{itemize}
    \item Pepperoni: 
    \begin{itemize}
        \item Medium - \$11.49 
        \item Large - \$13.49 
    \end{itemize}
    \item Breadsticks: 
    \begin{itemize}
        \item 5 for \$4.99 
    \end{itemize}
    \item \$2.49 delivery charge.  Don’t forget tip! 
\end{itemize}


You are a good person and do not plan to pocket any of the money that your friends are going to give you to pay for these pizzas and snacks.  Plus, you and your roommate are going to pay for your fairshare (also each pitch in \$10).  Decide what you’re going to get! 
\end{problem}
\begin{tcolorbox}[colback=yellow!15]
\begin{example}
Roughly estimate how many gallons of gasoline you might use to drive from here to Chicago, IL.

\begin{center}
    \includegraphics[width=4in]{ColumbusChicago.png}
\end{center}

\begin{explanation}
Distance to Chicago, IL: about 350 miles (google maps)
My vehicle gets about 30 miles per gallon (30 miles = 1 gallon), so $350$ mi $\times \frac{1 gal}{30 mi}$ is about ${360 \over 30} = 12$ gallons
\end{explanation}
\end{example}
\end{tcolorbox}

\begin{problem}
Roughly estimate how many seconds you’ve been alive. 
\begin{itemize}
    \item Determine a number that would definitely be way too low for even a rough estimate for each question, but still requires some calculation. Explain why you think this would be unreasonably low. (Still do not use a calculator, use mental arithmetic.) 
    \item Determine a number that would definitely be way too high for even a rough estimate for each question. Explain why you think this would be unreasonably high.
    \item For each of those problems, do you think your original rough estimation is an underestimate or overestimate? Explain why you think this based off of your calculations. Or if you’re not sure if it’s under or over, explain why you are unsure.
    \item If possible, determine a more exact value by using a calculator and not rounding values to get an exact number, or see if google has any estimate(s). Compare this with your original estimations: were you under or over estimating, or can’t tell? If you have values to compare, how different are the two values? 
\end{itemize}
\end{problem}

%\typeout{************************************************}
%\typeout{Units}
%\typeout{************************************************}

\section{Units}
Is 12 the same as 1?  As a mathematical value, no 12 and 1 are not the same value.  But if we give these values units, they actually can represent the same thing! \\
What units can we give to 12 and to 1 so that they are equal? \\
12 $\answer{\text{inches}}$ = 1 $\answer{\text{foot}}$
Units in everyday life are often used, even if we don’t necessarily think of them as units.  For example, you wouldn’t say “I went to the grocery store and bought a dozen.”  A dozen what?  You’ve given how many you’ve bought (the value), but not how many of what you have bought (the units).  Instead you would say “I went to the grocery store and bought a dozen eggs.”


%\typeout{************************************************}
%\typeout{Percentages}
%\typeout{************************************************}

\section{Percentages}
Percentages are everywhere!  They are used to describe discounts, markups, commissions, statistics, information, change, and on and on.  They are an important topic and so we want to make sure that we really understand them.  Let’s start by just breaking down the word “percent.”  “Per” and “cent.”  There are some specific words that often translate into certain mathematical operations.  For example, if a question asks, “how many apples and oranges are there?” what operation are you going to do with the number of apples and oranges?  Let’s make a short list of which words usually translate into which operation: \\
If you read a math problem: “There are 5 apples and 6 oranges.  How many apples and oranges are there?”  What math operation are you going to do with the number of apples (5) and oranges (6)?
\begin{center}
    “and” often translates to: $\answer{\text{addition}}$
\end{center}
Let’s change the problem a little bit: “There are 4 apples and 7 oranges.  What is the difference between the number of apples and number of oranges?”  What math operation are you going to do with the number of apples (4) and oranges (7)?
\begin{center}
    “difference” often translates to $\answer{\text{subtraction}}$
\end{center}

Let’s change it again: “We want each of the 3 people in our group to have 5 apples.  How many apples are we going to need?”  What math operation are you going to do with the number of people (3) and apples (5)?
\begin{center}
  “each of” often translates to: $\answer{multiplication}$
\end{center}

Note: We will also see in the following problems that specifically with percentages, “of” often translates to this same operation. \\
Another change: “We have a group of 5 people and a total of 15 apples.  How many apples per person are there?”  What math operation are you going to do with the number of apples (15) and people (5)?
\begin{center}
    “per” often translates to: $\answer{division}$
\end{center}
Let’s change it one last time: “The total number of apples and oranges is 13.  If the number of apples is “x” and the number of oranges is “y,” write an equation for the total number of apples and oranges.”  What math symbol did you put in for the word “is”?
\begin{center}
    “is” often translates to: $\answer{equals}$
\end{center}
Now “cent.”  Where have we seen this word before?  How many cents are in a dollar?  How many years are in a century?  How many centimeters are in a meter?  
\begin{center}
    Wherever you see the word “cent”, it is likely representing the number $\answer{100}$
\end{center}
\begin{center}
Hence, “percent” can be interpreted as $\answer{divide}$ by $\answer{100}$
\end{center}
So, for example, 13\% = $\answer{\frac{13}{100}}$ = 0.$\answer{13}$\\
\begin{tcolorbox}%[colback=blue!5]
\section{Percent Increase and Decrease}
In the problems where percent increases or decreases are calculated, we calculate not only the percent change, but also the “actual value” change as well.  There are specific vocabulary terms to describe these two “measures” of changes: absolute change and relative change. 
\begin{itemize}
    \item \textbf{Absolute change} is the “actual value” change that occurred.  For example, in question 2, the absolute change was \$14.69 -- the actual dollar amount that the price changed. 
    \item \textbf{Relative change} is the percent change that occurred.  For example, again in question 2, the relative change was 30\% decrease -- the percent or proportion that the dollar amount was changed. 
\end{itemize}

\end{tcolorbox}%[colback=blue!5]
%\typeout{************************************************}
%\typeout{Patterns in Tables}
%\typeout{************************************************}

%\section{Patterns in Tables}
%% From https://spot.pcc.edu/math/orcca/ed2/html/section-exploring-two-variable-data-and-rate-of-change.html
%\begin{example}
%What is the missing entry in each table? Can you describe each pattern in words and/or mathematics?
%
%\includegraphics{2-2table1.jpg}
%%how to do the tables in TeX so Accessible?
%
% \begin{explanation}
%We can view the table as assigning each input in the left column a corresponding output in the right column. It takes a number as input, and give twice that number as its output. Mathematically, we can describe the pattern as $y=2x$, where $x$  represents the input, and $y$ represents the output. Labeling the table mathematically, we have
%
%\includegraphics{2-2table1.2.jpg}
%%how to do the tables in TeX so Accessible?
%
%\end{explanation}
%\end{example}
%
%For each of the following tables, find an equation that describes the pattern you see. Numerical pattern recognition may or may not come naturally for you. Either way, pattern recognition is an important mathematical skill that anyone can develop. Solutions for these exercises provide some ideas for recognizing patterns.
%
%\begin{problem}
%Write an equation in the form $y=...$  suggested by the pattern in the table.
%
%\includegraphics{2-2table2.jpg}
%%how to do the tables in TeX so Accessible?
%
%$y=\answer{10x}$
%
%\begin{explanation}
%One approach to pattern recognition is to look for a relationship in each row. Here, the $y$-value in each row is always 10 more than the $x$-value. So the pattern is described by the equation $y=10x$
%\end{explanation}
%
%\end{problem}
%
%\begin{problem}
%Write an equation in the form $y=...$  suggested by the pattern in the table.
%
%\includegraphics{2-2table3.jpg}
%%how to do the tables in TeX so Accessible?
%
%$y=\answer{3x-1}$
%
%\begin{explanation}
%The relationship between $x$ and $y$  in each row is not as clear here. Another popular approach for finding patterns: in each column, consider how the values change from one row to the next. From row to row, the $x$-value increases by 1.   Also, the $y$-value increases by 3 from row to row.
%
%\includegraphics{2-2table4.jpg}
%%how to do the tables in TeX so Accessible?
%
%Since row-to-row change is always 1 for $x$ and is always 3 for $y$ the rate of change from one row to another row is always the same: 3 units of $y$ for every 1 unit of $x$. This suggests that $y=3x$ might be a good equation for the table pattern. But if we try to make a table with that pattern:
%
%\includegraphics{2-2table5.jpg}
%%how to do the tables in TeX so Accessible?
%
%We find that the values from $y=3x$ are 1 too large. So now we make an adjustment. The equation $y=3x-1$ describes the pattern in the table.
%
%\end{explanation}
%
%\end{problem}
%
%
%
%
%%\typeout{************************************************}
%%\typeout{Rates of Change}
%%\typeout{************************************************}
%
%\section{Rates of Change}
%% From https://spot.pcc.edu/math/orcca/ed2/html/section-slope.html
%
%For an hourly wage-earner, the amount of money they earn depends on how many hours they work. If a worker earns \$15per hour, then 10
%hours of work corresponds to \$150of pay. Working one additional hour will change 10 hours to 11 hours; and this will cause the \$150 in pay to rise by fifteen dollars to \$165 in pay. Any time we compare how one amount changes (dollars earned) as a consequence of another amount changing (hours worked), we are talking about a \textbf{rate of change}.
%
%Given a table of two-variable data, between any two rows we can compute a \textbf{rate of change}.
%
%\begin{example}
%The following data, given in both table and graphed form, gives the counts of invasive cancer diagnoses in Oregon over a period of time. (\url{wonder.cdc.gov})
%
%\includegraphics{2-2table7.jpg}
%%how to do the tables in TeX so Accessible?
%
%\begin{tikzpicture}
%    \begin{axis}[xlabel={year},
%            ylabel={count},
%            ytick=,
%            xmin=1995,xmax=2016,
%            ymin=15000,ymax=21000,
%            x axis line style={->},
%            y axis line style={->},
%            xtick={2000,2005,2010,2015},
%            xticklabels={2000,2005,2010,2015},
%            xticklabel style={rotate=-45},
%            axis y discontinuity=crunch,]
%        \addplot[only marks]coordinates{
%               (1999,17599)
%               (2000,17446)
%               (2001,17847)
%               (2002,17887)
%               (2003,17559)
%               (2004,18499)
%               (2005,18682)
%               (2006,19112)
%               (2007,19376)
%               (2008,20370)
%               (2009,19909)
%               (2010,19727)
%               (2011,20636)
%               (2012,20035)
%               (2013,20458)
%        };
%    \end{axis}
%\end{tikzpicture}
%
%What is the \textbf{rate of change} in Oregon invasive cancer diagnoses between 2000 and 2010? 
%
%\begin{explanation}
%The total (net) change in diagnoses over that timespan is
%$$19727-17446=2281$$
%meaning that there were 2281 more invasive cancer incidents in 2010 than in 2000. Since 10 years passed (which you can calculate as 2010-2000), the rate of change is 2281 diagnoses per 10 years, or
%$$\frac{2281 \text{ diagnoses}}{10 \text{ years}} = 228.1 \frac{\text{diagnoses}}{\text{year}}$$
%We read that last quantity as “228.1 diagnoses per year.” This rate of change means that between the years 2000 and 2010, there were 228.1 more diagnoses each year, on average. This is just an average over those ten years—it does not mean that the diagnoses grew by exactly this much each year.  We dare not interpret why that increase existed,  just that it did.  If you are interested in examining causal relationships that exist in real life,  we strongly recommend a statistics course or two in your future!
%\end{explanation}
%\end{example}
%
%\begin{tcolorbox}
%\begin{definition} 
% If $(x_1,y_1)$ and $(x_2,y_2)$ are two data points from a set of two-variable data, then the \textbf{rate of change} between them is
%$$ \frac{\text{change in } y}{\text{change in } x} = \frac{\Delta y}{\Delta x} = \frac{y_2-y_1}{x_2-x_1}$$
%(The Greek letter delta, $\Delta$ , is used to represent “change in” since it is the first letter of the Greek word for “difference.”)
%\end{definition}
%\end{tcolorbox}
%
%Here are some examples of rates of change from our example above.
%
%\begin{tikzpicture}
%    \begin{axis}[xlabel={year},
%            ylabel={count},
%            ytick=,
%            xmin=1995,xmax=2016,
%            ymin=15000,ymax=21000,
%            x axis line style={->},
%            y axis line style={->},
%            xtick={2000,2005,2010,2015},
%            xticklabels={2000,2005,2010,2015},
%            xticklabel style={rotate=-45},
%            axis y discontinuity=crunch,
%            clip=false]
%        \addplot[only marks]coordinates{
%               (1999,17599)
%               (2000,17446)
%               (2001,17847)
%               (2002,17887)
%               (2003,17559)
%               (2004,18499)
%               (2005,18682)
%               (2006,19112)
%               (2007,19376)
%               (2008,20370)
%               (2009,19909)
%               (2010,19727)
%               (2011,20636)
%               (2012,20035)
%               (2013,20458)
%        };
%        \addplot[firstcurve,-] coordinates {(2000,17446) (2010,19727)} node[pos=0.35, pin=100:{rate $228.1\,\frac{\text{diagnoses}}{\text{year}}$}]{};
%        \addplot[firstcurve,-] coordinates {(1999,17599) (2002,17887)} node[pos=0.5, pin=120:{rate $96\,\frac{\text{diagnoses}}{\text{year}}$}]{};
%        \addplot[firstcurve,-] coordinates {(2003,17559) (2011,20636)} node[pos=0.25, pin=-45:{rate $384.6\,\frac{\text{diagnoses}}{\text{year}}$}]{};
%    \end{axis}
%\end{tikzpicture}
%
%Note how the larger the numerical rate of change between two points, the steeper the line is that connects them in a graph. This is such an important observation, we'll put it in an official remark.
%
%\begin{tcolorbox}[colback=blue!5]
%\begin{remark}
%The rate of change between two data points is intimately related to the steepness of the line segment that connects those points.
%\begin{enumerate}
%\item The steeper the line, the larger the rate of change, and vice versa.
%\item If one rate of change between two data points equals another rate of change between two different data points, then the corresponding line segments will have the same steepness.
%\item We always measure rate of change from left to right. When a line segment between two data points slants up from left to right, the rate of change between those points will be positive. When a line segment between two data points slants down from left to right, the rate of change between those points will be negative.
%\end{enumerate}
%\end{remark}
%\end{tcolorbox}
%
%Let's revisit the earlier example $y=3x-1$
%
%\includegraphics{2-2table3.jpg}
%
%The key observation in this example was that the rate of change from one row to the next was constant: 3 units of increase in $y$ for every 1  unit of increase in $x$. Graphing this pattern in , we see that every line segment here has the same steepness, so the whole picture is a straight line.
%
%\begin{tikzpicture}
%    \begin{axis}[ymin=-5,ymax=9,ytick={-4,2,...,8},minor ytick={-5,-4,...,9},width=0.47\linewidth]
%        \addplot[only marks]coordinates{
%               (0,-1)
%               (1,2)
%               (2,5)
%               (3,8)
%        };
%        \addplot[firstcurve,-] coordinates {(0,-1) (3,8)};
%    \end{axis}
%\end{tikzpicture}
%
%Whenever the rate of change is constant no matter which two $(x,y)$-pairs (or data pairs) are chosen from a data set, then you can conclude the graph will be a straight line even without making the graph. We call this kind of relationship a \textbf{linear relationship}. We'll study linear relationships in more detail throughout this section.
%

\end{document}
