\documentclass{ximera}

\input{../preamble.tex}
\author{Elizabeth Miller, Ivo Terek}
\license{Creative Commons Attribution-ShareAlike 4.0 International License}
\acknowledgement{}
%Source: Stitz-Zeager
\title{Graphing Rational Functions}

\begin{document}
\licenseSZ
\begin{abstract}
\end{abstract}
\maketitle

\section{Behavior Near Points Not in the Domain}

We are now getting closer to understanding the properties of rational functions.  We have discussed how to find the domain, end behavior, and $x$-intercepts.  Finding the $y$-intercept just involves plugging $0$ into the function.  But there is one major feature of rational functions we still need to discuss before we can understand them fully.  What happens to the graph of a rational function near the points which are not in the domain?  

Recall that:

\begin{callout}
A rational functions is not defined when the polynomial in the denominator is equal to zero.  That is, if $r(x)=\frac{p(x)}{q(x)}$ is a rational function and both $p(x)$ and $q(x)$ are polynomials, than the domain of $r(x)$ is all values of $x$ except those where $q(x)=0$.  
\end{callout}

Rational functions can have two different kinds of behavior near a point where the denominator is zero.  

We know from investigating $f(x)=\frac{1}{x}$ that one possibility is a vertical asymptote.  Recall that:

\begin{definition}
The line $x=c$ is called a \emph{vertical asymptote} of the graph of a function $y=f(x)$ if as $x \to c^-$ or $x \to c^+$, either $f(x) \to +\infty$ or $f(x) \to -\infty$.  

Here, as $x \to c^-$ means for points near $x=c$ but less than $c$ and as $x \to c^+$ means for points near $x=c$ but greater than $c$.
\end{definition}

When you put points into a rational function that are close to the point where it is undefined, it will make the bottom of the fraction very small.  Usually, that will make the resulting fraction very big, resulting in a vertical asymptote on the graph.  The exception is when the top of the fraction is also getting very small.  Then, it is not quite clear what is happening to the value of the fraction overall.  Let's investigate.

\begin{example}
Let's return to the function $f(x) = \frac{x-1}{x^2-3x+2}$.  Earlier, we looked at the end behavior of this function and determined that $f(x) \to 0$ as $x \to +\infty$ and $f(x) \to 0$ as $x \to -\infty$.  Now, let's investigate the middle behavior of this function, by which I mean what it is doing between the two horizontal asymptotes.

First, we need to determine the points which are not in the domain of $f$.  That is, which values of $x$ make $f(x)$ undefined?  Note that we can factor the denominator of this rational function.  
$$x^2-3x+2 = (x-1)(x-2).$$
Setting the denominator equal to zero will let us find the points where $f(x)$ is undefined.  This means $(x-1)=0$ or $(x-2)=0$, giving us that $f(x)$ is undefined for $x=1$ and $x=2$. Now lets investigate what happens near each of these points.

Notice that we may write that $$   f(x) = \frac{x-1}{x^2-3x+2} = \frac{(x-1)}{(x-1)(x-2)} = \frac{1}{x-2} \text{\,\,\, \textbf{provided} \, } x \neq 1$$. 

This means that for all values of $x$ other than $x=1$, $f(x)$ gives the same $y$-value as $\frac{1}{x-2}$.  This means that the graph of $f(x)$ and the graph of $\frac{1}{x-2}$ should look identical except for at $x=1$.  

At $x=1$, $f(x)$ is not defined, but $x=1$ is not a special point at all for $\frac{1}{x-2}$.  The graph will continue as normal on either side of $x=1$ and there will be \textbf{no asymptote} there.  Therefore, the conclusion is that $f(x)$ will have a \textbf{hole} at $x=1$.   We can even find the height ($y$-value) at which to draw the hole by finding out what the $y$-value of $\frac{1}{x-2}$ is at $x=1$.  Since we keep referencing it, let's give $\frac{1}{x-2}$ a name.  It is not the same function as $f$ because it has a different domain.  Therefore, it needs a different name.  Let's say $g(x)=\frac{1}{x-2}$.  Then we have that 
$$g(1)=\frac{1}{1-2}=\frac{1}{-1}=-1.$$
Our conclusion is that the graph of the function $f(x)=\frac{(x-1)}{(x-1)(x-2)}$ will have a \textbf{hole at the point $(1,-1)$}.

Let's let's compare that with the behavior of $f(x)$ near $x=2$.  Recall that for all values of $x$ other than $x=1$, $f(x)=\frac{(x-1)}{(x-1)(x-2)}$ gives the same $y$-value as $g(x)=\frac{1}{x-2}$.  Since we are not investigating near one, we can just use the simpler function, $g(x)=\frac{1}{x-2}$.  Both $f$ and $g$ will have the same behavior everyone other than at $x=1$!.  But comparing the expression $g(x) = 1/(x-2)$ with what we have previously seen for the function $1/x$, we see that $f(x) \to +\infty$ as $x \to 2^+$ and $f(x) \to -\infty$ as $x \to 2^-$, so that the line $x=2$ is a vertical asymptote for $f(x)$. 
\end{example}

\section{Finding Vertical Asymptotes, Holes, and Zeros}

You may notice that the function $g$ in the previous example played an important role in our ability to determine whether our original function $f$ had a hole or an asymptote at each point which was not in the domain.  The function $g$ was special because it had no factors in common between the numerator and denominator of the rational function.  We give rational functions like this a special name. 

\begin{definition}  
A rational function is said to be in \textbf{lowest terms} if the numerator and the denominator have no factors in common.
\end{definition}

This allows us to state the following theorem.

\begin{theorem}
A rational function in lowest terms will have a vertical asymptote at every point which is not in its domain.
\end{theorem}

For rational functions which are not in lowest terms, we will want to factor the numerator and denominator completely to determine what cancels.  We can the write the rational function in lowest terms, being careful to keep track of any points which are not in the domain.  

\begin{theorem}
Every rational function, $f(x)=\frac{p(x)}{q(x)}$, can be written in lowest terms $g(x)=\frac{r(x)}{s(x)}$ where $r(x)$ and $s(x)$ have no common factors, provided the domain of the function is clearly stated so that any points which were not in the domain of, $f$, the original function are still excluded from $g$, the version of the function in lowest terms.  Any points which are excluded in this way and for which the  $s(x) \neq 0$ will be holes in the graph of the function.  Any points where $s(x)=0$ will be vertical asymptotes.
\end{theorem}

Let's try applying this theorem to an example.

\begin{example}
 Let $$f(x)=\frac{(x-5)(x^2+3)^2(x+3)(x+2)^3}{x(x+2)(x^2-9)(x-5)^5}$$

Determine where $f$ has vertical asymptotes, holes, and zeros.
\begin{explanation}
Let's break this analysis into steps.

\begin{enumerate}

\item[Step 1:] The first step is to make sure that both the numerator and denominator are completely factored into linear and \textbf{irreducible} quadratic terms.  Recall that a quadratic is irreducible when the discriminant, $b^2-4ac<0$.  Let's look at the two quadratic factor above.  

The factor $(x^2+3)$ is irreducible.  It has a disciminant of $0^2-4(1)(3)=-12<0$.  This means it cannot be factored and never equals zero.  Essentially, for the purposes of this problem, this factor is not going to contribute to any zeros, holes, or vertical asymptotes.  We can ignore it.

The factor $(x^2-9)$ is a difference of squares.  $(x^2-9)=(x-3)(x+3)$.  Thus, we want to replace $(x^2-9)$ with $(x+3)(x-3)$ in our formula for $f(x)$.

$$f(x)=\frac{(x-5)(x^2+3)^2(x+3)(x+2)^3}{x(x+2)(x+3)(x-3)(x-5)^5}$$

\item[Step 2:] The next step is to determine the points which are not in the domain of $f$.  To do this, we set the denominator equal to zero.  $x(x+2)(x+3)(x-3)(x-5)^5=0$ whenever one of the factors equals zero.  From the Polynomial Remainder Theorem, we can just read off the zeros for each term.  The zeros will be $x=0$ coming from the factor of $x=(x-0)$, $x=-2$ which comes from the factor $(x+2)$, $x=-3$ which comes from the factor $(x+3)$, $x=3$ which comes from the factor $(x-3)$ and $x=5$ which comes from the factor $(x-5)^5=0$.

\item[Step 3:]  Now that we know the domain, we can rewrite $f(x)$ in lowest terms, \textbf{provided} we keep track of which points are not in the domain.

\begin{align*}
f(x)&=\frac{(x-5)(x^2+3)^2(x+3)(x+2)^3}{x(x+2)(x+3)(x-3)(x-5)^5}\\
&=\frac{(x^2+3)^2(x+3)^{0}(x+2)^{3-1}}{x(x-3)(x-5)^{5-1}}  \text{\,\,\, \textbf{provided} \, } x \neq -3, -2, 0, 3, \text{or} \, \, 5 \\
&=\frac{(x^2+3)^2(x+2)^{2}}{x(x-3)(x-5)^{4}}  \text{\,\,\, \textbf{provided} \, } x \neq -3, -2, 0, 3, \text{or} \, \, 5  \\
\end{align*}

\item[Step 4:]  Now that $f(x)$ is written in lowest terms, anywhere the denominator is zero is a vertical asymptote.  Therefore, we want to consider $x(x-3)(x-5)^{4}=0$.  This gives us vertical asymptotes at $x=0, x=3,$ and $x=5$.

\item[Step 5:]  Now, let's determine the holes.  There is a hole at any $x$-values which are not in the domain but which are not vertical asymptotes.  Since the $x$-values $x =-3, -2, 0, 3,$ and 5 are not in the domain and only   $x=0, x=3,$ and $x=5$ are vertical asymptotes, we have that $x=-3$ and $x=-2$ will be $x$-values where holes are in the graph.

\item[Step 6:]  Finally, let's determine the zeros of $f$.  A rational function equals zero whenever the numerator equals zero, provided those points are in the domain of the function.  We can consider the numerator of version of the formula written in lowest terms.  We have $(x^2+3)^2(x+2)^{2}=0$.  We already determined that $(x^2+3)^2 \neq 0$.  Thus, we just have $x=-2$ coming from the term $(x+2)^2=0$.  We check to see whether $x=-2$ is in the domain of $f$.  It is not.  Therefore, it is not a zero.  $f$ has no zeros.
\end{enumerate}

In conclusion, we have interesting phenomena happening whenever the numerator or the denominator of the function equal zero.  For this function, we have vertical asymptotes at $x=0, x=3,$ and $x=5$, holes at $x=-3$ and $x=-2$, and no zeros.
\end{explanation}
\end{example}


\section{Graphing a Rational Function}

Now let's return the previous example and see if we can draw the graph of $f(x)=\frac{x-1}{x^2-3x+2}$.  We have already determined:

\begin{itemize}
\item As $x \to +\infty$ and $f(x) \to 0$ (or a borizontal asymptote at $y=0$)
\item As $x \to -\infty$ and $f(x) \to 0$ (or a borizontal asymptote at $y=0$)
\item The graph will have a hole at $(1,-1)$
\item The graph will have a vertical asymptote at $x=2$.
\end{itemize}

Now let's find a few additional points to help us graph the function.

To find the $y$-intercept of $f$, $f(0)=\frac{(0)-1}{(0)^2-3(0)+2}=\frac{-1}{2}$.  To find any $x$-intercepts, we set $y=0$, meaning we are looking at
$$0=\frac{x-1}{x^2-3x+2}.$$
A fraction equals zero when the numerator equals zero, so we need
\begin{align*}
x-1&=0\\
x&=1
\end{align*}
But, $x=1$ is not in the domain of our function so it is not actually an $x$-intercept!  There are no $x$-intercepts!

It is usually a good idea to find a point between each place where the function is zero or undefined.  This will help us see where the function is.  There is a result in calculus that the graph of a rational function can only change from being postive (above the $x$-axis) to negative (below the $x$-axis) or vice versa when the function is zero or undefined.  We will explore this more later, but this idea motivated the points we are going to choose to plot.

\begin{align*}
f(-1)&=\frac{(-1)-1}{(-1)^2-3(-1)+2}=\frac{-2}{6}=\frac{-1}{3}\\
f\left(\frac{3}{2}\right)&=\frac{\left(\frac{3}{2}\right)-1}{\left(\frac{3}{2}\right)^2-3\left(\frac{3}{2}\right)+2}=\frac{\frac{1}{2}}{\frac{9}{4}-\frac{9}{2}+2}=\frac{\frac{1}{2}}{\frac{-1}{4}}=\frac{1}{2} \cdot \frac{-4}{1}=-2\\
f(3)&=\frac{(3)-1}{(3)^2-3(3)+2}=\frac{2}{2}=1
\end{align*} 

Putting all of this together, we are able to draw the following graph:

      \begin{image}
        \begin{tikzpicture}
          \begin{axis}
            \addplot[samples=200,domain=-7:0.99]{1/(x-2)};
            \addplot[samples=200,domain=1.01:1.99]{1/(x-2)};
            \addplot[samples=200,domain=2.01:7]{1/(x-2)};
            \draw[fill=white](axis cs:1,-1)circle(1mm);
            \addplot[dashed, thick, color=penColor] coordinates {(2,-7) (2,7)};
            \addplot[thick, dashed, color=penColor, domain=-7:7]{0};
          \end{axis}
        \end{tikzpicture}
      \end{image}

It is important to note that many graphing calculators such as Desmos will not show the hole in the graph, but it is important to know that it is there.







\end{document}

%%%%%%%%%%%%%%%%%%%%%%%%%%%%%%%%%%%%%%%%%%%%%%%%%%%%%%%%%%%%%%%%%%%%%





\begin{example}
As for vertical asymptotes, we see that the only $x$-value for which $f(x)$ is undefined is $x=2$. So that's where we'll look, by choosing values for $x$ very close to $2$, but not equal to $2$. For example, we have that
    \begin{align*}
      f(2.1) &= \frac{1.1}{0.1} = 11 \\ f(2.01) &= \frac{1.01}{0.01} = 101 \\ f(2.001) &= \frac{1.001}{0.001} = 1001 
    \end{align*}
    indicates that $f(x) \to +\infty$ as $x \to 2^+$. Similarly, you can check that $f(x) \to -\infty$ as $x \to 2^-$, so the line $x=2$ is a vertical asymptote for $f(x)$ (in fact, the only one). Here's a graph:

\begin{image}
\begin{tikzpicture}
    \begin{axis}
      \addplot[samples=200,domain=2.01:7]{(x-1)/(x-2)};
      \addplot[samples=200,domain=-7:1.99]{(x-1)/(x-2)};
      \addplot[dashed, color=penColor] coordinates {(2,-7) (2,7)};
      \addplot[dashed, color=penColor, domain=-7:7]{1};
    \end{axis}
\end{tikzpicture}
\end{image}

   \item Consider now the rational function $f(x) = \frac{x-1}{x^2-3x+2}$. Let's do as above, and start looking for horizontal asymptotes.
  \begin{align*}
    f(100) &\approx 0.010... \\
    f(1000) &\approx 0.001... \\
    f(1000) &\approx 0.000...
  \end{align*}
  This suggests that $f(x) \to 0$ as $x \to +\infty$. Similarly, you can convince yourself that $f(x) \to 0$ as $x \to -\infty$, so that $y=0$ is the only horizontal asymptote. And for vertical asymptotes, we'll again find the $x$-values for which $f(x)$ is undefined, and see whether any of those indicates a vertical asymptote. Noting that $x^2-3x+2 = (x-1)(x-2)$, we can see that $f(x)$ is undefined for $x=1$ and $x=2$. However, we may write that $$   f(x) = \frac{x-1}{x^2-3x+2} = \frac{x-1}{(x-1)(x-2)} = \frac{1}{x-2},  $$\emph{provided $x \neq 1$}. And $1/(x-2)$ does not increase (or decrease) without bound as $x$ approaches $1$ from either side --- in fact, it approaches $-1$. So, even though $f(x)$ is undefined for the value $x=1$, the line $x=1$ is \emph{not} a vertical asymptote for $f(x)$. But comparing the expression $f(x) = 1/(x-2)$ (again, valid for $x \neq 1$) with what we have previously seen for the function $1/x$, we see that $f(x) \to +\infty$ as $x \to 2^+$ and $f(x) \to -\infty$ as $x \to 2^-$, so that the line $x=2$ is a vertical asymptote for $f(x)$. Here's the graph:

      \begin{image}
        \begin{tikzpicture}
          \begin{axis}
            \addplot[samples=200,domain=-7:0.99]{1/(x-2)};
            \addplot[samples=200,domain=1.01:1.99]{1/(x-2)};
            \addplot[samples=200,domain=2.01:7]{1/(x-2)};
            \draw[fill=white](axis cs:1,-1)circle(1mm);
            \addplot[dashed, thick, color=penColor] coordinates {(2,-7) (2,7)};
            \addplot[thick, dashed, color=penColor, domain=-7:7]{0};
          \end{axis}
        \end{tikzpicture}
      \end{image}

  \begin{callout}
    {\bf Warning:} In item (b) of the above example, the function $f(x)$ given is \emph{not} the same thing as the function $g(x) = 1/(x-2)$. The function $f(x)$ is undefined for $x=1$, but $g(1)$ is defined, and it equals $-1$. The domain is a crucial part of the data defining a function. We will address these issues on Unit 5. 
  \end{callout}
  \end{enumerate}
\end{example}



\begin{exploration}
  A mathematical model for the population $P$, in thousands, of a certain species of bacteria, $t$ days after it is introduced to an environment, is given by $P(t) = \frac{200}{(7-t)^2}$, $0 \leq t < 7$.
  \begin{enumerate}
  \item Find and interprete $P(0)$.
  \item When will the population reach $200,000$?
  \item Determine the behavior of $P$ as $t \to 7^-$. Interpret this result graphically and within the context of the problem.
  \end{enumerate}
\end{exploration}

Now, you must be asking yourself if every time we want to test for vertical or horizontal asymptotes, we need to keep plugging values and guessing. Fortunately, the answer is ``no''. Here's what you need to know (formal justificatives require Calculus --- we'll be content with getting intuition for now):

\begin{callout}
  {\bf Theorem (locating horizontal asymptotes):} Assume that $f(x) = p(x)/q(x)$ is a rational function, and that the leading coefficients of $p(x)$ and $q(x)$ are $a$ and $b$, respectively.
  \begin{itemize}
  \item If the degree of $p(x)$ is the same as the degree of $q(x)$, then $y=a/b$ is the unique horizontal asymptote of the graph of $y=f(x)$.
  \item If the degree of $p(x)$ is less than the degree of $q(x)$, then $y=0$ is the unique horizontal asymptote of the graph of $y=f(x)$.
  \item If the degree of $p(x)$ is greater than the degree of $q(x)$, then the graph of $y=f(x)$ has no horizontal asymptotes.
  \end{itemize}
\end{callout}


The above theorem essentially says that one can detect horizontal asymptotes by looking at degrees and leading coefficients. Only the leading terms of $p(x)$ and $q(x)$ matter, and it makes no difference whether one considers $x\to +\infty$ or $x\to -\infty$. For example, how would you apply this to study horizontal asymptotes for the following function? $$ f(x) =  \frac{3x^6-5x^4+3x^3-3x^2 + 10x + 1}{5x^6 + 10000x^5 - 5x+2}$$

\begin{explanation}
  The degree of the numerator $p(x) = 3x^6-5x^4+3x^3-3x^2 + 10x + 1$ and the degree of the denominator $q(x) = 5x^6 + 10000x^5 - 5x+2$ are equal, namely, to $6$. Thus, since the leading coefficient of $p(x)$ is $3$ and the leading coefficient of $q(x)$ is $5$, we conclude that the line $y=3/5$ is the only horizontal asymptote for this rational function.
\end{explanation}

\begin{callout}
  {\bf Theorem (locating vertical asymptotes):} Assume that $f(x) = p(x)/q(x)$ is a rational function written in lowest terms, that is, such that $p(x)$ and $q(x)$ have no common zeros. Let $c$ be a real number for which $f(c)$ is undefined.
  \begin{itemize}
  \item If $q(c) \neq 0$, then the graph of $y = f(x)$ has a hole at the point $(c,f(c))$.
  \item If $q(c) = 0$, then the line $x=c$ is a vertical asymptote of the graph of $y=f(x)$.
  \end{itemize}
\end{callout}

The above theorem tells us how to distinguish vertical asymptotes and holes in the graph of a rational function. For example, consider $$  f(x) = \frac{x^2-4x+3}{x^2-3x+2}.  $$Is $f(x)$ in lowest terms? For which values of $x$ is this function undefined? How to go from there?

\begin{explanation}
You can find out if $f(x)$ is in lowest terms by just factoring both numerator and denominator, and by factoring the denominator, you will also find out which values of $x$ we have $f(x)$ undefined. Since $$   f(x) = \frac{x^2-4x+3}{x^2-3x+2} = \frac{(x-1)(x-3)}{(x-1)(x-2)} = \frac{x-3}{x-2},  $$we see that the values $f(1)$ and $f(2)$ are undefined, while the above equality holds for all $x$ except for $x=1$. In this simplified form, we may recognize $q(x) = x-2$. Since $q(1) = -1 \neq 0$, the graph of $y=f(x)$ has a hole at the point $(1,f(1))$, but since $q(2) = 0$, we have that the line $x=2$ is a vertical asymptote for the graph of $y=f(x)$.
\end{explanation}

\section{Slant asymptotes}

We are now ready to address the last type of asymptote a rational function may or may not have. And the reasoning is somewhat simple: why should we restrict ourselves to only vertical or horizontal asymptotes? This question itself motivates the name ``slant'' asymptote. Now, you know that the general equation of a line has the form $y=mx+b$, where $m$ is some slope and $b$ is the $y$-intercept. When $m=0$, we have a horizontal line, so when discussing slant asymptotes, we'll always assume that $m \neq 0$.


\begin{definition}
  The line $y=mx+b$, where $m\neq 0$, is called a \emph{slant asymptote} of the graph of a function $y=f(x)$ if as $x \to -\infty$ or as $x\to +\infty$, we have $f(x) \to mx+b$.
\end{definition}

Note that saying that $y=mx+b$ is a slant asymptote for the graph of $y=f(x)$ is the same thing as saying that $y=0$ is a horizontal asymptote for the graph of the difference function $y=f(x)-(mx+b)$.




\begin{example} Do the following rational functions have slant asymptotes? If so, what are their line equations?
  \begin{enumerate}
  \item $f(x) = \frac{x^2-4x+2}{1-x}$. \\[1em]
    \begin{explanation}
      When trying to find slant asymptotes, long division is the way to go. Performing it, we see that $$    \frac{x^2-4x+2}{1-x} = -x+3 - \frac{1}{1-x}.  $$Since $1/(1-x) \to 0$ as $x \to \infty$ or $x \to -\infty$ and $y=-x+3$ describes a line, we conclude that $y=-x+3$ is a slant asymptote for the graph of $y=f(x)$.
  \begin{image}
 \begin{tikzpicture}
     \begin{axis}
       \addplot[samples=200,domain=1.01:7]{(x^2-4*x+2)/(1-x)};
       \addplot[samples=200,domain=-7:0.99]{(x^2-4*x+2)/(1-x)};
       \addplot[dashed,thick,domain=-7:7, color=penColor]{-x+3};
     \end{axis}
 \end{tikzpicture}
 \end{image}
    \end{explanation}
    
  \item $f(x) = \frac{x^2-4}{x-2}$. \\[1em]
    \begin{explanation}
      We may just simplify it to $f(x) = x+2$, valid for all $x \neq 2$. We may regard this as a long division for which the remainder is zero, as in $$  f(x) = x+2+\frac{0}{x-2},  $$and since $0/(x-2) \to 0$ as $x\to +\infty$ or $x\to -\infty$, it follows that $y=x+2$ is a slant asymptote for the graph of $y=f(x)$, even though the graph is just said line with a hole!
      \begin{image}
        \begin{tikzpicture}
          \begin{axis}
            \addplot[domain=-7:7]{x+2};
           \draw[fill=white](axis cs:2,4)circle(1mm);
          \end{axis}
        \end{tikzpicture}
      \end{image}      
    \end{explanation}
  \item $f(x)=\frac{x^3-4x^2+5x-1}{x-1}$. \\[1em]
    \begin{explanation}
      Performing the long division, as before, we see that $$   f(x) = \frac{x^3-4x^2+5x-1}{x-1} = x^2-3x+2 + \frac{1}{x-1}.  $$As expected, $1/(x-1) \to 0$ when $x \to +\infty$ or $x\to -\infty$, but $y=x^2-3x+2$ is not a line equation. Hence there are no slant asymptotes for the graph of $f(x)$ (loosely speaking, the graph cannot be simultaneously asymptotic to a parabola and to a straight line).
      \begin{image}
        \begin{tikzpicture}
          \begin{axis}
            \addplot[samples=200,domain=1.01:7]{(x^3-4*x^2+5*x-1)/(x-1)};
            \addplot[samples=200,domain=-7:0.99]{(x^3-4*x^2+5*x-1)/(x-1)};
            \addplot[dashed,thick,smooth,domain=-7:7,color=penColor]{x^2-3*x+2};
          \end{axis}
        \end{tikzpicture}
      \end{image}
    \end{explanation}
  \item $f(x) = \frac{x^3+2x^2+x+1}{x^3-2x^2-x+1}$. \\[1em]
   \begin{explanation}
      Long division shows that: $$   f(x) = \frac{x^3+2x^2+x+1}{x^3-2x^2-x+1} = 1 + \underbrace{\frac{4x^2+2x}{x^3-2x^2-x+1}}_{\to 0}.  $$The indicated remainder goes to zero when $x \to +\infty$ or $x\to -\infty$ simply because the degree of the numerator is lower than the degree of the numerator. The remaining quotient does give us the asymptote $y=1$. But this is not a slant asymptote, it is a horizontal asymptote (as you might have expected).
%            \begin{image}
%        \begin{tikzpicture}
%          \begin{axis}
%            \addplot[samples=300,domain=-7:-0.83]{(x^3+2*x^2+x+1)/(x^3-2*x^2-x+1)};
%            \addplot[samples=300,domain=-0.8:0.554]{(x^3+2*x^2+x+1)/(x^3-2*x^2-x+1)};
%             \addplot[samples=300,domain=0.556:2.24]{(x^3+2*x^2+x+1)/(x^3-2*x^2-x+1)};
%             \addplot[samples=300,domain=2.26:7]{(x^3+2*x^2+x+1)/(x^3-2*x^2-x+1)};
%             \addplot[dashed,thick,domain=-7:7, color=penColor]{1};
%          \end{axis}
%        \end{tikzpicture}
%      \end{image}
      Note that asymptotes are really concerned about the end behavior of the function. In the above example, the line $y=1$ does intersect the graph of $y=f(x)$, but this is fine --- the graph still only approaches said line as $x \to +\infty$ and $x \to -\infty$.
   \end{explanation}
  \item $f(x) = \frac{x^2+1}{x^5-4}$. \\[1em]
    \begin{explanation}
      Since the degree of the numerator is smaller than the degree of the denominator, you can think of the long division as already having been performed, as in $$ f(x) = \frac{x^2+1}{x^5-4} = 0 + \frac{x^2+1}{x^5-4}.  $$Again, the remainder goes to zero when $x \to +\infty$ or $x\to -\infty$. So, the line $y=0$ would be an asymptote, but it is horizontal, not slant, as in the previous item.

  \begin{image}
 \begin{tikzpicture}
     \begin{axis}
       \addplot[samples=200,domain=-7:1.318]{(x^2+1)/(x^5-4)};
       \addplot[samples=200,domain=1.32:7]{(x^2+1)/(x^5-4)};
       \addplot[dashed,thick,domain=-7:7, color=penColor]{0};
     \end{axis}
 \end{tikzpicture}
 \end{image}

      
    \end{explanation}
  \end{enumerate}

\end{example}

The above examples suggest that if the degree of the numerator is at least two higher than the degree of the numerator, what survives outside the remainder has degree higher than one, and thus does not describe a line equation --- meaning no slant asymptotes. Similarly, if the degree of the numerator is equal or lower to the degree of the denominator, there's ``not enough quotient left" to describe a line equation. This is not a coincidence, but a general fact.

\begin{callout}
  {\bf Theorem (on slant asymptotes):} Let $f(x) = p(x)/q(x)$ be a rational function for which the degree of $p(x)$ is exactly one higher than the degree of $q(x)$. Then the graph of $y=f(x)$ has the slant asymptote $y=L(x)$, where $L(x)$ is the quotient obtained by dividing $p(x)$ by $q(x)$. If the degree of $p(x)$ is not exactly one higher than the degree of $q(x)$, there is no slant asymptote whatsoever.
\end{callout}

  Unlike what happened for horizontal and vertical asymptotes, the above theorem does not \emph{immediately} tell you what is the line equation describing the slant asymptote. We must resort to long division.

%\typeout{************************************************}
%\typeout{Summary}
%\typeout{************************************************}

\begin{summary}\begin{itemize}
%\item A \emph{rational} function is, as the name suggests, a function defined as a \emph{ratio} $f(x) = p(x)/q(x)$ of two polynomials $p(x)$ and $q(x)$. It makes sense for all real values of $x$ \emph{except} for those such that $q(x) = 0$, as one cannot divide by zero.
\item There are three types of asymptotes for rational functions: vertical asymptotes, horizontal asymptotes, and slant asymptotes. The latter occurs when the degree of the numerator is exactly one higher than the degree of the denominator.
\end{itemize}\end{summary}



\end{document}
