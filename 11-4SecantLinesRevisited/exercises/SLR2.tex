\documentclass{ximera}

\input{../../preamble.tex}

\author{Elizabeth Campolongo}
\acknowledgement{https://www.stitz-zeager.com/szca07042013.pdf}

\begin{document}
\begin{exercise}
Represent the cost, in hundreds, to produce $x$ thousand pencils by $C(x) = x^2-10x+27$. 
\begin{enumerate}
\item What is the average rate of change as production increases from 3000 to 5000 pencils?\\
$\answer{-2}$.

\item In real terms, what does this mean? \\
As production is increased from 3000 to 5000 pens, the cost
\begin{multipleChoice}
\choice{decreases at an average rate of $\$2$ per pencil produced.}
\choice{decreases by $\$200$.}
\choice[correct]{decreases at an average rate of $\$200$ per 1000 pencils produced.}
\choice{decreases at an average rate of $\$200$ per 100 pencils produced.}
\choice{decreases by $\$2000$.}
\choice{decreases at an average rate of $\$2$ per 100 pencils produced.}
\end{multipleChoice}
\end{enumerate}
\end{exercise}
\end{document}