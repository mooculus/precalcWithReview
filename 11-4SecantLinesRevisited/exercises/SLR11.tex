\documentclass{ximera}

\input{../../preamble.tex}

\author{Elizabeth Campolongo}

\begin{document}
\begin{exercise}
For the following functions,
find and simplify the difference quotients. [Hint: Remember our tricks for simplification, like multiplying by the conjugate, given in 12-3-3 Algebra of Secant lines.]

\begin{enumerate}
\item $f(x)= mx + b$, for $m \neq 0$.

$\frac{f(x+h)-f(x)}{h} = \answer{m}$.  
%\begin{exercise}
%What is the equation of the secant line to $f$ when $x = 1$ and $h=1$ (in point-slope form)?\\
%$y-1= \answer{7}(x-\answer{1})$
%\end{exercise}

\item $f(x)= ax^2 + bx+c$, for $a \neq 0$.

$\frac{f(x+h)-f(x)}{h} = \answer{2a}x + \answer{ah +b}$.  
%\begin{exercise}
%What is the equation of the secant line to $f$ when $x=-1$ and $h = 2$ (in point-slope form)?\\
%$y-6= \answer{-3}(x +\answer{1})$
%\end{exercise}


\item $f(x)= \sqrt{9x-81}$.

$\frac{f(x+h)-f(x)}{h} = \frac{3}{\answer{\sqrt{x+h-9} + \sqrt{x-9}}}$.  
%\begin{exercise}
%What is the equation of the secant line to $f$ when $x = 4$ and $h=5$ (in point-slope form)?\\
%$y-6\sqrt{2} = \answer{\frac{2\sqrt{2}}{5}}(x-\answer{9})$
%\end{exercise}

\item $f(x)= \sqrt{ax+b}$ for $a \neq 0$.

$\frac{f(x+h)-f(x)}{h} = \frac{a}{\answer{\sqrt{ax+ah+b} + \sqrt{ax+b}}}$.  
%\begin{exercise}
%What is the equation of the secant line to $f$ when $x = 4$ and $h=5$ (in point-slope form)?\\
%$y-6\sqrt{2} = \answer{\frac{2\sqrt{2}}{5}}(x-\answer{9})$
%\end{exercise}

\item $f(x)= x\sqrt{x}$.

$\frac{f(x+h)-f(x)}{h} = \frac{\answer{3x^2+3xh+h^2}}{\answer{(x+h)^{3/2}} +x^{3/2}}$.  
%\begin{exercise}
%What is the equation of the secant line to $f$ when $x = 4$ and $h=5$ (in point-slope form)?\\
%$y-6\sqrt{2} = \answer{\frac{2\sqrt{2}}{5}}(x-\answer{9})$
%\end{exercise}


\item $f(x)= \sqrt[3]{x}$. [Hint: the relation $a^3-b^3 = (a-b)(a^2+ab+b^2)$ may be useful.]

$\frac{f(x+h)-f(x)}{h} = \frac{1}{\answer{(x+h)^{2/3}+(x+h)^{1/3}x^{1/3} + x^{2/3}}}$.  
%\begin{exercise}
%What is the equation of the secant line to $f$ when $x = 0$ and $h=1$ (in point-slope form)?\\
%$y+ \answer{1} = \answer{\frac{1}{2}}x$
%\end{exercise}
	
\end{enumerate}
\end{exercise}
\end{document}