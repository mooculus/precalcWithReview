\documentclass{ximera}

\input{../preamble}
\author{Elizabeth Campolongo}
\license{Creative Commons Attribution-ShareAlike 4.0 International License}
\acknowledgement{https://www.stitz-zeager.com/szprecalculus07042013.pdf}

\title{Algebra of Secant Lines}

\begin{document}
\begin{abstract}
  
\end{abstract}
\maketitle


%\typeout{************************************************}
%\typeout{Motivating Questions}
%\typeout{************************************************}

\begin{motivatingQuestions}\begin{itemize}
\item What are some algebra techniques that allow us to simplify the equation of a secant line?
\item Why is this important?
\end{itemize}\end{motivatingQuestions}


%\typeout{************************************************}
%\typeout{Subsection Introduction}
%\typeout{************************************************}

\section{Introduction}

Given the graph of a function $y = f(x)$, we have discussed methods to determine the slope of the secant line between two points, $(a,f(a))$ and $(b,f(b))$, on the graph. We know that this slope represents the average rate of change of the function $f$ on the interval $[a,b]$, denoted by $\av_{[a,b]}$. Both of these can be rewritten by letting $b = h-a$, so that we have the value $h$ representing the horizontal distance between the points. This means that as $h\rightarrow 0$, the secant line, or the average rate of change of the function, approaches a value known as the slope of the tangent line of $f$ at $a$. This will be discussed extensively in future calculus courses, but in this section we will focus on tools to simplify the expression $\av_{[a,a+h]}$, as they are essential to calculating this limit.

%We have discussed secant lines to the graph of a function $y=f(x)$. If $(a,f(a))$ and $(b,f(b))$ are two such points, the slope $$m=\frac{f(b)-f(a)}{b-a}$$of the secant line passing through such points is, in fact, the average rate of change of $f$ on the interval $[a,b]$. It is also convenient to let $h=b-a$, so the slope expression becomes $$m=\frac{f(a+h)-f(a)}{h},$$which is really equivalent to the previous one.
%
%\begin{image}[2in]
%		  \begin{tikzpicture}
%		  \draw [thick, scale = 3, domain = -.5:2, smooth, variable =\x] plot ({\x},  {3/4*exp(\x)*cos(deg(\x)) + 1/2*exp(\x)*sin(deg(\x))});%{\x*(3-2*\x*\x)});
%		    \coordinate (C) at (0,2.8);
%		    \coordinate (D) at (5,1);
%		    \coordinate (E) at (5,7.5);
%		   % \draw[decoration={brace,mirror,raise=.2cm},decorate,thin] (0,2)--(4,2);
%		   % \draw[decoration={brace,mirror,raise=.2cm},decorate,thin] (4,2)--(4,6.5);
%		  %  \draw[decoration={brace,raise=.2cm},decorate,thin] (0,2)--(4,6.5);
%		    \draw[blue] (C)--(E)--cycle;
%		 %   \draw[dotted] (C) edge (D);
%		%    \draw[dotted] (E) edge (D);
%		      \node at (2,7) {$y= f(x)$};
%		   \node at (.4,4) {$(a, f(a))$};
%		   % \node[rotate=-90] at (4+.5,4) {A'};
%		 %   \node[rotate=35] at (.8,4.8) {$f(x)-f(x_1) = m(x-x_1)$};
%		    \node at (5.1,7.8) {$(a+h,f(a+h))$};
%		%    \node at (-4.7, 1) {$(x_1,y_1)$};
%		  \end{tikzpicture}
%		\end{image}
%

%\begin{image}[2in]
%		  \begin{tikzpicture}
%		  \draw [thick, scale = 3, domain = -.5:2, smooth, variable =\x] plot ({\x},  {3/4*exp(\x)*cos(deg(\x)) + 1/2*exp(\x)*sin(deg(\x))});%{\x*(3-2*\x*\x)});
%		    \coordinate (C) at (0,2.8);
%		    \coordinate (D) at (5,1);
%		    \coordinate (E) at (5,7.5);
%		    \coordinate (A) at (0,2);
%		    \coordinate (B) at (2.4,6);
%		   % \draw[decoration={brace,mirror,raise=.2cm},decorate,thin] (0,2)--(4,2);
%		   % \draw[decoration={brace,mirror,raise=.2cm},decorate,thin] (4,2)--(4,6.5);
%		  %  \draw[decoration={brace,raise=.2cm},decorate,thin] (0,2)--(4,6.5);
%		    \draw[blue] (C)--(E)--cycle;
%		    \draw[red] (A)--(B)--cycle;
%		 %   \draw[dotted] (C) edge (D);
%		%    \draw[dotted] (E) edge (D);
%		      \node at (6.4,5.6) {$y= f(x)$};
%		   \node at (.4,4) {$(a, f(a))$};
%		   % \node[rotate=-90] at (4+.5,4) {A'};
%		 %   \node[rotate=35] at (.8,4.8) {$f(x)-f(x_1) = m(x-x_1)$};
%		    \node at (5.1,7.8) {$(a+h,f(a+h))$};
%		%    \node at (-4.7, 1) {$(x_1,y_1)$};
%		  \end{tikzpicture}
%		\end{image}

%tangent line parallel to secant line
%\begin{image}[2in]
%		  \begin{tikzpicture}
%		  \draw [thick, scale = 3, domain = -.5:2, smooth, variable =\x] plot ({\x},  {3/4*exp(\x)*cos(deg(\x)) + 1/2*exp(\x)*sin(deg(\x))});%{\x*(3-2*\x*\x)});
%		    \coordinate (C) at (0,2.8);
%		    \coordinate (D) at (5,1);
%		    \coordinate (E) at (5,7.5);
%		    \coordinate (A) at (1.5,5.41);
%		    \coordinate (B) at (4.5,8.23);
%		   % \draw[decoration={brace,mirror,raise=.2cm},decorate,thin] (0,2)--(4,2);
%		   % \draw[decoration={brace,mirror,raise=.2cm},decorate,thin] (4,2)--(4,6.5);
%		  %  \draw[decoration={brace,raise=.2cm},decorate,thin] (0,2)--(4,6.5);
%		    \draw[blue] (C)--(E)--cycle;
%		    \draw[red] (A)--(B)--cycle;
%		 %   \draw[dotted] (C) edge (D);
%		%    \draw[dotted] (E) edge (D);
%		      \node at (6.4,5.6) {$y= f(x)$};
%		   \node at (.4,4) {$(a, f(a))$};
%		   % \node[rotate=-90] at (4+.5,4) {A'};
%		 %   \node[rotate=35] at (.8,4.8) {$f(x)-f(x_1) = m(x-x_1)$};
%		    \node at (5.1,7.8) {$(a+h,f(a+h))$};
%		%    \node at (-4.7, 1) {$(x_1,y_1)$};
%		  \end{tikzpicture}
%		\end{image}




\section{Definitions and examples}
%relate to difference of squares: rethinking difference of squares through square roots
%simplfying polynomials and fractions!

Recall the special formula for difference of squares, $a^2-b^2 = (a-b)(a+b)$. For non-square values of $a$ and $b$ we can use the same idea to rationalize differences (or sums) of square roots through multiplication by the corresponding sum (or difference), which we call the {\it conjugate}.
Given any expression $\sqrt{a} \pm \sqrt{b}$, $a,b$ real numbers, the conjugate of this expression is $\sqrt{a} \mp \sqrt{b}$. Multiplying such an expression by its conjugate rationalizes it through the distributive property: $(\sqrt{a} + \sqrt{b})\cdot(\sqrt{a} - \sqrt{b}) = (\sqrt{a})^2 + \sqrt{a}\sqrt{b} - \sqrt{a}\sqrt{b} - (\sqrt{b})^2 = a - b$

\begin{callout}
  {\bf Definition:} 
  Given any difference of positive values $a-b$, we know from the difference of squares, that $a-b = (\sqrt{a}-\sqrt{b})(\sqrt{a}+\sqrt{b})$. \\
  The sum $\sqrt{a}+\sqrt{b}$ is the {\it conjugate} of the difference $\sqrt{a}-\sqrt{b}$.
  Likewise, the difference $\sqrt{a}-\sqrt{b}$ is the {\it conjugate} of the sum $\sqrt{a}+\sqrt{b}$.
 
 Multiplying such an expression by its conjugate will rationalize the expression.
%Given an expression of the form $\frac{\sqrt{x+h} - \sqrt{x}}{h}$, we can multiply by the {\bf conjugate} of the numerator to rationalize it. The conjugate of the numerator is $\sqrt{x+h}+\sqrt{x}$. 
  \end{callout}
%
Note that this is one of the most important tools in your simplification toolbox. Other tools include simplifying polynomials and fractions (finding the common denominator), moving coefficients inside or outside the square root, and the trigonometric identities we learned earlier.

%[moving coefficient inside or outside square root, absolute value functions]
%
\begin{example}
For the following, find the difference quotient. Simplify as much as possible
 %
  \begin{enumerate}
  \item $f(x) = \sqrt{x}$, $x \geq 0$

    \begin{explanation}
    We consider $h>0$ to avoid any potential undefined values plugged into our function $f$ since its domain is $[0,\infty)$.
    \begin{equation}\label{sqrtDQ}
     \frac{f(x+h) - f(x)}{h} = \frac{\sqrt{x+h} - \sqrt{x}}{h}
    \end{equation}
    %
    Observe that we cannot combine any terms in \eqref{sqrtDQ}, but the numerator is of the form $\sqrt{a} - \sqrt{b}$. Hence, we will multiply by the conjugate to rationalize the numerator:
   \begin{align}
   \frac{\sqrt{x+h} - \sqrt{x}}{h} & =  \frac{(\sqrt{x+h} - \sqrt{x})}{h} \cdot  \frac{(\sqrt{x+h} + \sqrt{x})}{(\sqrt{x+h} + \sqrt{x})} \label{multbyC} \\
   &=  \frac{(\sqrt{x+h})^2 - (\sqrt{x})^2}{h(\sqrt{x+h}+\sqrt{x})}\label{diffSq}
   \end{align}
   %
   Remember, in \eqref{multbyC}, that in order to avoid changing the value of the expression, we must multiply by the conjugate over itself, i.e., multiply by 1. Then \eqref{diffSq} has a difference of squares in the numerator and is equal to 
   %
   \begin{align*}
    \frac{(x+h) - x}{h(\sqrt{x+h} + \sqrt{x})} & = \frac{h}{h(\sqrt{x+h}+\sqrt{x})} \\
    &=\frac{1}{\sqrt{x+h}+\sqrt{x}},
   \end{align*} 
   by cancelling out the $h$ in the numerator and the denominator. 
   
   This expression that we have found is now in a form that allows us to consider what happens when $h \rightarrow 0$ by removing the $h$ from the denominator. 
       \end{explanation}
       %
  \item $g(x) = \sqrt{8-2x}$, $x \geq 4$.\\
  \begin{explanation}
  We consider $h>0$ to avoid any potential undefined values plugged into our function $g$ since its domain is $[4,\infty)$.

  \begin{equation}\label{sqDQ}
   \frac{g(x+h) - g(x)}{h} = \frac{\sqrt{8-4(x+h)} - \sqrt{8-4x}}{h}
  \end{equation}
  %
  Once again, we cannot combine any terms in the numerator of \eqref{sqDQ}, so we will multiply by the conjugate to rationalize the numerator, hoping we will be able to simplify the equation. \eqref{sqDQ} is equal to
  %
  \begin{align*}
 % \frac{\sqrt{8-2(x+h)} - \sqrt{8-2x}}{h} &=
 \frac{(\sqrt{8-4(x+h)} - \sqrt{8-4x})}{h} &\cdot \frac{(\sqrt{8-4(x+h)} + \sqrt{8-4x})}{(\sqrt{8-4(x+h)} + \sqrt{8-4x})} \\
  &=  \frac{(\sqrt{8-4(x+h)})^2 - (\sqrt{8-4x})^2}{h(\sqrt{8-4(x+h)} + \sqrt{8-4x})} \\
  &= \frac{(8-4x-4h) - (8-4x)}{h(\sqrt{4(2-(x+h))} + \sqrt{4(2-x)})}\\
  &= \frac{-4h}{2h(\sqrt{2-(x+h)} + \sqrt{2-x})}
  \end{align*}
  %
  Now, we simply cancel the $2h$ in the numerator and the denominator, giving
  $$ \frac{g(x+h) - g(x)}{h} = \frac{-2}{\sqrt{2-(x+h)} + \sqrt{2-x}}.$$
  \end{explanation}
  %
  \item $f(x) = \cos(2x)$
 \\
 \begin{explanation}
 Note that $\cos(z)$ is defined for all real numbers $z$, so we need not worry about the values of $x$ and $h$ plugged into the difference quotient formula.
 \begin{equation*}\label{cosDQ}
 \frac{f(x+h) - f(x)}{h} = \frac{\cos(2(x+h)) - \cos(2x)}{h}
 \end{equation*}
 Now, we can expand out using the angle sum formula for cosine:
 %
 \begin{align*}
 \frac{\cos(2x)\cos(2h)-\sin(2x)\sin(2h) - \cos(2x)}{h} 
 &= \frac{\cos(2x)(\cos(2h)-1)- \sin(2x)\sin(2h)}{h} \\
 &=  \cos(2x)\frac{\cos(2h)-1}{h}- \sin(2x)\frac{\sin(2h)}{h}
 \end{align*}
 
Here we can plug in decreasing values of $h$ for cosine and sine and start to notice a pattern.\\
Explore this further by changing the $x$ and $h$ values in the following Desmos graph:
\desmos{lf9wkgurwz}{800}{600}
 \end{explanation}
 %
 %
  \item $f(x) = |x-1|$ %for $|x|>1|$. 
  \\
  %
   \begin{explanation}
   We will consider two regions and ranges of $h$: (1) $x \in (-\infty,1)$ with $h<0$ and (2) $x\in(1, \infty)$ with $h>0$. 
   %Additionally, we will consider the case for $h$ positive and $h$ negative each separately. 
   
  Let's start with region (1), where $x<1$ and $h<0$. From this, we know that $x+h-1<x-1<0$, so $|x-1|=-(x-1)$. Hence we have
  \begin{align*}
 \frac{f(x+h) - f(x)}{h} &= \frac{|x+h-1| - |x-1|}{h}\\
 &= \frac{-x-h+1 + (x-1)}{h}\\
 &= \frac{-h}{h} = -1
 \end{align*}
 
 Alternatively, if we consider region (2), where $x>1$ and $h>0$, then we have $x+h-1>x-1>0$, so that
  \begin{align*}
 \frac{|x+h+1| - |x+1|}{h} &= \frac{x+h-1 - (x-1)}{h}\\
 &= \frac{h}{h} = 1
 \end{align*}
 
 Notice that this is not as clear-cut if we consider say $x<1$ and $h>0$. Then we would need to consider if $h$ is large enough that $x+h>1$. Let's explore this some more.
 
 Let $x<1$ and $h>0$. Further, assume $h>1-x$, then 
  \begin{align*}
 \frac{f(x+h) - f(x)}{h} &= \frac{|x+h-1| - |x-1|}{h}\\
 &= \frac{x+h+1 + (x-1)}{h}\\
 &= \frac{2x + h}{h}
 \end{align*}
Alternatively, if $h<1-x$, then
 \begin{align*}
 \frac{f(x+h) - f(x)}{h} &= \frac{|x+h-1| - |x-1|}{h}\\
 &= \frac{-x-h+1 + (x-1)}{h}\\
 &= \frac{-h}{h} = -1
 \end{align*}
As when $x<1$ and $h<0$.
 \end{explanation}
 
 \item $g(x) =  \frac{2x}{x^2 + 3}$\\
 \begin{explanation}
 First note that the denominator of our function $g$ is greater than zero for all real values of $x$, so the function is defined for all real numbers. Thus, we may calculate the difference quotient without concern for input values of $x$ and $h$.
 %
 \begin{equation*}
 \frac{g(x+h) - g(x)}{h} = \frac{\frac{2(x+h)}{(x+h)^2 + 3} - \frac{2x}{x^2+3}}{h}
 \end{equation*}
 %
 This expression for the difference quotient looks rather messy, so let's find the common denominator and see if we can cancel out some terms in the numerator by combining the fractions. We will leave the terms in the denominator in their current format, but multiply out the $(x+h)^2$ in the numerator for ease of simplification. 
 
 Note that the common denominator is $((x+h)^2+3)(x^2 + 3)$. Then we have
 \begin{align*}
  \frac{g(x+h) - g(x)}{h} &= \frac{\frac{(2x+2h)(x^2+3) - 2x(x^2+2xh+h^2+3)}{((x+h)^2+3)(x^2 + 3)}}{h} \\
  &= \frac{2x^3 +2x^2h + 6x + 6h - (2x^3 +4x^2h + 2xh^2 +6x)}{h\big((x+h)^2+3\big)(x^2 + 3)}
 \end{align*}
 Observe that we have combined the denominators now and have many common terms in the numerator that can be subtracted from each other, so that
 %
 \begin{equation*}
 \frac{g(x+h) - g(x)}{h} = \frac{-2x^2h + 6h -2xh^2}{h\big((x+h)^2+3\big)(x^2 + 3)}.
\end{equation*}
%
Now, all the terms in the numerator have a factor of $h$, so we can cancel the $h$ in the numerator and denominator for a final, simplified difference quotient of
 \begin{equation*}
\frac{-2x^2-2xh + 6}{\big((x+h)^2+3\big)(x^2 + 3)}.
\end{equation*}
 \end{explanation}
  \end{enumerate}
\end{example}



%\typeout{************************************************}
%\typeout{Summary}
%\typeout{************************************************}

\begin{summary}
Useful tools for simplification:
\begin{itemize}
\item Simplifying polynomials.
\item Simplifying fractions by finding common denominators.
\item Multiplying by the conjugate to rationalize the numerator.
\item Considering regions for absolute value functions. 
\end{itemize}\end{summary}




\end{document}
